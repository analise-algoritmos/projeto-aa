\begin{center}
\textbf{ABSTRACT}\
\end{center}
The analysis of algorithms, or complexity analysis, is a mechanism for understanding and evaluating an algorithm in two main aspects: correctness — assessing the accuracy of the method through a mathematical perspective — and efficiency, which examines the algorithm’s use of computational resources such as memory and execution time. This analysis contributes to the more appropriate and effective application of algorithms to practical problems. This monograph describes approximately fifty sorting algorithms developed across different historical periods — from the earliest ones, created in the nineteenth century, to the most recent proposals in the literature. The project is organized into two major parts: the first adopts a theoretical perspective, presenting the algorithms in terms of their construction, operation, and analyses of time and space complexity, as well as providing their implementations in three programming languages — Python, C, and C++; the second part focuses on an empirical perspective, in which computational experiments are used to explore the performance of these algorithms when applied to relatively large problem instances, allowing us to simulate their expected behavior in the asymptotic context. The results confirm the hierarchy predicted among complexity classes, while also highlighting the influence of factors such as internal optimizations, computational overhead, memory consumption, and data organization. As an additional contribution, the project produced libraries in the three aforementioned languages, which have been made publicly available in a GitHub repository.
\vspace{0.2cm}

\noindent\textbf{Keywords}: Sorting algorithms, optimization, memory, computational performance
