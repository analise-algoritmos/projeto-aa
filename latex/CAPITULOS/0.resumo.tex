\begin{center}
\textbf{RESUMO}\\
\end{center}
A análise de algoritmos ou análise de complexidade é um mecanismo para compreender e avaliar um algoritmo em dois aspectos principais: correção - avaliando a exatidão do método praticado utilizando-se do viés matemático para tal;e análise, onde se avalia a eficiência do algoritmo no viés de recursos consumidos, sejam estes memória ou tempo de execução. Essa análise contribui para a aplicação mais adequada e eficiente de algoritmos em problemas práticos. Esta monografia descreve aproximadamente cinquenta algoritmos de ordenação que foram desenvolvidos ao longo dde diferentes períodos históricos - desde os mais antigos, desenvolvidos no século XIX como os últimos propostos na literatura. Este projeto apresenta dois grandes blocos específicos: o primeiro sob a ótica teórica, apresentando os algoritmos em suas especificações de construção, funcionamento e análises das complexidades de tempo e espaço, além da apresentação de suas implementações em três linguagens,Python, C, C++; e o segundo com viés empírico, na qual é explorado através de experimentos computacionais as performances destes algoritmos atuando sob diferentes instâncias relativamente grandes de problemas, de modo a simularmos o comportamento experado no contexto assintótico. Os resultados confirmam a hierarquia prevista entre as classes de complexidade, mas também evidenciam a influência de fatores como otimizações internas, overhead computacional, consumo de memória e organização dos dados. Como contribuição adicional, o projeto resultou na criação de bibliotecas nas três linguagens citadas, disponibilizadas publicamente em um repositório no GitHub.
\vspace{0.2cm}

\noindent{\textbf{Palavras-chave}: Algoritmos de ordenação, otimização, memória, desempenho computacional}  