\chapter{Algoritmos com complexidades de tempo 
\texorpdfstring{$O(n)$}{O(n)}}

Nesta seção apresentamos algoritmos de ordenação cuja complexidade de tempo pode ser considerada linear em determinados cenários. Esses métodos exploram propriedades dos dados de entrada, como o tamanho do domínio ou a representação numérica. 

\textcolor{blue}{Sugiro que vocês expressem os cálculos das complexidades considerando as constantes envolvidas nas funções. Posso explicar isto na aula da segunda semana de outubro.}

\section{Counting sort}
\textbf{Descrição:} O Counting Sort não utiliza comparações entre elementos. Em vez disso, ele conta quantas vezes cada valor aparece e, com base nessas contagens, reconstrói o vetor ordenado. É especialmente eficiente quando os valores de entrada pertencem a um intervalo pequeno em relação ao número de elementos.

\begin{exmp}
Considere ordenar o vetor $A = [4, 2, 2, 8, 3, 3, 1]$ usando o \textit{Counting Sort}.  

\begin{enumerate}
    \item \textbf{Contagem da frequência:}  
    Criamos um vetor $count$ com tamanho igual ao maior valor de $A$ (neste caso, 8) e inicializamos com zeros.  
    Em seguida, contamos quantas vezes cada número aparece:  count = [0, 1, 2, 2, 1, 0, 0, 0, 1], onde $count[i]$ indica quantas vezes o número $i$ aparece em $A$.

    \item \textbf{Cálculo das posições acumuladas:}  
    Transformamos o vetor $count$ em um vetor de posições acumuladas, somando os valores anteriores: count = [0, 1, 3, 5, 6, 6, 6, 6, 7]. Agora, $count[i]$ indica a posição final do último elemento $i$ no vetor ordenado.

    \item \textbf{Reconstrução do vetor ordenado:}  
    Percorremos o vetor $A$ da direita para a esquerda e colocamos cada elemento na posição correta no vetor $output$:  output = [1, 2, 2, 3, 3, 4, 8]. Finalmente, copiamos o vetor $output$ de volta para $A$.

\end{enumerate}
\end{exmp}

\begin{center}
\begin{minipage}{.9\linewidth}
\begin{algorithm}[H]
\DontPrintSemicolon
\textbf{countingSort(values: array of int, n: integer, k: integer)}

\For{$i \gets 0$ \KwTo $k$}{
    $count[i] \gets 0$\;
}
\For{$i \gets 0$ \KwTo $n-1$}{
    $count[values[i]] \gets count[values[i]] + 1$\;
}
\For{$i \gets 1$ \KwTo $k$}{
    $count[i] \gets count[i] + count[i-1]$\;
}
\For{$i \gets n-1$ \KwTo $0$}{
    $output[count[values[i]]-1] \gets values[i]$\;
    $count[values[i]] \gets count[values[i]] - 1$\;
}
\For{$i \gets 0$ \KwTo $n-1$}{
    $values[i] \gets output[i]$\;
}
\caption{Counting sort.}
\label{lab:alg-countingSort}
\end{algorithm}
\end{minipage}
\end{center}

\vspace{1em}

\begin{lstlisting}[language=C,caption={Counting sort em C},captionpos=t]
void countingSort(int arr[], int n, int k) {
    int count[k+1];
    int output[n];
    for(int i=0; i<=k; i++) count[i] = 0;
    for(int i=0; i<n; i++) count[arr[i]]++;
    for(int i=1; i<=k; i++) count[i] += count[i-1];
    for(int i=n-1; i>=0; i--) {
        output[count[arr[i]]-1] = arr[i];
        count[arr[i]]--;
    }
    for(int i=0; i<n; i++) arr[i] = output[i];
}
\end{lstlisting}

\begin{lstlisting}[language=C++,caption={Counting sort em C++},captionpos=t]
#include <vector>
using namespace std;

void countingSort(vector<int>& arr, int k) {
    int n = arr.size();
    vector<int> count(k + 1, 0), output(n);

    for (int num : arr)
        count[num]++;

    for (int i = 1; i <= k; i++)
        count[i] += count[i - 1];

    for (int i = n - 1; i >= 0; i--) {
        output[count[arr[i]] - 1] = arr[i];
        count[arr[i]]--;
    }

    arr = output;
}
\end{lstlisting}

\begin{lstlisting}[language=Python,caption={Counting sort em Python},captionpos=t]
def counting_sort(arr, k):
    count = [0] * (k+1)
    output = [0] * len(arr)
    for num in arr:
        count[num] += 1
    for i in range(1, len(count)):
        count[i] += count[i-1]
    for num in reversed(arr):
        output[count[num]-1] = num
        count[num] -= 1
    return output
\end{lstlisting}

\subsection{Análise de complexidade do algoritmo \ref{lab:alg-countingSort}}

Consideramos que o vetor $values$ tem comprimento $n$ e que os valores são inteiros no intervalo $[0, k]$. O algoritmo possui os seguintes \textit{loops}:

\begin{enumerate}
    \item Inicialização do vetor $count$ (tamanho $k+1$):
    \[
        \text{executa } k+1 \text{ atribuições em } O(1) \Rightarrow O(k)
    \]
    
    \item Contagem das ocorrências dos elementos do vetor $values$:
    \[
        n \text{ iterações, cada uma com operação } O(1) \Rightarrow O(n)
    \]
    
    \item Acúmulo das contagens no vetor $count$:
    \[
        k \text{ iterações, cada uma com operação } O(1) \Rightarrow O(k)
    \]
    
    \item Construção do vetor ordenado $output$:
    \[
        n \text{ iterações, cada uma com operação } O(1) \Rightarrow O(n)
    \]
    
    \item Cópia do vetor $output$ de volta para $values$:
    \[
        n \text{ iterações, cada uma com operação } O(1) \Rightarrow O(n)
    \]
\end{enumerate}

Portanto, o número total de operações é proporcional a:
\[
T(n, k) = O(k) + O(n) + O(k) + O(n) + O(n) = O(n + k)
\]

\noindent Em termos assintóticos, podemos escrever:
\[
T(n, k) \leq c_1 \cdot n + c_2 \cdot k
\]
para constantes $c_1, c_2 > 0$. Logo:
\[
T(n, k) \in O(n + k)
\]

\begin{itemize}
    \item \textbf{Melhor caso:} $O(n + k)$, pois todos os \textit{loops} sempre executam a mesma quantidade de iterações, independentemente da ordem dos elementos.
    
    \item \textbf{Caso médio:} $O(n + k)$, pois a complexidade depende apenas do tamanho do vetor e do domínio $k$, não da disposição dos elementos.
    
    \item \textbf{Pior caso:} $O(n + k)$, mesmo raciocínio: a ordem dos elementos não altera o número de operações.
\end{itemize}


\section{Radix sort}
\textbf{Descrição:} O Radix Sort ordena números inteiros processando dígito por dígito, da menor para a maior ordem, utilizando Counting Sort como sub-rotina estável. É eficiente quando o número de dígitos $d$ não é muito grande e o domínio dos dígitos é pequeno.

\begin{exmp}
Considere ordenar o vetor $A = [170, 45, 75, 90, 802, 24, 2, 66]$ usando o \textit{Radix Sort}.  

\begin{enumerate}
    \item \textbf{Ordenação pelo dígito das unidades:}  
    Extraímos o dígito das unidades de cada elemento e aplicamos Counting Sort:  
    dígitos das unidades = [0,5,5,0,2,4,2,6]  
    Após o Counting Sort pelos dígitos das unidades, o vetor fica: $[170, 90, 802, 2, 24, 45, 75, 66]$.

    \item \textbf{Ordenação pelo dígito das dezenas:}  
    Extraímos o dígito das dezenas e aplicamos Counting Sort:  
    dígitos das dezenas = [7,9,0,0,2,4,7,6]  
    Após o Counting Sort pelos dígitos das dezenas, o vetor fica: $[802, 2, 24, 45, 66, 170, 75, 90]$.

    \item \textbf{Ordenação pelo dígito das centenas:}  
    Extraímos o dígito das centenas e aplicamos Counting Sort:  
    dígitos das centenas = [1,0,0,0,8,0,0,0]  
    Após o Counting Sort pelos dígitos das centenas, o vetor final ordenado é: $[2, 24, 45, 66, 75, 90, 170, 802]$.
\end{enumerate}
\end{exmp}


\begin{center}
\begin{minipage}{.9\linewidth}
\begin{algorithm}[H]
\DontPrintSemicolon
\textbf{radixSort(values: array of int, n: integer)}

$m \gets$ maior valor em values\;
$exp \gets 1$\;
\While{$m / exp > 0$}{
    countingSort(values, n, exp)\;
    $exp \gets exp \times 10$\;
}
\caption{Radix sort.}
\label{lab:alg-radixSort}
\end{algorithm}
\end{minipage}
\end{center}

\begin{lstlisting}[language=C,caption={Radix sort em C},captionpos=t]
void countingSortRadix(int arr[], int n, int exp) {
    int output[n], count[10] = {0};
    for(int i=0; i<n; i++)
        count[(arr[i]/exp)%10]++;
    for(int i=1; i<10; i++)
        count[i] += count[i-1];
    for(int i=n-1; i>=0; i--) {
        output[count[(arr[i]/exp)%10]-1] = arr[i];
        count[(arr[i]/exp)%10]--;
    }
    for(int i=0; i<n; i++) arr[i] = output[i];
}

void radixSort(int arr[], int n) {
    int m = arr[0];
    for(int i=1; i<n; i++)
        if(arr[i] > m) m = arr[i];
    int exp = 1;
    while(m/exp > 0) {
        countingSort(arr, n, exp);
        exp *= 10;
    }
}
\end{lstlisting}

\begin{lstlisting}[language=C++,caption={Radix sort em C++},captionpos=t]
#include <vector>
#include <algorithm>
using namespace std;

void countingSortRadix(vector<int>& arr, int exp) {
    int n = arr.size();
    vector<int> output(n);
    int count[10] = {0};

    for (int num : arr)
        count[(num / exp) % 10]++;

    for (int i = 1; i < 10; i++)
        count[i] += count[i - 1];

    for (int i = n - 1; i >= 0; i--) {
        int idx = (arr[i] / exp) % 10;
        output[count[idx] - 1] = arr[i];
        count[idx]--;
    }

    arr = output;
}

void radixSort(vector<int>& arr) {
    int m = *max_element(arr.begin(), arr.end());
    for (int exp = 1; m / exp > 0; exp *= 10)
        countingSortRadix(arr, exp);
}
\end{lstlisting}

\begin{lstlisting}[language=Python,caption={Radix sort em Python},captionpos=t]
def counting_sort_radix(arr, exp):
    n = len(arr)
    output = [0] * n
    count = [0] * 10
    for num in arr:
        index = (num // exp) % 10
        count[index] += 1
    for i in range(1, 10):
        count[i] += count[i-1]
    for num in reversed(arr):
        index = (num // exp) % 10
        output[count[index]-1] = num
        count[index] -= 1
    return output

def radix_sort(arr):
    m = max(arr)
    exp = 1
    while m // exp > 0:
        arr = counting_sort_radix(arr, exp)
        exp *= 10
    return arr
\end{lstlisting}

\subsection{Análise de complexidade do algoritmo \ref{lab:alg-radixSort}}

Seja $n$ o número de elementos e $d$ o número de dígitos no maior valor. Cada passo de Counting Sort tem complexidade $O(n + k)$, onde $k=10$ (base decimal). Assim:

\[
T(n) = d \cdot O(n + k) = O(d \cdot (n + k))
\]

\begin{itemize}
    \item \textbf{Melhor caso:} $O(d \cdot (n+k))$, todos os elementos já ordenados não alteram o número de iterações.
    \item \textbf{Caso médio:} $O(d \cdot (n+k))$, a complexidade depende do número de dígitos $d$.
    \item \textbf{Pior caso:} $O(d \cdot (n+k))$, ordem dos elementos não influencia.
\end{itemize}

Complexidade de espaço: $O(n+k)$ para armazenar o vetor de contagem e o vetor de saída.

\section{Bucket sort}
\textbf{Descrição:} O Bucket Sort distribui os elementos em "baldes" (\textit{buckets}), ordena cada balde individualmente (geralmente com Insertion Sort) e depois concatena os baldes. Funciona melhor quando os elementos estão distribuídos uniformemente.

\begin{exmp}
Considere ordenar o vetor $A = [0.78, 0.17, 0.39, 0.26, 0.72, 0.94, 0.21, 0.12, 0.23, 0.68]$ usando o \textit{Bucket Sort}.  

\begin{enumerate}
    \item \textbf{Distribuição nos baldes:}  
    Criamos $n = 10$ baldes (um para cada faixa de 0.0 a 1.0) e distribuímos cada elemento de $A$ no balde correspondente:  
    \begin{itemize}
        \item Balde 0: [0.12]  
        \item Balde 1: [0.17, 0.21, 0.23, 0.26]  
        \item Balde 2: [0.39]  
        \item Balde 6: [0.68, 0.72, 0.78]  
        \item Balde 9: [0.94]
    \end{itemize}

    \item \textbf{Ordenação de cada balde:}  
    Aplicamos Insertion Sort dentro de cada balde:
    \begin{itemize}
        \item Balde 1: [0.17, 0.21, 0.23, 0.26] (já ordenado)  
        \item Balde 6: [0.68, 0.72, 0.78] (já ordenado)  
        \item Os demais baldes contêm apenas um elemento.
    \end{itemize}

    \item \textbf{Concatenação dos baldes:}  
    Unimos todos os baldes em ordem, obtendo o vetor final ordenado:  
    $[0.12, 0.17, 0.21, 0.23, 0.26, 0.39, 0.68, 0.72, 0.78, 0.94]$.
\end{enumerate}
\end{exmp}

\begin{center}
\begin{minipage}{.9\linewidth}
\begin{algorithm}[H]
\DontPrintSemicolon
\textbf{bucketSort(values: array of float, n: integer)}

\For{$i \gets 0$ \KwTo $n-1$}{
    distribuir values[i] em um balde\;
}
\For{cada balde $b$}{
    ordenar $b$ com insertionSort\;
}
concatenar todos os baldes em ordem\;

\caption{Bucket sort.}
\label{lab:alg-bucketSort}
\end{algorithm}
\end{minipage}
\end{center}

\begin{lstlisting}[language=C,caption={Bucket sort em C},captionpos=t]
void insertionSort(float arr[], int n) {
    for(int i=1;i<n;i++){
        float key=arr[i];
        int j=i-1;
        while(j>=0 && arr[j]>key){
            arr[j+1]=arr[j];
            j--;
        }
        arr[j+1]=key;
    }
}

void bucketSort(float arr[], int n){
    float buckets[n][n]; int count[n]={0};
    for(int i=0;i<n;i++){
        int index = n*arr[i];
        buckets[index][count[index]++] = arr[i];
    }
    for(int i=0;i<n;i++)
        insertionSort(buckets[i], count[i]);
    int idx=0;
    for(int i=0;i<n;i++)
        for(int j=0;j<count[i];j++)
            arr[idx++] = buckets[i][j];
}
\end{lstlisting}

\begin{lstlisting}[language=C++,caption={Bucket sort em C++},captionpos=t]
#include <vector>
#include <algorithm>
using namespace std;

void insertionSort(vector<float>& arr) {
    for (int i = 1; i < arr.size(); i++) {
        float key = arr[i];
        int j = i - 1;
        while (j >= 0 && arr[j] > key) {
            arr[j + 1] = arr[j];
            j--;
        }
        arr[j + 1] = key;
    }
}

void bucketSort(vector<float>& arr) {
    int n = arr.size();
    vector<vector<float>> buckets(n);

    for (float num : arr) {
        int idx = n * num;
        buckets[idx].push_back(num);
    }

    for (auto& b : buckets)
        insertionSort(b);

    arr.clear();
    for (auto& b : buckets)
        arr.insert(arr.end(), b.begin(), b.end());
}
\end{lstlisting}

\begin{lstlisting}[language=Python,caption={Bucket sort em Python},captionpos=t]
def insertion_sort(arr):
    for i in range(1, len(arr)):
        key = arr[i]
        j = i-1
        while j >=0 and arr[j] > key:
            arr[j+1] = arr[j]
            j -= 1
        arr[j+1] = key
    return arr

def bucket_sort(arr):
    n = len(arr)
    buckets = [[] for _ in range(n)]
    for num in arr:
        index = int(n*num)
        buckets[index].append(num)
    for i in range(n):
        buckets[i] = insertion_sort(buckets[i])
    sorted_arr = []
    for bucket in buckets:
        sorted_arr.extend(bucket)
    return sorted_arr
\end{lstlisting}

\subsection{Análise de complexidade do algoritmo \ref{lab:alg-bucketSort}}

Seja $n$ o número de elementos:

\begin{itemize}
    \item Distribuição nos baldes: $O(n)$
    \item Ordenação dos baldes (Insertion Sort): no caso médio $O(n)$, assumindo distribuição uniforme
    \item Concatenação dos baldes: $O(n)$
\end{itemize}

Portanto, a complexidade total no **caso médio** é:

\[
T(n) = O(n) + O(n) + O(n) = O(n)
\]

\begin{itemize}
    \item \textbf{Melhor caso:} $O(n)$, quando os elementos são distribuídos uniformemente.
    \item \textbf{Caso médio:} $O(n)$, para distribuição uniforme.
    \item \textbf{Pior caso:} $O(n^2)$, se todos os elementos caírem no mesmo balde (Insertion Sort domina).
\end{itemize}

Complexidade de espaço: $O(n + k)$, onde $k$ é o número de baldes (geralmente $k=n$).
\vspace{1cm}

\textcolor{blue}{Aline, leia os textos nos \textit{links} abaixo e tente melhor o seu texto neste capítulo}: 
\begin{itemize}
    \item 
\href{https://www.ic.unicamp.br/~ra063658/disciplinas/stco02_2025s1/sort_linear.pdf}{Sorting in Linear Time};

\item \href{https://ocw.mit.edu/courses/6-006-introduction-to-algorithms-fall-2011/bf7d79105762bf79bbc0925438e1468a_MIT6_006F11_lec07.pdf}{Linear-time sorting};

\item  \href{https://iudatastructurescourse.github.io/course-web-page-fall-2024/lectures/sort-linear.html}{Sorting in linear time};

\item \href{http://personal.kent.edu/~amohamm4/daa-f2019/slides/ch4-2%20LinearTime%20Sorting.pdf}{Linear-time sorting};

\item
\href{https://www.dcc.fc.up.pt/~pribeiro/aulas/aed2425/slides/4_sorting.pdf}{Sorting Algorithms}.
\end{itemize}
