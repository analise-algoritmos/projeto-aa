\chapter{Experimentos computacionais}

Todos os algoritmos descritos nos capítulos anteriores foram implementados nas linguagens de programação C e Python. Utilizamos ... para 
Neste capítulo, analisamos experimentos 
\section{Ambiente Computacional}
\label{sec:Env}
Os experimentos foram executados em um ambiente computacional com a seguinte configuração: USAR O SEGUINTE MODELO 
\begin{itemize}
    \item[-] Computador: Processador Intel Xeon(R) CPU E5-1620 v2, 4 Núcleos de 3,70GHz e 8 GB de Memória RAM;
    \item[-] Sistema Operacional: Linux Ubuntu 16.04 LTS, 64 bits;
    \item[-] Linguagens de programação: \texttt{C} e \texttt{Python};
    \item[-] Compilador da linguagem C: gcc 5.4.0;
    \item[-] Interpretador da linguagem Python: ...;
    \item[-] Função para coleta do tempo de processamento na CPU na linguagem C: \texttt{gettimeofday()};
    \item[-] Função para a coleta do tempo de processamento na CPU na linguagem Python: \texttt{...}.
\end{itemize}

%Os Algoritmos foram implementados na linguagem de programação C e os tempos de processamento foram coletados usando a função \textit{gettimeofday()}.

Na próxima seção descrevemos as instâncias utilizadas nesta pesquisa.

\section{Instâncias}
\label{sec:Inst}
Para a realização dos experimentos, utilizamos os conjuntos de dados:
\begin{itemize}
\item 
\href{https://www.kaggle.com/datasets/bekiremirhanakay/benchmark-dataset-for-sorting-algorithms}{Benchmark Dataset for Sorting Algorithms}

\item 
\href{https://ieeexplore.ieee.org/document/7280062}{Analysis and Testing of Sorting Algorithms on a Standard Dataset}

\item 
\href{https://link.springer.com/chapter/10.1007/978-981-97-5703-9_8}{A Comparative Analysis of Sorting Algorithms for Large-Scale Data: Performance Metrics and Language Efficiency}

\item 
\href{https://www.kaggle.com/datasets/wazahathussain/data-set-for-sorting-algorithm}{Data set for sorting algorithm}

\end{itemize}

\section{Análises}
Criar tabelas, gráficos de barra, gráficos de pizza etc.. 
