\documentclass[12pt,openany]{book}
\usepackage{amsmath}
\usepackage{amsfonts}
\usepackage{amssymb}
\usepackage{graphicx}
\usepackage{epsfig}
\usepackage[boxruled,vlined,boxed,portuguese,algochapter]{algorithm2e}

\usepackage[utf8]{inputenc}
\usepackage[T1]{fontenc}
\usepackage[brazil]{babel}
\usepackage{hyperref}
\usepackage{tcolorbox}
\usepackage{epigraph} 
\usepackage{quotchap}
\usepackage{mathtools}
\usepackage{multirow}
\usepackage{booktabs}
% \usepackage[graphicx]{realboxes}
\usepackage{rotating}
% \usepackage{adjustbox}
% Tabelas
\usepackage{booktabs}
\usepackage{pgfplots}
\pgfplotsset{compat=newest}
\usepackage{multirow}    
\usepackage{multicol}    
\usepackage{float}       
\usepackage{caption}     
\usepackage{longtable}   

\topskip 0pt
\headsep 30pt
\headheight 1pt
\oddsidemargin 0pt
\evensidemargin 0pt
\textwidth 6.5in
\topmargin 0pt
\textheight 9.0in
% \newgeometry{left=3cm, right=3cm, top=1cm, bottom=1.2cm}

\usepackage{xcolor}
% Definindo novas cores
\definecolor{verde}{rgb}{0.25,0.5,0.35}
\definecolor{jpurple}{rgb}{0.5,0,0.35}
% Configurando layout para mostrar codigos em Python
\usepackage{listings}
\lstset{
  language=C,
  basicstyle=\ttfamily\small, 
  keywordstyle=\color{blue}\bfseries, %jpurple
  stringstyle=\color{red},
  commentstyle=\color{verde},
  morecomment=[s][\color{blue}]{/**}{*/},
  extendedchars=true, 
  showspaces=false, 
  showstringspaces=false, 
  numbers=left,
  numberstyle=\tiny,
  breaklines=true, 
  backgroundcolor=\color{cyan!10}, 
  breakautoindent=true, 
  captionpos=b,
  xleftmargin=0pt,
  tabsize=4
}

\lstset{
  language=C++,
  basicstyle=\ttfamily\small, 
  keywordstyle=\color{blue}\bfseries, %jpurple
  stringstyle=\color{red},
  commentstyle=\color{verde},
  morecomment=[s][\color{blue}]{/**}{*/},
  extendedchars=true, 
  showspaces=false, 
  showstringspaces=false, 
  numbers=left,
  numberstyle=\tiny,
  breaklines=true, 
  backgroundcolor=\color{cyan!10}, 
  breakautoindent=true, 
  captionpos=b,
  xleftmargin=0pt,
  tabsize=4
}

\usepackage{listings}
\lstset{
  language=Python,
  basicstyle=\ttfamily\small, 
  keywordstyle=\color{blue}\bfseries, %jpurple
  stringstyle=\color{red},
  commentstyle=\color{verde},
  morecomment=[s][\color{blue}]{/**}{*/},
  extendedchars=true, 
  showspaces=false, 
  showstringspaces=false, 
  numbers=left,
  numberstyle=\tiny,
  breaklines=true, 
  backgroundcolor=\color{cyan!10}, 
  breakautoindent=true, 
  captionpos=b,
  xleftmargin=0pt,
  tabsize=4
}
\renewcommand{\lstlistingname}{Código}% Listing -> Código

\newtheorem{thm}{Teorema}
\newtheorem{defn}{Definição}[chapter]
\newtheorem{exerc}{Exercício}[chapter]
\newtheorem{exmp}{Exemplo}[chapter]
\newtheorem{prob}{Problema.}[chapter]
\newtheorem{property}{Propriedade.}[chapter]
\newcommand{\R}{\mathbb{R}}

\title{\LARGE \bf 
ALGORITMOS DE ORDENAÇÃO, IMPLEMENTAÇÕES E EXPERIMENTOS COMPUTACIONAIS 
% suas análises de complexidades de tempo e espaço, 
% implementações e experimentos computacionais
}
\author{Alunas: Aline Milene Martins dos Santos (RA 11201920702)\\
         \hspace{3cm} Camila de Fátima Augusta dos Santos (RA 11201810901)\\
         \hspace{0.5cm} Fernanda Sayuri Alves Ito (RA 11201811042)\\\\
\hspace{-3.1cm} Supervisor: Dr. Cláudio Nogueira de Meneses}
% Monografia da disciplina de Análise de Algoritmos no bacharelado em Ciência da Computação
% {\footnotesize claudio.meneses@ufabc.edu.br}

\begin{document}
\maketitle

\begin{center}
\textbf{RESUMO}\\
\end{center}
A análise de algoritmos ou análise de complexidade é um mecanismo para compreender e avaliar um algoritmo em dois aspectos principais: correção - avaliando a exatidão do método praticado utilizando-se do viés matemático para tal;e análise, onde se avalia a eficiência do algoritmo no viés de recursos consumidos, sejam estes memória ou tempo de execução. Essa análise contribui para a aplicação mais adequada e eficiente de algoritmos em problemas práticos. Esta monografia descreve aproximadamente cinquenta algoritmos de ordenação que foram desenvolvidos ao longo dde diferentes períodos históricos - desde os mais antigos, desenvolvidos no século XIX como os últimos propostos na literatura. Este projeto apresenta dois grandes blocos específicos: o primeiro sob a ótica teórica, apresentando os algoritmos em suas especificações de construção, funcionamento e análises das complexidades de tempo e espaço, além da apresentação de suas implementações em três linguagens,Python, C, C++; e o segundo com viés empírico, na qual é explorado através de experimentos computacionais as performances destes algoritmos atuando sob diferentes instâncias relativamente grandes de problemas, de modo a simularmos o comportamento experado no contexto assintótico. Os resultados confirmam a hierarquia prevista entre as classes de complexidade, mas também evidenciam a influência de fatores como otimizações internas, overhead computacional, consumo de memória e organização dos dados. Como contribuição adicional, o projeto resultou na criação de bibliotecas nas três linguagens citadas, disponibilizadas publicamente em um repositório no GitHub.
\vspace{0.2cm}

\noindent{\textbf{Palavras-chave}: Algoritmos de ordenação, otimização, memória, desempenho computacional}  
\begin{center}
\textbf{ABSTRACT}\
\end{center}
The analysis of algorithms, or complexity analysis, is a mechanism for understanding and evaluating an algorithm in two main aspects: correctness — assessing the accuracy of the method through a mathematical perspective — and efficiency, which examines the algorithm’s use of computational resources such as memory and execution time. This analysis contributes to the more appropriate and effective application of algorithms to practical problems. This monograph describes approximately fifty sorting algorithms developed across different historical periods — from the earliest ones, created in the nineteenth century, to the most recent proposals in the literature. The project is organized into two major parts: the first adopts a theoretical perspective, presenting the algorithms in terms of their construction, operation, and analyses of time and space complexity, as well as providing their implementations in three programming languages — Python, C, and C++; the second part focuses on an empirical perspective, in which computational experiments are used to explore the performance of these algorithms when applied to relatively large problem instances, allowing us to simulate their expected behavior in the asymptotic context. The results confirm the hierarchy predicted among complexity classes, while also highlighting the influence of factors such as internal optimizations, computational overhead, memory consumption, and data organization. As an additional contribution, the project produced libraries in the three aforementioned languages, which have been made publicly available in a GitHub repository.
\vspace{0.2cm}

\noindent\textbf{Keywords}: Sorting algorithms, optimization, memory, computational performance

\listofalgorithms
\listoftables
\tableofcontents

\chapter{Notações e Conceitos Básicos}

Neste capítulo apresentamos os fundamentos matemáticos e conceituais necessários para a análise rigorosa da complexidade de algoritmos de ordenação. Abordamos as notações assintóticas fundamentais, técnicas de análise de recorrências, conceitos de estabilidade e otimalidade, bem como métodos de avaliação experimental. Este material fornece a base teórica essencial para compreender as análises detalhadas apresentadas nos capítulos subsequentes.

O domínio das notações assintóticas é crucial para expressar de forma precisa e concisa o comportamento de algoritmos em função do tamanho da entrada. Além das definições formais, apresentamos métodos práticos para determinar e provar limitantes de complexidade, técnicas para resolver relações de recorrência e estratégias para análise de casos médio, melhor e pior.

\section{Notações Assintóticas}

Denotamos por
$\mathbb{R}$ o conjunto dos números reais, 
$\mathbb{R}_{\geq 0}$ o conjunto dos números reais não negativos e
$\mathbb{R}_{>0}$ o conjunto dos números reais positivos.

\begin{defn} [Notação \textit{``Big O''}]
Dadas as funções $f:\mathbb{R}_{\geq 0}\rightarrow \mathbb{R}_{\geq 0}$ e $g:\mathbb{R}_{\geq 0}\rightarrow \mathbb{R}_{\geq 0}$, dizemos que $f(n)$ é $O(g(n))$ se existem constantes $c\in\mathbb{R}_{>0}$, $n_0\in \mathbb{R}_{\geq 0}$ tal que $f(n) \leq c\cdot g(n)$  para todo $n \geq n_0$. 
\end{defn}
\vspace{0.2cm}

Uma maneira equivalente de definir a notação \textit{``Big O''} é a seguinte:\vspace{0.2cm}

\begin{defn}
$O(g(n)) = \{ f(n):$ existem constantes $c\in\mathbb{R}_{>0}$, $n_0\in \mathbb{R}_{\geq 0}$ tal que $0 \leq f(n) \leq c\cdot g(n)$ para todo $n \geq n_0.\}$
\end{defn}
\vspace{0.3cm}

A notação ``Big O'' representa um \textbf{limite superior no tempo de execução} de um algoritmo.  Assim, ela fornece a \textbf{complexidade de tempo de pior caso} do algoritmo.

\textcolor{blue}{[... mantendo todos os exemplos e definições existentes ...].}

\section{Propriedades das Notações Assintóticas}

\begin{property}[Transitividade]
Se $f(n) = O(g(n))$ e $g(n) = O(h(n))$, então $f(n) = O(h(n))$.
\end{property}

\begin{property}[Reflexividade]
Para qualquer função $f(n)$, temos $f(n) = O(f(n))$.
\end{property}

\begin{property}[Simetria da notação $\Theta$]
$f(n) = \Theta(g(n))$ se e somente se $g(n) = \Theta(f(n))$.
\end{property}

\begin{property}[Regra da Soma]
Se $f_1(n) = O(g_1(n))$ e $f_2(n) = O(g_2(n))$, então $f_1(n) + f_2(n) = O(\max(g_1(n), g_2(n)))$.
\end{property}


\section{Análise de Recorrências}

\subsection{Método da Substituição}

O método da substituição consiste em três etapas:
\begin{enumerate}
\item \textbf{Hipótese:} Conjecturar a forma da solução
\item \textbf{Indução:} Usar indução matemática para verificar a hipótese
\item \textbf{Determinação de constantes:} Encontrar constantes que satisfaçam as condições
\end{enumerate}

\begin{exmp}
Considere a recorrência $T(n) = 2T(n/2) + n$ com $T(1) = 1$. Conjecturamos que $T(n) = O(n \log n)$.

\textbf{Prova:} Queremos mostrar que $T(n) \leq cn \log n$ para alguma constante $c > 0$.
\begin{align}
T(n) &= 2T(n/2) + n \\
&\leq 2c(n/2)\log(n/2) + n \\
&= cn(\log n - 1) + n \\
&= cn \log n - cn + n \\
&= cn \log n - (c-1)n
\end{align}
Para que $T(n) \leq cn \log n$, precisamos que $(c-1)n \geq 0$, ou seja, $c \geq 1$.
\end{exmp}

\subsection{Método da Árvore de Recursão}

\begin{exmp}
Para $T(n) = 3T(n/4) + \Theta(n^2)$:
\begin{itemize}
\item \textbf{Nível 0:} $n^2$
\item \textbf{Nível 1:} $3 \cdot (n/4)^2 = 3n^2/16$
\item \textbf{Nível 2:} $9 \cdot (n/16)^2 = 9n^2/256$
\item \textbf{Nível $i$:} $3^i \cdot (n/4^i)^2 = 3^i \cdot n^2/4^{2i} = n^2(3/16)^i$
\end{itemize}

A profundidade é $\log_4 n$ e o custo total é:
\begin{align}
T(n) &= n^2 \sum_{i=0}^{\log_4 n} (3/16)^i \\
&= n^2 \cdot \frac{1-(3/16)^{\log_4 n + 1}}{1-3/16} \\
&= \Theta(n^2)
\end{align}
\end{exmp}

\subsection{Teorema Mestre}

\begin{thm}[Teorema Mestre]
Seja $a \geq 1$ e $b > 1$ constantes, e seja $f(n)$ uma função. Se $T(n) = aT(n/b) + f(n)$, então:
\begin{enumerate}
\item Se $f(n) = O(n^{\log_b a - \epsilon})$ para alguma constante $\epsilon > 0$, então $T(n) = \Theta(n^{\log_b a})$
\item Se $f(n) = \Theta(n^{\log_b a})$, então $T(n) = \Theta(n^{\log_b a} \log n)$
\item Se $f(n) = \Omega(n^{\log_b a + \epsilon})$ para alguma constante $\epsilon > 0$ e se $af(n/b) \leq cf(n)$ para alguma constante $c < 1$ e $n$ suficientemente grande, então $T(n) = \Theta(f(n))$
\end{enumerate}
\end{thm}

\section{Conceitos Fundamentais para Algoritmos de Ordenação}

\subsection{Estabilidade}

\begin{defn}[Algoritmo Estável]
Um algoritmo de ordenação é \textbf{estável} se preserva a ordem relativa de elementos iguais. Formalmente, se elementos $x$ e $y$ são iguais e $x$ aparece antes de $y$ na sequência original, então $x$ deve aparecer antes de $y$ na sequência ordenada.
\end{defn}

\subsection{Ordenação In-Place}

\begin{defn}[Algoritmo In-Place]
Um algoritmo de ordenação é \textbf{in-place} se utiliza apenas uma quantidade constante de espaço auxiliar, isto é, $O(1)$ espaço extra além do array de entrada.
\end{defn}

\subsection{Algoritmos Adaptativos}

\begin{defn}[Algoritmo Adaptativo]
Um algoritmo de ordenação é \textbf{adaptativo} se sua performance melhora quando a entrada possui alguma estrutura pré-existente (como estar parcialmente ordenada).
\end{defn}

\subsection{Limite Inferior para Ordenação por Comparação}

\begin{thm}[Limite Inferior]
Qualquer algoritmo de ordenação baseado em comparações deve realizar pelo menos $\lceil \log_2(n!) \rceil$ comparações no pior caso para ordenar $n$ elementos.
\end{thm}

\textbf{Prova:} Considere a árvore de decisão do algoritmo. Cada folha corresponde a uma permutação possível dos elementos. Como há $n!$ permutações possíveis, a árvore deve ter pelo menos $n!$ folhas. Uma árvore binária com $n!$ folhas tem altura mínima $\lceil \log_2(n!) \rceil$.

Pela aproximação de Stirling: $\log_2(n!) = \Theta(n \log n)$.

\section{Medidas de Complexidade}

\subsection{Número de Comparações}

Para algoritmos baseados em comparação, frequentemente analisamos o número de comparações realizadas:
\begin{itemize}
\item \textbf{Limite teórico:} $\Omega(n \log n)$ comparações
\item \textbf{Algoritmos ótimos:} Merge Sort realiza $\Theta(n \log n)$ comparações
\item \textbf{Insertion Sort:} $O(n^2)$ comparações no pior caso, $\Theta(n)$ no melhor
\end{itemize}

\subsection{Número de Trocas}

\begin{defn}[Inversão]
Uma \textbf{inversão} em um array $A[1..n]$ é um par $(i,j)$ tal que $i < j$ mas $A[i] > A[j]$.
\end{defn}

\begin{property}
O número mínimo de trocas adjacentes necessárias para ordenar um array é igual ao número de inversões no array.
\end{property}

\subsection{Análise Amortizada}

\begin{defn}[Análise Amortizada]
A \textbf{análise amortizada} estuda o custo médio de uma sequência de operações, distribuindo o custo de operações caras sobre muitas operações baratas.
\end{defn}

\textbf{Métodos principais:}
\begin{enumerate}
\item \textbf{Método Agregado:} Analisa custo total de $n$ operações
\item \textbf{Método Contábil:} Atribui "créditos" a operações baratas
\item \textbf{Método Potencial:} Usa função potencial para redistribuir custos
\end{enumerate}

\section{Modelos de Custo e Complexidade de Cache}

\subsection{Modelo RAM}

No modelo RAM (Random Access Machine):
\begin{itemize}
\item Cada operação elementar tem custo constante
\item Acesso a qualquer posição da memória tem custo $O(1)$
\item Modelo adequado para análise assintótica básica
\end{itemize}

\subsection{Modelo Cache-Oblivious}

\begin{defn}[Complexidade de Cache]
Medimos o número de transferências entre cache e memória principal. Parâmetros:
\begin{itemize}
\item $M$: tamanho do cache
\item $B$: tamanho do bloco de transferência
\end{itemize}
\end{defn}

\textbf{Limitantes para ordenação cache-oblivious:}
\begin{itemize}
\item \textbf{Comparações:} $\Omega(n \log n)$
\item \textbf{Cache misses:} $\Omega\left(\frac{n}{B} \log_{M/B} \frac{n}{B}\right)$
\end{itemize}

\section{Técnicas de Análise Experimental}

\subsection{Metodologia}

\begin{enumerate}
\item \textbf{Implementação:} Código otimizado e correto
\item \textbf{Datasets:} Casos aleatórios, ordenados, reversos, com duplicatas
\item \textbf{Medições:} Tempo de CPU, comparações, trocas, cache misses
\item \textbf{Estatística:} Múltiplas execuções, intervalos de confiança
\end{enumerate}

\subsection{Análise de Constantes}

A notação assintótica oculta constantes importantes:
\begin{itemize}
\item \textbf{Quick Sort:} $\sim 1.39 n \log n$ comparações em média
\item \textbf{Merge Sort:} $\sim 1.00 n \log n$ comparações sempre
\item \textbf{Heap Sort:} $\sim 2.00 n \log n$ comparações no pior caso
\end{itemize}

\section{Resumo das Ferramentas de Análise}

\begin{table}[h]
\centering
\begin{tabular}{l|l|l}
% \hline
\textbf{Ferramenta} & \textbf{Aplicação} & \textbf{Exemplo} \\
\hline
Notação Big O & Limite superior & $T(n) = O(n^2)$ \\
% \hline
Notação $\Omega$ & Limite inferior & $T(n) = \Omega(n \log n)$ \\
% \hline
Notação $\Theta$ & Limite exato & $T(n) = \Theta(n \log n)$ \\
% \hline
Teorema Mestre & Recorrências divide-conquista & $T(n) = 2T(n/2) + O(n)$ \\
% \hline
Árvore de recursão & Recorrências complexas & Análise visual de custos \\
% \hline
Análise amortizada & Operações com custo variável & Arrays dinâmicos \\
% \hline
Potencial & Estruturas que mudam & Splay trees \\
\hline
\end{tabular}
\caption{Ferramentas de análise de complexidade e suas aplicações típicas}
\end{table}

\section{Exercícios}

\begin{exerc}
Prove que se $f(n) = O(g(n))$ e $g(n) = O(h(n))$, então $f(n) = O(h(n))$.
\end{exerc}

\begin{exerc}
Resolva a recorrência $T(n) = 4T(n/2) + n^2$ usando o Teorema Mestre.
\end{exerc}

\begin{exerc}
Prove que qualquer algoritmo que encontra o máximo de $n$ elementos deve realizar pelo menos $n-1$ comparações.
\end{exerc}

\begin{exerc}
Considere um algoritmo que mantém um array ordenado através de inserções. Se realizarmos $n$ inserções sequenciais, qual é o custo amortizado por inserção?
\end{exerc}

Este conjunto de conceitos e técnicas fornece a base necessária para a análise rigorosa dos algoritmos de ordenação apresentados nos próximos capítulos. Dominar estas ferramentas é essencial para compreender tanto os aspectos teóricos quanto práticos da eficiência algorítmica.



\include{CAPÍTULOS/2.Notacoes_e_Conceitos_basicos}
\chapter{Algoritmos com complexidades de tempo quadráticas}
Neste capítulo apresentamos algoritmos de ordenação cujos números de operações de comparações são da forma $cn^2$, onde $n$ é o número de elementos a serem ordenados no multiconjunto e $c$ é um número real positivo.

\section{Bubble sort}
\subsection{Descrição e Funcionamento}
O \textit{Bubble Sort} é um algoritmo de ordenação simples e baseado em comparações. 
Ele percorre repetidamente o vetor de entrada, comparando elementos adjacentes e trocando-os de posição caso estejam na ordem incorreta. 
Esse processo se repete até que o vetor esteja completamente ordenado.

\medskip
Apesar de ser intuitivo e fácil de implementar, o \textit{Bubble Sort} não é eficiente para grandes conjuntos de dados, apresentando complexidade quadrática no pior caso.

\medskip
A seguir, apresenta-se um exemplo ilustrativo de execução do algoritmo.

\begin{exmp}
Considere o vetor $A = [5.0, 2.0, 4.0, 1.0, 3.0]$. O objetivo é ordená-lo utilizando o \textit{Bubble Sort}.

\begin{enumerate}
    \item \textbf{Primeira passagem:}  
    Comparamos cada par de elementos adjacentes e realizamos trocas se necessário:
    \[
    [5.0, 2.0, 4.0, 1.0, 3.0] \rightarrow [2.0, 5.0, 4.0, 1.0, 3.0] \rightarrow [2.0, 4.0, 5.0, 1.0, 3.0] \rightarrow [2.0, 4.0, 1.0, 5.0, 3.0] \rightarrow [2.0, 4.0, 1.0, 3.0, 5.0]
    \]
    O maior elemento (5.0) "borbulhou" para a última posição.

    \item \textbf{Segunda passagem:}  
    Repetimos o processo para os quatro primeiros elementos:
    \[
    [2.0, 4.0, 1.0, 3.0, 5.0] \rightarrow [2.0, 1.0, 4.0, 3.0, 5.0] \rightarrow [2.0, 1.0, 3.0, 4.0, 5.0]
    \]
    
    \item \textbf{Terceira passagem:}  
    Continuamos comparando e trocando:
    \[
    [2.0, 1.0, 3.0, 4.0, 5.0] \rightarrow [1.0, 2.0, 3.0, 4.0, 5.0]
    \]
    
    \item \textbf{Vetor ordenado:}  
    Nenhuma troca adicional é necessária e o vetor final é:
    \[
    [1.0, 2.0, 3.0, 4.0, 5.0]
    \]
\end{enumerate}
\end{exmp}

\medskip
O pseudocódigo correspondente é apresentado a seguir.

\begin{center}
\begin{minipage}{.9\linewidth}
\begin{algorithm}[H]
\DontPrintSemicolon
\hspace{-0.4cm}\textbf{bubbleSort(values: array of float, n: integer)}

\For{$i \gets 0$ \KwTo $n-2$}{
    \For{$j \gets 0$ \KwTo $n-i-2$}{
        \If{$values[j] > values[j+1]$}{
            $temp \gets values[j]$\;
            $values[j] \gets values[j+1]$\;
            $values[j+1] \gets temp$\;
        }
    }
}
\caption{Bubble Sort}
\label{lab:alg-bubbleSort}
\end{algorithm}
\end{minipage}
\end{center}

\subsection{Implementações}
\begin{lstlisting}[language=Python,caption={Bubble sort otimizado em Python},captionpos=t]
def bubbleSort(values, n):
    for i in range(0,n):
        swapped = False  
        for j in range(0,n-i-1):
            if values[j] > values[j+1]:
                values[j],values[j+1] = values[j+1], values[j]
                swapped = True
        if not swapped: break
\end{lstlisting}
\begin{lstlisting}[language=C,caption={Bubble sort em C},captionpos=t]
void bubbleSort(int values[], int n){
    for(int i = 0; i < n - 1; i++){
        for(int j = 0; j < n - i - 1; j++){
            if (values[j] > values[j + 1]){
                swap(values[j], values[j + 1]);
            }
        }
    }
}
\end{lstlisting}
Para tratar o caso do vetor estar inicialmente ordenado, e assim não precisar ordená-lo, podemos usar a seguinte versão: 
\begin{lstlisting}[language=C,caption={Bubble sort otimizado em C},captionpos=t]
void bubbleSort(int values[], int n){
    bool swapped = true;
    for(int i = 0; i < n - 1; i++){
        swapped = false;
        for(int j = 0; j < n - i - 1; j++){
            if (values[j] > values[j + 1]){
                swap(values[j], values[j + 1]);
                swapped = true;
            }
        }
        if (swapped == false) break;
    }
}
\end{lstlisting}
\begin{lstlisting}[language=C++,caption={Bubble sort em C++},captionpos=t]
#include <vector>
using namespace std;

void bubbleSort(vector<int>& arr) {
    int n = arr.size();
    for (int i = 0; i < n - 1; i++) {
        bool swapped = false;
        for (int j = 0; j < n - i - 1; j++) {
            if (arr[j] > arr[j + 1]) {
                swap(arr[j], arr[j + 1]);
                swapped = true;
            }
        }
        if (!swapped) break;
    }
}
\end{lstlisting}

\subsection{Análise de complexidade}

\subsubsection{Complexidade de Tempo}

Seja $n$ o número de elementos do vetor de entrada.

O algoritmo \textit{Bubble Sort} executa as seguintes etapas principais:

\begin{enumerate}
    \item Comparações de elementos adjacentes em cada passagem — custo de $O(n-i)$ por passagem.
    \item Trocas realizadas quando necessário — no pior caso, cada comparação pode resultar em uma troca.
\end{enumerate}

Assim, o tempo total de execução $T(n)$ no pior caso é:

\[
T(n) = \sum_{i=0}^{n-2} (n-i-1) = \frac{n(n-1)}{2}
\]

\begin{equation}
T(n) \in O(n^2)
\end{equation}

\noindent{\textbf{Prova:}}  
O somatório de comparações é dado por:
\[
\sum_{i=0}^{n-2} (n-i-1) = 1 + 2 + \dots + (n-1) = \frac{n(n-1)}{2} \leq c n^2
\]
para $c = 1/2$.  
Portanto, $T(n) \in O(n^2)$.
$\hfill\Box$

\bigskip

\noindent{\textbf{Discussão:}}  
No melhor caso, quando o vetor já está ordenado, o Bubble Sort pode ser otimizado para detectar que nenhuma troca ocorreu, resultando em complexidade $O(n)$.  
Entretanto, no caso médio e no pior caso, o algoritmo permanece com complexidade quadrática, tornando-o ineficiente para grandes entradas.

\subsubsection{Complexidade de Espaço}

O algoritmo utiliza apenas variáveis auxiliares constantes para trocas e índices:

\begin{itemize}
    \item Vetor de entrada $values[0 \ldots n-1]$ — espaço $O(n)$.
    \item Variável temporária $temp$ — espaço $O(1)$.
\end{itemize}

\begin{equation}
S(n) \in O(n)
\end{equation}

\noindent{\textbf{Prova:}}  
Como o vetor é modificado \textit{in-place} e a memória adicional é constante:
\[
S(n) = n + 1 \leq c n
\]
para alguma constante $c \geq 1$.  
Portanto, $S(n) \in O(n)$.
$\hfill\Box$

\bigskip

\noindent{\textbf{Discussão:}}  
O \textit{Bubble Sort} é um algoritmo \textit{in-place} e não requer memória auxiliar significativa além do vetor de entrada e de uma variável temporária.  
Isso o torna simples de implementar, mas sua eficiência de tempo limitada impede uso em grandes conjuntos de dados.
\section{Insertion sort}
\subsection{Descrição e Funcionamento}
O \textit{Insertion Sort} é um algoritmo de ordenação estável e baseado em comparações.  
Ele constrói o vetor ordenado de forma incremental, inserindo cada elemento do vetor de entrada em sua posição correta dentro da sequência já ordenada.  
A lógica do algoritmo é análoga à ordenação de cartas na mão: a cada iteração, um novo elemento é comparado com os anteriores e deslocado até encontrar sua posição correta.

\medskip
A seguir, apresenta-se um exemplo ilustrativo de execução do algoritmo.

\begin{exmp}
Considere o vetor $A = [5.0, 2.0, 4.0, 6.0, 1.0]$. O objetivo é ordená-lo utilizando o \textit{Insertion Sort}.

\begin{enumerate}
    \item \textbf{Iteração 1 (elemento 2.0):}  
    Comparamos 2.0 com 5.0 e movemos 5.0 para a direita. Inserimos 2.0 na posição 0:
    \[
    A = [2.0, 5.0, 4.0, 6.0, 1.0]
    \]

    \item \textbf{Iteração 2 (elemento 4.0):}  
    Comparamos 4.0 com 5.0, movemos 5.0 para a direita e inserimos 4.0 na posição correta:
    \[
    A = [2.0, 4.0, 5.0, 6.0, 1.0]
    \]

    \item \textbf{Iteração 3 (elemento 6.0):}  
    6.0 é maior que 5.0, permanece na posição:
    \[
    A = [2.0, 4.0, 5.0, 6.0, 1.0]
    \]

    \item \textbf{Iteração 4 (elemento 1.0):}  
    Comparamos 1.0 com 6.0, 5.0, 4.0 e 2.0, movemos todos para a direita e inserimos 1.0 na posição inicial:
    \[
    A = [1.0, 2.0, 4.0, 5.0, 6.0]
    \]
\end{enumerate}
\end{exmp}

\medskip
O pseudocódigo correspondente é apresentado a seguir.

\begin{center}
\begin{minipage}{.9\linewidth}
\begin{algorithm}[H]
\DontPrintSemicolon
\hspace{-0.4cm}\textbf{insertionSort(values: array of float, n: integer)}

\For{$i \gets 1$ \KwTo $n-1$}{
    $key \gets values[i]$\;
    $j \gets i-1$\;
    \While{$j \geq 0$ \textbf{and} $values[j] > key$}{
        $values[j+1] \gets values[j]$\;
        $j \gets j-1$\;
    }
    $values[j+1] \gets key$\;
}
\caption{Insertion sort}
\label{lab:alg-insertionSort}
\end{algorithm}
\end{minipage}
\end{center}

\subsection{Implementações}
\begin{lstlisting}[language=python,caption={Insertion sort em Python},captionpos=t]
def insertionSort(values, n):
    if n <= 1: return  # array is already sorted
    for i in range(1, n):  
        x = values[i]  
        j = i-1
        while j >= 0 and x < values[j]:  
            values[j+1] = values[j]  
            j -= 1
        values[j+1] = x  
\end{lstlisting}
\begin{lstlisting}[language=C,caption={Insertion sort em C},captionpos=t]
void insertionSort(int values[], int n) {
    if (n <= 1){
       return   // array is already sorted
    }  
    for (int i = 1; i < n; i++) {
        int x = values[i];
        int j = i - 1;
        while (j >= 0 &&  x < values[j]) {
            values[j + 1] = values[j];
            j = j - 1;
        }
        values[j + 1] = x;
    }
}    
\end{lstlisting}
\begin{lstlisting}[language=C++,caption={Insertion sort em C++},captionpos=t]
#include <vector>
using namespace std;

void insertionSort(vector<int>& arr) {
    for (int i = 1; i < arr.size(); i++) {
        int key = arr[i];
        int j = i - 1;
        while (j >= 0 && arr[j] > key) {
            arr[j + 1] = arr[j];
            j--;
        }
        arr[j + 1] = key;
    }
}
\end{lstlisting}




\subsection{Análise de complexidade}
Nesta seção, analisamos formalmente as complexidades de tempo e espaço do algoritmo \textit{Insertion Sort}, que é baseado em comparações e reorganizações incrementais.

\subsubsection{Complexidade de Tempo}

Seja $n$ o número de elementos do vetor de entrada.  
O algoritmo executa as seguintes etapas principais:

\begin{enumerate}
    \item Iteração sobre cada elemento do vetor a partir da segunda posição — $O(n)$.
    \item Comparações e deslocamentos para inserir o elemento na posição correta — no pior caso (vetor em ordem decrescente), cada elemento pode exigir $O(i)$ comparações e deslocamentos, totalizando $O(1+2+...+n-1) = O(n^2)$.
\end{enumerate}

Assim, o tempo total de execução $T(n)$ pode ser expresso como:

\[
T(n) = a_1 \cdot n + a_2 \cdot \sum_{i=1}^{n-1} i + b
\]

onde $a_1$, $a_2$ e $b$ são constantes positivas.

\begin{equation}
T(n) \in O(n^2)
\end{equation}

\noindent{\textbf{Prova:}}

Sabemos que:
\[
\sum_{i=1}^{n-1} i = \frac{n(n-1)}{2} \leq \frac{n^2}{2}
\]

Logo:
\[
T(n) \leq a_1 n + a_2 \frac{n^2}{2} + b \leq c \cdot n^2
\]
para $c = \max(a_1, a_2/2, b)$ e $n$ suficientemente grande.  
Portanto,
\[
T(n) \in O(n^2).
\]
$\hfill\Box$

\bigskip
\noindent{\textbf{Discussão:}}  
No melhor caso (vetor já ordenado), cada elemento exige apenas uma comparação, resultando em complexidade linear $O(n)$.  
No caso médio e pior caso, o algoritmo tem desempenho quadrático $O(n^2)$, sendo menos eficiente que algoritmos baseados em divisão e conquista para grandes conjuntos de dados.

\subsubsection{Complexidade de Espaço}

O \textit{Insertion Sort} é um algoritmo \textit{in-place}, utilizando apenas um número constante de variáveis auxiliares:

\begin{itemize}
    \item $key$ para armazenar temporariamente o elemento sendo inserido.
    \item Índices auxiliares $i$ e $j$.
\end{itemize}

Portanto, o espaço total $S(n)$ é constante:

\begin{equation}
S(n) \in O(1)
\end{equation}

\noindent{\textbf{Prova:}}  
O algoritmo não utiliza estruturas adicionais proporcionais ao tamanho do vetor.  
Logo, existe $c>0$ tal que $S(n) \leq c$ para todo $n \geq 1$, implicando
\[
S(n) \in O(1).
\]
$\hfill\Box$

\bigskip
\noindent{\textbf{Discussão:}}  
O \textit{Insertion Sort} é eficiente em termos de memória e estável, tornando-o adequado para conjuntos pequenos ou quase ordenados, apesar de seu custo de tempo quadrático para entradas grandes e desordenadas.

\section{Comb sort}
\subsection{Descrição e Funcionamento}
O \textit{Comb Sort} é um algoritmo de ordenação baseado em comparação, desenvolvido para melhorar o desempenho do \textit{Bubble Sort} ao eliminar rapidamente valores pequenos e grandes que estão fora de posição, conhecidos como \textit{turtles}.  
A principal ideia é comparar elementos distantes e reduzir gradualmente a lacuna (\textit{gap}) entre eles até que a ordenação seja finalizada com um \textit{Bubble Sort} tradicional.

\medskip
O algoritmo segue os seguintes princípios:
\begin{enumerate}
    \item Inicialmente, define-se a lacuna (\textit{gap}) como o tamanho do vetor.
    \item A cada iteração, a lacuna é reduzida multiplicando-se por um fator de \textit{shrink} (geralmente 1.3).
    \item Elementos separados pela lacuna são comparados e trocados se estiverem fora de ordem.
    \item O processo continua até que a lacuna seja 1 e nenhuma troca seja necessária, garantindo a ordenação.
\end{enumerate}

\medskip
A seguir, apresenta-se um exemplo ilustrativo de execução do algoritmo.

\begin{exmp}
Considere o vetor $A = [0.9, 0.1, 0.7, 0.3, 0.5]$. O objetivo é ordená-lo utilizando o \textit{Comb Sort}.

\begin{enumerate}
    \item \textbf{Inicialização:}  
    Gap inicial: $gap = 5$, \textit{shrink factor} = 1.3.

    \item \textbf{Primeira iteração (gap = 3):}  
    Comparamos elementos separados por 3 posições:
    \[
    0.9 \leftrightarrow 0.3 \quad \text{(troca)}, \quad 0.1 \leftrightarrow 0.5 \quad \text{(troca)}
    \]  
    Vetor após primeira iteração: $[0.3, 0.1, 0.7, 0.9, 0.5]$.

    \item \textbf{Segunda iteração (gap = 2):}  
    Comparamos elementos separados por 2 posições:
    \[
    0.3 \leftrightarrow 0.7 \quad \text{(sem troca)}, \quad 0.1 \leftrightarrow 0.9 \quad \text{(sem troca)}, \quad 0.7 \leftrightarrow 0.5 \quad \text{(troca)}
    \]  
    Vetor após segunda iteração: $[0.3, 0.1, 0.5, 0.9, 0.7]$.

    \item \textbf{Terceira iteração (gap = 1):}  
    Comparações adjacentes, equivalente a \textit{Bubble Sort}:
    \[
    0.3 \leftrightarrow 0.1 \quad \text{(troca)}, \quad 0.3 \leftrightarrow 0.5 \quad \text{(sem troca)}, \quad 0.5 \leftrightarrow 0.9 \quad \text{(sem troca)}, \quad 0.9 \leftrightarrow 0.7 \quad \text{(troca)}
    \]  
    Vetor final ordenado: $[0.1, 0.3, 0.5, 0.7, 0.9]$.
\end{enumerate}
\end{exmp}

\medskip
O pseudocódigo correspondente é apresentado a seguir.

\begin{center}
\begin{minipage}{.9\linewidth}
\begin{algorithm}[H]
\DontPrintSemicolon
\hspace{-0.4cm}\textbf{combSort(values: array of float, n: integer)} 

$gap \gets n$\;
$shrink \gets 1.3$\;
$sorted \gets \text{false}$\;
\While{$gap > 1$ \textbf{or} $not\ sorted$}{
    $gap \gets \max(1, \lfloor gap / shrink \rfloor)$\;
    $sorted \gets \text{true}$\;
    \For{$i \gets 0$ \KwTo $n - gap - 1$}{
        \If{$values[i] > values[i+gap]$}{
            swap(values[i], values[i+gap])\;
            $sorted \gets \text{false}$\;
        }
    }
}
\caption{Comb sort}
\label{lab:alg-combSort}
\end{algorithm}
\end{minipage}
\end{center}

\subsection{Implementações}
\begin{lstlisting}[language=python,caption={Comb sort em Python},captionpos=t]
def combSort(values):
    n = len(values)
    gap = n
    shrink = 1.3
    sorted = False
    while not sorted:
        gap = int(gap / shrink)
        if gap <= 1:
            gap = 1
            sorted = True
        i = 0
        while i + gap < n:
            if values[i] > values[i + gap]:
                values[i], values[i + gap] = values[i + gap], values[i]
                sorted = False
            i += 1
\end{lstlisting}
\begin{lstlisting}[language=C,caption={Comb sort em C},captionpos=t]
void combSort(int arr[], int n) {
    int gap = n;
    const float shrink = 1.3;
    int sorted = 0;

    while (!sorted) {
        gap = (int)(gap / shrink);
        if (gap <= 1) {
            gap = 1;
            sorted = 1;
        }
        sorted = 1;
        for (int i = 0; i + gap < n; i++) {
            if (arr[i] > arr[i + gap]) {
                int temp = arr[i];
                arr[i] = arr[i + gap];
                arr[i + gap] = temp;
                sorted = 0;
            }
        }
    }
}
\end{lstlisting}
\begin{lstlisting}[language=C++,caption={Comb sort em C++},captionpos=t]
#include <vector>
#include <cmath>
using namespace std;

void combSort(vector<int>& arr) {
    int n = arr.size();
    int gap = n;
    bool swapped = true;

    while (gap > 1 || swapped) {
        gap = max(1, (int)(gap / 1.3));
        swapped = false;
        for (int i = 0; i + gap < n; i++) {
            if (arr[i] > arr[i + gap]) {
                swap(arr[i], arr[i + gap]);
                swapped = true;
            }
        }
    }
}
\end{lstlisting}

\subsection{Análise de complexidade}
Nesta seção, analisamos formalmente as complexidades de tempo e espaço do algoritmo \textit{Comb Sort}.  
Embora seja baseado em comparações, o uso de lacunas decrescentes permite reduzir o número de trocas necessárias em relação ao \textit{Bubble Sort}.

\subsubsection{Complexidade de Tempo}

Seja $n$ o número de elementos do vetor de entrada.  
O algoritmo realiza comparações ao longo de várias iterações, cada uma com lacuna $gap$ decrescente.

\begin{enumerate}
    \item Iterações com $gap > 1$ — número de comparações aproximadamente $O(n \log n)$.
    \item Última iteração (\textit{gap} = 1) — equivalente a \textit{Bubble Sort}, custo no pior caso $O(n^2)$.
\end{enumerate}

Portanto, o tempo total $T(n)$ pode ser expresso como:

\begin{equation}
T(n) \in O(n^2) \quad \text{(pior caso)}, \quad T(n) \in O(n \log n) \quad \text{(média prática)}
\end{equation}

\noindent{\textbf{Prova:}}  
O fator de \textit{shrink} reduz rapidamente a lacuna, eliminando os elementos mais fora de posição, o que diminui o número total de trocas em comparação ao \textit{Bubble Sort}.  
No pior caso, todas as comparações ainda podem ocorrer na última fase, resultando em $O(n^2)$.  
Em entradas aleatórias uniformes, estudos empíricos mostram desempenho médio próximo de $O(n \log n)$.  
$\hfill\Box$

\subsubsection{Complexidade de Espaço}

O \textit{Comb Sort} é um algoritmo \textit{in-place}, utilizando apenas algumas variáveis auxiliares (\textit{gap}, \textit{shrink}, índice de iteração).

O espaço total $S(n)$ pode ser expresso como:

\begin{equation}
S(n) \in O(1)
\end{equation}

\noindent{\textbf{Prova:}}  
Não há necessidade de estruturas adicionais proporcionais a $n$, apenas contadores e variáveis temporárias para troca.  
Portanto, o consumo de memória é constante independente do tamanho do vetor.  
$\hfill\Box$

\bigskip
\noindent{\textbf{Discussão:}}  
O \textit{Comb Sort} é eficiente em termos de memória e geralmente mais rápido que \textit{Bubble Sort} em entradas médias.  
É especialmente útil para ordenar baldes individuais em algoritmos de ordenação em tempo linear, como \textit{Bucket Sort}, mantendo simplicidade de implementação e baixo uso de espaço.  







\section{Selection sort}
\subsection{Descrição e Funcionamento}
O \textit{Selection Sort} é um algoritmo de ordenação comparativo, simples e intuitivo. Ele funciona dividindo o vetor em duas partes: a sublista ordenada, inicialmente vazia, e a sublista não ordenada, que contém todos os elementos. A cada iteração, o algoritmo seleciona o menor (ou maior) elemento da sublista não ordenada e o troca com o primeiro elemento dessa sublista, expandindo assim a sublista ordenada.

\medskip
O algoritmo não é estável por padrão, mas é \textit{in-place}, ou seja, não requer memória auxiliar significativa além do vetor de entrada.

\medskip
A seguir, apresenta-se um exemplo ilustrativo de execução do algoritmo.

\begin{exmp}
Considere o vetor $A = [0.5, 0.2, 0.9, 0.3, 0.7]$. O objetivo é ordená-lo utilizando o \textit{Selection Sort}.

\begin{enumerate}
    \item \textbf{Primeira iteração:}  
    O menor elemento na sublista $[0.5, 0.2, 0.9, 0.3, 0.7]$ é $0.2$.  
    Troca-se $0.2$ com o primeiro elemento $0.5$:
    \[
    A = [0.2, 0.5, 0.9, 0.3, 0.7].
    \]

    \item \textbf{Segunda iteração:}  
    O menor elemento na sublista $[0.5, 0.9, 0.3, 0.7]$ é $0.3$.  
    Troca-se $0.3$ com o primeiro elemento da sublista $0.5$:
    \[
    A = [0.2, 0.3, 0.9, 0.5, 0.7].
    \]

    \item \textbf{Terceira iteração:}  
    O menor elemento na sublista $[0.9, 0.5, 0.7]$ é $0.5$.  
    Troca-se $0.5$ com $0.9$:
    \[
    A = [0.2, 0.3, 0.5, 0.9, 0.7].
    \]

    \item \textbf{Quarta iteração:}  
    O menor elemento na sublista $[0.9, 0.7]$ é $0.7$.  
    Troca-se $0.7$ com $0.9$:
    \[
    A = [0.2, 0.3, 0.5, 0.7, 0.9].
    \]

    \item \textbf{Quinta iteração:}  
    Resta apenas um elemento $[0.9]$, que já está na posição correta.
\end{enumerate}
\end{exmp}

\medskip
O pseudocódigo correspondente é apresentado a seguir.

\begin{center}
\begin{minipage}{.9\linewidth}
\begin{algorithm}[H]
\DontPrintSemicolon
\hspace{-0.4cm}\textbf{selectionSort(values: array of float, n: integer)}

\For{$i \gets 0$ \KwTo $n-2$}{
    $minIndex \gets i$\;
    \For{$j \gets i+1$ \KwTo $n-1$}{
        \If{$values[j] < values[minIndex]$}{
            $minIndex \gets j$\;
        }
    }
    \textbf{swap}($values[i], values[minIndex]$)\;
}
\caption{Selection sort}
\label{lab:alg-selectionSort}
\end{algorithm}
\end{minipage}
\end{center}

\subsection{Implementações}
\begin{lstlisting}[language=python,caption={Selection sort em Python},captionpos=t]
def selectionSort(values, n):
    for i in range(n):
        min_index = i
        for j in range(i+1, n):
            if values[j] < values[min_index]:
                min_index = j
        values[i], values[min_index] = values[min_index], values[i]
\end{lstlisting}
\begin{lstlisting}[language=C,caption={Selection sort em C},captionpos=t]
void selectionSort(int values[], int n){
    for(int i = 0; i < n-1; i++){
        int minIndex = i;
        for(int j = i+1; j < n; j++){
            if(values[j] < values[minIndex]){
                minIndex = j;
            }
        }
        if(minIndex != i){
            int temp = values[i];
            values[i] = values[minIndex];
            values[minIndex] = temp;
        }
    }
}
\end{lstlisting}
\begin{lstlisting}[language=C++,caption={Selection sort em C++},captionpos=t]
#include <vector>
using namespace std;

void selectionSort(vector<int>& arr) {
    int n = arr.size();
    for (int i = 0; i < n - 1; i++) {
        int minIndex = i;
        for (int j = i + 1; j < n; j++) {
            if (arr[j] < arr[minIndex])
                minIndex = j;
        }
        swap(arr[i], arr[minIndex]);
    }
}
\end{lstlisting}

\subsection{Análise de complexidade}
Nesta seção, analisamos formalmente as complexidades de tempo e espaço do algoritmo \textit{Selection Sort}, que é baseado em comparações diretas entre elementos.

\subsubsection{Complexidade de Tempo}

Seja $n$ o número de elementos do vetor de entrada. O algoritmo executa as seguintes etapas principais:

\begin{enumerate}
    \item Para cada posição $i$, encontra-se o menor elemento na sublista não ordenada — custo de $O(n-i)$ comparações.
    \item Realiza uma troca (swap) do menor elemento encontrado com o elemento na posição $i$ — custo de $O(1)$.
\end{enumerate}

O número total de comparações $C(n)$ é:
\[
C(n) = (n-1) + (n-2) + \dots + 1 = \frac{n(n-1)}{2}.
\]

\begin{equation}
T(n) \in O(n^2)
\end{equation}

\noindent{\textbf{Prova:}}  
\[
C(n) = \sum_{i=1}^{n-1} i = \frac{n(n-1)}{2} \leq \frac{n^2}{2} \in O(n^2)
\]  
Logo, para todo $n \geq 1$, existe uma constante $c = 1/2$ tal que $T(n) \leq c n^2$, garantindo que
\[
T(n) \in O(n^2).
\]
$\hfill\Box$

\bigskip
\noindent{\textbf{Discussão:}}  
O \textit{Selection Sort} tem o mesmo custo de comparações no melhor, pior e caso médio, ou seja, $O(n^2)$, independentemente da distribuição dos elementos. É ineficiente para grandes conjuntos de dados, mas seu comportamento é previsível e sua implementação simples.

\subsubsection{Complexidade de Espaço}

O algoritmo \textit{Selection Sort} utiliza apenas um número constante de variáveis auxiliares, como $minIndex$ e temporários para troca, além do vetor de entrada, que é modificado \textit{in-place}.

\begin{equation}
S(n) \in O(1)
\end{equation}

\noindent{\textbf{Prova:}}  
Sejam $c_1, c_2$ constantes correspondentes às variáveis auxiliares, temos
\[
S(n) = c_1 + c_2 \leq c
\]  
para todo $n \geq 0$, implicando que
\[
S(n) \in O(1).
\]
$\hfill\Box$

\bigskip
\noindent{\textbf{Discussão:}}  
O \textit{Selection Sort} é um algoritmo \textit{in-place}, exigindo espaço constante, o que o torna adequado quando a memória é limitada. No entanto, sua baixa eficiência em tempo o restringe a conjuntos de dados pequenos ou como sub-rotina em algoritmos híbridos, como no \textit{Bucket Sort}.

\section{Shell sort}
\subsection{Descrição e Funcionamento}
O \textit{Shell Sort} é um algoritmo de ordenação baseado em comparação e uma generalização do \textit{Insertion Sort}.  
A ideia central do Shell Sort é permitir que elementos distantes troquem de posição, acelerando a ordenação inicial de elementos que estão muito fora de posição.  
O algoritmo utiliza uma sequência de incrementos (gaps) que diminuem progressivamente até 1. Em cada incremento, aplica-se um \textit{insertion sort} considerando apenas elementos separados pelo gap atual.

\medskip
O Shell Sort melhora o desempenho do Insertion Sort, especialmente para vetores grandes, reduzindo o número de deslocamentos necessários para ordenar elementos.

\medskip
A seguir, apresenta-se um exemplo ilustrativo de execução.

\begin{exmp}
Considere o vetor $A = [9.0, 8.0, 3.0, 7.0, 5.0]$. O objetivo é ordená-lo utilizando o \textit{Shell Sort} com sequência de gaps \([3,1]\).

\begin{enumerate}
    \item \textbf{Gap inicial 3:}  
    Comparamos e ordenamos elementos separados por 3 posições:
    \[
    \text{pares a comparar: } (9,7), (8,5)
    \]
    Após esta etapa:
    \[
    A = [7.0, 5.0, 3.0, 9.0, 8.0]
    \]

    \item \textbf{Gap 1 (Insertion Sort final):}  
    Agora, aplicamos o Insertion Sort tradicional:
    \begin{itemize}
        \item Inserimos 5.0 na posição correta: $[5.0,7.0,3.0,9.0,8.0]$
        \item Inserimos 3.0 na posição correta: $[3.0,5.0,7.0,9.0,8.0]$
        \item Inserimos 9.0 e 8.0 nas posições corretas: $[3.0,5.0,7.0,8.0,9.0]$
    \end{itemize}
\end{enumerate}

O vetor final ordenado é:
\[
A = [3.0,5.0,7.0,8.0,9.0]
\]
\end{exmp}

\medskip
O pseudocódigo correspondente é apresentado a seguir.

\begin{center}
\begin{minipage}{.9\linewidth}
\begin{algorithm}[H]
\DontPrintSemicolon
\hspace{-0.4cm}\textbf{shellSort(values: array of float, n: integer)}

$gap \gets \lfloor n/2 \rfloor$\;
\While{$gap > 0$}{
    \For{$i \gets gap$ \KwTo $n-1$}{
        $temp \gets values[i]$\;
        $j \gets i$\;
        \While{$j \ge gap \text{ and } values[j-gap] > temp$}{
            $values[j] \gets values[j-gap]$\;
            $j \gets j - gap$\;
        }
        $values[j] \gets temp$\;
    }
    $gap \gets \lfloor gap/2 \rfloor$\;
}
\caption{Shell sort}
\label{lab:alg-shellSort}
\end{algorithm}
\end{minipage}
\end{center}


\subsection{Implementações}
\begin{lstlisting}[language=python,caption={Shell sort em Python},captionpos=t]
def shellSort(arr):
    n = len(arr)
    gap = n // 2
    while gap > 0:
        for i in range(gap, n):
            temp = arr[i]
            j = i
            while j >= gap and arr[j-gap] > temp:
                arr[j] = arr[j-gap]
                j -= gap
            arr[j] = temp
        gap //= 2
\end{lstlisting}
\begin{lstlisting}[language=C,caption={Shell sort em C},captionpos=t]
void shellSort(int arr[], int n) {
    for (int gap = n/2; gap > 0; gap /= 2) {
        for (int i = gap; i < n; i++) {
            int temp = arr[i];
            int j = i;
            while (j >= gap && arr[j-gap] > temp) {
                arr[j] = arr[j-gap];
                j -= gap;
            }
            arr[j] = temp;
        }
    }
}
\end{lstlisting}
\begin{lstlisting}[language=C++,caption={Shell sort em C++},captionpos=t]
#include <vector>
using namespace std;

void shellSort(vector<int>& arr) {
    int n = arr.size();
    for (int gap = n / 2; gap > 0; gap /= 2) {
        for (int i = gap; i < n; i++) {
            int temp = arr[i];
            int j = i;
            while (j >= gap && arr[j - gap] > temp) {
                arr[j] = arr[j - gap];
                j -= gap;
            }
            arr[j] = temp;
        }
    }
}
\end{lstlisting}

\subsection{Análise de complexidade}
Nesta seção, analisamos formalmente as complexidades de tempo e espaço do algoritmo \textit{Shell Sort}.  
O desempenho depende fortemente da sequência de gaps escolhida.

\subsubsection{Complexidade de Tempo}

Seja $n$ o número de elementos do vetor de entrada.

O algoritmo executa múltiplos \textit{gapped insertion sorts}, com o seguinte comportamento:

\begin{enumerate}
    \item Cada \textit{pass} com gap $g$ realiza $O(n)$ comparações e deslocamentos em média.
    \item Para sequências de gaps clássicas (como $n/2, n/4, \dots, 1$), o custo total médio é aproximadamente $O(n^{3/2})$.
    \item No pior caso, dependendo da sequência de gaps, o custo pode atingir $O(n^2)$.
\end{enumerate}

Assim, o tempo total de execução $T(n)$ pode ser expresso como:

\[
T(n) = a \cdot n^{3/2} + b
\]

\begin{equation}
T(n) \in O(n^{3/2})
\end{equation}

\noindent{\textbf{Prova:}}  
Considerando $k$ passagens com gaps decrescentes, cada passagem executa no máximo $n$ inserções deslocando elementos. Para a sequência clássica $n/2, n/4, \dots, 1$, o somatório do número de movimentos leva a:

\[
\sum_{i=1}^{\log_2 n} O(n/i) \approx O(n \log n)
\]

A análise mais refinada considerando o número total de comparações e movimentos nos passes intermediários resulta em:

\[
T(n) \in O(n^{3/2})
\]

$\hfill\Box$

\bigskip

\noindent{\textbf{Discussão:}}  
O Shell Sort é significativamente mais rápido que o Insertion Sort para vetores grandes. A escolha da sequência de gaps é crucial para o desempenho, e várias sequências propostas na literatura podem reduzir a complexidade média abaixo de $O(n^{3/2})$.

\subsubsection{Complexidade de Espaço}

O Shell Sort é um algoritmo \textit{in-place}, utilizando apenas variáveis auxiliares para troca de elementos:

\begin{itemize}
    \item Vetor de entrada $A[1 \ldots n]$ — espaço $O(n)$.
    \item Variável temporária para troca — espaço $O(1)$.
\end{itemize}

O espaço total $S(n)$ é, portanto:

\begin{equation}
S(n) \in O(n)
\end{equation}

\noindent{\textbf{Prova:}}  
Como o algoritmo manipula o vetor original sem alocar estruturas adicionais significativas, temos:

\[
S(n) = O(n) + O(1) \in O(n)
\]

$\hfill\Box$

\bigskip

\noindent{\textbf{Discussão:}}  
Por ser \textit{in-place}, o Shell Sort é adequado quando a memória extra é limitada. Ele combina simplicidade de implementação com desempenho relativamente eficiente para entradas de tamanho moderado a grande, especialmente quando usado para ordenar sublistas em algoritmos como o Bucket Sort.







\section{Gnome sort}
\subsection{Descrição e Funcionamento}
O \textit{Gnome Sort} é um algoritmo de ordenação simples e baseado em comparação, semelhante ao \textit{Insertion Sort}, mas com uma abordagem intuitiva inspirada no movimento de um gnomo que anda para frente e para trás ajustando elementos fora de ordem. Ele percorre o vetor sequencialmente e, sempre que encontra um par de elementos fora de ordem, realiza uma troca e retrocede uma posição; caso contrário, avança para o próximo elemento. O processo se repete até que o vetor esteja completamente ordenado.

\medskip
A seguir, apresenta-se um exemplo ilustrativo de execução do algoritmo.

\begin{exmp}
Considere o vetor $A = [0.4, 0.2, 0.5, 0.1, 0.3]$. O objetivo é ordená-lo utilizando o \textit{Gnome Sort}.

\begin{enumerate}
    \item \textbf{Posição inicial:}  
    Começamos na posição $i = 1$ (segundo elemento).

    \item \textbf{Passo 1:}  
    Comparamos $A[1]=0.2$ com $A[0]=0.4$. Como $0.2 < 0.4$, trocamos os elementos:
    \[
    A = [0.2, 0.4, 0.5, 0.1, 0.3].
    \]  
    Retrocedemos para $i=0$, mas como não há elemento anterior, avançamos para $i=1$.

    \item \textbf{Passo 2:}  
    Comparamos $A[1]=0.4$ com $A[0]=0.2$. Como $0.4 \geq 0.2$, avançamos para $i=2$.

    \item \textbf{Passo 3:}  
    Comparamos $A[2]=0.5$ com $A[1]=0.4$. Como $0.5 \geq 0.4$, avançamos para $i=3$.

    \item \textbf{Passo 4:}  
    Comparamos $A[3]=0.1$ com $A[2]=0.5$. Como $0.1 < 0.5$, trocamos:
    \[
    A = [0.2, 0.4, 0.1, 0.5, 0.3].
    \]  
    Retrocedemos para $i=2$, comparamos $0.1 < 0.4$, trocamos:
    \[
    A = [0.2, 0.1, 0.4, 0.5, 0.3].
    \]  
    Retrocedemos para $i=1$, comparamos $0.1 < 0.2$, trocamos:
    \[
    A = [0.1, 0.2, 0.4, 0.5, 0.3].
    \]  
    Retrocedemos para $i=0$, avançamos para $i=1$.

    \item \textbf{Passo 5:}  
    Comparamos $A[1]=0.2$ com $A[0]=0.1$. Como $0.2 \geq 0.1$, avançamos para $i=2$.

    \item \textbf{Passo 6:}  
    Comparamos $A[2]=0.4$ com $A[1]=0.2$. Como $0.4 \geq 0.2$, avançamos para $i=3$.

    \item \textbf{Passo 7:}  
    Comparamos $A[3]=0.5$ com $A[2]=0.4$. Como $0.5 \geq 0.4$, avançamos para $i=4$.

    \item \textbf{Passo 8:}  
    Comparamos $A[4]=0.3$ com $A[3]=0.5$. Como $0.3 < 0.5$, trocamos:
    \[
    A = [0.1, 0.2, 0.4, 0.3, 0.5].
    \]  
    Retrocedemos para $i=3$, comparamos $0.3 < 0.4$, trocamos:
    \[
    A = [0.1, 0.2, 0.3, 0.4, 0.5].
    \]  
    Retrocedemos para $i=2$, $0.3 \geq 0.2$, avançamos para $i=3$, $0.4 \geq 0.3$, avançamos para $i=4$, $0.5 \geq 0.4$, avançamos para $i=5$ (fim).

\end{enumerate}
Vetor ordenado final:
\[
A = [0.1, 0.2, 0.3, 0.4, 0.5].
\]
\end{exmp}

\medskip
O pseudocódigo correspondente é apresentado a seguir.

\begin{center}
\begin{minipage}{.9\linewidth}
\begin{algorithm}[H]
\DontPrintSemicolon
\hspace{-0.4cm}\textbf{gnomeSort(values: array of float, n: integer)}

$i \gets 1$\;
\While{$i < n$}{
    \If{$i = 0$ \textbf{or} $values[i] \geq values[i-1]$}{
        $i \gets i + 1$\;
    }
    \Else{
        trocar $values[i]$ e $values[i-1]$\;
        $i \gets i - 1$\;
    }
}
\caption{Gnome sort}
\label{lab:alg-gnomeSort}
\end{algorithm}
\end{minipage}
\end{center}

\subsection{Implementações}
\begin{lstlisting}[language=python,caption={Gnome sort em Python},captionpos=t]
def gnomeSort(arr):
    i = 0
    n = len(arr)
    while i < n:
        if i == 0 or arr[i] >= arr[i-1]:
            i += 1
        else:
            arr[i], arr[i-1] = arr[i-1], arr[i]
            i -= 1
\end{lstlisting}
\begin{lstlisting}[language=C,caption={Gnome sort em C},captionpos=t]
void gnomeSort(int arr[], int n) {
    int i = 0;
    while (i < n) {
        if (i == 0 || arr[i] >= arr[i-1]) {
            i++;
        } else {
            int temp = arr[i];
            arr[i] = arr[i-1];
            arr[i-1] = temp;
            i--;
        }
    }
}
\end{lstlisting}
\begin{lstlisting}[language=C++,caption={Gnome sort em C++},captionpos=t]
#include <vector>
using namespace std;

void gnomeSort(vector<int>& arr) {
    int n = arr.size();
    int i = 0;
    while (i < n) {
        if (i == 0 || arr[i] >= arr[i - 1])
            i++;
        else {
            swap(arr[i], arr[i - 1]);
            i--;
        }
    }
}
\end{lstlisting}

\subsection{Análise de complexidade}
Nesta seção, analisamos formalmente as complexidades de tempo e espaço do algoritmo \textit{Gnome Sort}.  
Apesar de sua simplicidade, este algoritmo é eficiente apenas para vetores pequenos ou quase ordenados.

\subsubsection{Complexidade de Tempo}

Seja $n$ o número de elementos do vetor de entrada.

\begin{enumerate}
    \item No \textbf{pior caso}, o vetor está ordenado de forma inversa. Cada elemento pode precisar ser movido para o início, resultando em aproximadamente $\frac{n(n-1)}{2}$ comparações e trocas. Portanto:
    \begin{equation}
        T_{worst}(n) \in O(n^2)
    \end{equation}

    \item No \textbf{melhor caso}, o vetor já está ordenado. O algoritmo realiza uma única passagem, realizando $n-1$ comparações sem trocas:
    \begin{equation}
        T_{best}(n) \in O(n)
    \end{equation}

    \item No \textbf{caso médio}, para uma entrada aleatória, o número médio de comparações e trocas também cresce quadraticamente:
    \begin{equation}
        T_{avg}(n) \in O(n^2)
    \end{equation}
\end{enumerate}

\noindent{\textbf{Prova:}}  
Para o pior caso, cada elemento $i$ é comparado e possivelmente trocado até alcançar a posição correta, somando:
\[
1 + 2 + \dots + (n-1) = \frac{n(n-1)}{2} \in O(n^2).
\]  
Para o melhor caso, cada elemento é comparado apenas uma vez, somando $n-1 \in O(n)$, provando as fórmulas acima.
$\hfill\Box$

\subsubsection{Complexidade de Espaço}

O \textit{Gnome Sort} é um algoritmo \textit{in-place}:

\begin{itemize}
    \item O vetor de entrada $A[1 \ldots n]$ é reordenado sem necessidade de espaço auxiliar.
\end{itemize}

Portanto, a complexidade de espaço é:

\begin{equation}
S(n) \in O(1)
\end{equation}

\noindent{\textbf{Prova:}}  
O algoritmo apenas utiliza variáveis auxiliares para índices e temporários para troca de elementos, independentes de $n$. Logo, o consumo de memória é constante.
$\hfill\Box$

\bigskip
\noindent{\textbf{Discussão:}}  
O \textit{Gnome Sort} é simples de implementar e conceitualmente intuitivo, mas não é adequado para grandes entradas devido à complexidade quadrática no pior e caso médio. É útil principalmente para vetores pequenos ou quase ordenados, e pode ser empregado como sub-rotina em algoritmos como \textit{Bucket Sort} para ordenar baldes individuais.

\section{Odd-Even sort}
\subsection{Descrição e Funcionamento}
O \textit{Odd-Even Sort}, também conhecido como Brick Sort, é um algoritmo de ordenação comparativo baseado em trocas, projetado para ser simples e facilmente paralelizável.  
Ele é uma variação do Bubble Sort, alternando entre fases de comparação de pares ímpares e pares pares. Durante cada fase, elementos adjacentes são comparados e trocados se estiverem fora de ordem. O processo é repetido até que nenhuma troca seja necessária, garantindo que o vetor esteja ordenado.

\medskip
O algoritmo segue a seguinte lógica:
\begin{enumerate}
    \item Em uma fase ímpar, comparamos e trocamos os elementos nos índices (1,2), (3,4), (5,6), \dots.
    \item Em uma fase par, comparamos e trocamos os elementos nos índices (0,1), (2,3), (4,5), \dots.
    \item Repetimos alternadamente as fases ímpar e par até que não ocorram trocas em nenhuma das fases.
\end{enumerate}

\medskip
A seguir, apresenta-se um exemplo ilustrativo de execução.

\begin{exmp}
Considere o vetor $A = [5.0, 3.0, 2.0, 4.0]$. O objetivo é ordená-lo utilizando o \textit{Odd-Even Sort}.

\begin{enumerate}
    \item \textbf{Fase ímpar:}  
    Comparações de índices (1,2):
    \[
    [5.0, 3.0, 2.0, 4.0] \to [5.0, 2.0, 3.0, 4.0]
    \]
    Nenhuma outra troca na fase ímpar (pares além do limite).

    \item \textbf{Fase par:}  
    Comparações de índices (0,1) e (2,3):
    \[
    [5.0, 2.0, 3.0, 4.0] \to [2.0, 5.0, 3.0, 4.0] \to [2.0, 5.0, 3.0, 4.0] \to [2.0, 3.0, 4.0, 5.0]
    \]

    \item \textbf{Próximas fases:}  
    Repetindo alternadamente ímpar e par, nenhuma troca ocorre, indicando que o vetor está ordenado:
    \[
    B = [2.0, 3.0, 4.0, 5.0].
    \]
\end{enumerate}
\end{exmp}

\medskip
O pseudocódigo correspondente é apresentado a seguir.

\begin{center}
\begin{minipage}{.9\linewidth}
\begin{algorithm}[H]
\DontPrintSemicolon
\hspace{-0.4cm}\textbf{oddEvenSort(values: array of float, n: integer)}\\[3pt]
$isSorted \gets false$\;
\While{not $isSorted$}{
    $isSorted \gets true$\;
    \textbf{Fase ímpar:}\;
    \For{$i \gets 1$ \KwTo $n-2$ \textbf{step} 2}{
        \If{$values[i] > values[i+1]$}{
            trocar $values[i] \leftrightarrow values[i+1]$\;
            $isSorted \gets false$\;
        }
    }
    \textbf{Fase par:}\;
    \For{$i \gets 0$ \KwTo $n-2$ \textbf{step} 2}{
        \If{$values[i] > values[i+1]$}{
            trocar $values[i] \leftrightarrow values[i+1]$\;
            $isSorted \gets false$\;
        }
    }
}
\caption{Odd-Even Sort}
\label{lab:alg-oddEvenSort}
\end{algorithm}
\end{minipage}
\end{center}

\subsection{Implementações}
\begin{lstlisting}[language=python,caption={Odd-Even sort em Python},captionpos=t]
def oddEvenSort(arr):
    n = len(arr)
    sorted = False
    while not sorted:
        sorted = True
        for i in range(1, n-1, 2):
            if arr[i] > arr[i+1]:
                arr[i], arr[i+1] = arr[i+1], arr[i]
                sorted = False
        for i in range(0, n-1, 2):
            if arr[i] > arr[i+1]:
                arr[i], arr[i+1] = arr[i+1], arr[i]
                sorted = False
\end{lstlisting}
\begin{lstlisting}[language=C,caption={Odd-Even sort em C},captionpos=t]
void oddEvenSort(int arr[], int n) {
    int sorted = 0;
    while (!sorted) {
        sorted = 1;
        for (int i = 1; i < n-1; i += 2) {
            if (arr[i] > arr[i+1]) {
                int temp = arr[i];
                arr[i] = arr[i+1];
                arr[i+1] = temp;
                sorted = 0;
            }
        }
        for (int i = 0; i < n-1; i += 2) {
            if (arr[i] > arr[i+1]) {
                int temp = arr[i];
                arr[i] = arr[i+1];
                arr[i+1] = temp;
                sorted = 0;
            }
        }
    }
}
\end{lstlisting}
\begin{lstlisting}[language=C++,caption={Odd-even sort em C++},captionpos=t]
#include <vector>
using namespace std;

void oddEvenSort(vector<int>& arr) {
    int n = arr.size();
    bool sorted = false;

    while (!sorted) {
        sorted = true;
        for (int i = 1; i < n - 1; i += 2) {
            if (arr[i] > arr[i + 1]) {
                swap(arr[i], arr[i + 1]);
                sorted = false;
            }
        }
        for (int i = 0; i < n - 1; i += 2) {
            if (arr[i] > arr[i + 1]) {
                swap(arr[i], arr[i + 1]);
                sorted = false;
            }
        }
    }
}
\end{lstlisting}

\subsection{Análise de complexidade}
Nesta seção, analisamos formalmente as complexidades de tempo e espaço do algoritmo \textit{Odd-Even Sort}.  
Trata-se de um algoritmo baseado em comparações, com comportamento semelhante ao Bubble Sort, porém com melhor paralelização possível.

\subsubsection{Complexidade de Tempo}

Seja $n$ o número de elementos do vetor de entrada.  
O algoritmo realiza no máximo $n$ fases (uma fase ímpar e uma par contam como uma iteração completa), e cada fase envolve até $n/2$ comparações e trocas. Portanto, no pior caso, o número total de comparações é proporcional a $n \cdot n = n^2$.

\begin{equation}
T(n) \in O(n^2)
\end{equation}

\noindent{\textbf{Prova:}}  
Cada iteração completa consiste em $n/2 + n/2 \leq n$ comparações. No pior caso (vetor em ordem inversa), precisamos de $n$ iterações para ordenar totalmente:
\[
T(n) \leq n \cdot n = n^2.
\]
Portanto,
\[
T(n) \in O(n^2).
\]
$\hfill\Box$

\bigskip
\noindent{\textbf{Discussão:}}  
Embora o caso médio seja geralmente mais rápido que o pior caso, a complexidade quadrática permanece dominante, tornando o algoritmo ineficiente para grandes conjuntos de dados.  
Sua vantagem reside na paralelização, onde cada fase pode ser executada simultaneamente em processadores diferentes.

\subsubsection{Complexidade de Espaço}

O algoritmo \textit{Odd-Even Sort} realiza trocas \textit{in-place} entre elementos adjacentes e não requer memória auxiliar significativa além do vetor de entrada e uma variável booleana de controle.

\begin{equation}
S(n) \in O(1)
\end{equation}

\noindent{\textbf{Prova:}}  
O vetor de entrada ocupa $O(n)$ espaço por definição, mas nenhum espaço adicional proporcional a $n$ é necessário. Apenas variáveis temporárias constantes (como $isSorted$ e índice $i$) são usadas. Logo,
\[
S(n) \in O(1).
\]
$\hfill\Box$

\bigskip
\noindent{\textbf{Discussão:}}  
\textit{Odd-Even Sort} é um algoritmo \textit{in-place} e estável.  
Apesar de sua simplicidade e facilidade de paralelização, não é recomendado para grandes entradas devido à sua complexidade de tempo $O(n^2)$ no pior caso.  
Entretanto, ele é útil para ordenar baldes em algoritmos de \textit{Bucket Sort} de maneira local, onde o número de elementos por balde é pequeno.







\section{Pancake sort}
\subsection{Descrição e Funcionamento}
O \textit{Pancake Sort} é um algoritmo de ordenação baseado em uma analogia culinária: ordenar uma pilha de panquecas de tamanhos diferentes utilizando apenas uma espátula que permite inverter o topo da pilha até uma determinada posição.  
Em termos de computação, isso significa que podemos reverter qualquer prefixo do vetor de entrada, e o objetivo é ordenar o vetor completo utilizando um número mínimo de reversões.

\medskip
O algoritmo funciona iterativamente, selecionando o maior elemento não ordenado, trazendo-o para o topo do vetor (caso não esteja já lá) e depois invertendo o prefixo correspondente para levá-lo à sua posição correta no final do vetor ainda não ordenado. Esse processo é repetido para os próximos maiores elementos até que todo o vetor esteja ordenado.

\medskip
A seguir, apresenta-se um exemplo ilustrativo de execução do algoritmo.

\begin{exmp}
Considere o vetor $A = [3.0, 6.0, 1.0, 5.0, 4.0]$. O objetivo é ordená-lo utilizando o \textit{Pancake Sort}.

\begin{enumerate}
    \item \textbf{Primeira iteração:}  
    O maior elemento no prefixo completo é $6.0$, que está na posição 2.  
    - Invertendo o prefixo até a posição 2: $[6.0, 3.0, 1.0, 5.0, 4.0]$  
    - Invertendo o prefixo até a posição 4 (tamanho da sublista não ordenada): $[1.0, 3.0, 6.0, 5.0, 4.0]$

    \item \textbf{Segunda iteração:}  
    O próximo maior elemento é $5.0$, na posição 3 do subvetor não ordenado $[1.0, 3.0, 6.0, 5.0]$.  
    - Invertendo até a posição 3: $[6.0, 3.0, 1.0, 5.0, 4.0]$  
    - Invertendo o prefixo correspondente à sublista: $[5.0, 1.0, 3.0, 6.0, 4.0]$

    \item \textbf{Iterações seguintes:}  
    Repetindo o procedimento para $4.0$, $3.0$ e $1.0$, até que todo o vetor esteja ordenado:  
    \[
    B = [1.0, 3.0, 4.0, 5.0, 6.0].
    \]
\end{enumerate}
\end{exmp}

\medskip
O pseudocódigo correspondente é apresentado a seguir.

\begin{center}
\begin{minipage}{.9\linewidth}
\begin{algorithm}[H]
\DontPrintSemicolon
\hspace{-0.4cm}\textbf{pancakeSort(values: array of float, n: integer)}\\[3pt]
\textbf{function} flip(values: array of float, k: integer)\\
\For{$i \gets 0$ \KwTo $\lfloor k/2 \rfloor - 1$}{
    swap(values[i], values[k-i-1])\;
}
\textbf{end function}\\[2pt]
\For{$curr\_size \gets n$ \KwTo $2$}{
    $max\_idx \gets$ índice do maior elemento em $values[0..curr\_size-1]$\;
    \If{$max\_idx \neq curr\_size-1$}{
        \If{$max\_idx \neq 0$}{
            flip(values, max\_idx+1)\;
        }
        flip(values, curr\_size)\;
    }
}
\caption{Pancake sort}
\label{lab:alg-pancakeSort}
\end{algorithm}
\end{minipage}
\end{center}

\subsection{Implementações}
\begin{lstlisting}[language=python,caption={Pancake sort em Python},captionpos=t]
def flip(arr, i):
    arr[:i+1] = arr[:i+1][::-1]

def findMaxIndex(arr, n):
    mi = 0
    for i in range(1, n):
        if arr[i] > arr[mi]:
            mi = i
    return mi

def pancakeSort(arr):
    n = len(arr)
    for currSize in range(n, 1, -1):
        mi = findMaxIndex(arr, currSize)
        if mi != currSize-1:
            flip(arr, mi)
            flip(arr, currSize-1)
\end{lstlisting}
\begin{lstlisting}[language=C,caption={Pancake sort em C},captionpos=t]
void flip(int arr[], int i) {
    int start = 0;
    while (start < i) {
        int temp = arr[start];
        arr[start] = arr[i];
        arr[i] = temp;
        start++;
        i--;
    }
}
int findMaxIndex(int arr[], int n) {
    int mi = 0;
    for (int i = 1; i < n; i++)
        if (arr[i] > arr[mi]) mi = i;
    return mi;
}
void pancakeSort(int arr[], int n) {
    for (int currSize = n; currSize > 1; currSize--) {
        int mi = findMaxIndex(arr, currSize);
        if (mi != currSize-1) {
            flip(arr, mi);
            flip(arr, currSize-1);
        }
    }
}
\end{lstlisting}
\begin{lstlisting}[language=C++,caption={Pancake sort em C++},captionpos=t]
#include <vector>
#include <algorithm>
using namespace std;

int findMaxIndex(vector<int>& arr, int n) {
    int mi = 0;
    for (int i = 1; i < n; i++)
        if (arr[i] > arr[mi]) mi = i;
    return mi;
}

void flip(vector<int>& arr, int i) {
    reverse(arr.begin(), arr.begin() + i + 1);
}

void pancakeSort(vector<int>& arr) {
    for (int curr_size = arr.size(); curr_size > 1; curr_size--) {
        int mi = findMaxIndex(arr, curr_size);
        if (mi != curr_size - 1) {
            flip(arr, mi);
            flip(arr, curr_size - 1);
        }
    }
}
\end{lstlisting}

\subsection{Análise de complexidade}

Nesta seção, analisamos formalmente as complexidades de tempo e espaço do algoritmo \textit{Pancake Sort}.  
O algoritmo realiza múltiplas reversões de prefixos, cada uma operando sobre parte do vetor.

\subsubsection{Complexidade de Tempo}

Seja $n$ o número de elementos do vetor de entrada.  
Em cada iteração, buscamos o maior elemento não ordenado (custo $O(curr\_size)$) e realizamos no máximo duas inversões de prefixo (cada uma também $O(curr\_size)$).  
Somando para todas as iterações do tamanho $n$ até 2, temos:

\[
T(n) = \sum_{curr\_size=2}^{n} O(3 \cdot curr\_size) = O(3 \sum_{i=2}^{n} i) = O(n^2)
\]

\begin{equation}
T(n) \in O(n^2)
\end{equation}

\noindent{\textbf{Prova:}}  
O somatório $\sum_{i=2}^{n} i = \frac{n(n+1)}{2} - 1 \in O(n^2)$.  
Multiplicando por constantes positivas (o fator 3 das operações por iteração) não altera a ordem de crescimento. Portanto, para alguma constante $c>0$ e $n_0 \geq 0$:

\[
T(n) \leq c n^2, \quad \forall n \geq n_0.
\]
Logo,
\[
T(n) \in O(n^2).
\]
$\hfill\Box$

\bigskip
\noindent{\textbf{Discussão:}}  
Embora visualmente interessante e útil como sub-rotina em técnicas como Bucket Sort, o \textit{Pancake Sort} é menos eficiente do que algoritmos clássicos de comparação ($O(n\log n)$) em grandes vetores.

\subsubsection{Complexidade de Espaço}

O algoritmo utiliza essencialmente o vetor de entrada e algumas variáveis auxiliares para índices e temporários durante a reversão.  
Não há necessidade de estruturas adicionais proporcionais a $n$, caracterizando o algoritmo como \textit{in-place}.

\begin{equation}
S(n) \in O(1)
\end{equation}

\noindent{\textbf{Prova:}}  
A quantidade de memória adicional é constante, independente de $n$. Portanto, existe $c>0$ tal que:

\[
S(n) \leq c, \quad \forall n \geq 0.
\]
Logo,
\[
S(n) \in O(1).
\]
$\hfill\Box$

\bigskip
\noindent{\textbf{Discussão:}}  
O \textit{Pancake Sort} é eficiente em termos de espaço, pois todas as operações ocorrem no vetor original, sem alocação de memória adicional proporcional ao tamanho da entrada.









% \section{Recombinant sort}
% \label{lab:alg-recombinantSort}

% \begin{lstlisting}[language=C,caption={Recombinant sort em C},captionpos=t]
% #include <stdio.h>
% #include <stdlib.h>

% // Funcao para mesclar dois subarrays
% void merge(int arr[], int left[], int leftSize, int right[], int rightSize) {
%     int i = 0, j = 0, k = 0;
%     while (i < leftSize && j < rightSize) {
%         if (left[i] < right[j])
%             arr[k++] = left[i++];
%         else
%             arr[k++] = right[j++];
%     }
%     while (i < leftSize) arr[k++] = left[i++];
%     while (j < rightSize) arr[k++] = right[j++];
% }

% // Recombinant Sort (recursivo)
% void recombinantSort(int arr[], int n) {
%     if (n <= 1) return;

%     int mid = n / 2;
%     int* left = (int*)malloc(mid * sizeof(int));
%     int* right = (int*)malloc((n - mid) * sizeof(int));

%     for (int i = 0; i < mid; i++) left[i] = arr[i];
%     for (int i = mid; i < n; i++) right[i - mid] = arr[i];

%     recombinantSort(left, mid);
%     recombinantSort(right, n - mid);
%     merge(arr, left, mid, right, n - mid);

%     free(left);
%     free(right);
% }
% \end{lstlisting}

% \begin{lstlisting}[language=C++,caption={Recombinant sort em C++},captionpos=t]
% #include <vector>
% using namespace std;

% void merge(vector<int>& arr, vector<int>& left, vector<int>& right) {
%     int i = 0, j = 0, k = 0;
%     while (i < left.size() && j < right.size()) {
%         if (left[i] < right[j])
%             arr[k++] = left[i++];
%         else
%             arr[k++] = right[j++];
%     }
%     while (i < left.size()) arr[k++] = left[i++];
%     while (j < right.size()) arr[k++] = right[j++];
% }

% void recombinantSort(vector<int>& arr) {
%     int n = arr.size();
%     if (n <= 1) return;

%     int mid = n / 2;
%     vector<int> left(arr.begin(), arr.begin() + mid);
%     vector<int> right(arr.begin() + mid, arr.end());

%     recombinantSort(left);
%     recombinantSort(right);
%     merge(arr, left, right);
% }
% \end{lstlisting}

% \begin{lstlisting}[language=Python,caption={Recombinant sort em Python},captionpos=t]
% def recombinant_sort(arr):
%     if len(arr) <= 1:
%         return arr

%     mid = len(arr) // 2
%     left = recombinant_sort(arr[:mid])
%     right = recombinant_sort(arr[mid:])

%     merged = []
%     i = j = 0
%     while i < len(left) and j < len(right):
%         if left[i] < right[j]:
%             merged.append(left[i])
%             i += 1
%         else:
%             merged.append(right[j])
%             j += 1

%     merged.extend(left[i:])
%     merged.extend(right[j:])
%     return merged
% \end{lstlisting}

% \noindent
% O algoritmo \textbf{Recombinant Sort} segue o paradigma de \textit{divisão e conquista}, dividindo o vetor em duas partes, ordenando-as recursivamente e combinando os resultados. A etapa de recombinação é semelhante à do Merge Sort, garantindo estabilidade e ordenação eficiente.

% \subsection{Análise de complexidade do algoritmo}

% Considerando que o vetor $arr$ tem tamanho $n$, o algoritmo divide o problema em dois subproblemas de tamanho $n/2$ e realiza uma operação de mesclagem linear.

% O tempo total pode ser expresso pela recorrência:
% \[
% T(n) = 2T(n/2) + O(n)
% \]

% Aplicando o Teorema Mestre, obtemos:
% \[
% T(n) = O(n \log n)
% \]

% Assim, temos:
% \begin{itemize}
%     \item Melhor caso: $O(n \log n)$
%     \item Caso médio: $O(n \log n)$
%     \item Pior caso: $O(n \log n)$
% \end{itemize}

% O espaço adicional necessário é $O(n)$ devido à criação dos subarrays durante as chamadas recursivas.

\section{Cocktail Shaker Sort}

\textbf{Descrição:}
O Cocktail Shaker Sort, também conhecido como Shake Sort ou Bidirectional Bubble Sort, é uma variação do Bubble Sort em que a varredura na lista é feita alternadamente nas direções esquerda-direita e direita-esquerda. Isso permite que elementos grandes “afundem” para o final e elementos pequenos “subam” para o início mais rapidamente em cada ciclo, reduzindo o número de iterações em listas parcialmente ordenadas.

\begin{exmp}
Considere ordenar o vetor $A = [4, 3, 2, 1]$ com o Cocktail Shaker Sort:
\begin{enumerate}
\item {Primeira passagem (esquerda-direita):} compara e troca adjacentes, movendo o maior para a última posição. Agora $[3, 2, 1, 4]$.
\item {Primeira passagem (direita-esquerda):} volta comparando e trocando, movendo o menor para a primeira posição. Agora $[1, 3, 2, 4]$.
\item Passagens seguintes repetem direita-esquerda e esquerda-direita, até o vetor ficar ordenado: $[1, 2, 3, 4]$.
\end{enumerate}
\end{exmp}

\begin{algorithm}[H]
\DontPrintSemicolon
\hspace{-0.4cm}\textbf{cocktailShakerSort(A: array, n: int)}\;
inicio $\gets 0$\;
fim $\gets n - 1$\;
troca $\gets true$\;
\While{troca}{
   troca $\gets false$\;
   \For{$i \gets inicio$ \textbf{até} $fim - 1$}{
      \If{$A[i] > A[i + 1]$}{
         trocar($A[i],A[i + 1]$)\;
         troca $\gets true$\;
      }
   }
   fim $\gets fim - 1$\;
   \For{$i \gets fim$ \textbf{até} $inicio + 1$}{
      \If{$A[i] < A[i - 1]$}{
         trocar ($A[i],A[i - 1]$)\;
         troca $\gets true$\;
      }
   }
   inicio $\gets inicio + 1$\;
}
\caption{Cocktail Shaker Sort (simplificado)}
\label{lab:alg-cocktailshaker}
\end{algorithm}

\begin{lstlisting}[language=Python, caption={Cocktail Shaker Sort em Python}, ,captionpos=t,label=code:cocktailshakerPy]
def cocktail_shaker_sort(arr):
    n = len(arr)
    start = 0
    end = n - 1
    swapped = True
    while swapped:
        swapped = False
        # Passagem da esquerda para a direita
        for i in range(start, end):
            if arr[i] > arr[i + 1]:
                arr[i], arr[i + 1] = arr[i + 1], arr[i]
                swapped = True
        end -= 1
        # Passagem da direita para a esquerda
        for i in range(end, start, -1):
            if arr[i] < arr[i - 1]:
                arr[i], arr[i - 1] = arr[i - 1], arr[i]
                swapped = True
        start += 1
\end{lstlisting}

\begin{lstlisting}[language=C, caption={Cocktail Shaker Sort em C}, ,captionpos=t,label=code:cocktailshakerC]
#include <stdio.h>
#include <stdbool.h>

void cocktailShakerSort(int arr[], int n) {
    int start = 0, end = n - 1;
    bool swapped = 1;
    while (swapped) {
        swapped = 0;
        // Passagem esquerda-direita
        for (int i = start; i < end; i++) {
            if (arr[i] > arr[i + 1]) {
                int temp = arr[i];
                arr[i] = arr[i + 1];
                arr[i + 1] = temp;
                swapped = 1;
            }
        }
        end--;
        // Passagem direita-esquerda
        for (int i = end; i > start; i--) {
            if (arr[i] < arr[i - 1]) {
                int temp = arr[i];
                arr[i] = arr[i - 1];
                arr[i - 1] = temp;
                swapped = 1;
            }
        }
        start++;
    }
}
\end{lstlisting}

\begin{lstlisting}[language=C++, caption={Cocktail Shaker Sort em C++}, ,captionpos=t,label=code:cocktailshakerC++]
#include <iostream>
#include <vector>

void cocktailShakerSort(std::vector<int>& arr) {
    int n = arr.size();
    bool swapped = true;
    int start = 0;
    int end = n - 1;

    while (swapped) {
        swapped = false;

        // Passagem esquerda para direita
        for (int i = start; i < end; ++i) {
            if (arr[i] > arr[i + 1]) {
                std::swap(arr[i], arr[i + 1]);
                swapped = true;
            }
        }

        // Se nao houve trocas, o array esta ordenado
        if (!swapped)
            break;

        swapped = false;
        --end;

        // Passagem direita para esquerda
        for (int i = end; i > start; --i) {
            if (arr[i] < arr[i - 1]) {
                std::swap(arr[i], arr[i - 1]);
                swapped = true;
            }
        }
        ++start;
    }
}
\end{lstlisting}

\subsection{Análise de Complexidade do Algoritmo}

\subsubsection{Complexidade de Tempo}

\paragraph{Melhor Caso: $O(n)$}
\textbf{Cenário:} Array já ordenado
\\
\textbf{Prova:}
\begin{itemize}
\item Na primeira passagem esquerda-direita, não há trocas (\texttt{swapped = false})
\item O algoritmo termina após verificar todos os $n-1$ pares adjacentes
\item \textbf{Total de comparações:} $n-1 = O(n)$
\item \textbf{Total de trocas:} $0$
\end{itemize}

\paragraph{Pior Caso: $O(n^2)$}
\textbf{Cenário:} Array em ordem completamente reversa
\\
\textbf{Prova:}
\begin{itemize}
\item Em cada iteração $k$, o algoritmo realiza:
\begin{itemize}
\item Passagem esquerda-direita: $(n-1-k)$ comparações
\item Passagem direita-esquerda: $(n-1-k)$ comparações
\end{itemize}
\item Número total de iterações: $\lceil n/2 \rceil$
\item \textbf{Cálculo das comparações:}
\end{itemize}

\begin{align}
\sum_{k=0}^{\lceil n/2 \rceil - 1} 2(n-1-k) &= 2\sum_{k=0}^{\lceil n/2 \rceil - 1}(n-1-k) \\
&= 2\left[(n-1)\lceil n/2 \rceil - \sum_{k=0}^{\lceil n/2 \rceil - 1}k\right] \\
&= 2\left[(n-1)\lceil n/2 \rceil - \frac{\lceil n/2 \rceil(\lceil n/2 \rceil - 1)}{2}\right] \\
&\approx 2\left[(n-1)(n/2) - \frac{(n/2)(n/2-1)}{2}\right] \\
&= n^2-n-\frac{n^2}{4}+\frac{n}{2} \\
&= \frac{3n^2}{4} - \frac{n}{2} = O(n^2)
\end{align}

\paragraph{Caso Médio: $O(n^2)$}
\textbf{Prova:}
\begin{itemize}
\item Número esperado de inversões em um array aleatório: $\frac{n(n-1)}{4}$
\item Cada passagem bidirecional remove uma quantidade constante de inversões
\item Número esperado de passagens: $O(n)$
\item \textbf{Complexidade total:} $O(n^2)$
\end{itemize}

\subsubsection{Complexidade de Espaço}

\paragraph{Espaço Auxiliar: $O(1)$}
\textbf{Prova:}
\begin{itemize}
\item Variáveis utilizadas:
\begin{itemize}
\item \texttt{start}, \texttt{end}: ponteiros para limites $\rightarrow O(1)$
\item \texttt{swapped}: flag booleana $\rightarrow O(1)$
\item \texttt{i}: contador de loop $\rightarrow O(1)$
\item Variável temporária para swap $\rightarrow O(1)$
\end{itemize}
\item \textbf{Total:} $O(1)$ - algoritmo in-place
\end{itemize}

\subsubsection{Análise Prática}

\textbf{Constante multiplicativa reduzida:}
O Cocktail Shaker Sort reduz aproximadamente pela metade o número de passadas necessárias em comparação ao Bubble Sort tradicional. Para arrays parcialmente ordenados, pode ser significativamente mais rápido na prática, embora mantenha a mesma complexidade assintótica.

\subsection{Análise de Complexidade do Algoritmo}

\subsubsection{Complexidade de Tempo}

\paragraph{Melhor Caso: $O(n)$}
\textbf{Cenário:} Array já ordenado
\\
\textbf{Prova:}
\begin{itemize}
\item Na primeira passagem esquerda-direita, não há trocas (\texttt{swapped = false})
\item O algoritmo termina após verificar todos os $n-1$ pares adjacentes
\item \textbf{Total de comparações:} $n-1 = O(n)$
\item \textbf{Total de trocas:} $0$
\end{itemize}

\paragraph{Pior Caso: $O(n^2)$}
\textbf{Cenário:} Array em ordem completamente reversa
\\
\textbf{Prova:}
\begin{itemize}
\item Em cada iteração $k$, o algoritmo realiza:
\begin{itemize}
\item Passagem esquerda-direita: $(n-1-k)$ comparações
\item Passagem direita-esquerda: $(n-1-k)$ comparações
\end{itemize}
\item Número total de iterações: $\lceil n/2 \rceil$
\item \textbf{Cálculo das comparações:}
\end{itemize}

\begin{align}
\sum_{k=0}^{\lceil n/2 \rceil - 1} 2(n-1-k) &= 2\sum_{k=0}^{\lceil n/2 \rceil - 1}(n-1-k) \\
&= 2\left[(n-1)\lceil n/2 \rceil - \sum_{k=0}^{\lceil n/2 \rceil - 1}k\right] \\
&= 2\left[(n-1)\lceil n/2 \rceil - \frac{\lceil n/2 \rceil(\lceil n/2 \rceil - 1)}{2}\right] \\
&\approx 2\left[(n-1)(n/2) - \frac{(n/2)(n/2-1)}{2}\right] \\
&= n^2-n-\frac{n^2}{4}+\frac{n}{2} \\
&= \frac{3n^2}{4} - \frac{n}{2} = O(n^2)
\end{align}

\paragraph{Caso Médio: $O(n^2)$}
\textbf{Prova:}
\begin{itemize}
\item Número esperado de inversões em um array aleatório: $\frac{n(n-1)}{4}$
\item Cada passagem bidirecional remove uma quantidade constante de inversões
\item Número esperado de passagens: $O(n)$
\item \textbf{Complexidade total:} $O(n^2)$
\end{itemize}

\subsubsection{Complexidade de Espaço}

\paragraph{Espaço Auxiliar: $O(1)$}
\textbf{Prova:}
\begin{itemize}
\item Variáveis utilizadas:
\begin{itemize}
\item \texttt{start}, \texttt{end}: ponteiros para limites $\rightarrow O(1)$
\item \texttt{swapped}: flag booleana $\rightarrow O(1)$
\item \texttt{i}: contador de loop $\rightarrow O(1)$
\item Variável temporária para swap $\rightarrow O(1)$
\end{itemize}
\item \textbf{Total:} $O(1)$ - algoritmo in-place
\end{itemize}

\subsubsection{Características do Algoritmo}

\begin{itemize}
\item \textbf{Estabilidade:} Estável - elementos iguais mantêm sua ordem relativa original
\item \textbf{In-place:} Sim - ordena o array sem usar estruturas auxiliares significativas
\item \textbf{Adaptativo:} Sim - performance melhora com arrays parcialmente ordenados
\item \textbf{Método:} Troca (exchanging)
\item \textbf{Comparações:} Sempre $O(n^2)$ no pior caso
\item \textbf{Trocas:} Mínimo $0$ (melhor caso), máximo $O(n^2)$ (pior caso)
\end{itemize}

\textbf{Resumo Final:}
\begin{itemize}
\item Melhor caso: $O(n)$ - vetor já ordenado
\item Caso médio: $O(n^2)$
\item Pior caso: $O(n^2)$ - vetor em ordem reversa
\item Espaço auxiliar: $O(1)$ - algoritmo in-place
\item Vantagem: Elementos pequenos "sobem" mais rapidamente que no Bubble Sort tradicional
\end{itemize}

\section{Cycle Sort}

\textbf{Descrição:}
O Cycle Sort é um algoritmo de ordenação por comparação que procura minimizar o número de movimentações (trocas) realizadas. Ele identifica ciclos de permutação nos elementos e realiza uma série de trocas para colocar cada elemento em sua posição correta final, movimentando cada valor o mínimo necessário. É especialmente eficiente quando o custo de escrita é alto, pois cada elemento é movido apenas uma vez, sempre que possível. O Cycle Sort é também um algoritmo in-place.

\begin{exmp}
Considere ordenar o vetor $A = [3, 1, 4, 2]$ por Cycle Sort:

\begin{enumerate}
\item O 3 deve ir para a posição 2 (pois há dois menores que ele). Troca 3 com 4: $[4, 1, 3, 2]$.
\item O 3 deve ir para a posição 2 (continua). Troca 3 com 2: $[4, 1, 2, 3]$.
\item O 3 já está na posição correta. Agora analisa o 4, 1, e 2... até todos estarem ordenados: $[1, 2, 3, 4]$.
\end{enumerate}
\end{exmp}

\begin{algorithm}[H]
\DontPrintSemicolon
\textbf{cycleSort(A: array, n: int)};
\For{$cycle_start = 0$ até $n-2$}{
   $item \gets A[cycle_start]$;
   pos $\gets$ número de elementos menores que item a partir de $cycle_start$;
   \If{$pos == cycle_start$}{
      continuar para próximo ciclo;
   }
   \While{$item == A[pos]$}{
      $pos \gets pos + 1$;
   }
   trocar item $\leftrightarrow$ A[pos];
   \While{$pos != cycle_start$}{
      pos $\gets$ número de elementos menores que item a partir de $cycle_start$;
      \While{$item == A[pos]$}{
         $pos \gets pos + 1$;
      }
    trocar item $\leftrightarrow A[pos]$;
   }
}
\caption{Cycle Sort (simplificado)}
\label{lab:alg-cyclesort}
\end{algorithm}

\begin{lstlisting}[language=Python, caption={Cycle Sort em Python},captionpos=t, label=code:cyclesortPy]
def cycle_sort(arr):
    n = len(arr)
    for cycle_start in range(n - 1):
        item = arr[cycle_start]
        pos = cycle_start
        # Encontra a posicao correta do item
        for i in range(cycle_start + 1, n):
            if arr[i] < item:
                pos += 1
        # Pula se o item ja esta na posicao correta
        if pos == cycle_start:
            continue
        while item == arr[pos]:
            pos += 1
        arr[pos], item = item, arr[pos]
        # Continua com novos ciclos ate retornar ao inicio
        while pos != cycle_start:
            pos = cycle_start
            for i in range(cycle_start + 1, n):
                if arr[i] < item:
                    pos += 1
            while item == arr[pos]:
                pos += 1
            arr[pos], item = item, arr[pos]
\end{lstlisting}

\begin{lstlisting}[language=C, caption={Cycle Sort em C},,captionpos=t, label=code:cyclesortC]
#include <stdio.h>

void cycleSort(int arr[], int n) {
    for (int cycle_start = 0; cycle_start < n - 1; cycle_start++) {
        int item = arr[cycle_start];
        int pos = cycle_start;
        // Encontra a posicao correta do item
        for (int i = cycle_start + 1; i < n; i++) {
            if (arr[i] < item)
                pos++;
        }
        // Se ja esta no lugar certo, pula
        if (pos == cycle_start)
            continue;
        // Pula valores duplicados
        while (item == arr[pos])
            pos++;
        // Troca
        int temp = arr[pos];
        arr[pos] = item;
        item = temp;
        // Continua o ciclo ate retornar ao inicio
        while (pos != cycle_start) {
            pos = cycle_start;
            for (int i = cycle_start + 1; i < n; i++) {
                if (arr[i] < item)
                    pos++;
            }
            while (item == arr[pos])
                pos++;
            temp = arr[pos];
            arr[pos] = item;
            item = temp;
        }
    }
}
\end{lstlisting}

\begin{lstlisting}[language=C++, caption={Cycle Sort em C++},,captionpos=t, label=code:cyclesortC++]
#include <iostream>
#include <vector>
using namespace std;

void cycleSort(int arr[], int n) {
    for (int start = 0; start < n - 1; ++start) {
        int item = arr[start];
        int pos = start;

        // Encontra a posicao correta para o elemento atual
        for (int i = start + 1; i < n; ++i)
            if (arr[i] < item)
                ++pos;

        // Se o elemento ja esta na posicao correta, continue
        if (pos == start)
            continue;

        // Ignora duplicatas
        while (item == arr[pos])
            ++pos;

        // Faz o swap do elemento para a posicao correta
        if (pos != start)
            swap(item, arr[pos]);

        // Rotaciona o resto do ciclo
        while (pos != start) {
            pos = start;
            for (int i = start + 1; i < n; ++i)
                if (arr[i] < item)
                    ++pos;
            while (item == arr[pos])
                ++pos;
            if (item != arr[pos])
                swap(item, arr[pos]);
        }
    }
}
\end{lstlisting}

\subsection{Análise de Complexidade do Cycle Sort}

\subsubsection{Complexidade de Tempo}

O Cycle Sort é único entre os algoritmos de ordenação por comparação, pois minimiza o número de escritas (trocas) no array. Vamos analisar sua complexidade em detalhes:

\paragraph{Análise Geral das Operações}

O algoritmo executa os seguintes passos para cada elemento:
\begin{enumerate}
\item Encontra a posição correta do elemento atual
\item Conta quantos elementos são menores que ele
\item Realiza a troca para a posição correta
\item Continua o ciclo até retornar à posição inicial
\end{enumerate}

\paragraph{Melhor Caso: $O(n^2)$}
\textbf{Cenário:} Array já ordenado em ordem crescente
\\
\textbf{Prova:}

Para cada elemento na posição $i$ (onde $i = 0, 1, \ldots, n-2$):
\begin{itemize}
\item \textbf{Contagem da posição:} Compara com $(n-1-i)$ elementos à direita
\item \textbf{Trocas:} $0$ (elemento já está na posição correta)
\item \textbf{Verificação de duplicatas:} $O(1)$ em média
\end{itemize}

Total de comparações:
\begin{align}
C_{\text{melhor}}(n) &= \sum_{i=0}^{n-2} (n-1-i) \\
&= \sum_{k=1}^{n-1} k \quad \text{(substituindo } k = n-1-i\text{)} \\
&= \frac{(n-1)n}{2} = O(n^2)
\end{align}

\paragraph{Pior Caso: $O(n^2)$}
\textbf{Cenário:} Array em qualquer configuração
\\
\textbf{Prova:}

O Cycle Sort sempre executa o mesmo número de comparações, independente da configuração inicial:

\begin{itemize}
\item Para cada posição inicial $i$, conta elementos menores em posições $j > i$
\item Número de comparações por posição $i$: $(n-1-i)$
\item \textbf{Comparações totais:} $\sum_{i=0}^{n-2}(n-1-i) = \frac{n(n-1)}{2}$
\end{itemize}

\textbf{Análise das escritas (característica única):}
\begin{itemize}
\item \textbf{Melhor caso:} $0$ escritas (array ordenado)
\item \textbf{Pior caso:} $n-1$ escritas (cada elemento fora do lugar)
\item \textbf{Escritas mínimas:} O Cycle Sort garante o número mínimo de escritas possível
\end{itemize}

\paragraph{Caso Médio: $O(n^2)$}
\textbf{Cenário:} Array com elementos em ordem aleatória
\\
\textbf{Prova:}
\begin{itemize}
\item \textbf{Comparações:} $\frac{n(n-1)}{2} = O(n^2)$ (sempre constante)
\item \textbf{Escritas esperadas:} 
\end{itemize}

Para um elemento na posição $i$, a probabilidade de estar na posição errada é $\frac{n-1}{n}$.

Número esperado de escritas:
\begin{align}
E[W(n)] &= \sum_{i=0}^{n-1} P(\text{elemento } i \text{ fora da posição}) \\
&= \sum_{i=0}^{n-1} \frac{n-1}{n} \\
&= n \cdot \frac{n-1}{n} = n-1 = O(n)
\end{align}

\subsubsection{Análise Detalhada do Algoritmo}

\textbf{Complexidade por ciclo:}

Para cada ciclo que começa na posição $start$:
\begin{enumerate}
\item \textbf{Encontrar posição:} $O(n-start)$ comparações
\item \textbf{Lidar com duplicatas:} $O(k)$ onde $k$ é número de duplicatas
\item \textbf{Realizar troca:} $O(1)$
\item \textbf{Continuar ciclo:} Repetir até retornar ao início
\end{enumerate}

\textbf{Teorema da Complexidade Total:}
\begin{align}
T(n) &= \sum_{\text{ciclos}} (\text{comparações por ciclo}) \\
&= \sum_{i=0}^{n-2} \sum_{j=i+1}^{n-1} 1 \\
&= \frac{n(n-1)}{2} = O(n^2)
\end{align}

\subsubsection{Complexidade de Espaço}

\paragraph{Espaço Auxiliar: $O(1)$}
\textbf{Prova:}
\begin{itemize}
\item \textbf{Variáveis utilizadas:}
\begin{itemize}
\item \texttt{start}: posição inicial do ciclo $\rightarrow O(1)$
\item \texttt{pos}: posição correta do elemento $\rightarrow O(1)$
\item \texttt{item}: elemento sendo processado $\rightarrow O(1)$
\item Contadores de loop $\rightarrow O(1)$
\end{itemize}
\item \textbf{Espaço total:} $O(1)$ - algoritmo in-place
\item \textbf{Nenhuma estrutura auxiliar} é necessária
\end{itemize}

\subsubsection{Características Distintivas do Cycle Sort}

\begin{thm}[Minimalidade das Escritas]
O Cycle Sort realiza o número mínimo de escritas necessárias para ordenar um array, sendo este número igual ao número de elementos que não estão em suas posições corretas.
\end{thm}

\textbf{Prova:} Cada elemento é movido diretamente para sua posição final através de ciclos, sem escritas desnecessárias.

\begin{table}[h]
\centering
\begin{tabular}{|l|c|c|c|}
\hline
\textbf{Métrica} & \textbf{Melhor Caso} & \textbf{Caso Médio} & \textbf{Pior Caso} \\
\hline
Comparações & $\frac{n(n-1)}{2}$ & $\frac{n(n-1)}{2}$ & $\frac{n(n-1)}{2}$ \\
\hline
Escritas & $0$ & $n-1$ & $n-1$ \\
\hline
Complexidade Temporal & $O(n^2)$ & $O(n^2)$ & $O(n^2)$ \\
\hline
Complexidade Espacial & $O(1)$ & $O(1)$ & $O(1)$ \\
\hline
\end{tabular}
\caption{Análise Completa do Cycle Sort}
\end{table}

\subsubsection{Vantagens e Aplicações Específicas}

\textbf{Vantagens únicas:}
\begin{itemize}
\item \textbf{Escritas mínimas:} Ideal quando escritas são custosas (ex: flash memory)
\item \textbf{Determinístico:} Sempre o mesmo número de comparações
\item \textbf{In-place:} Não requer memória adicional
\item \textbf{Estável em escritas:} Minimiza desgaste em dispositivos de armazenamento
\end{itemize}

\textbf{Casos de uso ideais:}
\begin{itemize}
\item Sistemas embarcados com memória flash limitada
\item Situações onde o custo de escrita é muito maior que o de leitura
\item Quando o número de escritas deve ser minimizado
\end{itemize}

\subsubsection{Características Finais do Algoritmo}

\begin{itemize}
\item \textbf{Estabilidade:} Não estável - ordem de elementos iguais pode mudar
\item \textbf{In-place:} Sim - utiliza apenas $O(1)$ espaço extra
\item \textbf{Adaptativo:} Não - sempre $O(n^2)$ comparações
\item \textbf{Método:} Ciclos e posicionamento direto
\item \textbf{Especialização:} Otimizado para minimizar escritas
\end{itemize}

\textbf{Resumo Final:}
\begin{itemize}
\item \textbf{Todos os casos:} $O(n^2)$ em tempo, $O(1)$ em espaço
\item \textbf{Característica única:} Número mínimo de escritas ($\leq n-1$)
\item \textbf{Aplicação ideal:} Situações onde escritas são custosas
\item \textbf{Invariante:} Sempre $\frac{n(n-1)}{2}$ comparações, independente dos dados
\end{itemize}


\section{Spaghetti Sort}

\textbf{Descrição:}
O Spaghetti Sort é um algoritmo de ordenação não convencional, inspirado em uma analogia física: imagine que cada elemento do vetor é representado por um espaguete de comprimento proporcional ao seu valor. Ao segurar todos os espaguetes verticalmente sobre uma superfície, o mais longo toca a superfície primeiro, o segundo mais longo em seguida, e assim por diante, permitindo a ordenação ao “retirar” os espaguetes em ordem. No contexto computacional, simula-se esse processo selecionando repetidamente o maior valor ainda não escolhido.

\begin{exmp}
Para ordenar $A = [3, 1, 4, 2]$ com Spaghetti Sort:
\begin{enumerate}
\item Cada número é representado por um espaguete do respectivo comprimento.
\item O mais longo ($4$) “encosta” primeiro, depois $3$, depois $2$, depois $1$.
\item Ordem crescente: $[1, 2, 3, 4]$.
\end{enumerate}
\end{exmp}

\begin{algorithm}[H]
\DontPrintSemicolon
\textbf{spaghettiSort(A: array, n: int)};
marcar todos os elementos como "não utilizados";
\For{$k$ de $1$ até $n$}{
encontrar o maior elemento ainda não utilizado;
colocar esse elemento na próxima posição do resultado;
marcar o elemento como “utilizado”;
}
copiar o resultado ordenado de volta para $A$;
\caption{Spaghetti Sort (simulação computacional)}
\label{lab:alg-spaghettisort}
\end{algorithm}

\begin{lstlisting}[language=Python, caption={Spaghetti Sort em Python (simulação)},captionpos=t, label=code:spaghettisortPy]
def spaghetti_sort(arr):
    n = len(arr)
    used = [False] * n
    result = []
    for _ in range(n):
        max_val = float('-inf')
        idx = -1
        for i in range(n):
            if not used[i] and arr[i] > max_val:
                max_val = arr[i]
                idx = i
        result.insert(0, max_val)  # Insere no inicio para ficar em ordem crescente
        used[idx] = True
    return result
\end{lstlisting}

\begin{lstlisting}[language=C, caption={Spaghetti Sort em C (simulado)},captionpos=t, label=code:spaghettisortC]
#include <stdio.h>
#include <stdlib.h>
#include <stdbool.h>

void spaghettiSort(int *arr, int n, int *out) {
    bool *used = (bool*)calloc(n, sizeof(bool));
    for (int k = n - 1; k >= 0; k--) {
        int max_val = arr[0], idx = 0;
        for (int i = 0; i < n; i++) {
            if (!used[i] && arr[i] > max_val) {
                max_val = arr[i];
                idx = i;
            }
        }
        out[k] = max_val;
        used[idx] = true;
    }
    free(used);
}
\end{lstlisting}

\begin{lstlisting}[language=C++, caption={Spaghetti Sort em C++ (simulado)},captionpos=t, label=code:spaghettisortC++]
#include <iostream>
#include <vector>
#include <algorithm>
using namespace std;

void spaghettiSort(vector<int>& arr) {
    if (arr.empty()) return;
    
    // Encontra o valor maximo para determinar a "altura"
    int maxVal = *max_element(arr.begin(), arr.end());
    
    vector<int> sorted;
    vector<bool> used(arr.size(), false);
    
    // Simula a "queda" dos espaguetes
    // Comeca do menor valor 
    for (int height = 0; height <= maxVal; ++height) {
        for (int i = 0; i < arr.size(); ++i) {
            if (!used[i] && arr[i] == height) {
                sorted.push_back(arr[i]);
                used[i] = true;
            }
        }
    }
    
    arr = sorted;
}
\end{lstlisting}

\subsection{Análise de Complexidade do Spaghetti Sort}

\subsubsection{Complexidade de Tempo}

O Spaghetti Sort é baseado em uma analogia física onde "espaguetes" de diferentes comprimentos são soltos verticalmente, e os mais curtos tocam o chão primeiro. Na implementação, funciona como uma variante do algoritmo de contagem (Counting Sort).

\paragraph{Análise das Fases do Algoritmo}

O algoritmo executa as seguintes fases sequenciais:
\begin{enumerate}
\item \textbf{Encontrar valor máximo:} $O(n)$
\item \textbf{Inicializar array de contagem:} $O(k)$ onde $k = \max(A) + 1$
\item \textbf{Contar ocorrências:} $O(n)$
\item \textbf{Reconstruir array ordenado:} $O(n + k)$
\end{enumerate}

\paragraph{Melhor Caso: $O(n)$}
\textbf{Cenário:} Quando o range dos valores é pequeno e proporcional a $n$
\\
\textbf{Prova:}

Seja $k = \max(A) - \min(A) + 1$ o range dos valores distintos.

No melhor caso, $k = O(n)$:
\begin{align}
T_{\text{melhor}}(n) &= O(n) + O(k) + O(n) + O(n + k) \\
&= O(n) + O(n) + O(n) + O(n + n) \\
&= O(n) + O(n) + O(n) + O(n) \\
&= O(n)
\end{align}

\textbf{Exemplo:} Array $[1, 2, 3, 4, 5]$ tem $n = 5$ e $k = 5$.

\paragraph{Pior Caso: $O(k)$}
\textbf{Cenário:} Quando o range dos valores é muito maior que $n$
\\
\textbf{Prova:}

No pior caso, $k \gg n$ (valores muito esparsos):
\begin{align}
T_{\text{pior}}(n) &= O(n) + O(k) + O(n) + O(n + k) \\
&= O(n) + O(k) + O(n) + O(k) \quad \text{(pois } k \gg n\text{)} \\
&= O(k)
\end{align}

\textbf{Exemplo:} Array $[1, 1000000]$ tem $n = 2$ mas $k = 1000000$.

\paragraph{Caso Médio: $O(n + k)$}
\textbf{Cenário:} Distribuição típica de valores
\\
\textbf{Prova:}

Para uma distribuição geral, ambos $n$ e $k$ contribuem significativamente:
\begin{align}
T_{\text{médio}}(n) &= O(n) + O(k) + O(n) + O(n + k) \\
&= O(n + k)
\end{align}

\textbf{Análise detalhada de cada fase:}
\begin{itemize}
\item \textbf{Busca do máximo:} $\sum_{i=0}^{n-1} O(1) = O(n)$
\item \textbf{Inicialização contadores:} $\sum_{i=0}^{k-1} O(1) = O(k)$
\item \textbf{Contagem:} $\sum_{i=0}^{n-1} O(1) = O(n)$
\item \textbf{Reconstrução:} $O(n)$ elementos + $O(k)$ iteração sobre contadores $= O(n + k)$
\end{itemize}

\subsubsection{Análise Dependente do Range}

\paragraph{Classificação por Range}
\textbf{Caso 1: Range pequeno ($k = O(n)$)}
\begin{align}
T(n) &= O(n + k) = O(n + n) = O(n) \\
\text{Eficiência} &= \text{Linear}
\end{align}

\textbf{Caso 2: Range moderado ($k = O(n \log n)$)}
\begin{align}
T(n) &= O(n + k) = O(n + n \log n) = O(n \log n) \\
\text{Eficiência} &= \text{Linearítmica}
\end{align}

\textbf{Caso 3: Range grande ($k = O(n^c)$ para $c > 1$)}
\begin{align}
T(n) &= O(n + k) = O(n + n^c) = O(n^c) \\
\text{Eficiência} &= \text{Polinomial}
\end{align}

\textbf{Caso 4: Range exponencial ($k = O(2^n)$)}
\begin{align}
T(n) &= O(n + k) = O(n + 2^n) = O(2^n) \\
\text{Eficiência} &= \text{Exponencial}
\end{align}

\subsubsection{Complexidade de Espaço}

\paragraph{Espaço Auxiliar: $O(k)$}
\textbf{Prova:}
\begin{itemize}
\item \textbf{Array de contagem:} \texttt{count[0..k-1]} $\rightarrow O(k)$
\item \textbf{Variáveis auxiliares:}
\begin{itemize}
\item \texttt{maxVal}: valor máximo $\rightarrow O(1)$
\item \texttt{index}: índice de reconstrução $\rightarrow O(1)$
\item Contadores de loop: $i, j$ $\rightarrow O(1)$
\end{itemize}
\item \textbf{Espaço total:} $O(k) + O(1) = O(k)$
\end{itemize}

\paragraph{Análise do Espaço por Cenário}
\textbf{Melhor caso (espaço):} $k = O(n) \Rightarrow S(n) = O(n)$
\\
\textbf{Pior caso (espaço):} $k = O(\max(A)) \Rightarrow S(n) = O(\max(A))$

\subsubsection{Características da Analogia Física}

\paragraph{Modelo Matemático da ``Queda dos Espaguetes''}
\textbf{Interpretação física:}
\begin{itemize}
\item Espaguete de comprimento $\ell$ representa valor $\ell$
\item Tempo para tocar o chão: $t(\ell) = \sqrt{\frac{2h}{\ell}} \propto \ell^{-1/2}$
\item Ordem de chegada: crescente por comprimento
\end{itemize}

\textbf{Mapeamento computacional:}
\begin{itemize}
\item "Empilhar espaguetes": fase de contagem $O(n)$
\item "Soltar espaguetes": ordenação implícita $O(1)$
\item "Coletar por ordem de chegada": reconstrução $O(n + k)$
\end{itemize}

\subsubsection{Análise de Estabilidade Temporal}

\paragraph{Invariantes do Algoritmo}
\begin{itemize}
\item \textbf{Preservação de frequências:} $\sum_{i} \text{count}[i] = n$
\item \textbf{Ordenação por construção:} elementos reconstruídos em ordem
\item \textbf{Completude:} todos elementos originais são preservados
\end{itemize}

\paragraph{Limitações Dependentes de Dados}
\textbf{Teorema da Dependência do Range:}
\begin{thm}
O Spaghetti Sort tem complexidade temporal $\Theta(n + k)$ onde $k$ é o range dos valores de entrada, tornando-o eficiente apenas quando $k = O(n^c)$ para $c \leq 1$.
\end{thm}

\textbf{Prova:} A necessidade de inicializar e iterar sobre o array de contagem de tamanho $k$ torna impossível reduzir a dependência de $k$.

\subsubsection{Casos Especiais de Desempenho}

\paragraph{Valores Negativos}
\textbf{Ajuste para range $[\min, \max]$:}
\begin{align}
k &= \max(A) - \min(A) + 1 \\
\text{offset} &= -\min(A) \\
T(n) &= O(n + k) \text{ (inalterado)}
\end{align}

\paragraph{Valores Duplicados}
\textbf{Comportamento com multiplicidades:}
\begin{itemize}
\item Contagem preserva multiplicidades: $O(1)$ por elemento
\item Reconstrução mantém estabilidade: $O(\text{multiplicidade})$ por valor
\item Complexidade total: $O(n + k)$ (inalterado)
\end{itemize}

\subsubsection{Características Finais do Algoritmo}

\begin{itemize}
\item \textbf{Estabilidade:} Estável - preserva ordem relativa de elementos iguais
\item \textbf{In-place:} Não - requer $O(k)$ espaço adicional
\item \textbf{Adaptativo:} Não - complexidade independe da configuração inicial
\item \textbf{Método:} Contagem e distribuição
\item \textbf{Restrições:} Eficiente apenas para ranges pequenos
\end{itemize}

\textbf{Resumo Final:}
\begin{itemize}
\item \textbf{Melhor caso:} $O(n)$ quando $k = O(n)$
\item \textbf{Caso geral:} $O(n + k)$ onde $k$ é o range dos valores
\item \textbf{Pior caso:} $O(k)$ quando $k \gg n$
\item \textbf{Espaço auxiliar:} $O(k)$ - não é in-place
\item \textbf{Aplicabilidade:} Ideal para valores inteiros com range limitado
\end{itemize}




\section{I Can't Believe It Can Sort}

\textbf{Descrição:}
O I Can't Believe It Can Sort (ICBICS, também conhecido como ICBINS, “I Can't Believe It's Not Sorting”) é um algoritmo de ordenação por comparação criado como curioso experimento algorítmico. A ideia é partir de um array já ordenado chamado “ordenação canônica” e gerar permutações através de trocas parciais usando algoritmos derivados de busca binária, até retornar à ordenação desejada. Apesar do nome e da intenção humorística, ele funciona corretamente para ordenar, mas sem ganho de eficiência prática ou teórica. O algoritmo ilustra propriedades de permutação, busca e é um exemplo didático notório.

\begin{exmp}
Considere ordenar $A = [5, 1, 4, 2, 3]$ com ICBICS.
\begin{enumerate}
\item Gera-se um “array canônico” $B = [1, 2, 3, 4, 5]$.
\item Para cada elemento de $A$, busca-se a posição correta em $B$ e realiza-se o movimento equivalente no $A$, por meio de trocas com busca binária.
\item Repetido para todos os elementos, o array original vai convergindo até se tornar igual ao array ordenado.
\end{enumerate}
\end{exmp}

\begin{algorithm}[H]
\DontPrintSemicolon
\textbf{ICBICS(A: array, n: int)};
B $\gets$ array canônico de $A$, ordenado;
\For{$i$ de $0$ até $n-1$}{
buscar onde $A[i]$ deveria ir em $B$ (busca binária);
trocar $A[i]$ e $A[pos]$ se necessário;
}
repetir até $A = B$
\caption{I Can't Believe It Can Sort}
\label{lab:alg-icbics}
\end{algorithm}

\begin{lstlisting}[language=Python, caption={I Can’t Believe It Can Sort em Python},captionpos=t,label=code:icbicsPy]
def icbics(arr):
    B = sorted(arr)
    n = len(arr)
    while arr != B:
        for i in range(n):
            # Busca binaria na ordenacao canonica
            left, right = 0, n - 1
            while left <= right:
                mid = (left + right) // 2
                if B[mid] == arr[i]:
                    if i != mid:
                        arr[i], arr[mid] = arr[mid], arr[i]
                    break
                elif B[mid] < arr[i]:
                    left = mid + 1
                else:
                    right = mid - 1
    return arr
\end{lstlisting}

\begin{lstlisting}[language=C, caption={I Can't Believe It Can Sort em C (simplificado)},captionpos=t, label=code:icbicsC]
#include <stdio.h>
#include <stdlib.h>
#include <stdbool.h>
#include <string.h>

void icbics(int *arr, int n) {
    int *B = malloc(n * sizeof(int));
    memcpy(B, arr, n * sizeof(int));

    // Ordenacao canonica (Bubble Sort didatico)
    for (int i = 0; i < n - 1; i++)
        for (int j = 0; j < n - i - 1; j++)
            if (B[j] > B[j + 1]) {
                int temp = B[j];
                B[j] = B[j + 1];
                B[j + 1] = temp;
            }

    bool sorted = false;
    while (!sorted) {
        sorted = true;
        for (int i = 0; i < n; i++) {
            // Busca binaria de arr[i] em B
            int left = 0, right = n - 1;
            while (left <= right) {
                int mid = (left + right) / 2;
                if (B[mid] == arr[i]) {
                    if (i != mid) {
                        int temp = arr[i];
                        arr[i] = arr[mid];
                        arr[mid] = temp;
                        sorted = false;
                    }
                    break;
                } else if (B[mid] < arr[i]) {
                    left = mid + 1;
                } else {
                    right = mid - 1;
                }
            }
        }
    }
    free(B);
}
\end{lstlisting}

\begin{lstlisting}[language=C++, caption={I Canct Believe It Can Sort em C++ (simplificado)},captionpos=t, label=code:icbicsC++]
#include <iostream>
#include <vector>
#include <cstdlib>
#include <ctime>
#include <algorithm>
using namespace std;

// Funcao para checar se o vetor esta ordenado
bool isSorted(const vector<int>& arr) {
    for (size_t i = 1; i < arr.size(); ++i)
        if (arr[i-1] > arr[i]) return false;
    return true;
}

void icbicsSort(vector<int>& arr) {
    srand(time(0));
    while (!isSorted(arr)) {
        int i = rand() % arr.size();
        int j = rand() % arr.size();
        swap(arr[i], arr[j]);
    }
}

int main() {
    vector<int> arr = {3, 1, 4, 2};

    icbicsSort(arr);

    cout << "Sorted array: ";
    for (auto n : arr) cout << n << " ";
    cout << endl;
    return 0;
}
\end{lstlisting}

\subsection{Análise de Complexidade do I Can't Believe It Can Sort}

\subsubsection{Complexidade de Tempo}

O "I Can't Believe It Can Sort" (ICBICS) é um algoritmo probabilístico que embaralha aleatoriamente o array e verifica se está ordenado, repetindo até conseguir a ordenação desejada.

\paragraph{Análise Probabilística Fundamental}

Para um array de $n$ elementos distintos:
\begin{itemize}
\item \textbf{Total de permutações possíveis:} $n!$
\item \textbf{Permutações ordenadas:} $1$ (apenas a sequência crescente)
\item \textbf{Probabilidade de sucesso por tentativa:} $P = \frac{1}{n!}$
\end{itemize}

\paragraph{Melhor Caso: $O(n)$}
\textbf{Cenário:} Array já ordenado ou ordenado na primeira tentativa
\\
\textbf{Prova:}
\begin{itemize}
\item \textbf{Tentativas necessárias:} $1$
\item \textbf{Custo de verificação:} $O(n)$ para checar se está ordenado
\item \textbf{Custo de embaralhamento:} $O(n)$ (se necessário)
\item \textbf{Complexidade total:} $O(n) + O(n) = O(n)$
\end{itemize}

\paragraph{Pior Caso: $O(\infty)$}
\textbf{Cenário:} Teoricamente, o algoritmo pode nunca terminar
\\
\textbf{Prova:}
\begin{itemize}
\item Cada tentativa é independente das anteriores
\item Não há garantia de que uma permutação ordenada será gerada
\item \textbf{Limite superior:} $\lim_{k \to \infty} O(k \cdot n) = O(\infty)$
\end{itemize}

\textbf{Análise probabilística do pior caso:}
A probabilidade de não conseguir ordenar após $k$ tentativas é:
\begin{align}
P(\text{falha após } k \text{ tentativas}) &= \left(1 - \frac{1}{n!}\right)^k \\
&= \left(\frac{n! - 1}{n!}\right)^k
\end{align}

Para $k \to \infty$: $P(\text{falha}) \to 0$, mas nunca é exatamente zero.

\paragraph{Caso Médio (Esperado): $O(n! \cdot n)$}
\textbf{Cenário:} Análise do número esperado de tentativas
\\
\textbf{Prova:}

Seja $X$ a variável aleatória que representa o número de tentativas até o sucesso. $X$ segue uma distribuição geométrica com parâmetro $p = \frac{1}{n!}$.

\textbf{Número esperado de tentativas:}
\begin{align}
E[X] &= \frac{1}{p} = \frac{1}{\frac{1}{n!}} = n!
\end{align}

\textbf{Custo por tentativa:}
\begin{itemize}
\item Embaralhamento: $O(n)$
\item Verificação de ordenação: $O(n)$
\item Total por tentativa: $O(n)$
\end{itemize}

\textbf{Complexidade temporal esperada:}
\begin{align}
E[T(n)] &= E[X] \times \text{custo por tentativa} \\
&= n! \times O(n) \\
&= O(n! \cdot n)
\end{align}

\subsubsection{Análise Detalhada por Operação}

\paragraph{Embaralhamento Aleatório}
\textbf{Algoritmo Fisher-Yates (usado internamente):}
\begin{align}
T_{\text{shuffle}}(n) &= \sum_{i=0}^{n-1} O(1) = O(n)
\end{align}

\paragraph{Verificação de Ordenação}
\textbf{Verificação sequencial:}
\begin{align}
T_{\text{check}}(n) &= \sum_{i=0}^{n-2} O(1) = O(n-1) = O(n)
\end{align}

\paragraph{Análise da Variância}
Para uma distribuição geométrica com parâmetro $p = \frac{1}{n!}$:
\begin{align}
\text{Var}[X] &= \frac{1-p}{p^2} = \frac{1-\frac{1}{n!}}{\left(\frac{1}{n!}\right)^2} \\
&= (n!)^2 - n! \\
&= n!(n! - 1)
\end{align}

\textbf{Desvio padrão:} $\sigma = \sqrt{n!(n! - 1)} \approx n!$

\subsubsection{Complexidade de Espaço}

\paragraph{Espaço Auxiliar: $O(1)$}
\textbf{Prova:}
\begin{itemize}
\item \textbf{Variáveis utilizadas:}
\begin{itemize}
\item Índices para embaralhamento: $i, j$ $\rightarrow O(1)$
\item Variável temporária para troca: \texttt{temp} $\rightarrow O(1)$
\item Flag booleana para verificação: \texttt{sorted} $\rightarrow O(1)$
\item Gerador de números aleatórios (estado): $\rightarrow O(1)$
\end{itemize}
\item \textbf{Modificação in-place:} O array é embaralhado no próprio espaço
\item \textbf{Espaço total:} $O(1)$ - algoritmo in-place
\end{itemize}

\subsubsection{Análise Assintótica Detalhada}

\paragraph{Crescimento Factorial}
\textbf{Comportamento de $n!$:}
\begin{align}
n! &\approx \sqrt{2\pi n} \left(\frac{n}{e}\right)^n \quad \text{(Fórmula de Stirling)} \\
\log(n!) &= O(n \log n) \\
n! &= \Omega(2^n) \text{ e } n! = O(n^n)
\end{align}

\textbf{Implicações práticas:}
\begin{itemize}
\item Para $n = 10$: $E[X] = 3.628.800$ tentativas
\item Para $n = 15$: $E[X] \approx 1.3 \times 10^{12}$ tentativas
\item Para $n = 20$: $E[X] \approx 2.4 \times 10^{18}$ tentativas
\end{itemize}

\subsubsection{Características Probabilísticas}

\paragraph{Distribuição do Tempo de Execução}
\textbf{Função de distribuição acumulada:}
\begin{align}
P(X \leq k) &= 1 - \left(1 - \frac{1}{n!}\right)^k \\
&= 1 - \left(\frac{n! - 1}{n!}\right)^k
\end{align}

\paragraph{Mediana do Número de Tentativas}
\textbf{Valor mediano:}
\begin{align}
P(X \leq m) &= 0.5 \\
\left(\frac{n! - 1}{n!}\right)^m &= 0.5 \\
m &= \frac{\ln(0.5)}{\ln\left(\frac{n! - 1}{n!}\right)} \\
&\approx n! \ln(2) \approx 0.693 \cdot n!
\end{align}

\subsubsection{Limitações e Impracticabilidade}

\begin{thm}[Impracticabilidade Assintótica]
Para $n \geq 10$, o tempo esperado de execução do ICBICS torna-se computacionalmente impraticável, crescendo mais rapidamente que qualquer polinômio.
\end{thm}

\textbf{Prova:} $O(n! \cdot n)$ cresce mais rapidamente que $O(n^k)$ para qualquer $k$ constante.

\subsubsection{Características Finais do Algoritmo}

\begin{itemize}
\item \textbf{Estabilidade:} Não aplicável - embaralhamento aleatório
\item \textbf{In-place:} Sim - utiliza apenas $O(1)$ espaço extra
\item \textbf{Adaptativo:} Não - comportamento independe da configuração inicial
\item \textbf{Método:} Embaralhamento aleatório repetitivo
\item \textbf{Terminação:} Quase certamente termina, mas sem garantia temporal
\end{itemize}

\textbf{Resumo Final:}
\begin{itemize}
\item \textbf{Melhor caso:} $O(n)$ - sucesso imediato
\item \textbf{Caso esperado:} $O(n! \cdot n)$ - número esperado de tentativas
\item \textbf{Pior caso:} $O(\infty)$ - pode nunca terminar
\item \textbf{Espaço auxiliar:} $O(1)$ - algoritmo in-place
\item \textbf{Natureza:} Algoritmo probabilístico de interesse puramente teórico
\end{itemize}


\section{Exchange Sort}
\textbf{Descrição:}
O Exchange Sort é um algoritmo simples de ordenação por comparação semelhante ao Selection Sort. O algoritmo compara cada elemento com todos os seguintes e troca elementos quando necessário, até todo o vetor estar ordenado.

\begin{exmp}
Considere ordenar o vetor $A = [4, 2, 3, 1]$ com Exchange Sort:
\begin{enumerate}
\item O 4 é comparado com 2, 3 e 1. Sempre que encontra um valor menor, realiza a troca imediatamente.
\item O 2 é comparado e trocado se necessário; segue até o fim.
\item O processo se repete até o vetor ficar ordenado: $[1, 2, 3, 4]$.
\end{enumerate}
\end{exmp}

\begin{algorithm}[H]
\DontPrintSemicolon
\textbf{exchangeSort(A: array, n: int)};
\For{$i = 0$ até $n-2$}{
\For{$j = i+1$ até $n-1$}{
\If{$A[i] > A[j]$}{
trocar $A[i] \leftrightarrow A[j]$;
}
}
}
\caption{Exchange Sort}
\label{lab:alg-exchangesort}
\end{algorithm}

\begin{lstlisting}[language=Python, caption={Exchange Sort em Python},captionpos=t, label=code:exchangesortPy]
def exchange_sort(arr):
    n = len(arr)
    for i in range(n - 1):
        for j in range(i + 1, n):
            if arr[i] > arr[j]:
                arr[i], arr[j] = arr[j], arr[i]
\end{lstlisting}

\begin{lstlisting}[language=C, caption={Exchange Sort em C},captionpos=t, label=code:exchangesortC]
#include <stdio.h>

void exchangeSort(int arr[], int n) {
    for (int i = 0; i < n - 1; i++) {
        for (int j = i + 1; j < n; j++) {
            if (arr[i] > arr[j]) {
                int temp = arr[i];
                arr[i] = arr[j];
                arr[j] = temp;
            }
        }
    }
}
\end{lstlisting}

\begin{lstlisting}[language=C++, caption={Exchange Sort em C++},captionpos=t, label=code:exchangesortC++]
#include <iostream>
using namespace std;

void exchangeSort(int arr[], int n) {
    for (int i = 0; i < n - 1; ++i) {
        for (int j = i + 1; j < n; ++j) {
            if (arr[i] > arr[j]) {
                // Troca se fora de ordem
                swap(arr[i], arr[j]);
            }
        }
    }
}
\end{lstlisting}

\subsection{Análise de Complexidade do Exchange Sort}

\subsubsection{Complexidade de Tempo}

O Exchange Sort possui uma característica única: o número de comparações é sempre o mesmo, independente da configuração inicial do array. Vamos analisar cada caso:

\paragraph{Análise Geral das Comparações}
\textbf{Cálculo do número total de comparações:}

O algoritmo executa dois loops aninhados:
\begin{itemize}
\item Loop externo: $i$ varia de $0$ até $n-2$
\item Loop interno: $j$ varia de $i+1$ até $n-1$
\end{itemize}

O número total de comparações é:
\begin{align}
C(n) &= \sum_{i=0}^{n-2} \sum_{j=i+1}^{n-1} 1 \\
&= \sum_{i=0}^{n-2} (n-1-i) \\
&= \sum_{i=0}^{n-2} (n-1) - \sum_{i=0}^{n-2} i \\
&= (n-1)^2 - \frac{(n-2)(n-1)}{2} \\
&= \frac{2(n-1)^2 - (n-2)(n-1)}{2} \\
&= \frac{(n-1)[2(n-1) - (n-2)]}{2} \\
&= \frac{(n-1)(n)}{2} \\
&= \frac{n(n-1)}{2}
\end{align}

Portanto, $C(n) = \frac{n(n-1)}{2} = O(n^2)$ para todos os casos.

\paragraph{Melhor Caso: $O(n^2)$}
\textbf{Cenário:} Array já ordenado em ordem crescente
\\
\textbf{Prova:}
\begin{itemize}
\item \textbf{Comparações:} $\frac{n(n-1)}{2} = O(n^2)$
\item \textbf{Trocas:} $0$ (nenhuma troca necessária)
\item \textbf{Complexidade total:} $O(n^2) + O(0) = O(n^2)$
\end{itemize}

\paragraph{Pior Caso: $O(n^2)$}
\textbf{Cenário:} Array em ordem completamente decrescente
\\
\textbf{Prova:}
\begin{itemize}
\item \textbf{Comparações:} $\frac{n(n-1)}{2} = O(n^2)$
\item \textbf{Trocas:} $\frac{n(n-1)}{2}$ (todas as comparações resultam em troca)
\item \textbf{Complexidade total:} $O(n^2) + O(n^2) = O(n^2)$
\end{itemize}

\textbf{Demonstração das trocas no pior caso:}
Para um array decrescente $[n, n-1, n-2, \ldots, 1]$:
\begin{itemize}
\item Elemento na posição $i$ é trocado com todos os elementos nas posições $j > i$
\item Número de trocas para elemento $i$: $(n-1-i)$
\item Total de trocas: $\sum_{i=0}^{n-2}(n-1-i) = \frac{n(n-1)}{2}$
\end{itemize}

\paragraph{Caso Médio: $O(n^2)$}
\textbf{Cenário:} Array com elementos em ordem aleatória
\\
\textbf{Prova:}
\begin{itemize}
\item \textbf{Comparações:} $\frac{n(n-1)}{2} = O(n^2)$ (sempre o mesmo)
\item \textbf{Trocas esperadas:} 
\end{itemize}

Para cada par $(i,j)$ onde $i < j$, a probabilidade de $A[i] > A[j]$ é $\frac{1}{2}$.

Número esperado de trocas:
\begin{align}
E[T(n)] &= \sum_{i=0}^{n-2} \sum_{j=i+1}^{n-1} P(A[i] > A[j]) \\
&= \sum_{i=0}^{n-2} \sum_{j=i+1}^{n-1} \frac{1}{2} \\
&= \frac{1}{2} \cdot \frac{n(n-1)}{2} \\
&= \frac{n(n-1)}{4} = O(n^2)
\end{align}

\textbf{Complexidade total:} $O(n^2) + O(n^2) = O(n^2)$

\subsubsection{Complexidade de Espaço}

\paragraph{Espaço Auxiliar: $O(1)$}
\textbf{Prova:}
\begin{itemize}
\item \textbf{Variáveis utilizadas:}
\begin{itemize}
\item $i$, $j$: contadores de loop $\rightarrow O(1)$
\item \texttt{temp}: variável temporária para troca $\rightarrow O(1)$
\item Nenhum array ou estrutura auxiliar é criada
\end{itemize}
\item \textbf{Espaço total:} $O(1)$ - algoritmo in-place
\item \textbf{Modificações:} O algoritmo modifica o array original, sem criar cópias
\end{itemize}

\subsubsection{Características Distintivas do Exchange Sort}

\begin{thm}[Invariabilidade das Comparações]
O Exchange Sort sempre realiza exatamente $\frac{n(n-1)}{2}$ comparações, independente da configuração inicial do array.
\end{thm}

\textbf{Prova:} A estrutura de loops aninhados garante que cada par $(i,j)$ com $i < j$ seja comparado exatamente uma vez.

\begin{table}[h]
\centering
\begin{tabular}{|l|c|c|c|}
\hline
\textbf{Métrica} & \textbf{Melhor Caso} & \textbf{Caso Médio} & \textbf{Pior Caso} \\
\hline
Comparações & $\frac{n(n-1)}{2}$ & $\frac{n(n-1)}{2}$ & $\frac{n(n-1)}{2}$ \\
\hline
Trocas & $0$ & $\frac{n(n-1)}{4}$ & $\frac{n(n-1)}{2}$ \\
\hline
Complexidade & $O(n^2)$ & $O(n^2)$ & $O(n^2)$ \\
\hline
\end{tabular}
\caption{Análise Detalhada do Exchange Sort}
\end{table}

\subsubsection{Comparação com Outros Algoritmos}

\begin{table}[h]
\centering
\begin{tabular}{|l|c|c|c|}
\hline
\textbf{Algoritmo} & \textbf{Melhor Caso} & \textbf{Pior Caso} & \textbf{Espaço} \\
\hline
Exchange Sort & $O(n^2)$ & $O(n^2)$ & $O(1)$ \\
\hline
Selection Sort & $O(n^2)$ & $O(n^2)$ & $O(1)$ \\
\hline
Bubble Sort & $O(n)$ & $O(n^2)$ & $O(1)$ \\
\hline
\end{tabular}
\caption{Comparação entre Algoritmos Similares}
\end{table}

\subsubsection{Características do Algoritmo}

\begin{itemize}
\item \textbf{Estabilidade:} Não estável - elementos iguais podem ter ordem alterada
\item \textbf{In-place:} Sim - ordena sem espaço auxiliar significativo
\item \textbf{Adaptativo:} Não - sempre $O(n^2)$, mesmo para arrays ordenados
\item \textbf{Método:} Comparação e troca direta (direct exchange)
\item \textbf{Comportamento:} Determinístico - sempre o mesmo número de comparações
\end{itemize}

\textbf{Resumo Final:}
\begin{itemize}
\item Melhor caso: $O(n^2)$ - array já ordenado
\item Caso médio: $O(n^2)$ - distribuição aleatória
\item Pior caso: $O(n^2)$ - array em ordem reversa
\item Espaço auxiliar: $O(1)$ - algoritmo in-place
\item \textbf{Peculiaridade:} Único algoritmo simples com complexidade sempre $O(n^2)$
\end{itemize}


\section{Strand Sort}

\textbf{Descrição:}
O Strand Sort é um algoritmo de ordenação por comparação que constrói repetidamente sublistas ordenadas (strands) a partir da lista original. Em cada etapa, extrai-se uma strand crescente da lista de entrada, mescla-se com a lista resultante ordenada, e repete-se até a lista original estar vazia.

\begin{exmp}
Considere ordenar o vetor $A = [4, 2, 3, 1]$ com Strand Sort:
\begin{enumerate}
\item A primeira strand extraída é $[4]$. A nova lista fica $[2,3,1]$.
\item Na segunda passagem, a strand $[2,3]$ é retirada. Mescla-se com $[4]$ para $[2,3,4]$; nova lista $[1]$.
\item O $[1]$ é a última strand, e a mesclagem final resulta em $[1,2,3,4]$.
\end{enumerate}
\end{exmp}

\begin{algorithm}[H]
\DontPrintSemicolon
\textbf{strandSort(A: lista)}\;
result $\gets$ lista vazia\;
\While{A não está vazia}{
extrair primeira strand ordenada de A para S;
result $\gets$ merge(result, S)
}
\textbf{retorn} result
\caption{Strand Sort (simplificado)}
\label{lab:alg-strandsort}
\end{algorithm}

\begin{lstlisting}[language=Python, caption={Strand Sort em Python},captionpos=t, label=code:strandsortPy]
def merge(a, b):
    result = []
    i = j = 0
    while i < len(a) and j < len(b):
        if a[i] < b[j]:
            result.append(a[i])
            i += 1
        else:
            result.append(b[j])
            j += 1
    result.extend(a[i:])
    result.extend(b[j:])
    return result

def strand_sort(arr):
    result = []
    arr = arr[:]  # Evita modificar lista original
    while arr:
        strand = [arr.pop(0)]
        i = 0
        while i < len(arr):
            if arr[i] >= strand[-1]:
                strand.append(arr.pop(i))
            else:
                i += 1
        result = merge(result, strand)
    return result
\end{lstlisting}

\begin{lstlisting}[language=C, caption={Strand Sort em C (simplificado)},captionpos=t, label=code:strandsortC]
#include <stdio.h>
#include <stdlib.h>

void merge(int *a, int na, int *b, int nb, int *out) {
    int i = 0, j = 0, k = 0;
    while (i < na && j < nb) {
        if (a[i] < b[j])
            out[k++] = a[i++];
        else
            out[k++] = b[j++];
    }
    while (i < na) out[k++] = a[i++];
    while (j < nb) out[k++] = b[j++];
}

void strandSort(int *arr, int n, int *out, int *outSz) {
    int *input = malloc(n * sizeof(int));
    for (int i = 0; i < n; i++)
        input[i] = arr[i];
    int inSz = n, resSz = 0;
    int *result = malloc(n * sizeof(int));
    while (inSz > 0) {
        int *strand = malloc(n * sizeof(int));
        int strandSz = 0;
        int last = input[0], si = 1;
        strand[strandSz++] = last;
        for (int i = 1; i < inSz; i++) {
            if (input[i] >= last) {
                strand[strandSz++] = input[i];
                last = input[i];
            } else {
                input[si++] = input[i];
            }
        }
        inSz = si - 1;
        int *merged = malloc(n * sizeof(int));
        merge(result, resSz, strand, strandSz, merged);
        resSz = resSz + strandSz;
        for (int i = 0; i < resSz; i++)
            result[i] = merged[i];
        free(merged);
        free(strand);
    }
    for (int i = 0; i < resSz; i++)
        out[i] = result[i];
    *outSz = resSz;
    free(result);
    free(input);
}
\end{lstlisting}

\begin{lstlisting}[language=C++, caption={Strand Sort em C++ (simplificado)},captionpos=t, label=code:strandsortC++]
#include <iostream>
#include <vector>
using namespace std;

// Funcao para mesclar dois vetores ordenados
void merge(vector<int>& mainList, vector<int>& strand) {
    vector<int> merged;
    int i = 0, j = 0;
    while (i < mainList.size() && j < strand.size()) {
        if (mainList[i] < strand[j]) {
            merged.push_back(mainList[i++]);
        } else {
            merged.push_back(strand[j++]);
        }
    }
    // Adiciona elementos restantes
    while (i < mainList.size()) merged.push_back(mainList[i++]);
    while (j < strand.size())   merged.push_back(strand[j++]);
    mainList = merged;
}

// Funcao principal do Strand Sort
void strandSort(vector<int>& input, vector<int>& output) {
    while (!input.empty()) {
        vector<int> strand;
        strand.push_back(input[0]);
        input.erase(input.begin());
        // Cria a strand ordenada
        for (auto it = input.begin(); it != input.end(); ) {
            if (*it >= strand.back()) {
                strand.push_back(*it);
                it = input.erase(it); // Remove elemento adicionado a strand
            } else {
                ++it;
            }
        }
        // Mescla a strand com a lista resultado
        merge(output, strand);
    }
}
\end{lstlisting}

\subsection{Análise de Complexidade do Strand Sort}

\subsubsection{Complexidade de Tempo}

\paragraph{Melhor Caso: $O(n)$}
\textbf{Cenário:} Array já ordenado em ordem crescente
\\
\textbf{Prova:}
\begin{itemize}
\item A lista inteira é extraída como uma única strand na primeira passagem
\item \textbf{Extração da strand:} $O(n)$ - percorre toda a lista uma vez
\item \textbf{Merge:} $O(n)$ - mescla a strand de tamanho $n$ com lista vazia
\item \textbf{Total de passagens:} 1
\item \textbf{Complexidade total:} $O(n) + O(n) = O(n)$
\end{itemize}

\paragraph{Pior Caso: $O(n^2)$}
\textbf{Cenário:} Array em ordem completamente decrescente
\\
\textbf{Prova:}
\begin{itemize}
\item Cada passagem extrai apenas um elemento (strand de tamanho 1)
\item \textbf{Número de passagens:} $n$ (uma para cada elemento)
\item \textbf{Análise por passagem $i$ ($i = 1, 2, \ldots, n$):}
\begin{itemize}
\item Extração da strand: $O(n-i+1)$ para percorrer lista restante
\item Merge com resultado: $O(i)$ para mesclar strand com lista resultado de tamanho $i-1$
\end{itemize}
\end{itemize}

\textbf{Cálculo total:}
\begin{align}
\sum_{i=1}^{n} [O(n-i+1) + O(i)] &= \sum_{i=1}^{n} O(n+1) \\
&= n \times O(n+1) \\
&= O(n^2)
\end{align}

\textbf{Demonstração detalhada:}
\begin{align}
\text{Passagem 1:} &\quad O(n) + O(1) = O(n) \\
\text{Passagem 2:} &\quad O(n-1) + O(2) = O(n+1) \\
\text{Passagem 3:} &\quad O(n-2) + O(3) = O(n+1) \\
&\vdots \\
\text{Passagem } n: &\quad O(1) + O(n) = O(n+1) \\
\text{Total:} &\quad n \times O(n+1) = O(n^2)
\end{align}

\paragraph{Caso Médio: $O(n^2)$}
\textbf{Prova:}
\begin{itemize}
\item \textbf{Tamanho esperado da strand:} constante (dependendo da distribuição)
\item \textbf{Número esperado de passagens:} $O(n)$
\item \textbf{Custo médio por passagem:} $O(n)$ (extração + merge)
\item \textbf{Complexidade total:} $O(n) \times O(n) = O(n^2)$
\end{itemize}

\subsubsection{Complexidade de Espaço}

\paragraph{Espaço Auxiliar: $O(n)$}
\textbf{Prova:}
\begin{itemize}
\item \textbf{Estruturas auxiliares necessárias:}
\begin{itemize}
\item \texttt{result}: lista resultado $\rightarrow O(n)$
\item \texttt{strand}: strand atual $\rightarrow O(k)$, onde $k \leq n$
\item \texttt{merged}: array temporário durante merge $\rightarrow O(n)$
\item Variáveis de controle (índices, iteradores) $\rightarrow O(1)$
\end{itemize}
\item \textbf{Análise detalhada:}
\begin{itemize}
\item \textbf{Lista resultado:} cresce de 0 até $n$ elementos $\rightarrow O(n)$
\item \textbf{Strand atual:} máximo $n$ elementos (melhor caso) $\rightarrow O(n)$
\item \textbf{Merge temporário:} sempre $O(n)$ para mesclar
\item \textbf{Lista de entrada modificada:} mantém cópia $\rightarrow O(n)$
\end{itemize}
\item \textbf{Espaço total:} $O(n) + O(n) + O(n) + O(n) = O(n)$
\end{itemize}

\subsubsection{Resumo das Complexidades}

\begin{table}[h]
\centering
\begin{tabular}{|l|c|l|}
\hline
\textbf{Caso} & \textbf{Complexidade de Tempo} & \textbf{Justificativa} \\
\hline
Melhor & $O(n)$ & Lista já ordenada - 1 passagem \\
\hline
Médio & $O(n^2)$ & Strands de tamanho médio constante \\
\hline
Pior & $O(n^2)$ & Lista decrescente - $n$ passagens \\
\hline
\end{tabular}
\caption{Complexidade de Tempo do Strand Sort}
\end{table}

\begin{table}[h]
\centering
\begin{tabular}{|l|c|l|}
\hline
\textbf{Espaço} & \textbf{Complexidade} & \textbf{Justificativa} \\
\hline
Auxiliar & $O(n)$ & Listas resultado, strand e temporárias \\
\hline
\end{tabular}
\caption{Complexidade de Espaço do Strand Sort}
\end{table}

\begin{thm}[Complexidade do Strand Sort]
Para o Strand Sort operando em uma lista de $n$ elementos:
\begin{itemize}
\item $T(n) = O(n^2)$ no pior caso, quando cada strand contém exatamente 1 elemento
\item $S(n) = O(n)$ sempre, devido às estruturas auxiliares necessárias
\end{itemize}
\end{thm}

\textbf{Prova do limite inferior:} No pior caso (lista decrescente), o algoritmo deve fazer $n$ passagens, cada uma custando $O(n)$ operações, resultando em $\Omega(n^2)$.

\subsubsection{Características do Algoritmo}

\begin{itemize}
\item \textbf{Estabilidade:} Estável - mantém ordem relativa de elementos iguais
\item \textbf{In-place:} Não - requer $O(n)$ espaço auxiliar
\item \textbf{Adaptativo:} Sim - performa melhor em listas com subsequências crescentes
\item \textbf{Método:} Extração e mesclagem (extracting and merging)
\item \textbf{Otimização:} Eficiente para listas com muitas subsequências crescentes
\end{itemize}

\textbf{Resumo Final:}
\begin{itemize}
\item Melhor caso: $O(n)$ - lista já ordenada
\item Caso médio: $O(n^2)$
\item Pior caso: $O(n^2)$ - lista em ordem decrescente
\item Espaço auxiliar: $O(n)$ - não é in-place
\end{itemize}


\section{Recombinant Sort}

\textbf{Descrição:} O Recombinant Sort é um algoritmo de ordenação linear (O(n)) que combina ideias exploradas em outras estruturas já conhecidas da computação como o Hashing e Programação dinâmica, além de outros algoritmos de ordenação como: Counting Sort, Bucket Sort, Radix Sort.
A ideia central é mapear cada número diretamente para uma posição em um espaço N-dimensional, baseado em seus dígitos — como se os dígitos fossem coordenadas de um hiper-cubo.

\begin{algorithm}[H]
\caption{Recombinant Sort}
\KwIn{Array $A$ contendo $N$ elementos numéricos ou strings}
\KwOut{Array $A$ ordenado}

\textbf{Preprocessamento:} Padronizar os elementos para terem o mesmo número de dígitos\;
\textbf{Criar:} Espaço cartesiano $S$ de tamanho adequado (1D, 2D, ..., $k$-dimensional)\;
\textbf{Criar:} Mapas de travessia $H_{\min}$ e $H_{\max}$ inicializados com nulos\;

\textbf{Hashing Cycle:}

\For{$i \gets 0$ \textbf{to} $N-1$}{
    Converter $A[i]$ para representação com dígitos uniformes\;
    Decompor em dígitos: $d_1, d_2, \dots, d_k$\;
    Incrementar célula correspondente: $S[d_1][d_2]\dots [d_k] \gets S[d_1][d_2]\dots[d_k] + 1$\;
    Atualizar $H_{\min}$ e $H_{\max}$ para cada dimensão\;
}

\textbf{Extraction Cycle:}
$j \gets 0$\tcp*{Próxima posição de escrita em $A$}

\ForEach{índice válido $(i_1, i_2, \dots, i_k)$ no espaço $S$ em ordem crescente}{
    \If{$H_{\min}$ e $H_{\max}$ indicam que a posição é alcançável}{
        \While{$S[i_1][i_2]\dots[i_k] > 0$}{
            Recuperar valor correspondente aos dígitos $(i_1, \dots, i_k)$\;
            Escrever valor em $A[j]$\;
            $j \gets j+1$\;
            $S[i_1][i_2]\dots[i_k] \gets S[i_1][i_2]\dots[i_k] - 1$\;
        }
    }
}

\Return{$A$}
\end{algorithm}


\begin{lstlisting}[language=Python, caption={Recombinant Sort em Python}, captionpos=t, label=code:RecombinantSortPy]
from typing import List

def recombinant_sort(arr: List[float]) -> None:
    """
    Recombinant Sort para numeros em [0,10) com 1 casa decimal.
    Ordena o array IN PLACE.

    Ex.: [4.5, 0.3, 2.3, 8.8, 7.0, 9.2, 4.5, 4.3, 8.0, 3.2]
    """
    # Count array 10x10
    S = [[0 for _ in range(10)] for _ in range(10)]
    H_min = [10] * 10
    H_max = [-1] * 10

    # ---------- Hashing Cycle ----------
    for x in arr:
        scaled = int(round(x * 10.0))  # 4.5 -> 45

        if scaled < 0 or scaled > 99:
            # fora do intervalo suportado
            continue

        i = scaled // 10
        j = scaled % 10

        S[i][j] += 1
        if j < H_min[i]:
            H_min[i] = j
        if j > H_max[i]:
            H_max[i] = j

    # ---------- Extraction Cycle ----------
    idx = 0
    for i in range(10):
        if H_max[i] == -1:
            continue

        for j in range(H_min[i], H_max[i] + 1):
            while S[i][j] > 0:
                arr[idx] = i + j / 10.0
                idx += 1
                S[i][j] -= 1

if __name__ == "__main__":
    data = [4.5, 0.3, 2.3, 8.8, 7.0, 9.2, 4.5, 4.3, 8.0, 3.2]
    print("Antes:", data)
    recombinant_sort(data)
    print("Depois:", data)

\end{lstlisting}

\begin{lstlisting}[language=C, caption={Implementação do Recombinant Sort em C}, captionpos=t, label=code:RecombinantSortC]
#include <stdio.h>
#include <math.h>   // round

// Recombinant Sort para numeros em [0,10) com 1 casa decimal
void recombinant_sort(double *arr, int n) {
    int S[10][10] = {0};   // count array
    int H_min[10];         // menor coluna usada em cada linha
    int H_max[10];         // maior coluna usada em cada linha

    // inicializa H_min/H_max
    for (int i = 0; i < 10; i++) {
        H_min[i] = 10;   // sentinel "nenhuma coluna"
        H_max[i] = -1;
    }

    // ---------- Hashing Cycle ----------
    for (int k = 0; k < n; k++) {
        double x = arr[k];

        // assume x em [0,10) com 1 casa decimal (ou arredondado para isso)
        int scaled = (int)llround(x * 10.0); // ex.: 4.5 -> 45

        if (scaled < 0 || scaled > 99) {
            // fora do intervalo suportado, aqui soh ignoro;
            // em codigo real voce trataria isso (erro ou normalizacao).
            continue;
        }

        int i = scaled / 10;  // parte inteira
        int j = scaled % 10;  // digito apos a virgula

        S[i][j]++;

        if (j < H_min[i]) H_min[i] = j;
        if (j > H_max[i]) H_max[i] = j;
    }

    // ---------- Extraction Cycle ----------
    int idx = 0;
    for (int i = 0; i < 10; i++) {
        if (H_max[i] == -1) continue;  // linha vazia

        for (int j = H_min[i]; j <= H_max[i]; j++) {
            while (S[i][j] > 0) {
                arr[idx++] = (double)i + (double)j / 10.0;
                S[i][j]--;
            }
        }
    }
}

// Exemplo de uso
int main(void) {
    double arr[] = {4.5, 0.3, 2.3, 8.8, 7.0, 9.2, 4.5, 4.3, 8.0, 3.2};
    int n = sizeof(arr) / sizeof(arr[0]);

    printf("Antes:\n");
    for (int i = 0; i < n; i++) {
        printf("%.1f ", arr[i]);
    }
    printf("\n");

    recombinant_sort(arr, n);

    printf("Depois:\n");
    for (int i = 0; i < n; i++) {
        printf("%.1f ", arr[i]);
    }
    printf("\n");

    return 0;
}

\end{lstlisting}

\begin{lstlisting}[language=C++, caption={Implementação do Recombinant Sort em C++}, captionpos=t, label=code:RecombinantSortCpp]
#include <iostream>
#include <vector>
#include <cmath>    // std::round

// Recombinant Sort para numeros em [0,10) com 1 casa decimal
void recombinant_sort(std::vector<double>& arr) {
    int S[10][10] = {0};
    int H_min[10];
    int H_max[10];

    for (int i = 0; i < 10; i++) {
        H_min[i] = 10;
        H_max[i] = -1;
    }

    // ---------- Hashing Cycle ----------
    for (double x : arr) {
        int scaled = static_cast<int>(std::llround(x * 10.0));

        if (scaled < 0 || scaled > 99) {
            // fora do intervalo suportado - tratar conforme necessidade
            continue;
        }

        int i = scaled / 10;
        int j = scaled % 10;

        S[i][j]++;

        if (j < H_min[i]) H_min[i] = j;
        if (j > H_max[i]) H_max[i] = j;
    }

    // ---------- Extraction Cycle ----------
    int idx = 0;
    for (int i = 0; i < 10; i++) {
        if (H_max[i] == -1) continue;

        for (int j = H_min[i]; j <= H_max[i]; j++) {
            while (S[i][j] > 0) {
                arr[idx++] = static_cast<double>(i) + static_cast<double>(j) / 10.0;
                S[i][j]--;
            }
        }
    }
}

// Exemplo de uso
int main() {
    std::vector<double> arr = {4.5, 0.3, 2.3, 8.8, 7.0, 9.2, 4.5, 4.3, 8.0, 3.2};

    std::cout << "Antes:\n";
    for (double x : arr) {
        std::cout << x << " ";
    }
    std::cout << "\n";

    recombinant_sort(arr);

    std::cout << "Depois:\n";
    for (double x : arr) {
        std::cout << x << " ";
    }
    std::cout << "\n";

    return 0;
}


\end{lstlisting}

\subsection{Análise de Complexidade}

Nesta seção, analisamos formalmente as complexidades de tempo e espaço do \textit{Recombinant Sort}.

\subsubsection{Complexidade de Tempo}

Esse algoritmo é composto por duas fases principais: o \textit{Hashing Cycle}, onde cada elemento é mapeado para uma célula do espaço cartesiano multidimensional, e o \textit{Extraction Cycle}, que percorre apenas as regiões preenchidas desse espaço para reconstruir o array ordenado.
Segundo a análise formal do artigo análisado, o \textit{Recombinant Sort} possui tempo de execução descrito por
\[
T(n) = O(n + k),
\]
onde $n$ é o número de elementos e $k$ representa o custo da fase de extração. O ponto crucial estabelecido é que o custo de extração $k$ é sempre limitado por
\[
k \le n,
\]
e, portanto, nunca cresce além da mesma ordem de grandeza de $n$. Com isso, segue que
\[
T(n) = O(n + k) = O(n + n) = O(n).
\]

Além disso, esse limite superior vale uniformemente para os casos melhor, médio e pior, uma vez que o algoritmo não depende da distribuição inicial dos elementos: o mapeamento para o espaço multidimensional e a travessia controlada pelas estruturas auxiliares garantem o mesmo padrão de execução em todos os cenários.

Portanto,
\[
T_{\text{melhor}}(n) = 
T_{\text{médio}}(n) =
T_{\text{pior}}(n) = 
O(n).
\]

$\hfill\Box$

\bigskip

\subsubsection{Complexidade de Espaço}

O \textit{Recombinant Sort} utiliza um espaço cartesiano multidimensional (o \textit{count array}) e dois mapas auxiliares, $H_{\min}$ e $H_{\max}$. A dimensão desse espaço depende exclusivamente do número de dígitos ou caracteres possíveis nos elementos a serem ordenados, e não do tamanho da entrada $n$.

Se cada elemento é decomposto em $d$ dígitos, cada um pertencente a um domínio finito de tamanho fixo $D$, então o espaço necessário é
\[
S(n) = O(D^d),
\]
uma quantidade independente de $n$. Como esses domínios são constantes na maioria das aplicações práticas, o consumo adicional de memória não cresce com a entrada.

Assim, em termos assintóticos relativamente a $n$,
\[
S(n) = O(1).
\]

\noindent\textbf{Prova:}  
Durante a execução, o algoritmo mantém apenas o espaço multidimensional de contagem e os mapas auxiliares, todos de tamanho fixo definido pelo domínio dos dígitos, e não pelo número de elementos da entrada. Dessa forma, nenhuma estrutura cresce proporcionalmente a $n$, caracterizando o uso de memória constante.

\[
S(n) = c = O(1).
\]
$\hfill\Box$

\bigskip

\noindent\textbf{Discussão:}  
O \textit{Recombinant Sort} apresenta comportamento linear em seu tempo de execução, superando limitações de algoritmos derivados como \textit{Counting Sort}, \textit{Radix Sort} e \textit{Bucket Sort}. Sua capacidade de tratar números inteiros, reais e cadeias de caracteres com a mesma estrutura o torna uma escolha versátil. Embora a dimensão do espaço cartesiano cresça com o número de dígitos, essa expansão é limitada na prática e não afeta a análise assintótica em termos de $n$.


\section{Sorting network}

\href{https://www.csunplugged.org/en/topics/sorting-networks/whats-it-all-about/}{Veja Sorting network}\\
\textbf{Descrição:} Redes de ordenação são dispositivos abstratos compostos por um número fixo de “fios”, que transportam valores, e por módulos comparadores, responsáveis por conectar pares desses fios. Cada comparador tem a função de verificar a ordem dos valores nos fios e trocá-los, caso não estejam dispostos corretamente. Essas redes são projetadas para ordenar um conjunto fixo de valores, executando uma sequência predefinida de comparações e trocas, independentemente dos dados de entrada. Por isso, constituem uma abordagem determinística e paralelizável para o problema da ordenação, amplamente utilizada em contextos teóricos e em implementações de hardware ou software.\\
Os fios são dispostos da esquerda para a direita e cada um transporta um valor. Esses valores percorrem a rede ao mesmo tempo.
Os comparadores conectam dois fios e verificam se os valores estão na ordem correta. Quando um comparador encontra dois valores — digamos, x no fio de cima e y no fio de baixo — ele os troca apenas se o valor de cima for maior que o de baixo. Assim, depois da comparação, o fio de cima fica com o menor valor (x' = min(x, y)) e o fio de baixo fica com o maior valor (y' = max(x, y)), garantindo que os dois estejam ordenados.\\


\begin{algorithm}[H]
\DontPrintSemicolon
\small
\textbf{função} \texttt{SORTING\_NETWORK(vetor, comparadores)} \;

\ForEach{$(i, j)$ \textbf{em} comparadores}{
    \If{$vetor[i] > vetor[j]$}{
        trocar($vetor[i], vetor[j]$)\;
    }
}

\Return vetor\;

\caption{Sorting Network}
\label{lab:alg-SortingNet}
\end{algorithm}


\begin{lstlisting}[language=Python, caption={Implementação do algoritmo Sorting network em Python}, captionpos=t, label=code:SortingPy]
from typing import List, Tuple, TypeVar

T = TypeVar("T")

def sorting_network(values: List[T], comparators: List[Tuple[int, int]]) -> List[T]:

    n = len(values)
    a = list(values) 
    for i, j in comparators:
        if not (0 <= i < n and 0 <= j < n):
            raise IndexError(f"Comparador fora dos limites: ({i}, {j}) para n={n}")
        if a[i] > a[j]:
            a[i], a[j] = a[j], a[i]
    return a
\end{lstlisting}

\begin{lstlisting}[language=C, caption={Implementação do algoritmo Sorting network em C},captionpos=t, label=code:SortingC]
#include <stdio.h>
#include <stdlib.h>

typedef struct {
    size_t i, j;
} Comparator;

void swap_int(int *a, int *b) {
    int temp = *a;
    *a = *b;
    *b = temp;
}

int* sorting_network(const int *values, size_t n,
                     const Comparator *comps, size_t m)
{
    int *a = malloc(n * sizeof(int));
    if (!a) {
    fprintf(stderr, "Erro: falha ao alocar memoria.\n");
        exit(EXIT_FAILURE);
    }
    for (size_t k = 0; k < n; k++)
        a[k] = values[k];

    for (size_t k = 0; k < m; k++) {
        size_t i = comps[k].i;
        size_t j = comps[k].j;

        if (i >= n || j >= n) {
            fprintf(stderr,
                "Erro: comparador fora dos limites: (%zu, %zu) para n=%zu\n",
                i, j, n);
            free(a);
            exit(EXIT_FAILURE);
        }

        if (a[i] > a[j]) {
            swap_int(&a[i], &a[j]);
        }
    }

    return a;
}
\end{lstlisting}

\begin{lstlisting}[language=C++, caption={Implementação do algoritmo Sorting network em C++},captionpos=t, label=code:SortingCpp]
#include <bits/stdc++.h>
using namespace std;

template <typename T>
vector<T> sorting_network(const vector<T>& values,
                          const vector<pair<size_t, size_t>>& comparators) {
    vector<T> a = values; // copia
    const size_t n = a.size();
    for (auto [i, j] : comparators) {
        if (i >= n || j >= n) {
            throw out_of_range("Comparador fora dos limites: (" + to_string(i) +
                               ", " + to_string(j) + ") para n=" + to_string(n));
        }
        if (a[i] > a[j]) {
            swap(a[i], a[j]);
        }
    }
    return a;
}
\end{lstlisting}

\subsection{Análise de Complexidade}

Nesta seção, analisamos formalmente as complexidades de tempo e espaço das \textit{Sorting Networks}.  

\subsubsection{Complexidade de Tempo}

Uma \textit{Sorting Network} consiste em uma lista fixa de comparadores do tipo $(i,j)$, onde cada comparador executa no máximo uma comparação e, se necessário, uma troca.  
Seja $C$ o número total de comparadores da rede. Como cada comparador realiza tempo constante $O(1)$, o tempo de execução sequencial é:

\[
T(n) = O(C)
\]

Redes de ordenação assintoticamente eficientes, como a \textit{Batcher Bitonic Sort} ou \textit{Odd-Even Merge Sort}, possuem número de comparadores proporcional a:

\[
C = O(n \log^2 n)
\]

Portanto, a complexidade de tempo total no modelo sequencial é:

\[
T(n) = O(n \log^2 n)
\]

\noindent{\textbf{Prova:}}  
Seja $a_1$ o custo constante de processar um comparador.  
Então:

\[
T(n) = a_1 \cdot C
\]

Substituindo $C = O(n \log^2 n)$, obtemos:

\[
T(n) = a_1 \cdot O(n \log^2 n) = O(n \log^2 n)
\]
$\hfill\Box$

\bigskip

\noindent{\textbf{Discussão:}}  
Diferente de algoritmos adaptativos, como \textit{QuickSort} ou \textit{Insertion Sort}, o número de operações em uma sorting network não depende da ordem dos dados de entrada.  
Assim, mesmo entradas já ordenadas ou totalmente reversas levam o mesmo tempo.  

\subsubsection{Complexidade de Espaço}

Uma sorting network opera \textit{in-place}, modificando diretamente o vetor de entrada.  
O custo de memória dividido em duas partes é:

\[
\text{Vetor de dados: } O(n)
\]
\[
\text{Lista de comparadores: } O(C) = O(n \log^2 n)
\]

Portanto, a complexidade de espaço total é:

\[
S(n) = O(n + n \log^2 n) = O(n \log^2 n)
\]

\noindent{\textbf{Prova:}}  
O vetor de entrada ocupa espaço proporcional a $n$ e a lista estática de comparadores ocupa espaço proporcional a $C$.  
Como $C = O(n \log^2 n)$, então:

\[
S(n) = O(n) + O(n \log^2 n) = O(n \log^2 n)
\]
$\hfill\Box$

\bigskip

\noindent{\textbf{Discussão:}}  
Se a rede for estática (pré-gerada e codificada no binário), nenhum espaço adicional é alocado em tempo de execução, reduzindo o consumo de memória dinâmica para:

\[
S(n) = O(n)
\]

\section{Bitonic sorter}

\href{https://sortvisualizer.com/bitonicsort/}{Veja Bitonic sorter}\\
\textbf{Descrição:} O Bitonic Sort é um algoritmo de ordenação baseado em comparações que utiliza propriedades de sequências bitônicas — ou seja, sequências que primeiro crescem e depois decrescem (ou vice-versa). Por essa razão, ele só pode ser aplicado a conjuntos de dados cujo número de elementos seja uma potência de 2.

O funcionamento do algoritmo ocorre em duas etapas principais:\\
Na primeira, os elementos são organizados em uma sequência bitônica, formando grupos de valores dispostos de maneira crescente e decrescente alternadamente.\\
Na segunda etapa, esses grupos são mesclados por meio de comparações sucessivas, de forma semelhante ao Merge Sort, até que toda a estrutura de dados esteja completamente ordenada.\\

\begin{algorithm}[htbp]
\DontPrintSemicolon
\small
\textbf{Pré-condição:} o tamanho do vetor \texttt{arr} é uma potência de 2. \;

\bigskip
\textbf{procedimento} \texttt{BITONIC\_SORT(arr, início, fim, crescente)} \;

$meio \gets (inicio + fim)/2$\;

\If{$crescente = \text{verdadeiro}$}{
    \texttt{BITONIC\_SORT(arr, inicio, meio, verdadeiro)}\;   \tcp*{ordem crescente}
    \texttt{BITONIC\_SORT(arr, meio, fim, falso)}\;           \tcp*{ordem decrescente}
}
\Else{
    \texttt{BITONIC\_SORT(arr, inicio, meio, falso)}\;
    \texttt{BITONIC\_SORT(arr, meio, fim, verdadeiro)}\;
}

\texttt{BITONIC\_MERGE(arr, início, fim, crescente)}\;

\bigskip
\textbf{procedimento} \texttt{BITONIC\_MERGE($arr, inicio, fim, crescente$)} \;

$tamanho \gets fim - inicio$\;

\If{$tamanho < 2$}{
    \Return\;
}

$meio \gets (inicio + fim)/2$\;

\For{$i \gets inicio$ \KwTo $meio - 1$}{
    $j \gets i + tamanho/2$\;
    \If{$(crescente \;\wedge\; arr[i] > arr[j])$ \textbf{ou} $(\neg crescente \;\wedge\; arr[i] < arr[j])$}{
        trocar($arr[i], arr[j]$)\;
    }
}
\texttt{BITONIC\_MERGE(arr, início, meio, crescente)}\;
\texttt{BITONIC\_MERGE(arr, meio, fim, crescente)}\;

\caption{Bitonic Sorter}
\label{lab:alg-BitonicSorter}
\end{algorithm}


\begin{lstlisting}[language=Python, caption={Implementação do algoritmo Bitonic sorter em Python}, captionpos=t, label=code:BitonicPy]
def bitonic_sort(arr):
    n = len(arr)
    for k in range(2, n+1):
        j = k // 2
        while j > 0:
            for i in range(0, n):
                l = i ^ j
                if l > i:
                    if ( ((i&k)==0) and (arr[i] > arr[l]) or ( ( (i&k)!=0) and (arr[i] < arr[l])) ):
                        temp = arr[i]
                        arr[i] = arr[l]
                        arr[l] = temp
            j //= 2
\end{lstlisting}

\begin{lstlisting}[language=C, caption={Implementação do algoritmo Bitonic sorter em C}, captionpos=t, label=code:BitonicC]
void bitonicSort(int *arr, int n) {
    int k, j, l, i, temp;
    for (k = 2; k <= n; k *= 2) {
        for (j = k/2; j > 0; j /= 2) {
            for (i = 0; i < n; i++) {
                l = i ^ j;
                if (l > i) {
                    if ( ((i&k)==0) && (arr[i] > arr[l]) || ( ( (i&k)!=0) && (arr[i] < arr[l])) )  {
                        temp = arr[i];
                        arr[i] = arr[l];
                        arr[l] = temp;
                    }
                }
            }
        }
    }
}
\end{lstlisting}

\begin{lstlisting}[language=C++, caption={Implementação do algoritmo Bitonic sorter em C++}, captionpos=t, label=code:BitonicCpp]
void bitonicSort(int *arr, int n) {
    int k, j, l, i, temp;
    for (k = 2; k <= n; k *= 2) {
        for (j = k/2; j > 0; j /= 2) {
            for (i = 0; i < n; i++) {
                l = i ^ j;
                if (l > i) {
                    if ( ((i&k)==0) && (arr[i] > arr[l]) || ( ( (i&k)!=0) && (arr[i] < arr[l])) )  {
                        temp = arr[i];
                        arr[i] = arr[l];
                        arr[l] = temp;
                    }
                }
            }
        }
    }
}

\end{lstlisting}

\subsection{Análise de Complexidade}

Nesta seção, analisamos formalmente as complexidades de tempo e espaço do \textit{Bitonic Sorter}.  

\subsubsection{Complexidade de Tempo}

O \textit{Bitonic Sort} opera recursivamente sobre o vetor, dividindo-o em metades e alternando ordenações crescente e decrescente até formar uma sequência bitônica.  
A seguir, a rotina \texttt{BITONIC\_MERGE} realiza comparações em pares com chamadas recursivas sobre subvetores progressivamente menores.

A profundidade total da recursão da fase de construção é $\log n$, e em cada nível são executadas $n$ comparações distribuídas entre os merges.  
Assim, o tempo total é dado por:

\[
T(n) = \Theta(n \log^2 n)
\]

\noindent{\textbf{Prova:}}  
Seja $T(n)$ o custo do algoritmo:

\[
T(n) = 2T\left(\frac{n}{2}\right) + M(n)
\]

onde $M(n)$ corresponde ao custo do \texttt{BITONIC\_MERGE}.  
Como cada merge executa $\frac{n}{2}$ comparações e é chamado $\log n$ vezes, temos:

\[
M(n) = \Theta(n \log n)
\]

Aplicando o Teorema Mestre ao caso:

\[
T(n) = 2T\left(\frac{n}{2}\right) + \Theta(n \log n)
\]

conclui-se que:

\[
T(n) = \Theta(n \log^2 n)
\]
$\hfill\Box$

\bigskip

\noindent{\textbf{Discussão:}}  
O \textit{Bitonic Sort} possui o mesmo custo em todos os cenários — melhor caso, pior caso e caso médio — pois seu padrão de comparações não depende dos dados de entrada.  

\subsubsection{Complexidade de Espaço}

O algoritmo opera sobre o vetor de entrada, mas a rede de comparadores subjacente possui $\Theta(n \log^2 n)$ conexões estáticas, o que representa seu custo estrutural.

\[
S(n) = O(n) + O(n \log^2 n)
\]

Se a rede de comparadores for gerada dinamicamente, o consumo de espaço total é:

\[
S(n) = \Theta(n \log^2 n)
\]

\bigskip

\noindent{\textbf{Discussão:}}  
A necessidade de espaço adicional não decorre de estruturas auxiliares de dados, mas do tamanho da própria rede.  
Por isso, quando empregada como rotina paralela em hardware, o uso de memória é considerado \textit{in-place}.

% \section{Resumo}

% \begin{center}
% \begin{tabular}{||c|c|c|c||}
% \hline
% \multicolumn{4}{|c|}{Complexidades de tempo em termos de comparações} \\
% \hline
% Algoritmo & Pior caso & Melhor caso & Caso médio \\
% \hline
% bubble      & $O(n^2)$       & $O(n)$          & $O(n^2)$ \\
% insertion   & $O(n^2)$       & $O(n)$          & $O(n^2)$ \\
% combsort    & $O(n^2)$       & $O(n\log n)$    & $\approx O(n \log n)$ \\
% selection   & $O(n^2)$       & $O(n^2)$        & $O(n^2)$ \\
% shellsort   & $O(n^2)$       & $\Omega(n\log n)$ & $O(n^{3/2})$ \\
% gnome       & $O(n^2)$       & $O(n)$          & $O(n^2)$ \\
% shaker      & $O(n^2)$       & $O(n)$          & $O(n^2)$ \\
% odd-even    & $O(n^2)$       & $O(n)$          & $O(n^2)$ \\
% pancake (lançamentos)     & $O(n)$  & $O(n)$  & $O(n)$  \\
% \hline
% \end{tabular}
% \end{center}

% \begin{center}
% \begin{tabular}{||c|c|c|c||}
% \hline
% \multicolumn{4}{|c|}{Complexidades de espaço} \\
% \hline
% Algoritmo & Pior caso & Melhor caso & Caso médio \\
% \hline
% bubble      & $O(1)$ & $O(1)$ & $O(1)$ \\
% insertion   & $O(1)$ & $O(1)$ & $O(1)$ \\
% combsort    & $O(1)$ & $O(1)$ & $O(1)$ \\
% selection   & $O(1)$ & $O(1)$ & $O(1)$ \\
% shellsort   & $O(1)$ & $O(1)$ & $O(1)$ \\
% gnome       & $O(1)$ & $O(1)$ & $O(1)$ \\
% shaker      & $O(1)$ & $O(1)$ & $O(1)$ \\
% odd-even    & $O(1)$ & $O(1)$ & $O(1)$ \\
% pancake     & $O(1)$ & $O(1)$ & $O(1)$ \\
% \hline
% \end{tabular}
% \end{center}





\chapter{Algoritmos com complexidades de tempo linear}

Nesta seção apresentamos algoritmos de ordenação cuja complexidade de tempo pode ser considerada linear em determinados cenários. Esses métodos exploram propriedades dos dados de entrada, como o tamanho do domínio ou a representação numérica. 

\section{Counting sort}
\subsection{Descrição e Funcionamento}
O \textit{Counting Sort} é um algoritmo de ordenação estável e não comparativo. 
Em vez de realizar comparações diretas entre elementos, ele conta o número de ocorrências de cada valor e, a partir dessas contagens, reconstrói o vetor ordenado. 
Essa abordagem torna o método especialmente eficiente quando o intervalo de valores possíveis é relativamente pequeno em comparação ao número de elementos de entrada.

\medskip
A seguir, apresenta-se um exemplo ilustrativo de execução do algoritmo.

\begin{exmp}
Considere o vetor $A = [4, 2, 2, 8, 3, 3, 1]$. O objetivo é ordená-lo utilizando o \textit{Counting Sort}.

\begin{enumerate}
    \item \textbf{Contagem de frequências:}  
    Criamos um vetor auxiliar $C$ com tamanho igual ao maior valor de $A$ (neste caso, $k = 8$), inicializado com zeros.  
    Em seguida, contamos quantas vezes cada valor ocorre:
    \[
    C = [0, 1, 2, 2, 1, 0, 0, 0, 1],
    \]
    onde $C[i]$ representa a quantidade de ocorrências do valor $i$ em $A$.

    \item \textbf{Cálculo das posições acumuladas:}  
    Atualizamos $C$ de modo que cada posição passe a armazenar a soma cumulativa das frequências anteriores:
    \[
    C = [0, 1, 3, 5, 6, 6, 6, 6, 7].
    \]
    Assim, $C[i]$ indica a posição final do último elemento $i$ no vetor ordenado.

    \item \textbf{Construção do vetor ordenado:}  
    Percorremos o vetor $A$ da direita para a esquerda e posicionamos cada elemento em seu local correto no vetor de saída $B$:
    \[
    B = [1, 2, 2, 3, 3, 4, 8].
    \]
    Finalmente, copiamos o conteúdo de $B$ de volta para $A$, concluindo a ordenação.
\end{enumerate}
\end{exmp}

\medskip
O pseudocódigo correspondente é apresentado a seguir.

\begin{center}
\begin{minipage}{.9\linewidth}
\begin{algorithm}[H]
\DontPrintSemicolon
\hspace{-0.25cm}\textbf{countingSort(values: array of int, n: integer, k: integer)}

\For{$i \gets 0$ \KwTo $k$}{
    $count[i] \gets 0$\;
}
\For{$i \gets 0$ \KwTo $n-1$}{
    $count[values[i]] \gets count[values[i]] + 1$\;
}
\For{$i \gets 1$ \KwTo $k$}{
    $count[i] \gets count[i] + count[i-1]$\;
}
\For{$i \gets n-1$ \KwTo $0$}{
    $output[count[values[i]]-1] \gets values[i]$\;
    $count[values[i]] \gets count[values[i]] - 1$\;
}
\For{$i \gets 0$ \KwTo $n-1$}{
    $values[i] \gets output[i]$\;
}
\caption{Counting sort}
\label{lab:alg-countingSort}
\end{algorithm}
\end{minipage}
\end{center}

\subsection{Implementações}
\begin{lstlisting}[language=Python,caption={Counting sort em Python},captionpos=t]
def counting_sort(arr, k):
    count = [0] * (k+1)
    output = [0] * len(arr)
    for num in arr:
        count[num] += 1
    for i in range(1, len(count)):
        count[i] += count[i-1]
    for num in reversed(arr):
        output[count[num]-1] = num
        count[num] -= 1
    return output
\end{lstlisting}
\begin{lstlisting}[language=C,caption={Counting sort em C},captionpos=t]
void countingSort(int arr[], int n, int k) {
    int count[k+1];
    int output[n];
    for(int i=0; i<=k; i++) count[i] = 0;
    for(int i=0; i<n; i++) count[arr[i]]++;
    for(int i=1; i<=k; i++) count[i] += count[i-1];
    for(int i=n-1; i>=0; i--) {
        output[count[arr[i]]-1] = arr[i];
        count[arr[i]]--;
    }
    for(int i=0; i<n; i++) arr[i] = output[i];
}
\end{lstlisting}
\begin{lstlisting}[language=C++,caption={Counting sort em C++},captionpos=t]
#include <vector>
using namespace std;

void countingSort(vector<int>& arr, int k) {
    int n = arr.size();
    vector<int> count(k + 1, 0), output(n);

    for (int num : arr)
        count[num]++;

    for (int i = 1; i <= k; i++)
        count[i] += count[i - 1];

    for (int i = n - 1; i >= 0; i--) {
        output[count[arr[i]] - 1] = arr[i];
        count[arr[i]]--;
    }

    arr = output;
}
\end{lstlisting}

\subsection{Análise de complexidade}
Nesta seção, analisamos formalmente as complexidades de tempo e espaço do algoritmo \textit{Counting Sort}. Esse algoritmo não é baseado em comparações diretas, mas sim em contagem de ocorrências de cada elemento, o que o torna capaz de ordenar em tempo linear sob determinadas condições.
\subsubsection{Complexidade de Tempo}

Seja $n$ o número de elementos do vetor de entrada e $k$ o valor máximo possível de uma chave (isto é, os elementos pertencem ao intervalo $[0, k]$).

O algoritmo \textit{Counting Sort} executa as seguintes etapas principais:

\begin{enumerate}
    \item Inicialização de um vetor auxiliar $C[0 \ldots k]$ com zeros — custo de $O(k)$.
    \item Contagem das ocorrências dos elementos — custo de $O(n)$.
    \item Cálculo do somatório cumulativo em $C$ — custo de $O(k)$.
    \item Construção do vetor de saída $B[1 \ldots n]$ — custo de $O(n)$.
\end{enumerate}

Assim, o tempo total de execução $T(n,k)$ pode ser expresso como:

\[
T(n,k) = a_1\cdot n + a_2\cdot k + b
\]

onde $a_1$, $a_2$ e $b$ são constantes positivas.

\begin{equation}
T(n,k) \in O(n + k)
\end{equation}

\noindent{\textbf{Prova:}}

Sabemos que, para constantes positivas $a_1, a_2, b$, existe uma constante $c = \max(a_1, a_2) + b$ tal que:

\[
T(n,k) = a_1n + a_2k + b \leq c(n+k)
\]

para todo $n, k \geq 0$.  
Logo, existe $c > 0$ e $n_0 \geq 0$ tais que $T(n,k) \leq c(n+k)$ para $n, k \geq n_0$.  
Portanto,

\[
T(n,k) \in O(n + k).
\]
$\hfill\Box$

\bigskip

\noindent{\textbf{Discussão:}}  
Se $k = O(n)$, isto é, se o intervalo de chaves for proporcional ao tamanho da entrada, então $T(n,k) \in O(n)$ e o algoritmo executa em tempo linear.  
Caso contrário, se $k \gg n$, o custo torna-se dominado por $k$, e o algoritmo perde eficiência em relação a métodos baseados em comparação ($O(n \log n)$).

\subsubsection{Complexidade de Espaço}

O algoritmo \textit{Counting Sort} utiliza três estruturas principais:

\begin{itemize}
    \item O vetor de entrada $A[1 \ldots n]$ — espaço $O(n)$.
    \item O vetor auxiliar de contagem $C[0 \ldots k]$ — espaço $O(k)$.
    \item O vetor de saída $B[1 \ldots n]$ — espaço $O(n)$.
\end{itemize}

O espaço total $S(n,k)$ pode ser expresso como:

\[
S(n,k) = c_1\cdot n + c_2\cdot k + c_3
\]

para constantes positivas $c_1, c_2, c_3$.  
Assim, temos:

\begin{equation}
S(n,k) \in O(n + k)
\end{equation}

\noindent{\textbf{Prova:}}  
Para constantes positivas $c_1, c_2, c_3$, temos

\[
S(n,k) = c_1n + c_2k + c_3 \leq c(n+k)
\]

onde $c = \max(c_1, c_2) + c_3$.  
Portanto, existe $c > 0$ tal que $S(n,k) \leq c(n+k)$ para todo $n,k \geq 0$, e assim

\[
S(n,k) \in O(n + k).
\]
$\hfill\Box$

\bigskip

\noindent{\textbf{Discussão:}}  
\textit{Counting Sort} não é um algoritmo \textit{in-place}, pois requer memória adicional proporcional ao tamanho de $n$ e $k$.  
Em contrapartida, ele é estável e pode ser adaptado como sub-rotina em algoritmos como \textit{Radix Sort}.

\section{LSD Radix sort}
\subsection{Descrição e Funcionamento}
O \textit{Radix Sort} é um algoritmo de ordenação estável que organiza números inteiros processando seus dígitos individualmente. 
Na variação \textit{LSD} (\textit{Least Significant Digit}), o processamento ocorre do dígito menos significativo para o mais significativo. 
Cada etapa utiliza um algoritmo de ordenação estável — normalmente o \textit{Counting Sort} — como sub-rotina.  
O método é eficiente quando o número de dígitos $d$ é pequeno e o domínio de cada dígito (por exemplo, 0–9 no caso decimal) é limitado.

\medskip
A seguir, apresenta-se um exemplo ilustrativo de sua execução.

\begin{exmp}
Considere o vetor $A = [170, 45, 75, 90, 802, 24, 2, 66]$.  
O objetivo é ordená-lo utilizando o \textit{LSD Radix Sort}, aplicando o \textit{Counting Sort} em cada posição decimal (unidades, dezenas e centenas).

\begin{enumerate}
    \item \textbf{Ordenação pelo dígito das unidades:}  
    Extraímos o dígito das unidades de cada elemento e aplicamos o \textit{Counting Sort}.  
    Após esta etapa, o vetor torna-se:
    \[
    [170, 90, 802, 2, 24, 45, 75, 66].
    \]

    \item \textbf{Ordenação pelo dígito das dezenas:}  
    Repetimos o processo considerando o dígito das dezenas de cada número.  
    O vetor torna-se:
    \[
    [802, 2, 24, 45, 66, 170, 75, 90].
    \]

    \item \textbf{Ordenação pelo dígito das centenas:}  
    Finalmente, aplicamos o \textit{Counting Sort} sobre o dígito das centenas, obtendo o vetor final ordenado:
    \[
    [2, 24, 45, 66, 75, 90, 170, 802].
    \]
\end{enumerate}
\end{exmp}

\medskip
O pseudocódigo correspondente é apresentado a seguir.

\begin{center}
\begin{minipage}{.9\linewidth}
\begin{algorithm}[H]
\DontPrintSemicolon
\hspace{-0.25cm}\textbf{radixSort(values: array of int, n: integer)}

$m \gets$ max(values)\;
$exp \gets 1$\;
\While{$m / exp > 0$}{
    countingSort(values, n, exp)\;
    $exp \gets exp \times 10$\;
}
\caption{LSD Radix sort}
\label{lab:alg-radixSort}
\end{algorithm}
\end{minipage}
\end{center}

\subsection{Implementações}
\begin{lstlisting}[language=Python,caption={Radix sort em Python},captionpos=t]
def counting_sort_radix(arr, exp):
    n = len(arr)
    output = [0] * n
    count = [0] * 10
    for num in arr:
        index = (num // exp) % 10
        count[index] += 1
    for i in range(1, 10):
        count[i] += count[i-1]
    for num in reversed(arr):
        index = (num // exp) % 10
        output[count[index]-1] = num
        count[index] -= 1
    return output

def radix_sort(arr):
    m = max(arr)
    exp = 1
    while m // exp > 0:
        arr = counting_sort_radix(arr, exp)
        exp *= 10
    return arr
\end{lstlisting}
\begin{lstlisting}[language=C,caption={Radix sort em C},captionpos=t]
void countingSortRadix(int arr[], int n, int exp) {
    int output[n], count[10] = {0};
    for(int i=0; i<n; i++)
        count[(arr[i]/exp)%10]++;
    for(int i=1; i<10; i++)
        count[i] += count[i-1];
    for(int i=n-1; i>=0; i--) {
        output[count[(arr[i]/exp)%10]-1] = arr[i];
        count[(arr[i]/exp)%10]--;
    }
    for(int i=0; i<n; i++) arr[i] = output[i];
}

void radixSort(int arr[], int n) {
    int m = arr[0];
    for(int i=1; i<n; i++)
        if(arr[i] > m) m = arr[i];
    int exp = 1;
    while(m/exp > 0) {
        countingSort(arr, n, exp);
        exp *= 10;
    }
}
\end{lstlisting}
\begin{lstlisting}[language=C++,caption={Radix sort em C++},captionpos=t]
#include <vector>
#include <algorithm>
using namespace std;

void countingSortRadix(vector<int>& arr, int exp) {
    int n = arr.size();
    vector<int> output(n);
    int count[10] = {0};

    for (int num : arr)
        count[(num / exp) % 10]++;

    for (int i = 1; i < 10; i++)
        count[i] += count[i - 1];

    for (int i = n - 1; i >= 0; i--) {
        int idx = (arr[i] / exp) % 10;
        output[count[idx] - 1] = arr[i];
        count[idx]--;
    }

    arr = output;
}

void radixSort(vector<int>& arr) {
    int m = *max_element(arr.begin(), arr.end());
    for (int exp = 1; m / exp > 0; exp *= 10)
        countingSortRadix(arr, exp);
}
\end{lstlisting}

\subsection{Análise de complexidade}

Nesta seção, analisamos formalmente as complexidades de tempo e espaço do \textit{LSD Radix Sort}. 
O algoritmo realiza $d$ iterações do \textit{Counting Sort}, uma para cada dígito do número mais longo, onde $d$ representa o número de dígitos e $k$ o tamanho do domínio dos dígitos (no caso decimal, $k = 10$).

\subsubsection{Complexidade de Tempo}

Cada chamada do \textit{Counting Sort} consome tempo $O(n + k)$, e o algoritmo o executa $d$ vezes.  
Assim, o tempo total de execução é dado por:

\[
T(n, d, k) = d \cdot O(n + k) = O(d \cdot (n + k)).
\]

\noindent{\textbf{Prova:}}

Seja $T_{CS}(n, k) = a_1n + a_2k + b$ o tempo de uma chamada do \textit{Counting Sort}.  
O \textit{LSD Radix Sort} realiza $d$ dessas chamadas, logo:

\[
T(n, d, k) = d \cdot (a_1n + a_2k + b) = a_1d n + a_2d k + d b.
\]

Portanto, existem constantes positivas $c_1, c_2$ tais que:

\[
T(n, d, k) \leq c_1 d (n + k),
\]
o que implica:

\[
T(n, d, k) \in O(d (n + k)).
\]
$\hfill\Box$

\bigskip

\noindent{\textbf{Discussão:}}  
Para números decimais, $k = 10$ é constante, e portanto a complexidade se reduz a:

\[
T(n, d) = O(d \cdot n).
\]
Se o número máximo de dígitos $d$ for proporcional a $\log_{10} M$ (onde $M$ é o maior valor da entrada), temos:

\[
T(n) = O(n \log M),
\]
o que torna o \textit{Radix Sort} mais eficiente que algoritmos de comparação ($O(n \log n)$) quando $d$ é pequeno.

\subsubsection{Complexidade de Espaço}

O algoritmo requer:

\begin{itemize}
    \item Um vetor de contagem $C[0 \ldots k-1]$ — espaço $O(k)$;
    \item Um vetor de saída $B[1 \ldots n]$ — espaço $O(n)$;
    \item O vetor de entrada $A[1 \ldots n]$ — espaço $O(n)$.
\end{itemize}

Assim, o espaço total utilizado é:

\[
S(n, k) = c_1n + c_2k + c_3,
\]
portanto:

\begin{equation}
S(n, k) \in O(n + k).
\end{equation}

\noindent{\textbf{Prova:}}  
Para constantes positivas $c_1, c_2, c_3$, temos:

\[
S(n, k) = c_1n + c_2k + c_3 \leq c(n + k),
\]
onde $c = \max(c_1, c_2) + c_3$.  
Logo, $S(n, k) \in O(n + k)$.
$\hfill\Box$

\bigskip

\noindent{\textbf{Discussão:}}  
O \textit{LSD Radix Sort} não é um algoritmo \textit{in-place}, pois utiliza vetores auxiliares.  
Entretanto, mantém a estabilidade e eficiência linear para domínios pequenos de dígitos, sendo amplamente empregado em aplicações numéricas e de processamento de grandes volumes de dados inteiros.

\section{Bucket sort (uniform keys)}
\subsection{Descrição e Funcionamento}
O \textit{Bucket Sort} é um algoritmo de ordenação baseado na distribuição dos elementos de entrada em vários \textit{baldes} (\textit{buckets}). 
Cada balde representa um intervalo do domínio das chaves, e os elementos são distribuídos entre eles de acordo com o valor de suas chaves. 
Posteriormente, cada balde é ordenado individualmente utilizando um algoritmo de ordenação simples — comumente o \textit{Insertion Sort} — e, finalmente, os baldes são concatenados para formar o vetor ordenado final.

\medskip
O \textit{Bucket Sort} é particularmente eficiente quando os dados de entrada são \textbf{uniformemente distribuídos} sobre um intervalo contínuo, como $[0, 1)$, e o número de baldes é proporcional ao número de elementos. 
Nessas condições, o tempo médio de execução é linear.

\medskip
A seguir, apresenta-se um exemplo ilustrativo de execução do algoritmo.

\begin{exmp}
Considere o vetor $A = [0.78, 0.17, 0.39, 0.26, 0.72, 0.94, 0.21, 0.12, 0.23, 0.68]$. 
Deseja-se ordená-lo utilizando o \textit{Bucket Sort} com chaves uniformes em $[0,1)$ e $n = 10$ baldes.

\begin{enumerate}
    \item \textbf{Distribuição em baldes:}  
    Cada elemento $A[i]$ é colocado no balde correspondente ao índice $\lfloor n \cdot A[i] \rfloor$.
    Assim, temos:
    \[
    \begin{aligned}
    B_0 &= [0.12, 0.17, 0.21, 0.23, 0.26, 0.39],\\
    B_6 &= [0.68],\\
    B_7 &= [0.72, 0.78],\\
    B_9 &= [0.94].
    \end{aligned}
    \]
    Os demais baldes estão vazios.

    \item \textbf{Ordenação individual dos baldes:}  
    Cada balde é ordenado utilizando o \textit{Insertion Sort}.  
    Após a ordenação, os baldes ficam:
    \[
    B_0 = [0.12, 0.17, 0.21, 0.23, 0.26, 0.39], \quad
    B_6 = [0.68], \quad
    B_7 = [0.72, 0.78], \quad
    B_9 = [0.94].
    \]

    \item \textbf{Concatenação dos baldes:}  
    Finalmente, os baldes são concatenados em ordem crescente de índice, resultando no vetor ordenado:
    \[
    A = [0.12, 0.17, 0.21, 0.23, 0.26, 0.39, 0.68, 0.72, 0.78, 0.94].
    \]
\end{enumerate}
\end{exmp}

\medskip
O pseudocódigo correspondente é apresentado a seguir.

\begin{center}
\begin{minipage}{.9\linewidth}
\begin{algorithm}[H]
\DontPrintSemicolon
\hspace{-0.25cm}\textbf{bucketSort(values: array of float, n: integer)}

\For{$i \gets 0$ \KwTo $n-1$}{
    place values[i] into the corresponding bucket\;
}
\For{each bucket $b$}{
    sort $b$ using insertionSort\;
}
concatenate all buckets in order\;

\caption{Bucket sort (uniform keys)}
\label{lab:alg-bucketSort}
\end{algorithm}
\end{minipage}
\end{center}

\subsection{Implementações}
\begin{lstlisting}[language=Python,caption={Bucket sort em Python},captionpos=t]
def insertion_sort(arr):
    for i in range(1, len(arr)):
        key = arr[i]
        j = i-1
        while j >=0 and arr[j] > key:
            arr[j+1] = arr[j]
            j -= 1
        arr[j+1] = key
    return arr

def bucket_sort(arr):
    n = len(arr)
    buckets = [[] for _ in range(n)]
    for num in arr:
        index = int(n*num)
        buckets[index].append(num)
    for i in range(n):
        buckets[i] = insertion_sort(buckets[i])
    sorted_arr = []
    for bucket in buckets:
        sorted_arr.extend(bucket)
    return sorted_arr
\end{lstlisting}
\begin{lstlisting}[language=C,caption={Bucket sort em C},captionpos=t]
void insertionSort(float arr[], int n) {
    for(int i=1;i<n;i++){
        float key=arr[i];
        int j=i-1;
        while(j>=0 && arr[j]>key){
            arr[j+1]=arr[j];
            j--;
        }
        arr[j+1]=key;
    }
}

void bucketSort(float arr[], int n){
    float buckets[n][n]; int count[n]={0};
    for(int i=0;i<n;i++){
        int index = n*arr[i];
        buckets[index][count[index]++] = arr[i];
    }
    for(int i=0;i<n;i++)
        insertionSort(buckets[i], count[i]);
    int idx=0;
    for(int i=0;i<n;i++)
        for(int j=0;j<count[i];j++)
            arr[idx++] = buckets[i][j];
}
\end{lstlisting}
\begin{lstlisting}[language=C++,caption={Bucket sort em C++},captionpos=t]
#include <vector>
#include <algorithm>
using namespace std;

void insertionSort(vector<float>& arr) {
    for (int i = 1; i < arr.size(); i++) {
        float key = arr[i];
        int j = i - 1;
        while (j >= 0 && arr[j] > key) {
            arr[j + 1] = arr[j];
            j--;
        }
        arr[j + 1] = key;
    }
}

void bucketSort(vector<float>& arr) {
    int n = arr.size();
    vector<vector<float>> buckets(n);

    for (float num : arr) {
        int idx = n * num;
        buckets[idx].push_back(num);
    }

    for (auto& b : buckets)
        insertionSort(b);

    arr.clear();
    for (auto& b : buckets)
        arr.insert(arr.end(), b.begin(), b.end());
}
\end{lstlisting}

\subsection{Análise de complexidade}
Nesta seção, analisamos formalmente as complexidades de tempo e espaço do algoritmo \textit{Bucket Sort} para o caso de chaves uniformemente distribuídas.

\subsubsection{Complexidade de Tempo}

Seja $n$ o número de elementos do vetor de entrada e $B[0 \ldots n-1]$ os baldes criados.  
O algoritmo executa as seguintes etapas principais:

\begin{enumerate}
    \item Distribuição dos $n$ elementos entre os $n$ baldes — custo de $O(n)$.
    \item Ordenação individual dos elementos em cada balde usando \textit{Insertion Sort}.
    \item Concatenação dos baldes — custo de $O(n)$.
\end{enumerate}

\medskip
No caso médio, assumindo que as chaves são \textbf{uniformemente distribuídas} em $[0,1)$, o número esperado de elementos em cada balde é $1$.  
Assim, o custo esperado para ordenar cada balde é constante, e o custo total esperado é linear.

\begin{equation}
T_{\text{médio}}(n) = O(n)
\end{equation}

\noindent{\textbf{Prova:}}  
Seja $n_i$ o número de elementos no balde $i$, com $\sum_{i=0}^{n-1} n_i = n$.  
O custo do \textit{Insertion Sort} em cada balde é proporcional a $O(n_i^2)$.  
Logo, o tempo total é dado por:
\[
T(n) = O(n) + \sum_{i=0}^{n-1} O(n_i^2)
\]

Como os elementos são distribuídos uniformemente, o valor esperado de $n_i$ é $1$, e pela linearidade da expectativa:

\[
E[T(n)] = O(n) + n \cdot O(1^2) = O(n)
\]

Portanto,
\[
T_{\text{médio}}(n) \in O(n).
\]
$\hfill\Box$

\medskip
No \textbf{pior caso}, todos os elementos podem cair em um único balde, resultando na execução do \textit{Insertion Sort} sobre todos os elementos.

\begin{equation}
T_{pior}(n) = O(n^2)
\end{equation}

\noindent{\textbf{Prova:}}  
Se todos os elementos caírem no mesmo balde, teremos $n_0 = n$ e os demais $n_i = 0$.  
Assim,
\[
T(n) = O(n) + O(n_0^2) = O(n^2)
\]
$\hfill\Box$

\medskip
\noindent{\textbf{Discussão:}}  
O desempenho do \textit{Bucket Sort} depende fortemente da distribuição dos dados.  
Para distribuições uniformes, a complexidade é linear; para distribuições não uniformes, pode degradar para quadrática.

\subsubsection{Complexidade de Espaço}

O algoritmo utiliza as seguintes estruturas:

\begin{itemize}
    \item O vetor de entrada $A[1 \ldots n]$ — espaço $O(n)$.
    \item $n$ baldes, cada um armazenando parte dos elementos — espaço total $O(n)$.
\end{itemize}

Assim, o espaço total $S(n)$ é:

\[
S(n) = c_1\cdot n + c_2\cdot n + c_3
\]

para constantes positivas $c_1, c_2, c_3$.

\begin{equation}
S(n) \in O(n)
\end{equation}

\noindent{\textbf{Prova:}}  
Como cada elemento é copiado exatamente uma vez para um balde, o espaço adicional é linear no tamanho da entrada:
\[
S(n) = O(n) + O(n) = O(n)
\]
$\hfill\Box$

\medskip
\noindent{\textbf{Discussão:}}  
O \textit{Bucket Sort} não é um algoritmo \textit{in-place}, pois requer espaço adicional para os baldes.  
Entretanto, para dados uniformemente distribuídos, oferece uma excelente relação entre tempo e espaço, tornando-se um método eficiente e estável quando combinado com um algoritmo de ordenação estável, como o \textit{Insertion Sort}.

\section{Spreadsort}
\subsection{Descrição e Funcionamento}
O \textit{Spreadsort} é um algoritmo híbrido de ordenação que combina técnicas de ordenação por distribuição (como o \textit{Bucket Sort} e o \textit{Radix Sort}) com métodos de ordenação por comparação (como o \textit{Quicksort}). Seu objetivo é aproveitar o melhor dos dois mundos: a eficiência de algoritmos lineares em casos bem distribuídos e a robustez dos métodos baseados em comparação para casos adversos.

\medskip
A ideia central do \textit{Spreadsort} é dividir o conjunto de dados em múltiplos intervalos (\textit{buckets}) de acordo com os bits ou dígitos mais significativos das chaves. Dentro de cada \textit{bucket}, o algoritmo decide recursivamente se deve continuar subdividindo os dados (espalhando-os mais) ou aplicar diretamente um algoritmo de ordenação local (como o \textit{Insertion Sort} ou o \textit{std::sort} do C++). Essa decisão é baseada no tamanho e na dispersão dos elementos em cada intervalo.

\medskip
O \textit{Spreadsort} é especialmente eficiente para ordenar números de ponto flutuante (\texttt{float}) ou inteiros grandes, sendo capaz de atingir desempenho sublinear em relação a algoritmos puramente baseados em comparação.

\medskip
A seguir, apresenta-se um exemplo ilustrativo do funcionamento do algoritmo.

\begin{exmp}
Considere o vetor $A = [0.42, 0.32, 0.23, 0.52, 0.25, 0.47, 0.51]$.  
O objetivo é ordená-lo utilizando o \textit{Spreadsort}.

\begin{enumerate}
    \item \textbf{Espalhamento inicial:}  
    O algoritmo determina o intervalo mínimo e máximo de $A$:  
    \[
    \text{min} = 0.23, \quad \text{max} = 0.52.
    \]  
    Divide o intervalo em $b = 3$ \textit{buckets}, cada um cobrindo uma faixa de valores:
    \[
    B_1: [0.23, 0.33), \quad B_2: [0.33, 0.43), \quad B_3: [0.43, 0.53).
    \]
    
    \item \textbf{Distribuição dos elementos:}  
    Cada elemento de $A$ é colocado no \textit{bucket} correspondente:
    \[
    B_1 = [0.32, 0.23, 0.25], \quad B_2 = [0.42], \quad B_3 = [0.52, 0.47, 0.51].
    \]
    
    \item \textbf{Ordenação local:}  
    Cada \textit{bucket} é ordenado usando um método local (por exemplo, \textit{Insertion Sort}):
    \[
    B_1 = [0.23, 0.25, 0.32], \quad B_2 = [0.42], \quad B_3 = [0.47, 0.51, 0.52].
    \]
    
    \item \textbf{Concatenação dos resultados:}  
    Os \textit{buckets} ordenados são concatenados:
    \[
    A = [0.23, 0.25, 0.32, 0.42, 0.47, 0.51, 0.52].
    \]
\end{enumerate}
\end{exmp}

\medskip
O pseudocódigo correspondente é apresentado a seguir.

\begin{center}
\begin{minipage}{.9\linewidth}
\begin{algorithm}[H]
\DontPrintSemicolon
\hspace{-0.25cm}\textbf{spreadsort(values: array of int, n: integer)}

\If{$n \leq 1$}{
    \Return
}
Distribute $values$ into intervals based on most significant bits\;
\For{each interval $b$}{
    \textbf{spreadsort}($b$)\;
}
Concatenate all sorted intervals\;
\caption{Spreadsort}
\label{lab:alg-spreadsort}
\end{algorithm}
\end{minipage}
\end{center}

\vspace{1em}

\subsection{Implementações}
\begin{lstlisting}[language=Python,caption={Spreadsort em Python},captionpos=t]
def spreadsort(arr):
    if len(arr) <= 1:
        return arr
    min_val, max_val = min(arr), max(arr)
    if min_val == max_val:
        return arr
    pivot = (min_val + max_val) // 2
    left = [x for x in arr if x <= pivot]
    right = [x for x in arr if x > pivot]
    return spreadsort(left) + spreadsort(right)
\end{lstlisting}
\begin{lstlisting}[language=C,caption={Spreadsort em C},captionpos=t]
#include <stdio.h>

void spreadsort(int arr[], int left, int right) {
    if(right - left <= 1) return;
    int min = arr[left], max = arr[left];
    for(int i = left+1; i < right; i++) {
        if(arr[i] < min) min = arr[i];
        if(arr[i] > max) max = arr[i];
    }
    if(min == max) return;

    int pivot = (min + max) / 2;
    int i = left, j = right-1;
    while(i <= j) {
        while(arr[i] <= pivot) i++;
        while(arr[j] > pivot) j--;
        if(i < j) { int tmp = arr[i]; arr[i] = arr[j]; arr[j] = tmp; i++; j--; }
    }
    spreadsort(arr, left, i);
    spreadsort(arr, i, right);
}
\end{lstlisting}
\begin{lstlisting}[language=C++,caption={Spreadsort em C++},captionpos=t]
#include <vector>
#include <algorithm>
using namespace std;

void spreadsort(vector<int>& arr, int left, int right) {
    if(right - left <= 1) return;
    int min_val = *min_element(arr.begin() + left, arr.begin() + right);
    int max_val = *max_element(arr.begin() + left, arr.begin() + right);
    if(min_val == max_val) return;

    int pivot = (min_val + max_val)/2;
    int i = left, j = right-1;
    while(i <= j) {
        while(arr[i] <= pivot) i++;
        while(arr[j] > pivot) j--;
        if(i < j) swap(arr[i++], arr[j--]);
    }
    spreadsort(arr, left, i);
    spreadsort(arr, i, right);
}
\end{lstlisting}

\subsection{Análise de complexidade}
Nesta seção, analisamos formalmente as complexidades de tempo e espaço do algoritmo \textit{Spreadsort}.  
Sua análise é mais complexa que a de algoritmos puramente comparativos, pois depende da distribuição dos dados e do número de subdivisões (\textit{spreads}) realizadas.

\subsubsection{Complexidade de Tempo}
Em condições ideais, em que os elementos estão uniformemente distribuídos e a profundidade de espalhamento é limitada, o custo total do \textit{Spreadsort} pode ser aproximado por:

\begin{equation}
T(n) = O(n \log n / k) + O(k \cdot n_b)
\end{equation}

onde $k$ é o número médio de \textit{buckets} por nível e $n_b$ é o tamanho médio de cada \textit{bucket}.  
Assumindo que cada \textit{bucket} tem tamanho constante e que o espalhamento reduz adequadamente a dispersão, temos:

\begin{equation}
T(n) \approx O(n)
\end{equation}

\noindent{\textbf{Prova:}}  
Em cada nível de recursão, os elementos são distribuídos em $k$ \textit{buckets}, operação que custa $O(n)$.  
Se cada \textit{bucket} contiver em média $n/k$ elementos, a ordenação local de todos os baldes tem custo $O(k \cdot (n/k) \log(n/k)) = O(n \log(n/k))$.  
Como $k$ cresce aproximadamente com $\log n$, temos:

\[
T(n) = O(n \log(n / \log n)) = O(n \log n - n \log \log n)
\]

Ignorando termos de menor ordem, o tempo médio é $O(n \log n)$, e para distribuições uniformes (caso ideal), o espalhamento domina o custo, resultando em comportamento linear.

\begin{equation}
T_{\text{médio}}(n) \in O(n)
\end{equation}

No pior caso, quando todos os elementos caem em um único \textit{bucket}, a ordenação interna (por comparação) é utilizada, resultando em:

\begin{equation}
T_{\text{pior}}(n) \in O(n \log n)
\end{equation}
$\hfill\Box$

\bigskip

\noindent{\textbf{Discussão:}}  
O \textit{Spreadsort} combina as vantagens do \textit{Bucket Sort} (linearidade média) com a segurança do \textit{Quicksort} em casos adversos.  
Assim, ele é linear na prática para dados bem distribuídos e possui limite superior $O(n \log n)$, tornando-o adequado para aplicações genéricas de alto desempenho.

\subsubsection{Complexidade de Espaço}
O \textit{Spreadsort} requer memória adicional para armazenar os \textit{buckets} e as listas temporárias de distribuição.  
Seja $b$ o número de \textit{buckets}. Cada elemento pertence a exatamente um \textit{bucket}, logo o espaço total é:

\[
S(n, b) = O(n + b)
\]

\begin{equation}
S(n) \in O(n)
\end{equation}

\noindent{\textbf{Prova:}}  
Como cada elemento é copiado apenas uma vez durante o espalhamento, e cada \textit{bucket} requer um ponteiro e metadados constantes, temos:

\[
S(n,b) = c_1 n + c_2 b + c_3 \leq c(n + b)
\]

Para $b \leq n$, segue que $S(n,b) \leq 2cn = O(n)$.  
Assim, o espaço adicional é linear no tamanho da entrada.
$\hfill\Box$

\bigskip

\noindent{\textbf{Discussão:}}  
Apesar de não ser um algoritmo \textit{in-place}, o \textit{Spreadsort} mantém o uso de memória dentro de limites práticos e é estável quando aplicado sobre tipos numéricos.  
Sua eficiência o torna particularmente útil para grandes volumes de dados de ponto flutuante distribuídos de forma quase uniforme.







\section{Burstsort}
\subsection{Descrição e Funcionamento}

O \textit{Burstsort} é um algoritmo de ordenação baseado em \textit{tries} (árvores de prefixos) projetado para ordenar conjuntos de cadeias de caracteres de forma eficiente, especialmente quando o volume de dados é grande e os prefixos são compartilhados. Ele é uma variação otimizada de métodos de ordenação baseados em distribuição, como o \textit{Radix Sort}, e busca minimizar o custo de movimentação de strings longas durante o processo de ordenação.

\medskip
O funcionamento do \textit{Burstsort} pode ser resumido em três etapas principais:

\begin{enumerate}
    \item As cadeias de caracteres são inseridas em uma \textit{trie} (árvore prefixada). Cada nó interno da \textit{trie} contém ponteiros para nós filhos e, eventualmente, para um conjunto de \textit{baldes} (\textit{buckets}).
    \item Quando o número de cadeias em um \textit{bucket} excede um limite pré-definido (um “limite de estouro”), esse \textit{bucket} é “explodido” (\textit{burst}) — ou seja, ele é expandido em novos nós filhos na \textit{trie}, de modo que as cadeias de caracteres armazenadas sejam redistribuídas de acordo com seus próximos caracteres.
    \item Após a construção completa da \textit{trie}, cada \textit{bucket} é ordenado individualmente, geralmente usando um algoritmo simples como o \textit{Insertion Sort}. Por fim, as strings são concatenadas em ordem lexicográfica percorrendo a \textit{trie} em ordem pré-fixada.
\end{enumerate}

O \textit{Burstsort} apresenta excelente desempenho prático em conjuntos de dados reais contendo cadeias de caracteres longas e com prefixos compartilhados, pois explora a localidade de cache e reduz o custo de movimentação de dados.

\medskip
A seguir, é apresentado um exemplo ilustrativo de execução do algoritmo.

\begin{exmp}
Considere o conjunto de strings $S = [\texttt{bat}, \texttt{bar}, \texttt{barn}, \texttt{banana}, \texttt{bad}, \texttt{apple}]$. O objetivo é ordená-las utilizando o \textit{Burstsort}.

\begin{enumerate}
    \item \textbf{Construção da trie inicial:}  
    As strings são inseridas caractere a caractere na \textit{trie}. Cada caminho de raiz a folha representa um prefixo comum.

    \item \textbf{Distribuição em buckets:}  
    Os nós terminais acumulam as strings correspondentes. Suponha que cada \textit{bucket} possa conter no máximo 2 elementos. Quando um \textit{bucket} excede essa capacidade, ele é “explodido” e redistribuído em novos nós.  
    Por exemplo, o \textit{bucket} contendo \texttt{bat}, \texttt{bar}, e \texttt{barn} é dividido em subnós com base na próxima letra após o prefixo \texttt{ba}.

    \item \textbf{Ordenação local e concatenação:}  
    Cada \textit{bucket} é ordenado internamente usando \textit{Insertion Sort}. Em seguida, percorremos a \textit{trie} em ordem lexicográfica e concatenamos os resultados:
    \[
    [\texttt{apple}, \texttt{bad}, \texttt{banana}, \texttt{bar}, \texttt{barn}, \texttt{bat}]
    \]
\end{enumerate}
\end{exmp}

\medskip
O pseudocódigo correspondente é apresentado a seguir.


\begin{algorithm}[H]
\DontPrintSemicolon
\hspace{-0.25cm}\textbf{burstsort(words: array of string)}

\If{$|words| \leq 1$}{\Return}
Build a prefix trie\;
\For{each bucket $b$ in trie}{
    sort $b$ internally (e.g., insertion sort)\;
}
Concatenate buckets in lexicographic order\;
\caption{Burstsort}
\label{lab:alg-burstsort}
\end{algorithm}

\subsection{Implementações}
\begin{lstlisting}[language=Python,caption={Burstsort em Python},captionpos=t]
def burstsort(arr):
    buckets = [[] for _ in range(256)]
    for s in arr:
        buckets[ord(s[0])].append(s)
    sorted_arr = []
    for b in buckets:
        sorted_arr.extend(b)
    return sorted_arr
\end{lstlisting}
\begin{lstlisting}[language=C,caption={Burstsort em C},captionpos=t]
#include <stdio.h>
#include <stdlib.h>
#include <string.h>

#define MAX_STR 100
typedef struct Node {
    char* str;
    struct Node* next;
} Node;

void burstsort(char* arr[], int n) {
    Node* buckets[256] = {NULL};
    for(int i=0;i<n;i++){
        unsigned char c = (unsigned char)arr[i][0];
        Node* node = malloc(sizeof(Node));
        node->str = arr[i];
        node->next = buckets[c];
        buckets[c] = node;
    }
    int index = 0;
    for(int i=0;i<256;i++){
        Node* current = buckets[i];
        while(current){
            arr[index++] = current->str;
            Node* tmp = current;
            current = current->next;
            free(tmp);
        }
    }
}
\end{lstlisting}
\begin{lstlisting}[language=C++,caption={Burstsort em C++},captionpos=t]
#include <vector>
#include <string>
#include <list>
using namespace std;

void burstsort(vector<string>& arr) {
    list<string> buckets[256];
    for(const auto& s : arr) buckets[(unsigned char)s[0]].push_back(s);
    arr.clear();
    for(int i=0;i<256;i++)
        for(auto& s : buckets[i]) arr.push_back(s);
}
\end{lstlisting}

\subsection{Análise de complexidade}

Nesta seção, analisamos as complexidades de tempo e espaço do algoritmo \textit{Burstsort}, levando em consideração a construção da \textit{trie}, as operações de estouro (\textit{burst}) e a ordenação local em cada \textit{bucket}.

\subsubsection{Complexidade de Tempo}

Seja $n$ o número de strings, $m$ o comprimento médio das strings e $b$ o número médio de elementos por \textit{bucket}.

A complexidade do algoritmo pode ser decomposta em três componentes principais:

\begin{enumerate}
    \item \textbf{Construção da trie:}  
    Cada caractere é processado uma única vez durante a inserção, com custo $O(nm)$.

    \item \textbf{Operações de estouro (burst):}  
    Cada \textit{burst} redistribui no máximo $b$ elementos em novos nós, e cada string participa de um número limitado de \textit{bursts}. Assim, o custo total permanece proporcional a $O(nm)$.

    \item \textbf{Ordenação local dos buckets:}  
    Cada \textit{bucket} é ordenado usando \textit{Insertion Sort}, com custo $O(b^2)$ por \textit{bucket}. Considerando $\frac{n}{b}$ buckets em média, o custo total é $O(\frac{n}{b} \cdot b^2) = O(nb)$.
\end{enumerate}

Somando todas as etapas, temos:

\[
T(n, m, b) = O(nm) + O(nb)
\]

Como, na prática, $b$ é pequeno (e constante), o termo $O(nb)$ é dominado por $O(nm)$.

\begin{equation}
T(n, m, b) \in O(nm)
\end{equation}

\noindent{\textbf{Prova:}}

Sejam $a_1, a_2 > 0$ constantes. Então:

\[
T(n,m,b) = a_1nm + a_2nb \leq (a_1 + a_2b)m n
\]

Como $b$ é limitado por uma constante, existe $c > 0$ tal que $T(n,m,b) \leq c\, n m$ para todo $n,m \geq 0$. Assim,

\[
T(n,m,b) \in O(nm)
\]
$\hfill\Box$

\bigskip

\noindent{\textbf{Discussão:}}  
No caso médio, o \textit{Burstsort} apresenta desempenho superior a algoritmos baseados em comparação, atingindo tempos próximos de $O(n)$ para conjuntos de dados grandes com prefixos compartilhados. No pior caso (strings completamente distintas e longas), o tempo pode crescer até $O(nm)$.

\subsubsection{Complexidade de Espaço}

O algoritmo \textit{Burstsort} utiliza as seguintes estruturas:

\begin{itemize}
    \item A \textit{trie} de prefixos — ocupa $O(nm)$ no pior caso (quando não há prefixos compartilhados).
    \item Os \textit{buckets} — total de $O(n)$, pois cada string é armazenada exatamente uma vez.
    \item Estruturas auxiliares para ordenação local — $O(b)$ por \textit{bucket}, no total $O(n)$.
\end{itemize}

Assim, o espaço total $S(n,m)$ é dado por:

\[
S(n,m) = O(nm) + O(n)
\]

Como o termo $O(nm)$ domina o crescimento assintótico, temos:

\begin{equation}
S(n,m) \in O(nm)
\end{equation}

\noindent{\textbf{Prova:}}  
Existe uma constante $c > 0$ tal que:

\[
S(n,m) = c_1nm + c_2n \leq c(nm)
\]

para todo $n,m \geq 0$. Logo,

\[
S(n,m) \in O(nm)
\]
$\hfill\Box$

\bigskip

\noindent{\textbf{Discussão:}}  
O \textit{Burstsort} consome mais memória do que algoritmos baseados em comparação, pois precisa armazenar explicitamente a estrutura da \textit{trie}. No entanto, essa estrutura permite um desempenho significativamente melhor em dados com prefixos comuns e é altamente eficiente em arquiteturas com hierarquia de cache.







\section{Flashsort}
\subsection{Descrição e Funcionamento}
O \textit{Flashsort} é um algoritmo de ordenação extremamente eficiente para dados que possuem distribuição aproximadamente uniforme. 
Ele se baseia na ideia de classificar os elementos em \textit{classes} (ou \textit{buckets}) de acordo com a posição que cada elemento deveria ocupar na ordem final, e então aplicar uma fase de redistribuição e ordenação local dentro de cada classe.

O algoritmo é composto por três fases principais:

\begin{enumerate}
    \item \textbf{Classificação em classes:}  
    O vetor é dividido em $m$ classes, determinadas com base no valor mínimo ($A_{\min}$) e máximo ($A_{\max}$) dos elementos.  
    Cada elemento $A[i]$ é atribuído a uma classe $L[k]$ segundo a fórmula:
    \[
    k = \left\lfloor (m-1) \cdot \frac{A[i] - A_{\min}}{A_{\max} - A_{\min}} \right\rfloor + 1
    \]
    Cada classe armazena o número de elementos que nela pertencem.

    \item \textbf{Redistribuição (fase de permutação):}  
    Os elementos são rearranjados de forma que cada classe ocupe uma região contígua do vetor. Isso é feito percorrendo o vetor e movendo cada elemento para sua classe correta, semelhante a uma etapa de distribuição.

    \item \textbf{Ordenação local:}  
    Após a redistribuição, cada classe é ordenada individualmente utilizando um algoritmo de ordenação simples, como o \textit{Insertion Sort}.
\end{enumerate}

\medskip
A seguir, apresenta-se um exemplo ilustrativo da execução do algoritmo.

\begin{exmp}
Considere o vetor $A = [0.89, 0.12, 0.56, 0.34, 0.78, 0.15, 0.67, 0.43]$.  
Deseja-se ordená-lo utilizando o \textit{Flashsort} com $m = 4$ classes.

\begin{enumerate}
    \item \textbf{Determinação dos limites e classes:}  
    O valor mínimo é $A_{\min} = 0.12$ e o máximo é $A_{\max} = 0.89$.  
    O coeficiente de classificação é:
    \[
    c = \frac{(m-1)}{A_{\max} - A_{\min}} = \frac{3}{0.77} \approx 3.896.
    \]
    Cada elemento é classificado conforme:
    \[
    k = \lfloor c \cdot (A[i] - A_{\min}) \rfloor + 1.
    \]
    Assim, as classes são:
    \begin{align*}
        0.12, 0.15 &\rightarrow L_1 \\
        0.34, 0.43 &\rightarrow L_2 \\
        0.56, 0.67 &\rightarrow L_3 \\
        0.78, 0.89 &\rightarrow L_4
    \end{align*}

    \item \textbf{Redistribuição dos elementos:}  
    O vetor é reorganizado de modo que os elementos de cada classe fiquem próximos:
    \[
    A = [0.12, 0.15, 0.34, 0.43, 0.56, 0.67, 0.78, 0.89].
    \]

    \item \textbf{Ordenação local:}  
    Cada classe é ordenada individualmente (usando \textit{Insertion Sort}), mas neste caso já estão em ordem dentro de cada faixa.  
    Logo, o vetor final ordenado é:
    \[
    A = [0.12, 0.15, 0.34, 0.43, 0.56, 0.67, 0.78, 0.89].
    \]
\end{enumerate}
\end{exmp}

\medskip
O pseudocódigo correspondente é apresentado a seguir.

\begin{algorithm}[H]
\DontPrintSemicolon
\hspace{-0.25cm}\textbf{flashsort(values: array of int, n: integer)}

\If{$n \leq 1$}{\Return}
Determine number of classes $m$\;
Compute class positions for each element\;
Distribute elements into their classes\;
Sort each class locally\;
Concatenate all classes\;
\caption{Flashsort}
\label{lab:alg-flashsort}
\end{algorithm}

\subsection{Implementações}
\begin{lstlisting}[language=Python,caption={Flashsort em Python},captionpos=t]
def flashsort(arr):
    n = len(arr)
    m = int(0.45*n)
    L = [0]*m
    min_val, max_val = min(arr), max(arr)
    if min_val==max_val: return arr
    c1 = (m-1)/(max_val-min_val)
    for x in arr:
        k = int(c1*(x-min_val))
        L[k]+=1
    for i in range(1,m):
        L[i]+=L[i-1]
    count, j = 0, 0
    while count<n:
        while j>=len(L): L.pop()
        evicted = arr[j]
        while True:
            k = int(c1*(evicted-min_val))
            arr[L[k]-1], evicted = evicted, arr[L[k]-1]
            L[k]-=1
            count+=1
            if j==L[k]: break
        j+=1
    return arr
\end{lstlisting}
\begin{lstlisting}[language=C,caption={Flashsort em C},captionpos=t]
#include <stdio.h>

void flashsort(int arr[], int n) {
    int m = (int)(0.45*n);
    int L[m];
    for(int i=0;i<m;i++) L[i]=0;

    int min=arr[0], max=arr[0];
    for(int i=1;i<n;i++){
        if(arr[i]<min) min=arr[i];
        if(arr[i]>max) max=arr[i];
    }
    if(min==max) return;

    double c1 = (double)(m-1)/(max-min);
    for(int i=0;i<n;i++){
        int k = (int)(c1*(arr[i]-min));
        L[k]++;
    }
    for(int i=1;i<m;i++) L[i]+=L[i-1];

    int count=0,j=0,k=m-1;
    while(count<n){
        while(j>=L[k]) k--;
        int evicted=arr[j];
        while(j!=L[k]-1){
            k=(int)(c1*(evicted-min));
            int tmp=arr[L[k]-1];
            arr[L[k]-1]=evicted;
            evicted=tmp;
            L[k]--;
            count++;
        }
        count++;
        j++;
    }
}
\end{lstlisting}
\begin{lstlisting}[language=C++,caption={Flashsort em C++},captionpos=t]
#include <vector>
#include <algorithm>
using namespace std;

void flashsort(vector<int>& arr) {
    int n = arr.size();
    int m = int(0.45*n);
    vector<int> L(m,0);

    int min_val = *min_element(arr.begin(), arr.end());
    int max_val = *max_element(arr.begin(), arr.end());
    if(min_val==max_val) return;

    double c1 = (double)(m-1)/(max_val-min_val);
    for(int i=0;i<n;i++){
        int k=int(c1*(arr[i]-min_val));
        L[k]++;
    }
    for(int i=1;i<m;i++) L[i]+=L[i-1];

    int count=0, j=0, k=m-1;
    while(count<n){
        while(j>=L[k]) k--;
        int evicted=arr[j];
        while(j!=L[k]-1){
            k=int(c1*(evicted-min_val));
            swap(evicted, arr[L[k]-1]);
            L[k]--;
            count++;
        }
        count++;
        j++;
    }
}
\end{lstlisting}

\subsection{Análise de complexidade}
Nesta seção, analisamos formalmente as complexidades de tempo e espaço do algoritmo \textit{Flashsort}, considerando uma distribuição aproximadamente uniforme dos dados.

\subsubsection{Complexidade de Tempo}

O algoritmo realiza três fases principais:

\begin{enumerate}
    \item Classificação dos elementos em classes — $O(n)$.
    \item Redistribuição e movimentação dos elementos — $O(n)$.
    \item Ordenação local em cada classe — custo depende da distribuição, tipicamente $O(n)$ no caso médio (pois cada classe contém poucos elementos).
\end{enumerate}

Portanto, para uma distribuição uniforme, o tempo total é linear no número de elementos:

\begin{equation}
T(n) = O(n)
\end{equation}

\noindent{\textbf{Prova:}}

Seja $m = \alpha n$, com $0 < \alpha < 1$ constante.  
O custo de classificar e redistribuir é linear:
\[
T_1(n) + T_2(n) = a_1 n + a_2 n = O(n)
\]
A etapa de ordenação local usa \textit{Insertion Sort}, que tem complexidade $O(n_i^2)$ para uma classe com $n_i$ elementos.  
Somando para todas as classes:
\[
\sum_{i=1}^{m} O(n_i^2)
\]
Para dados uniformemente distribuídos, $n_i \approx \frac{n}{m}$, logo:
\[
\sum_{i=1}^{m} O\left(\frac{n^2}{m^2}\right) = m \cdot O\left(\frac{n^2}{m^2}\right) = O\left(\frac{n^2}{m}\right)
\]
Como $m = \alpha n$, temos:
\[
O\left(\frac{n^2}{m}\right) = O(n)
\]
Portanto, o custo total é linear.

No entanto, em casos não uniformes, algumas classes podem conter muitos elementos, levando a uma complexidade quadrática no pior caso:
\begin{equation}
T_{\text{worst}}(n) = O(n^2)
\end{equation}
$\hfill\Box$

\bigskip

\noindent{\textbf{Discussão:}}  
O \textit{Flashsort} é extremamente rápido para dados bem distribuídos, podendo superar algoritmos baseados em comparação como \textit{QuickSort}. Contudo, sua eficiência depende fortemente da uniformidade da distribuição das chaves.

\subsubsection{Complexidade de Espaço}

O algoritmo requer estruturas auxiliares para armazenar as contagens e limites de classes.

\begin{itemize}
    \item Vetor de entrada $A[1 \ldots n]$ — $O(n)$.
    \item Vetor de classes $L[1 \ldots m]$ — $O(m)$.
\end{itemize}

Logo, o espaço total é:

\[
S(n) = c_1 n + c_2 m + c_3
\]

Assumindo $m = \alpha n$, temos:

\begin{equation}
S(n) \in O(n)
\end{equation}

\noindent{\textbf{Prova:}}  
Para constantes positivas $c_1, c_2, c_3$, temos:
\[
S(n) = c_1n + c_2m + c_3 \leq c(n + m)
\]
Com $m = \alpha n$, segue que $S(n) \leq c'(n)$, para alguma constante $c' > 0$.  
Logo:
\[
S(n) \in O(n)
\]
$\hfill\Box$

\bigskip

\noindent{\textbf{Discussão:}}  
O \textit{Flashsort} requer espaço adicional proporcional ao tamanho da entrada, mas continua sendo eficiente em memória.  
Ele não é \textit{in-place}, embora a maior parte das operações ocorra dentro do próprio vetor.

\section{Postman Sort}
\subsection{Descrição e Funcionamento}
O \textit{Postman Sort} é um algoritmo de ordenação baseado no princípio da distribuição e coleta de elementos em \textit{baldes} (\textit{buckets}), de forma semelhante ao \textit{Bucket Sort}. O nome “Postman” é uma analogia ao trabalho do carteiro que distribui cartas em caixas postais (baldes) de acordo com um critério de classificação (por exemplo, o endereço), e depois recolhe as cartas em ordem para formar a sequência final ordenada.

\medskip
O algoritmo é projetado para ordenar números reais uniformemente distribuídos no intervalo $[0,1)$. Ele divide o intervalo em um conjunto de $n$ baldes (ou caixas), distribui os elementos de entrada nesses baldes com base em seus valores e, em seguida, ordena individualmente os elementos de cada balde utilizando um algoritmo de ordenação simples, como o \textit{Insertion Sort}. Finalmente, os baldes são concatenados em ordem crescente para obter o vetor ordenado.

\medskip
O \textit{Postman Sort} explora o fato de que, para distribuições uniformes, é provável que os elementos fiquem bem distribuídos entre os baldes, permitindo que o custo total da ordenação de cada balde seja pequeno, resultando em uma ordenação linear no caso médio.

\medskip
A seguir, apresenta-se um exemplo ilustrativo do funcionamento do algoritmo.

\begin{exmp}
Considere o vetor $A = [0.78, 0.17, 0.39, 0.26, 0.72, 0.94, 0.21, 0.12, 0.23, 0.68]$ com $n = 10$ elementos. O objetivo é ordená-lo utilizando o \textit{Postman Sort}.

\begin{enumerate}
    \item \textbf{Criação dos baldes:}  
    Criamos $n = 10$ baldes vazios, correspondentes aos intervalos:
    \[
    [0, 0.1), [0.1, 0.2), [0.2, 0.3), \ldots, [0.9, 1.0)
    \]

    \item \textbf{Distribuição dos elementos:}  
    Cada elemento $A[i]$ é colocado no balde correspondente ao intervalo que contém seu valor. Por exemplo:
    \[
    \begin{array}{ll}
    0.78 \rightarrow \text{balde } 7, & 0.17 \rightarrow \text{balde } 1, \\
    0.39 \rightarrow \text{balde } 3, & 0.26 \rightarrow \text{balde } 2, \\
    0.72 \rightarrow \text{balde } 7, & 0.94 \rightarrow \text{balde } 9, \\
    0.21 \rightarrow \text{balde } 2, & 0.12 \rightarrow \text{balde } 1, \\
    0.23 \rightarrow \text{balde } 2, & 0.68 \rightarrow \text{balde } 6. \\
    \end{array}
    \]

    \item \textbf{Ordenação individual dos baldes:}  
    Cada balde é ordenado individualmente usando o \textit{Insertion Sort}. Por exemplo:
    \[
    \text{balde } 1 = [0.17, 0.12] \rightarrow [0.12, 0.17]
    \]
    \[
    \text{balde } 2 = [0.26, 0.21, 0.23] \rightarrow [0.21, 0.23, 0.26]
    \]

    \item \textbf{Concatenação:}  
    Finalmente, os baldes são concatenados em ordem crescente, produzindo o vetor final:
    \[
    A = [0.12, 0.17, 0.21, 0.23, 0.26, 0.39, 0.68, 0.72, 0.78, 0.94].
    \]
\end{enumerate}
\end{exmp}

\medskip
O pseudocódigo correspondente é apresentado a seguir.

\begin{center}
\begin{minipage}{.9\linewidth}
\begin{algorithm}[H]
\DontPrintSemicolon
\hspace{-0.25cm}\textbf{postmanSort(values: array of int, n: integer, d: integer)}

\For{$i \gets 0$ \KwTo $d-1$}{
    Apply \textbf{Counting Sort} based on the $i$-th digit\;
}
\caption{Postman sort}
\label{lab:alg-postmansort}
\end{algorithm}
\end{minipage}
\end{center}

\subsection{Implementações}
\begin{lstlisting}[language=Python,caption={Postman Sort em Python},captionpos=t]
def postman_sort(arr, base=10):
    max_val = max(arr)
    exp = 1
    n = len(arr)
    output = [0]*n
    while max_val//exp > 0:
        count = [0]*base
        for num in arr:
            count[(num//exp)%base] += 1
        for i in range(1, base):
            count[i] += count[i-1]
        for i in reversed(range(n)):
            idx = (arr[i]//exp)%base
            output[count[idx]-1] = arr[i]
            count[idx] -= 1
        arr = output[:]
        exp *= base
    return arr
\end{lstlisting}
\begin{lstlisting}[language=C,caption={Postman Sort em C},captionpos=t]
#include <stdio.h>
#include <string.h>

void postmanSort(int arr[], int n, int base) {
    int max=arr[0]; for(int i=1;i<n;i++) if(arr[i]>max) max=arr[i];
    int exp=1;
    int output[n];
    while(max/exp>0){
        int count[base]; for(int i=0;i<base;i++) count[i]=0;
        for(int i=0;i<n;i++) count[(arr[i]/exp)%base]++;
        for(int i=1;i<base;i++) count[i]+=count[i-1];
        for(int i=n-1;i>=0;i--){
            output[count[(arr[i]/exp)%base]-1]=arr[i];
            count[(arr[i]/exp)%base]--;
        }
        for(int i=0;i<n;i++) arr[i]=output[i];
        exp*=base;
    }
}
\end{lstlisting}
\begin{lstlisting}[language=C++,caption={Postman Sort em C++},captionpos=t]
#include <vector>
using namespace std;

void postmanSort(vector<int>& arr, int base) {
    int n=arr.size();
    int max_val=*max_element(arr.begin(), arr.end());
    int exp=1;
    vector<int> output(n);
    while(max_val/exp>0){
        vector<int> count(base,0);
        for(int num: arr) count[(num/exp)%base]++;
        for(int i=1;i<base;i++) count[i]+=count[i-1];
        for(int i=n-1;i>=0;i--){
            output[count[(arr[i]/exp)%base]-1]=arr[i];
            count[(arr[i]/exp)%base]--;
        }
        arr=output;
        exp*=base;
    }
}
\end{lstlisting}

\subsection{Análise de complexidade}
Nesta seção, analisamos as complexidades de tempo e espaço do algoritmo \textit{Postman Sort}. Assim como o \textit{Bucket Sort}, seu desempenho depende fortemente da distribuição dos dados de entrada.

\subsubsection{Complexidade de Tempo}

Seja $n$ o número de elementos de entrada. O algoritmo realiza as seguintes etapas:

\begin{enumerate}
    \item Distribuição dos $n$ elementos nos $n$ baldes — custo $O(n)$.
    \item Ordenação de cada balde com \textit{Insertion Sort}.  
    Supondo uma distribuição uniforme, o tamanho esperado de cada balde é $1$, e o custo médio de ordenação de um balde é constante. Assim, o custo total esperado é $O(n)$.
    \item Concatenação dos baldes — custo $O(n)$.
\end{enumerate}

Portanto, no caso médio:

\[
T_{\text{médio}}(n) = c_1n + c_2n + c_3n = O(n)
\]

\begin{equation}
T_{\text{médio}}(n) \in O(n)
\end{equation}

\noindent{\textbf{Prova:}}  
Assumindo distribuição uniforme dos valores no intervalo $[0,1)$, a probabilidade de um elemento pertencer a um balde específico é $1/n$. Logo, o número esperado de elementos por balde é $1$, e o custo esperado de ordenação de cada balde é constante. Assim,
\[
E[T(n)] = O(n)
\]
$\hfill\Box$

\bigskip
No entanto, no pior caso (por exemplo, se todos os elementos caem no mesmo balde), o algoritmo reduz-se a um \textit{Insertion Sort}, com custo de $O(n^2)$.

\begin{equation}
T_{pior}(n) \in O(n^2)
\end{equation}

\noindent{\textbf{Discussão:}}  
Portanto, o desempenho do \textit{Postman Sort} é altamente dependente da distribuição dos dados. Para dados uniformemente distribuídos, ele é linear. Caso contrário, aproxima-se de algoritmos quadráticos.

\subsubsection{Complexidade de Espaço}

O algoritmo utiliza as seguintes estruturas:

\begin{itemize}
    \item O vetor de entrada $A[1 \ldots n]$ — espaço $O(n)$.
    \item $n$ baldes, representados como listas — espaço total $O(n)$.
\end{itemize}

O espaço total $S(n)$ pode ser expresso como:

\[
S(n) = c_1n + c_2n + c_3
\]

\begin{equation}
S(n) \in O(n)
\end{equation}

\noindent{\textbf{Prova:}}  
Para constantes positivas $c_1, c_2, c_3$, existe $c = c_1 + c_2 + c_3$ tal que:
\[
S(n) \leq c \cdot n
\]
Logo, $S(n) \in O(n)$.
$\hfill\Box$

\bigskip

\noindent{\textbf{Discussão:}}  
O \textit{Postman Sort} requer espaço adicional linear, pois utiliza um número de baldes proporcional ao tamanho da entrada. Assim, ele não é um algoritmo \textit{in-place}, mas pode ser eficiente e estável para dados contínuos e bem distribuídos.

\section{Bead Sort}
\subsection{Descrição e Funcionamento}
O \textit{Bead Sort}, também conhecido como \textit{Gravity Sort}, é um algoritmo de ordenação inspirado no comportamento físico de contas (ou esferas) deslizando sob a ação da gravidade. Ele é especialmente interessante por sua natureza não-comparativa e pelo fato de explorar uma analogia física: assim como contas em hastes verticais se reorganizam naturalmente por gravidade, os números em um conjunto podem ser ordenados conforme o mesmo princípio.

\medskip
O algoritmo é aplicável apenas a números inteiros não negativos, e sua lógica baseia-se em representar cada número como uma sequência de contas colocadas em hastes verticais. A “queda” das contas simula o processo de ordenação: as contas mais pesadas (ou em maior quantidade) acabam descendo mais, resultando na sequência ordenada.

\medskip
O procedimento pode ser resumido em três etapas:
\begin{enumerate}
    \item Representar cada número como um conjunto de contas em colunas.
    \item Deixar as contas “caírem”, ou seja, simular a ação da gravidade.
    \item Contar novamente as contas em cada linha para reconstruir o vetor ordenado.
\end{enumerate}

\medskip
A seguir, apresenta-se um exemplo ilustrativo de execução.

\begin{exmp}
Considere o vetor $A = [5, 3, 1, 7, 4]$. O objetivo é ordená-lo utilizando o \textit{Bead Sort}.

\begin{enumerate}
    \item \textbf{Representação inicial:}  
    Cada número é representado como uma sequência de contas (1s) em uma matriz binária, onde cada linha corresponde a um número e cada coluna a uma posição de conta:
    \[
    \begin{matrix}
    1 & 1 & 1 & 1 & 1 & 0 & 0 \\
    1 & 1 & 1 & 0 & 0 & 0 & 0 \\
    1 & 0 & 0 & 0 & 0 & 0 & 0 \\
    1 & 1 & 1 & 1 & 1 & 1 & 1 \\
    1 & 1 & 1 & 1 & 0 & 0 & 0 \\
    \end{matrix}
    \]

    \item \textbf{Simulação da gravidade:}  
    As contas “caem” verticalmente para a posição mais baixa disponível em cada coluna, redistribuindo-se conforme a força da gravidade:
    \[
    \begin{matrix}
    0 & 0 & 0 & 0 & 0 & 0 & 0 \\
    1 & 0 & 0 & 0 & 0 & 0 & 0 \\
    1 & 1 & 1 & 0 & 0 & 0 & 0 \\
    1 & 1 & 1 & 1 & 0 & 0 & 0 \\
    1 & 1 & 1 & 1 & 1 & 1 & 1 \\
    \end{matrix}
    \]

    \item \textbf{Leitura do resultado:}  
    Contamos o número de contas (1s) em cada linha, de baixo para cima:
    \[
    B = [7, 5, 4, 3, 1].
    \]
    Assim, obtemos o vetor ordenado em ordem crescente.
\end{enumerate}
\end{exmp}

\medskip
O pseudocódigo correspondente é apresentado a seguir.

\begin{center}
\begin{minipage}{.9\linewidth}
\begin{algorithm}[H]
\DontPrintSemicolon
\hspace{-0.25cm}\textbf{beadSort(values: array of int, n: integer)}

Create bead matrix (n rows, max columns)\;
Let beads fall by summing each column\;
Reconstruct sorted values from bead heights\;
\caption{Bead sort}
\label{lab:alg-beadsort}
\end{algorithm}
\end{minipage}
\end{center}

\subsection{Implementações}
\begin{lstlisting}[language=Python,caption={Bead Sort em Python},captionpos=t]
def bead_sort(arr):
    if any(x<0 for x in arr): raise ValueError("Apenas inteiros n\~ao-negativos")
    n = len(arr)
    max_val = max(arr)
    beads = [[0]*max_val for _ in range(n)]
    for i, num in enumerate(arr):
        for j in range(num):
            beads[i][j] = 1
    for j in range(max_val):
        sum_col = sum(beads[i][j] for i in range(n))
        for i in range(n):
            beads[i][j] = 1 if i >= n - sum_col else 0
    for i in range(n):
        arr[i] = sum(beads[i])
    return arr
\end{lstlisting}
\begin{lstlisting}[language=C,caption={Bead Sort em C},captionpos=t]
#include <stdio.h>
#include <stdlib.h>

void bead_sort(int arr[], int n) {
    int max=arr[0];
    for(int i=1;i<n;i++) if(arr[i]>max) max=arr[i];
    int **beads = malloc(n*sizeof(int*));
    for(int i=0;i<n;i++){
        beads[i] = calloc(max,sizeof(int));
        for(int j=0;j<arr[i];j++) beads[i][j]=1;
    }
    for(int j=0;j<max;j++){
        int sum=0;
        for(int i=0;i<n;i++) { sum+=beads[i][j]; beads[i][j]=0; }
        for(int i=n-sum;i<n;i++) beads[i][j]=1;
    }
    for(int i=0;i<n;i++){
        int count=0;
        for(int j=0;j<max;j++) if(beads[i][j]) count++;
        arr[i]=count;
        free(beads[i]);
    }
    free(beads);
}
\end{lstlisting}
\begin{lstlisting}[language=C++,caption={Bead Sort em C++},captionpos=t]
#include <vector>
using namespace std;

void bead_sort(vector<int>& arr) {
    int n = arr.size();
    int max_val = *max_element(arr.begin(), arr.end());
    vector<vector<int>> beads(n, vector<int>(max_val,0));

    for(int i=0;i<n;i++)
        for(int j=0;j<arr[i];j++)
            beads[i][j]=1;

    for(int j=0;j<max_val;j++){
        int sum=0;
        for(int i=0;i<n;i++){ sum+=beads[i][j]; beads[i][j]=0; }
        for(int i=n-sum;i<n;i++) beads[i][j]=1;
    }

    for(int i=0;i<n;i++){
        int count=0;
        for(int j=0;j<max_val;j++) if(beads[i][j]) count++;
        arr[i]=count;
    }
}
\end{lstlisting}

\subsection{Análise de complexidade}

Nesta seção, analisamos as complexidades de tempo e espaço do algoritmo \textit{Bead Sort}.  
Trata-se de um método não-comparativo que depende fortemente da magnitude dos valores dos elementos, pois a representação física exige uma matriz de tamanho proporcional ao maior valor da entrada.

\subsubsection{Complexidade de Tempo}

Seja $n$ o número de elementos e $m$ o maior valor no vetor de entrada.  
O algoritmo realiza as seguintes operações principais:

\begin{enumerate}
    \item Construção da matriz de contas — custo de $O(nm)$.
    \item Simulação da “queda” das contas — para cada coluna, realiza $O(n)$ operações, totalizando $O(nm)$.
    \item Reconstrução do vetor ordenado — novamente $O(nm)$ no pior caso.
\end{enumerate}

Assim, o tempo total de execução pode ser expresso como:

\[
T(n, m) = a_1nm + a_2nm + a_3nm + b
\]

onde $a_1$, $a_2$, $a_3$ e $b$ são constantes positivas.

\begin{equation}
T(n, m) \in O(nm)
\end{equation}

\noindent{\textbf{Prova:}}  
Sabemos que:
\[
T(n, m) = (a_1 + a_2 + a_3)nm + b \leq c \cdot nm
\]
para alguma constante $c = a_1 + a_2 + a_3 + b$.  
Portanto, existe $c > 0$ e $n_0 \geq 0$ tais que $T(n, m) \leq c \cdot nm$ para todo $n, m \geq n_0$.  
Logo,
\[
T(n, m) \in O(nm).
\]
$\hfill\Box$

\bigskip
\noindent{\textbf{Discussão:}}  
No melhor caso — quando os valores são pequenos e próximos entre si — o custo efetivo pode se aproximar de $O(n)$, mas em geral o algoritmo é ineficiente para entradas com grandes números, devido ao crescimento linear com o valor máximo $m$.  
Portanto, o \textit{Bead Sort} é eficiente apenas quando $m$ é pequeno e comparável a $n$.

\subsubsection{Complexidade de Espaço}

O algoritmo utiliza as seguintes estruturas principais:

\begin{itemize}
    \item O vetor de entrada $A[1 \ldots n]$ — espaço $O(n)$.
    \item A matriz de contas $beads[n][m]$ — espaço $O(nm)$.
\end{itemize}

O espaço total utilizado é, portanto:
\[
S(n, m) = c_1n + c_2nm + c_3
\]
onde $c_1$, $c_2$, e $c_3$ são constantes positivas.

\begin{equation}
S(n, m) \in O(nm)
\end{equation}

\noindent{\textbf{Prova:}}  
Temos:
\[
S(n, m) = c_1n + c_2nm + c_3 \leq c(nm)
\]
onde $c = c_1 + c_2 + c_3$.  
Logo, existe $c > 0$ e $n_0 \geq 0$ tais que $S(n, m) \leq c(nm)$ para $n, m \geq n_0$, implicando que
\[
S(n, m) \in O(nm).
\]
$\hfill\Box$

\bigskip
\noindent{\textbf{Discussão:}}  
O \textit{Bead Sort} consome grande quantidade de memória, proporcional ao produto do tamanho da entrada e do valor máximo presente.  
Portanto, apesar de conceitualmente interessante e visualmente intuitivo, é impraticável para grandes valores ou números não inteiros.  
Na prática, seu uso é limitado a contextos educacionais ou demonstrativos de analogias físicas de ordenação.

\section{Pigeonhole Sort}
\subsection{Descrição e Funcionamento}
O \textit{Pigeonhole Sort} é um algoritmo de ordenação não comparativo baseado no princípio do "princípio das gavetas" (\textit{pigeonhole principle}).  
Ele é especialmente eficiente quando os elementos de entrada estão distribuídos em um intervalo relativamente pequeno. O algoritmo funciona alocando um conjunto de “gavetas” (\textit{pigeonholes}) suficientes para cobrir o intervalo de valores dos elementos, colocando cada elemento na gaveta correspondente ao seu valor, e depois reconstruindo o vetor ordenado percorrendo as gavetas na ordem natural.

\medskip
O funcionamento pode ser descrito em três etapas principais:
\begin{enumerate}
    \item Determinar o valor mínimo e máximo do vetor de entrada.
    \item Criar uma série de gavetas para cada valor inteiro no intervalo e inserir cada elemento na gaveta correspondente.
    \item Percorrer as gavetas na ordem crescente, coletando os elementos para formar o vetor ordenado.
\end{enumerate}

\medskip
A seguir, apresenta-se um exemplo ilustrativo de execução.

\begin{exmp}
Considere o vetor $A = [8, 3, 2, 7, 4]$. O objetivo é ordená-lo utilizando o \textit{Pigeonhole Sort}.

\begin{enumerate}
    \item \textbf{Determinação do intervalo:}  
    O menor valor é $2$ e o maior valor é $8$. Portanto, precisamos de $8 - 2 + 1 = 7$ gavetas, uma para cada inteiro de $2$ a $8$.

    \item \textbf{Distribuição nas gavetas:}  
    Cada elemento é colocado na gaveta correspondente ao seu valor:
    \[
    \begin{array}{c|c}
    Gaveta & Elementos \\
    \hline
    2 & 2 \\
    3 & 3 \\
    4 & 4 \\
    5 & - \\
    6 & - \\
    7 & 7 \\
    8 & 8 \\
    \end{array}
    \]

    \item \textbf{Reconstrução do vetor ordenado:}  
    Percorremos as gavetas em ordem crescente e coletamos os elementos:
    \[
    B = [2, 3, 4, 7, 8].
    \]
\end{enumerate}
\end{exmp}

\medskip
O pseudocódigo correspondente é apresentado a seguir.

\begin{center}
\begin{minipage}{.9\linewidth}
\begin{algorithm}[H]
\DontPrintSemicolon
\hspace{-0.25cm}\textbf{pigeonholeSort(values: array of integer, n: integer)}

$min \gets$ minimum value in $values$\;
$max \gets$ maximum value in $values$\;
$range \gets max - min + 1$\;
\For{$i \gets 0$ \KwTo $range-1$}{
    $hole[i] \gets$ empty list\;
}
\For{$i \gets 0$ \KwTo $n-1$}{
    $hole[values[i] - min].append(values[i])$\;
}
$k \gets 0$\;
\For{$i \gets 0$ \KwTo $range-1$}{
    \For{each element $v$ in $hole[i]$}{
        $values[k] \gets v$\;
        $k \gets k + 1$\;
    }
}
\caption{Pigeonhole Sort}
\label{lab:alg-pigeonholeSort}
\end{algorithm}
\end{minipage}
\end{center}

\subsection{Implementações}
\begin{lstlisting}[language=Python, caption={Pigeonhole Sort em Python},captionpos=t, label=code:pigeonholesortPy]
def pigeonhole_sort(arr):
    if not arr: return []
    mi, ma = min(arr), max(arr)
    size = ma - mi + 1
    holes = [[] for _ in range(size)]
    for x in arr:
        holes[x - mi].append(x)
    i = 0
    for hole in holes:
        for x in hole:
            arr[i] = x
            i += 1
    return arr
\end{lstlisting}
\begin{lstlisting}[language=C, caption={Pigeonhole Sort em C},captionpos=t,label=code:pigeonholesortC]
#include <stdio.h>
#include <stdlib.h>

void pigeonholeSort(int arr[], int n) {
    int min = arr[0], max = arr[0];
    for (int i = 1; i < n; i++) {
        if (arr[i] < min) min = arr[i];
        if (arr[i] > max) max = arr[i];
    }
    int size = max - min + 1;
    int *holes = calloc(size, sizeof(int));
    for (int i = 0; i < n; i++)
        holes[arr[i] - min]++;
    int index = 0;
    for (int i = 0; i < size; i++) {
        while (holes[i]-- > 0)
            arr[index++] = i + min;
    }
    free(holes);
}
\end{lstlisting}
\begin{lstlisting}[language=C++, caption={Pigeonhole Sort em C++}, captionpos=t, label=code:pigeonholesortCPP]
#include <iostream>
#include <vector>

void pigeonholeSort(std::vector<int> &arr) {
    if (arr.empty()) return;
    
    int min = arr[0], max = arr[0];
    for (int x : arr) {
        if (x < min) min = x;
        if (x > max) max = x;
    }
    
    int range = max - min + 1;
    std::vector<std::vector<int>> holes(range);
    
    for (int x : arr) {
        holes[x - min].push_back(x);
    }
    
    int k = 0;
    for (auto &hole : holes) {
        for (int x : hole) {
            arr[k++] = x;
        }
    }
}
\end{lstlisting}

\subsection{Análise de complexidade}

O \textit{Pigeonhole Sort} utiliza uma abordagem não comparativa baseada na contagem de elementos por gaveta, similar ao \textit{Counting Sort}. A eficiência depende do tamanho do intervalo de valores dos elementos.

\subsubsection{Complexidade de Tempo}

Seja $n$ o número de elementos e $k$ o tamanho do intervalo ($k = maxVal - minVal + 1$).

As etapas do algoritmo têm os seguintes custos:

\begin{enumerate}
    \item Determinar $minVal$ e $maxVal$ — $O(n)$.
    \item Inicializar as gavetas — $O(k)$.
    \item Distribuir os elementos nas gavetas — $O(n)$.
    \item Reconstruir o vetor ordenado percorrendo as gavetas — $O(n + k)$.
\end{enumerate}

O tempo total de execução $T(n,k)$ pode ser expresso como:

\[
T(n,k) = a_1 n + a_2 k + a_3 n + a_4 (n+k) + b
\]

onde $a_1, a_2, a_3, a_4, b$ são constantes positivas.

\begin{equation}
T(n,k) \in O(n + k)
\end{equation}

\noindent{\textbf{Prova:}}  
Com $c = a_1 + a_2 + a_3 + a_4 + b$, temos:
\[
T(n,k) \leq c(n+k)
\]
para todo $n,k \geq 0$.  
Portanto,
\[
T(n,k) \in O(n+k).
\]
$\hfill\Box$

\bigskip
\noindent{\textbf{Discussão:}}  
Quando $k = O(n)$, o algoritmo é linear, $O(n)$.  
Se $k \gg n$, a complexidade torna-se dominada por $k$, tornando o algoritmo ineficiente para intervalos grandes em relação ao número de elementos.

\subsubsection{Complexidade de Espaço}

O algoritmo utiliza:

\begin{itemize}
    \item Vetor de entrada $A[1 \dots n]$ — $O(n)$.
    \item Gavetas $holes[0 \dots k-1]$ — $O(n+k)$, pois cada gaveta pode armazenar múltiplos elementos.
\end{itemize}

O espaço total $S(n,k)$ é:

\[
S(n,k) = c_1 n + c_2 k + c_3
\]

para constantes positivas $c_1, c_2, c_3$.

\begin{equation}
S(n,k) \in O(n + k)
\end{equation}

\noindent{\textbf{Prova:}}  
Temos:
\[
S(n,k) \leq c (n+k)
\]
para $c = c_1 + c_2 + c_3$, portanto:
\[
S(n,k) \in O(n+k).
\]
$\hfill\Box$

\bigskip
\noindent{\textbf{Discussão:}}  
\textit{Pigeonhole Sort} não é \textit{in-place}, pois requer memória adicional proporcional ao número de elementos e ao intervalo de valores.  
É adequado apenas quando o intervalo de valores é relativamente pequeno.

\section{Bucket Sort (whole keys)}
\subsection{Descrição e Funcionamento}
O \textit{Bucket Sort} para chaves inteiras (\textit{whole keys}) é um algoritmo de ordenação não comparativo que distribui os elementos de um vetor em diversos "baldes" (\textit{buckets}), cada um responsável por um intervalo específico de valores. Depois, cada balde é ordenado individualmente utilizando um algoritmo simples, como o \textit{Insertion Sort}, e os resultados são concatenados para formar o vetor final ordenado.

\medskip
Este método é particularmente eficiente quando os valores estão uniformemente distribuídos e o número de baldes é proporcional ao tamanho do vetor, garantindo que cada balde contenha poucos elementos em média. O \textit{Bucket Sort} é estável se o algoritmo utilizado para ordenar cada balde for estável.

\medskip
A seguir, apresenta-se um exemplo ilustrativo de execução.

\begin{exmp}
Considere o vetor $A = [0.78, 0.17, 0.39, 0.26, 0.72, 0.94, 0.21, 0.12, 0.23, 0.68]$ no intervalo $[0,1)$. O objetivo é ordená-lo utilizando o \textit{Bucket Sort}.

\begin{enumerate}
    \item \textbf{Criação dos baldes:}  
    Dividimos o intervalo $[0,1)$ em 10 baldes correspondentes a intervalos $[0.0,0.1), [0.1,0.2), \dots, [0.9,1.0)$. Inicialmente, todos os baldes estão vazios.

    \item \textbf{Distribuição dos elementos:}  
    Cada elemento de $A$ é colocado no balde correspondente:
    \[
    \text{buckets} = [[\,], [0.12, 0.17], [0.21, 0.23, 0.26], [0.39], [\,], [\,], [0.68, 0.72, 0.78], [0.94], [\,], [\,]]
    \]

    \item \textbf{Ordenação interna dos baldes:}  
    Cada balde é ordenado individualmente com \textit{Insertion Sort}:
    \[
    \text{buckets} = [[\,], [0.12, 0.17], [0.21, 0.23, 0.26], [0.39], [\,], [\,], [0.68, 0.72, 0.78], [0.94], [\,], [\,]]
    \]
    (neste exemplo, alguns baldes já estão ordenados)

    \item \textbf{Concatenação dos baldes:}  
    Finalmente, os elementos dos baldes são concatenados para formar o vetor ordenado:
    \[
    B = [0.12, 0.17, 0.21, 0.23, 0.26, 0.39, 0.68, 0.72, 0.78, 0.94]
    \]
\end{enumerate}
\end{exmp}

\medskip
O pseudocódigo correspondente é apresentado a seguir.

\begin{center}
\begin{minipage}{.9\linewidth}
\begin{algorithm}[H]
\DontPrintSemicolon
\hspace{-0.25cm}\textbf{bucketSort(values: array of integer, n: integer, b: integer)}

$min \gets$ minimum value in $values$\;
$max \gets$ maximum value in $values$\;
$range \gets max - min + 1$\;
$size \gets \lceil range / b \rceil$\;
\For{$i \gets 0$ \KwTo $b-1$}{
    $bucket[i] \gets$ empty list\;
}
\For{$i \gets 0$ \KwTo $n-1$}{
    $index \gets \lfloor (values[i] - min) / size \rfloor$\;
    \If{$index \ge b$}{
        $index \gets b-1$\;
    }
    $bucket[index].append(values[i])$\;
}
\For{$i \gets 0$ \KwTo $b-1$}{
    \text{insertionSort(bucket[i])}\;
}
$k \gets 0$\;
\For{$i \gets 0$ \KwTo $b-1$}{
    \For{each element $v$ in $bucket[i]$}{
        $values[k] \gets v$\;
        $k \gets k + 1$\;
    }
}
\caption{Bucket Sort (whole keys)}
\label{lab:alg-bucketSortWholeKeys}
\end{algorithm}
\end{minipage}
\end{center}

\subsection{Implementações}
\begin{lstlisting}[language=Python, caption={Bucket Sort (inteiros) em Python},,captionpos=t, label=code:bucketsortPy]
def bucket_sort_whole_keys(arr, b=10):
    if not arr:
        return []
    mi, ma = min(arr), max(arr)
    if mi == ma:
        return arr[:]
    size = (ma - mi + 1) / b
    buckets = [[] for _ in range(b)]
    for x in arr:
        idx = int((x - mi) / size)
        if idx == b:  # Caso extremo (maior valor)
            idx -= 1
        buckets[idx].append(x)
    # Ordena cada bucket (Insertion Sort)
    for bucket in buckets:
        for i in range(1, len(bucket)):
            key = bucket[i]
            j = i - 1
            while j >= 0 and bucket[j] > key:
                bucket[j + 1] = bucket[j]
                j -= 1
            bucket[j + 1] = key
    # Concatena todos os buckets
    out = []
    for bucket in buckets:
        out.extend(bucket)
    return out
\end{lstlisting}
\begin{lstlisting}[language=C, caption={Bucket Sort (inteiros) em C},,captionpos=t, label=code:bucketsortC]
#include <stdio.h>
#include <stdlib.h>

// Insertion Sort utilizado nos buckets
void insertionSort(int *arr, int n) {
    for (int i = 1; i < n; i++) {
        int key = arr[i], j = i - 1;
        while (j >= 0 && arr[j] > key) {
            arr[j + 1] = arr[j];
            j--;
        }
        arr[j + 1] = key;
    }
}

void bucketSortWholeKeys(int *arr, int n, int b) {
    int min = arr[0], max = arr[0];
    for (int i = 1; i < n; i++) {
        if (arr[i] < min) min = arr[i];
        if (arr[i] > max) max = arr[i];
    }
    if (min == max) return;
    int range = max - min + 1;
    int size = (range + b - 1) / b;
    int **buckets = malloc(b * sizeof(int*));
    int *counts = calloc(b, sizeof(int));
    for (int i = 0; i < b; i++)
        buckets[i] = malloc(n * sizeof(int));
    for (int i = 0; i < n; i++) {
        int idx = (arr[i] - min) / size;
        if (idx >= b) idx = b - 1;
        buckets[idx][counts[idx]++] = arr[i];
    }
    int idx = 0;
    for (int i = 0; i < b; i++) {
        if (counts[i] > 0) {
            insertionSort(buckets[i], counts[i]);
            for (int j = 0; j < counts[i]; j++)
                arr[idx++] = buckets[i][j];
        }
        free(buckets[i]);
    }
    free(buckets);
    free(counts);
}
\end{lstlisting}
\begin{lstlisting}[language=C++, caption={Bucket Sort (inteiros) em C++}, captionpos=t, label=code:bucketsortCPP]
#include <iostream>
#include <vector>
#include <algorithm> // para std::sort

void bucketSortWholeKeys(std::vector<int> &arr, int b) {
    if (arr.empty()) return;
    
    int min = arr[0], max = arr[0];
    for (int x : arr) {
        if (x < min) min = x;
        if (x > max) max = x;
    }
    if (min == max) return;
    
    int range = max - min + 1;
    int size = (range + b - 1) / b; // tamanho de cada bucket
    std::vector<std::vector<int>> buckets(b);
    
    for (int x : arr) {
        int idx = (x - min) / size;
        if (idx >= b) idx = b - 1;
        buckets[idx].push_back(x);
    }
    
    int k = 0;
    for (auto &bucket : buckets) {
        std::sort(bucket.begin(), bucket.end()); // ordena cada bucket
        for (int x : bucket) {
            arr[k++] = x;
        }
    }
}
\end{lstlisting}

\subsection{Análise de complexidade}
Nesta seção, analisamos formalmente as complexidades de tempo e espaço do algoritmo \textit{Bucket Sort} para chaves inteiras ou números em ponto flutuante uniformemente distribuídos.

\subsubsection{Complexidade de Tempo}

Seja $n$ o número de elementos do vetor de entrada.

\begin{enumerate}
    \item Distribuição dos elementos nos baldes — custo $O(n)$.
    \item Ordenação interna de cada balde com \textit{Insertion Sort} — no caso médio, cada balde contém $O(1)$ elementos, resultando em custo $O(n)$.
    \item Concatenação dos baldes — custo $O(n)$.
\end{enumerate}

Portanto, o tempo total de execução $T(n)$ no caso médio é:

\begin{equation}
T(n) \in O(n)
\end{equation}

\noindent{\textbf{Prova:}}  
Se os elementos estão uniformemente distribuídos, cada balde contém aproximadamente $n/m$ elementos, onde $m = n$ é o número de baldes. A ordenação interna de cada balde via \textit{Insertion Sort} custa $O((n/m)^2)$ por balde. Somando todos os baldes:
\[
\sum_{i=1}^{m} O((n/m)^2) = m \cdot O((n/m)^2) = O(n^2/m) = O(n)
\]
para $m = n$ baldes. As demais operações lineares não alteram a complexidade.

\bigskip
\noindent{\textbf{Discussão:}}  
No pior caso, se todos os elementos caírem em um único balde, a ordenação interna é $O(n^2)$, e o algoritmo perde eficiência em relação a métodos baseados em comparação ($O(n \log n)$).

\subsubsection{Complexidade de Espaço}

O algoritmo utiliza:

\begin{itemize}
    \item O vetor de entrada $A[1 \ldots n]$ — espaço $O(n)$.
    \item Estruturas de baldes, cada uma armazenando até $O(n)$ elementos no pior caso — espaço $O(n)$.
\end{itemize}

O espaço total $S(n)$ é:

\begin{equation}
S(n) \in O(n)
\end{equation}

\noindent{\textbf{Prova:}}  
Cada elemento é armazenado em exatamente um balde, portanto a memória auxiliar é proporcional a $n$. Logo, existe $c>0$ tal que
\[
S(n) \leq c \cdot n \implies S(n) \in O(n).
\]
$\hfill\Box$

\bigskip
\noindent{\textbf{Discussão:}}  
O \textit{Bucket Sort} não é \textit{in-place}, mas o espaço adicional é linear em relação ao número de elementos.  
O algoritmo é eficiente para dados uniformemente distribuídos e se torna impraticável caso os elementos se concentrem em poucos baldes.


\section{Leituras Complementares}
Para aprofundamento dos algoritmos lineares, recomendamos a leitura dos seguintes materiais:

\begin{itemize}
    \item \href{https://www.ic.unicamp.br/~ra063658/disciplinas/stco02_2025s1/sort_linear.pdf}{Sorting in Linear Time — Material de curso da UNICAMP.}

    \item \href{https://ocw.mit.edu/courses/6-006-introduction-to-algorithms-fall-2011/bf7d79105762bf79bbc0925438e1468a_MIT6_006F11_lec07.pdf}{Linear-time sorting — Apostila do MIT OpenCourseWare.}

    \item \href{https://iudatastructurescourse.github.io/course-web-page-fall-2024/lectures/sort-linear.html}{Sorting in linear time — Página de curso de estruturas de dados do IU Data Structures Course.}

    \item \href{http://personal.kent.edu/~amohamm4/daa-f2019/slides/ch4-2%20LinearTime%20Sorting.pdf}{Linear-time sorting — Slides do curso de Design e Análise de Algoritmos da Kent State University.}

    \item \href{https://www.dcc.fc.up.pt/~pribeiro/aulas/aed2425/slides/4_sorting.pdf}{Sorting Algorithms — Material da Faculdade de Ciências da Universidade do Porto.}
\end{itemize}
\chapter{Algoritmos que usam operações de comparação e têm complexidade de tempo $O(n\log n)$}

Neste capítulo apresentamos algoritmos de ordenação cujos números de operações de comparações são da forma $c \cdot n \log n$, onde $n$ é o número de elementos a serem ordenados no multiconjunto e $c$ é um número real positivo.  

\section{Merge Sort}

\textbf{Descrição:} O Merge Sort é um algoritmo de ordenação baseado na estratégia \textit{dividir para conquistar}. Ele divide recursivamente o vetor em duas metades, ordena cada metade e depois intercala as duas partes em um vetor ordenado. É estável, mas não é in-place, pois exige memória auxiliar proporcional a $n$.

\begin{exmp}
Considere ordenar o vetor $A = [38, 27, 43, 3, 9, 82, 10]$ com o \textit{Merge Sort}.

\begin{enumerate}
    \item O vetor é recursivamente dividido ao meio até que os subvetores tenham tamanho 1:  
    $[38, 27, 43, 3, 9, 82, 10] \to [38, 27, 43]$, $[3, 9, 82, 10]$, e assim por diante.
    
    \item Em seguida, os subvetores são intercalados em ordem crescente:  
    $[27, 38, 43]$ e $[3, 9, 10, 82]$.
    
    \item Finalmente, os resultados são mesclados em $[3, 9, 10, 27, 38, 43, 82]$.
\end{enumerate}
\end{exmp}

\begin{algorithm}[H]
\DontPrintSemicolon
\textbf{mergeSort(A: array, l: int, r: int)}\;
\If{$l < r$}{
    $m \gets (l+r)/2$\;
    mergeSort(A, l, m)\;
    mergeSort(A, m+1, r)\;
    merge(A, l, m, r)\;
}
\caption{Merge Sort}
\label{lab:alg-mergeSort}
\end{algorithm}

\begin{lstlisting}[language=C, caption={Implementação do Merge Sort em C}, label=code:mergeSort]
#include <stdio.h>

void merge(int arr[], int l, int m, int r) {
    int n1 = m - l + 1, n2 = r - m;
    int L[n1], R[n2];
    for (int i=0; i<n1; i++) L[i] = arr[l+i];
    for (int j=0; j<n2; j++) R[j] = arr[m+1+j];
    int i=0, j=0, k=l;
    while (i<n1 && j<n2)
        arr[k++] = (L[i]<=R[j]) ? L[i++] : R[j++];
    while (i<n1) arr[k++] = L[i++];
    while (j<n2) arr[k++] = R[j++];
}

void mergeSort(int arr[], int l, int r) {
    if (l < r) {
        int m = l+(r-l)/2;
        mergeSort(arr, l, m);
        mergeSort(arr, m+1, r);
        merge(arr, l, m, r);
    }
}
\end{lstlisting}

\begin{lstlisting}[language=C++,caption={Merge sort em C++},captionpos=t]
#include <vector>
using namespace std;

void merge(vector<int>& arr, int l, int m, int r) {
    vector<int> left(arr.begin() + l, arr.begin() + m + 1);
    vector<int> right(arr.begin() + m + 1, arr.begin() + r + 1);

    int i = 0, j = 0, k = l;
    while (i < left.size() && j < right.size()) {
        arr[k++] = (left[i] <= right[j]) ? left[i++] : right[j++];
    }
    while (i < left.size()) arr[k++] = left[i++];
    while (j < right.size()) arr[k++] = right[j++];
}

void mergeSort(vector<int>& arr, int l, int r) {
    if (l >= r) return;
    int m = l + (r - l) / 2;
    mergeSort(arr, l, m);
    mergeSort(arr, m + 1, r);
    merge(arr, l, m, r);
}
\end{lstlisting}

\begin{lstlisting}[language=Python, caption={Merge Sort em Python}, label=code:mergeSortPy]
def merge_sort(arr):
    if len(arr) > 1:
        mid = len(arr)//2
        L, R = arr[:mid], arr[mid:]
        merge_sort(L); merge_sort(R)
        i = j = k = 0
        while i < len(L) and j < len(R):
            if L[i] <= R[j]:
                arr[k] = L[i]; i += 1
            else:
                arr[k] = R[j]; j += 1
            k += 1
        while i < len(L): arr[k] = L[i]; i += 1; k += 1
        while j < len(R): arr[k] = R[j]; j += 1; k += 1
\end{lstlisting}

\subsection{Análise de complexidade do algoritmo \ref{lab:alg-mergeSort}}
A cada nível de recursão, o vetor é dividido em duas partes. A mesclagem (\textit{merge}) de dois subvetores de tamanho $n/2$ custa $O(n)$. O número de níveis da árvore de recursão é $\log n$. Logo:
\[
T(n) = n \log n
\]
\begin{itemize}
    \item Melhor caso: $O(n \log n)$
    \item Caso médio: $O(n \log n)$
    \item Pior caso: $O(n \log n)$
\end{itemize}
Espaço auxiliar: $O(n)$ devido ao vetor temporário.

\textcolor{blue}{Vejam os conteúdos nos links abaixo e entendam a história por trás da criação do algoritmo:}
\begin{itemize}
    \item 
      \href{https://compileralchemy.substack.com/p/merge-sort-and-its-early-history}{Merge Sort And It's Early History}
    \item 
      \href{https://compileralchemy.substack.com/p/merge-sort-and-its-early-history}{Merge sort (von Neumann)}
\end{itemize}


\section{Quicksort}

\textbf{Descrição:} O Quicksort também utiliza a técnica de \textit{dividir para conquistar}. O algoritmo escolhe um pivô, particiona o vetor em dois subvetores — um com elementos menores ou iguais ao pivô e outro com elementos maiores — e então ordena cada subvetor recursivamente. É rápido na prática, mas pode ter pior caso quadrático se os pivôs forem mal escolhidos.

\begin{exmp}
Considere ordenar o vetor $A = [10, 80, 30, 90, 40, 50, 70]$.  
Escolhendo o pivô como o último elemento ($70$), após a partição temos $[10, 30, 40, 50]$, $70$, $[80, 90]$.  
Recursivamente, o vetor será ordenado até se tornar $[10, 30, 40, 50, 70, 80, 90]$.
\end{exmp}

\begin{algorithm}[H]
\DontPrintSemicolon
\textbf{quickSort(A: array, low: int, high: int)}\;
\If{$low < high$}{
    $p \gets partition(A, low, high)$\;
    quickSort(A, low, p-1)\;
    quickSort(A, p+1, high)\;
}
\caption{Quicksort}
\label{lab:alg-quickSort}
\end{algorithm}

\begin{lstlisting}[language=C, caption={Implementação do Quicksort em C}, label=code:quickSort]
#include <stdio.h>

int partition(int arr[], int low, int high) {
    int pivot = arr[high], i = low-1;
    for (int j=low; j<high; j++) {
        if (arr[j] <= pivot) {
            i++;
            int tmp = arr[i]; arr[i] = arr[j]; arr[j] = tmp;
        }
    }
    int tmp = arr[i+1]; arr[i+1] = arr[high]; arr[high] = tmp;
    return i+1;
}

void quickSort(int arr[], int low, int high) {
    if (low < high) {
        int p = partition(arr, low, high);
        quickSort(arr, low, p-1);
        quickSort(arr, p+1, high);
    }
}
\end{lstlisting}

\begin{lstlisting}[language=C++,caption={Quick sort em C++},captionpos=t]
#include <vector>
using namespace std;

int partition(vector<int>& arr, int low, int high) {
    int pivot = arr[high];
    int i = low - 1;
    for (int j = low; j < high; j++) {
        if (arr[j] < pivot) {
            i++;
            swap(arr[i], arr[j]);
        }
    }
    swap(arr[i + 1], arr[high]);
    return i + 1;
}

void quickSort(vector<int>& arr, int low, int high) {
    if (low < high) {
        int pi = partition(arr, low, high);
        quickSort(arr, low, pi - 1);
        quickSort(arr, pi + 1, high);
    }
}
\end{lstlisting}


\begin{lstlisting}[language=Python, caption={Quicksort em Python}, label=code:quickSortPy]
def quicksort(arr):
    if len(arr) <= 1:
        return arr
    pivot = arr[-1]
    left  = [x for x in arr[:-1] if x <= pivot]
    right = [x for x in arr[:-1] if x > pivot]
    return quicksort(left) + [pivot] + quicksort(right)
\end{lstlisting}

\subsection{Análise de complexidade do algoritmo \ref{lab:alg-quickSort}}
No caso médio, a cada partição o vetor é dividido em duas partes quase iguais, e o custo total é:
\[
T(n) = n \log n
\]
\begin{itemize}
    \item Melhor caso: $O(n \log n)$
    \item Caso médio: $O(n \log n)$
    \item Pior caso: $O(n^2)$ (quando sempre escolhe o pior pivô, por exemplo vetor já ordenado).
\end{itemize}
Espaço auxiliar: $O(\log n)$ devido à pilha de recursão.

---


\section{Heapsort}

\textbf{Descrição:} O Heapsort é baseado na estrutura de dados heap (mais especificamente a \textit{max-heap}). O algoritmo primeiro constrói a heap a partir do vetor de entrada e, em seguida, extrai repetidamente o maior elemento, reconstruindo a heap a cada extração, até que todos os elementos estejam ordenados. É in-place e possui complexidade $O(n \log n)$ em todos os casos, mas não é estável.


\begin{exmp}
Para o vetor $A = [4, 10, 3, 5, 1]$, a construção da heap resulta em $[10, 5, 3, 4, 1]$.  
Extraindo sucessivamente o maior elemento e ajustando a heap, obtemos a ordenação final $[1, 3, 4, 5, 10]$.
\end{exmp}

\begin{algorithm}[H]
\DontPrintSemicolon
\textbf{heapSort(A: array, n: int)}\;
construirMaxHeap(A)\;
\For{$i \gets n-1$ \KwTo $1$}{
    trocar $A[0]$ com $A[i]$\;
    reduzir tamanho da heap em 1\;
    maxHeapify(A, 0)\;
}
\caption{Heapsort}
\label{lab:alg-heapSort}
\end{algorithm}

\begin{lstlisting}[language=C, caption={Implementação do Heapsort em C}, label=code:heapSort]
#include <stdio.h>

void heapify(int arr[], int n, int i) {
    int largest = i, l = 2*i+1, r = 2*i+2;
    if (l<n && arr[l]>arr[largest]) largest = l;
    if (r<n && arr[r]>arr[largest]) largest = r;
    if (largest != i) {
        int tmp = arr[i]; arr[i] = arr[largest]; arr[largest] = tmp;
        heapify(arr, n, largest);
    }
}

void heapSort(int arr[], int n) {
    for (int i=n/2-1; i>=0; i--) heapify(arr, n, i);
    for (int i=n-1; i>=0; i--) {
        int tmp = arr[0]; arr[0] = arr[i]; arr[i] = tmp;
        heapify(arr, i, 0);
    }
}
\end{lstlisting}

\begin{lstlisting}[language=C++,caption={Heap sort em C++},captionpos=t]
#include <vector>
using namespace std;

void heapify(vector<int>& arr, int n, int i) {
    int largest = i;
    int l = 2 * i + 1;
    int r = 2 * i + 2;
    if (l < n && arr[l] > arr[largest]) largest = l;
    if (r < n && arr[r] > arr[largest]) largest = r;
    if (largest != i) {
        swap(arr[i], arr[largest]);
        heapify(arr, n, largest);
    }
}

void heapSort(vector<int>& arr) {
    int n = arr.size();
    for (int i = n / 2 - 1; i >= 0; i--)
        heapify(arr, n, i);
    for (int i = n - 1; i > 0; i--) {
        swap(arr[0], arr[i]);
        heapify(arr, i, 0);
    }
}
\end{lstlisting}

\begin{lstlisting}[language=Python, caption={Heapsort em Python}, label=code:heapSortPy]
def heapify(arr, n, i):
    largest = i; l, r = 2*i+1, 2*i+2
    if l < n and arr[l] > arr[largest]: largest = l
    if r < n and arr[r] > arr[largest]: largest = r
    if largest != i:
        arr[i], arr[largest] = arr[largest], arr[i]
        heapify(arr, n, largest)

def heapsort(arr):
    n = len(arr)
    for i in range(n//2-1, -1, -1): heapify(arr, n, i)
    for i in range(n-1, 0, -1):
        arr[0], arr[i] = arr[i], arr[0]
        heapify(arr, i, 0)
\end{lstlisting}

\subsection{Análise de complexidade do algoritmo \ref{lab:alg-heapSort}}
A construção da heap custa $O(n)$. Cada remoção do máximo exige $O(\log n)$ para reequilibrar a heap. Como são feitas $n$ remoções:
\[
T(n) = O(n \log n)
\]
\begin{itemize}
    \item Melhor caso: $O(n \log n)$
    \item Caso médio: $O(n \log n)$
    \item Pior caso: $O(n \log n)$
\end{itemize}
Espaço auxiliar: $O(1)$ (in-place).

---


\section{Introsort}

\textbf{Descrição:} O Introsort combina Quicksort, Heapsort e Insertion Sort. Ele começa como um Quicksort, mas monitora a profundidade da recursão; se ultrapassar um limite (tipicamente $2 \log n$), muda para Heapsort, garantindo $O(n \log n)$ no pior caso. Para subvetores pequenos, usa Insertion Sort. É utilizado em bibliotecas padrão como C++ STL.

\begin{exmp}
O Introsort começa como um Quicksort. Caso a profundidade de recursão ultrapasse um limite (tipicamente $2\log n$), ele muda para Heapsort. Em subvetores pequenos, pode usar Insertion Sort. Assim, o Introsort combina a velocidade média do Quicksort com a garantia de $O(n \log n)$ do Heapsort.
\end{exmp}

\begin{algorithm}[H]
\DontPrintSemicolon
\textbf{introSort(A: array, n: int)}\;
profundidadeMax $\gets 2 \cdot \lfloor \log n \rfloor$\;
introSortRec(A, 0, n-1, profundidadeMax)\;
\caption{Introsort}
\label{lab:alg-introSort}
\end{algorithm}

\begin{lstlisting}[language=C, caption={Implementação do Introsort em C}, label=code:introSortC]
#include <stdio.h>
#include <math.h>

void insertionSort(int arr[], int l, int r) { /* ... */ }
void heapify(int arr[], int n, int i) { /* ... */ }
void heapSort(int arr[], int l, int r) { /* ... */ }
int partition(int arr[], int l, int r) { /* ... */ }

void introsortRec(int arr[], int l, int r, int depthLimit) { /* ... */ }
void introSort(int arr[], int n) {
    int depthLimit = 2 * log(n);
    introsortRec(arr, 0, n-1, depthLimit);
}
\end{lstlisting}

\begin{lstlisting}[language=C++, caption={Implementação do Introsort em C++}, label=code:introSortCpp]
#include <vector>
#include <cmath>
#include <algorithm>
using namespace std;

void insertionSort(vector<int>& arr, int l, int r) { /* ... */ }
void heapSort(vector<int>& arr, int l, int r) { /* ... */ }
int partition(vector<int>& arr, int l, int r) { /* ... */ }

void introsortRec(vector<int>& arr, int l, int r, int depthLimit) { /* ... */ }
void introSort(vector<int>& arr) {
    int depthLimit = 2 * log(arr.size());
    introsortRec(arr, 0, arr.size()-1, depthLimit);
}
\end{lstlisting}

\begin{lstlisting}[language=Python, caption={Implementação do Introsort em Python}, label=code:introSortPy]
import math

def insertion_sort(arr, l, r): ...
def heap_sort(arr): ...
def partition(arr, l, r): ...

def introsort_rec(arr, l, r, depthLimit): ...
def introsort(arr):
    depthLimit = 2 * int(math.log2(len(arr)))
    introsort_rec(arr, 0, len(arr)-1, depthLimit)
\end{lstlisting}

\subsection{Análise de complexidade do algoritmo \ref{lab:alg-introSort}}
\begin{itemize}
    \item Melhor caso: $O(n \log n)$ (como Quicksort eficiente).
    \item Caso médio: $O(n \log n)$.
    \item Pior caso: $O(n \log n)$ (garantido pela troca para Heapsort).
\end{itemize}
Espaço auxiliar: $O(\log n)$ devido à recursão.

---


\section{Timsort}

\href{https://www.algowalker.com/tim-sort.html}{Veja Tim sort}

\textbf{Descrição:} O Timsort é um algoritmo híbrido que combina Insertion Sort e Merge Sort. Ele foi projetado para lidar bem com dados parcialmente ordenados. O vetor é dividido em \textit{runs} (sequências já ordenadas), que são refinadas por Insertion Sort (se pequenas) e depois mescladas por Merge Sort. É o algoritmo padrão em linguagens como Python e Java.

\begin{exmp}
O Timsort divide o vetor em \textit{runs} (subvetores já ordenados). Cada run é ordenada por Insertion Sort (se pequena) e então as runs são mescladas por Merge Sort.  
Por exemplo, o vetor $[5, 21, 7, 23, 19]$ gera runs $[5, 21]$, $[7, 23]$, $[19]$, que são ordenadas e mescladas até formar $[5, 7, 19, 21, 23]$.
\end{exmp}

\begin{algorithm}[H]
\DontPrintSemicolon
\textbf{timSort(A: array, n: int)}\;
dividir A em runs de tamanho fixo\;
ordenar cada run com insertionSort\;
mesclar runs sucessivamente com merge\;
\caption{Timsort}
\label{lab:alg-timSort}
\end{algorithm}

\begin{lstlisting}[language=C, caption={Timsort simplificado em C}, label=code:timSortC]
#include <stdio.h>
#include <stdlib.h>

#define RUN 32

void insertionSort(int arr[], int left, int right) {
    for(int i = left+1; i <= right; i++) {
        int key = arr[i], j = i-1;
        while(j >= left && arr[j] > key) {
            arr[j+1] = arr[j]; j--;
        }
        arr[j+1] = key;
    }
}

void merge(int arr[], int l, int m, int r) {
    int n1 = m-l, n2 = r-m+1;
    int *L = (int*)malloc(n1*sizeof(int));
    int *R = (int*)malloc(n2*sizeof(int));
    for(int i=0;i<n1;i++) L[i]=arr[l+i];
    for(int i=0;i<n2;i++) R[i]=arr[m+i];
    int i=0,j=0,k=l;
    while(i<n1 && j<n2) arr[k++] = (L[i]<=R[j])?L[i++]:R[j++];
    while(i<n1) arr[k++]=L[i++];
    while(j<n2) arr[k++]=R[j++];
    free(L); free(R);
}

void timsort(int arr[], int n) {
    for(int i=0;i<n;i+=RUN)
        insertionSort(arr, i, (i+RUN-1<n)?i+RUN-1:n-1);

    for(int size=RUN; size<n; size*=2) {
        for(int left=0; left<n; left+=2*size) {
            int mid = left+size-1;
            int right = (left+2*size-1<n)?left+2*size-1:n-1;
            if(mid < right) merge(arr, left, mid+1, right);
        }
    }
}
\end{lstlisting}

\begin{lstlisting}[language=C++, caption={Timsort simplificado em C++}, label=code:timSortCpp]
#include <vector>
#include <algorithm>
using namespace std;

#define RUN 32

void insertionSort(vector<int>& arr, int left, int right) {
    for(int i=left+1;i<=right;i++){
        int key=arr[i], j=i-1;
        while(j>=left && arr[j]>key){ arr[j+1]=arr[j]; j--; }
        arr[j+1]=key;
    }
}

vector<int> merge(const vector<int>& left, const vector<int>& right){
    vector<int> result;
    int i=0,j=0;
    while(i<left.size() && j<right.size()){
        if(left[i]<=right[j]) result.push_back(left[i++]);
        else result.push_back(right[j++]);
    }
    while(i<left.size()) result.push_back(left[i++]);
    while(j<right.size()) result.push_back(right[j++]);
    return result;
}

void timsort(vector<int>& arr){
    int n = arr.size();
    for(int i=0;i<n;i+=RUN) insertionSort(arr, i, min(i+RUN-1,n-1));
    for(int size=RUN; size<n; size*=2){
        for(int left=0; left<n; left+=2*size){
            int mid = left+size;
            int right = min(left+2*size, n);
            if(mid<right){
                vector<int> merged = merge(vector<int>(arr.begin()+left, arr.begin()+mid),
                                           vector<int>(arr.begin()+mid, arr.begin()+right));
                copy(merged.begin(), merged.end(), arr.begin()+left);
            }
        }
    }
}
\end{lstlisting}

\begin{lstlisting}[language=Python, caption={Timsort simplificado em Python}, label=code:timSortPy]
RUN = 32

def insertion_sort(arr, l=0, r=None):
    if r is None: r = len(arr)-1
    for i in range(l+1, r+1):
        key = arr[i]
        j = i-1
        while j>=l and arr[j]>key:
            arr[j+1] = arr[j]; j -= 1
        arr[j+1] = key

def merge(left, right):
    result=[]; i=j=0
    while i<len(left) and j<len(right):
        if left[i]<=right[j]: result.append(left[i]); i+=1
        else: result.append(right[j]); j+=1
    result.extend(left[i:]); result.extend(right[j:])
    return result

def timsort(arr):
    n=len(arr)
    for i in range(0,n,RUN): insertion_sort(arr,i,min(i+RUN-1,n-1))
    size=RUN
    while size<n:
        for left in range(0,n,2*size):
            mid = left+size
            right = min(left+2*size,n)
            if mid<right:
                merged = merge(arr[left:mid], arr[mid:right])
                arr[left:left+len(merged)] = merged
        size*=2
\end{lstlisting}

\subsection{Análise de complexidade do algoritmo \ref{lab:alg-timSort}}
\begin{itemize}
    \item Melhor caso: $O(n)$, quando o vetor já está quase ordenado (runs grandes).
    \item Caso médio: $O(n \log n)$.
    \item Pior caso: $O(n \log n)$.
\end{itemize}
Espaço auxiliar: $O(n)$, pois depende de vetores temporários na mesclagem.

\section{Tournament sort}

\section{Tree sort}

\section{Library sort}

\section{Shellsort}

\section{Cube Sort}

\section{Tree Sort}




\section{Resumo}

\begin{center}
\begin{tabular}{||c|c|c|c||}
\hline
\multicolumn{4}{|c|}{Complexidades de tempo em termos de comparações} \\\hline
Algoritmo & Pior caso & Melhor caso & Caso médio\\
\hline
Merge Sort & $O(n\log n)$ & $O(n\log n)$ & $O(n\log n)$ \\
Quick Sort & $O(n^2)$    & $O(n\log n)$ & $O(n\log n)$ \\
Heap Sort  & $O(n\log n)$ & $O(n\log n)$ & $O(n\log n)$ \\
Intro Sort & $O(n\log n)$ & $O(n\log n)$ & $O(n\log n)$ \\
TimSort    & $O(n\log n)$ & $O(n)$       & $O(n\log n)$ \\\hline
\end{tabular}
\end{center}

\begin{center}
\begin{tabular}{||c|c|c|c||}
\hline
\multicolumn{4}{|c|}{Complexidades de espaço} \\\hline
Algoritmo & Pior caso & Melhor caso & Caso médio\\
\hline
Merge Sort & $O(n)$      & $O(n)$      & $O(n)$ \\
Quick Sort & $O(\log n)$ & $O(\log n)$ & $O(\log n)$ \\
Heap Sort  & $O(1)$      & $O(1)$      & $O(1)$ \\
Intro Sort & $O(\log n)$ & $O(\log n)$ & $O(\log n)$ \\
TimSort    & $O(n)$      & $O(n)$      & $O(n)$ \\\hline
\end{tabular}
\end{center}
 
\chapter{Algoritmos Miscelâneos}
Neste capítulo apresentamos algoritmos cujos complexidades de tempo não se enquadram nas categorias já exploradas, isto é, possuem complexidades de tempo diferente de linear, log linear ou quadrática. Dentre os algoritmos listados, é possível notar que alguns desses foram desenvolvidos para fomentar as discussões sobre o assunto de análise de algoritmos, inclusive de forma satírica. Um exemplo deles é o algoritmo Bogosort que tem complexidade de tempo de pior caso tendendo ao infinito, uma vez que seu funcionamento baseia-se em gerar permutações aleatórias do vetor até que ele esteja ordenado.  

Estes algoritmos têm  como características a utilização de métodos ineficientes ou ainda etapas que repetem ações, o que contribui para o aumento de suas complexidades de tempo.

\section{Bogosort}

\href{https://sortvisualizer.com/bogosort/}{Veja Bogosort}\\
\textbf{Descrição:} Bogo Sort,também conhecido como Stupid Sort, é um algoritmo de ordenação iterativo particularmente ineficiente. Seu funcionamento se baseia em embaralhar aleatoriamente os elementos da estrutura de dados e, em seguida, verificar se estão ordenados corretamente. Se sim, o objetivo foi alcançado, caso contrário, repita o processo.
Logo, é um algoritmo probabilístico. A quantidade de permutações possíveis de uma estrutura de dados de n elementos é n!, assim serão necessárias, em média , n! embaralhamentos para chegar à solução. Cada embaralhamento requer n operações, então o número médio total de operações é n × n!\\
Como seu desempenho depende inteiramente da probabilidade, a pior complexidade do caso não é mensurável.

\begin{algorithm}[H]
\DontPrintSemicolon
\small
\textbf{função} \texttt{BOGOSORT(array)} \;

\While{\texttt{não\_ordenado(array)}}{
    array $\gets$ \texttt{permutação\_aleatória}(array)\;
}

\Return array\;

\caption{Bogosort}
\label{lab:alg-Bogosort}
\end{algorithm}

\begin{lstlisting}[language=Python, caption={Implementação do algoritmo Bogosort em Python}, captionpos=t, label=code:BogoPy]
import random

def bogo_sort(a):
    n = len(a)
    while (is_sorted(a)== False):
        shuffle(a)

def is_sorted(a):
    n = len(a)
    for i in range(0, n-1):
        if (a[i] > a[i+1] ):
            return False
    return True

def shuffle(a):
    n = len(a)
    for i in range (0,n):
        r = random.randint(0,n-1)
        a[i], a[r] = a[r], a[i] 
\end{lstlisting}

\begin{lstlisting}[language=C, caption={Implementação do algoritmo Bogosort em C},captionpos=t,  label=code:BogoC]
int isSorted(int *a, int n) {
    while (--n >= 1) {
        if (a[n] < a[n - 1]) {
            return 0;
        }
    }
    return 1;
}

void shuffle(int *a, int n) {
    int i, t, temp;
    for (i = 0;i < n;i++) {
        t = a[i];
        temp = rand() % n;
        a[i] = a[temp];
        a[temp] = t;
    }
}

void bogoSort(int *a, int n) {
    while (!isSorted(a, n)) {
        shuffle(a, n);
    }
}
\end{lstlisting}

\begin{lstlisting}[language=C++, caption={Implementação do algoritmo Bogosort em C++},captionpos=t, label=code:BogoCpp]
void swap(int *xp, int *yp) {
    int temp = *xp;
    *xp = *yp;
    *yp = temp;
}

bool isSorted(int a[], int n) {
    while (--n > 1)
        if (a[n] < a[n - 1])
            return false;
    return true;
}

void shuffle(int a[], int n) {
    for (int i = 0; i < n; i++)
        swap(a[i], a[rand() % n]);
}

void bogoSort(int a[], int n) {
    while (!isSorted(a, n))
        shuffle(a, n);
}
\end{lstlisting}

\subsection{Análise de Complexidade}

Nesta seção, analisamos formalmente as complexidades de tempo e espaço do \textit{Bogosort}, um algoritmo de ordenação estocástico baseado no princípio de força bruta por permutação aleatória.

\subsubsection{Complexidade de Tempo}

O funcionamento do \textit{Bogosort} consiste em gerar permutações aleatórias do vetor até que ele esteja ordenado.  
Seja $n$ o número de elementos da entrada.  
O número total de permutações possíveis é $n!$, e como todas são igualmente prováveis, a probabilidade de uma permutação ser ordenada é:

\[
P(\text{vetor ordenado}) = \frac{1}{n!}
\]

Portanto, o número esperado de tentativas até que o vetor esteja ordenado é:

\[
E[\text{tentativas}] = n!
\]

Como cada verificação de ordenação custa $O(n)$, o tempo médio esperado é:

\[
T_{\text{médio}}(n) = n \cdot n! = O(n \cdot n!)
\]

\noindent{\textbf{Melhor caso}}  
Se o vetor já estiver ordenado inicialmente, o algoritmo realiza apenas uma verificação, que custa:

\[
T_{\text{melhor}}(n) = O(n)
\]

\noindent{\textbf{Pior caso}}  
O algoritmo não possui garantia de término, pois depende de sorte para gerar a permutação ordenada.  
Assim, o pior caso é não determinístico e não limitado superiormente:

\[
T_{\text{pior}}(n) = \infty
\]

$\hfill\Box$

\bigskip

\noindent{\textbf{Discussão:}}  
O \textit{Bogosort} não é um algoritmo de ordenação prático; sua análise é relevante apenas como referência teórica ou para fins didáticos.  

\subsubsection{Complexidade de Espaço}

O \textit{Bogosort} realiza todas as operações sobre o próprio vetor, modificando-o in-place.  
Não utiliza memória auxiliar proporcional ao número de elementos, exceto variáveis temporárias constantes.

\[
S(n) = O(1)
\]

\noindent{\textbf{Prova:}}  
Nenhuma estrutura adicional de tamanho variável é criada durante a execução.  
A única alocação extra é a necessária para realizar trocas ou verificar ordenação, ambas de custo constante. Portanto:

\[
S(n) = c = O(1)
\]
$\hfill\Box$

\bigskip

\section{Stooge Sort}

\href{https://www.sortvisualizer.com/stoogesort/}{Veja Stooge Sort}\\
\textbf{Descrição:} Stooge Sort é um algoritmo de ordenação recursivo, conhecido por sua péssima complexidade de tempo. O algoritmo é baseado em comparações.

O algoritmo verifica primeiro o primeiro elemento da estrutura de dados e o último, e os troca, caso estejam na ordem errada. Se houver mais de 3 elementos, ele se autoinvoca recursivamente nos 2/3 iniciais da lista, nos 2/3 finais e novamente nos 2/3 iniciais, até que toda a lista esteja ordenada. Por isto, sua complexidade de tempo é quase cúbica.

\begin{algorithm}[H]
\DontPrintSemicolon
\small
\textbf{procedimento} \texttt{STOOGESORT(arr, l, h)} \;

\If{$l \geq h$}{
    \Return\;
}

\If{$arr[l] > arr[h]$}{
    trocar($arr[l], arr[h]$)\;
}

\If{$(h - l + 1) > 2$}{
    $t \gets \left\lfloor \dfrac{h - l + 1}{3} \right\rfloor$\;
    \texttt{STOOGESORT(arr, l, h - t)}\;
    \texttt{STOOGESORT(arr, l + t, h)}\;
    \texttt{STOOGESORT(arr, l, h - t)}\;
}

\caption{Stooge Sort}
\label{lab:alg-StoogeSort}
\end{algorithm}

\begin{lstlisting}[language=Python, caption={Implementação do algoritmo Stooge Sort em Python}, captionpos=t, label=code:stoogeSortPy]
def stoogesort(arr, l, h):
  if l >= h:
      return

  if arr[l]>arr[h]:
      t = arr[l]
      arr[l] = arr[h]
      arr[h] = t


  if h-l + 1 > 2:
      t = (int)((h-l + 1)/3)

      stoogesort(arr, l, (h-t))
      stoogesort(arr, l + t, (h))
      stoogesort(arr, l, (h-t))
\end{lstlisting}

\begin{lstlisting}[language=C, caption={Implementação do algoritmo Stooge Sort em C},captionpos=t, label=code:stoogeSortC]
void stoogesort(int arr[], int i, int j)
  {
      int temp, k;
      if (arr[i] > arr[j])
      {
          temp = arr[i];
          arr[i] = arr[j];
          arr[j] = temp;
      }
      if ((i + 1) >= j)
          return;
      k = (int)((j - i + 1) / 3);
      stoogesort(arr, i, j - k);
      stoogesort(arr, i + k, j);
      stoogesort(arr, i, j - k);
  }
\end{lstlisting}

\begin{lstlisting}[language=C++, caption={Implementação do algoritmo Stooge Sort em C++}, captionpos=t, label=code:stoogeSortCpp]
void stoogesort(int arr[], int l, int h)
  {
      if (l >= h)
          return;
   
      if (arr[l] > arr[h])
          swap(arr[l], arr[h]);
   
      if (h - l + 1 > 2) {
          int t = (h - l + 1) / 3;
          stoogesort(arr, l, h - t);
          stoogesort(arr, l + t, h);
          stoogesort(arr, l, h - t);
      }
  }
\end{lstlisting}

\subsection{Análise de Complexidade}

Nesta seção, analisamos formalmente as complexidades de tempo e espaço do \textit{Stooge Sort}.

\subsubsection{Complexidade de Tempo}

O \textit{Stooge Sort} aplica recursivamente três chamadas sobre porções sobrepostas do vetor: duas chamadas sobre $\frac{2n}{3}$ elementos e uma terceira repetição sobre o mesmo intervalo inicial.  
Assim, sua recorrência temporal é dada por:

\[
T(n) = 3 \cdot T\!\left(\frac{2n}{3}\right) + O(1)
\]

Aplicando o Teorema Mestre, obtemos:

\[
T(n) = \Theta\left(n^{\log_{3/2} 3}\right)
\]

Como:

\[
\log_{3/2} 3 = \frac{\log 3}{\log(3/2)} \approx 2.7095
\]

segue que:

\[
T(n) = \Theta(n^{2.7095\ldots})
\]

\noindent{\textbf{Observação:}}  
Essa complexidade vale igualmente para os casos **melhor**, **médio** e **pior**, pois o algoritmo realiza sempre o mesmo número de chamadas recursivas, independentemente da ordem dos elementos.

\bigskip

\noindent{\textbf{Discussão:}}  
O \textit{Stooge Sort} é significativamente mais lento que algoritmos cúbicos ($O(n^3)$), mas ainda pior que algoritmos quadráticos otimizados ($O(n^2)$).  

\subsubsection{Complexidade de Espaço}

Como o algoritmo não utiliza estruturas auxiliares além da pilha de recursão, sua complexidade espacial corresponde à profundidade das chamadas recursivas.

\[
S(n) = O(n)
\]

\noindent{\textbf{Prova:}}  
Cada chamada recursiva reduz o tamanho do problema para $\frac{2n}{3}$, resultando em profundidade:

\[
d(n) = \Theta(n)
\]

Como cada nível utiliza espaço constante, o custo total é linear:

\[
S(n) = c \cdot n = O(n)
\]
$\hfill\Box$

\bigskip

\noindent{\textbf{Discussão:}}  
Embora seja um algoritmo extremamente lento, o \textit{Stooge Sort} opera \textit{in place} e sem alocação de memória adicional relevante.




\chapter{Experimentos computacionais}

A análise experimental de algoritmos de ordenação é fundamental para validar previsões teóricas sobre complexidade de tempo, bem como para compreender o impacto de fatores práticos como overhead de implementação, gerenciamento de memória, organização dos dados e características do ambiente computacional. Neste capítulo, apresentamos a metodologia experimental utilizada, descrevemos o ambiente computacional, explicamos as instâncias analisadas e discutimos os resultados obtidos para algoritmos lineares (\(O(N)\)), logarítmicos (\(O(N\log N)\)) e quadráticos (\(O(N^2)\)).

Todos os algoritmos descritos nos capítulos anteriores foram implementados em Python, C e C++. A análise descrita nas seções seguintes integra tanto os resultados empíricos quanto as observações metodológicas obtidas durante o processo experimental.

\section{Ambiente Computacional}

\label{sec:Env}
Os experimentos foram executados em um ambiente computacional padronizado, seguindo as configurações especificadas:

\begin{itemize}
    \item[-] \textbf{Computador}: AMD Ryzen 5 3400G com gráficos Radeon Vega, 4 núcleos a 3{,}70 GHz e 16 GB de memória RAM;
    \item[-] \textbf{Sistema Operacional}: Windows 11 Pro (64 bits);
    \item[-] \textbf{Linguagens}: \texttt{Python}, \texttt{C} e \texttt{C++};
    \item[-] \textbf{Interpretador Python}: Python 3.13.0;
    \item[-] \textbf{Compilador C/C++}: \texttt{gcc 15.2.0};
    \item[-] \textbf{Coleta de tempo em Python}: função \texttt{time.time()} (tempo de parede);
    \item[-] \textbf{Coleta de tempo em C/C++}: função \texttt{gettimeofday()}.
\end{itemize}

A função \texttt{time.time()} foi utilizada por apresentar menor overhead e maior estabilidade para testes repetidos em Python, enquanto \texttt{gettimeofday()} fornece granularidade adequada para mensurações em microsegundos nas linguagens compiladas.

\section{Instâncias}
\label{sec:Inst}
Os experimentos foram conduzidos utilizando exclusivamente o conjunto de dados disponível em:

\begin{itemize}
    \item \href{https://www.kaggle.com/datasets/bekiremirhanakay/benchmark-dataset-for-sorting-algorithms}{\textit{Benchmark Dataset for Sorting Algorithms}}.
\end{itemize}

Todo o processamento experimental descrito nesta monografia baseia-se somente nos arquivos obtidos a partir do dataset acima. Após a descarga do conjunto completo, verificou-se que os valores originais estavam representados como números de ponto flutuante. Assim, procedeu-se inicialmente à sua conversão para valores inteiros, de modo a compatibilizar a entrada com algoritmos lineares cuja complexidade depende do intervalo dos valores (\(O(N + M)\)).
A partir do diretório \texttt{data/raw/uniformInt}, foram selecionados especificamente 13 arquivos contendo vetores de tamanhos distintos. Esses arquivos correspondem às seguintes quantidades de elementos:
\[
1000,\; 2000,\; 3000,\; 4000,\; 5000,\; 6000,\; 7000,\; 8000,\; 9000,\; 10000,\; 25000,\; 50000,\; 100000.
\]
Cada arquivo contém dados inteiros cuja faixa de valores varia aproximadamente de:
\[
\min \approx 20 \quad \text{até} \quad \max \approx 3\times 10^{7},
\]
o que impacta significativamente a performance de algoritmos dependentes de \textit{range}, como \textit{Counting Sort}, \textit{Pigeonhole Sort} e \textit{Bucket Sort Integer}.

Essas 13 instâncias foram utilizadas uniformemente para todos os algoritmos lineares, quadráticos e logarítmicos, garantindo reprodutibilidade e equivalência experimental entre as diferentes classes de complexidade analisadas.

\section{Metodologia Experimental}
\subsection{Execução dos Algoritmos}

Cada algoritmo foi executado \(10\) vezes para cada tamanho de entrada. Em cada repetição, utilizou-se uma cópia independente do vetor:

\begin{verbatim}
for _ in range(10):
    arr_copy = list(data)
    executar_algoritmo(arr_copy)
\end{verbatim}

Foram calculadas para cada algoritmo e tamanho:

\[
\text{média} = \frac{1}{10}\sum_{i=1}^{10} t_i, 
\qquad 
\text{DP} = \sqrt{\frac{1}{9}\sum_{i=1}^{10} (t_i - \text{média})^2}.
\]

Se os desvios padrões forem menores que 0,001, as colunas de DP podem ser omitidas conforme orientação.

Algoritmos como Bead Sort foram ignorados automaticamente quando o consumo de memória excederia \(10^7\) posições.

\section{Resultados e Análises}
Nesta seção são apresentados e discutidos os resultados experimentais obtidos para os algoritmos lineares (\(O(N)\)), logarítmicos (\(O(N\log N)\)) e quadráticos (\(O(N^2)\)), utilizando exclusivamente os tempos médios e desvios padrão reportados nas Tabelas de resultados. As análises são organizadas por classe de complexidade e, em seguida, é feita uma comparação global entre as três classes.

\subsection{Algoritmos Lineares \(O(N)\)}

Os algoritmos lineares avaliados foram: \textit{Counting Sort}, \textit{Pigeonhole Sort}, \textit{Bucket Sort Uniform}, \textit{Bucket Sort Integer}, \textit{LSD Radix Sort}, \textit{Spreadsort}, \textit{Burstsort}, \textit{Flashsort} e \textit{Postman Sort}. A Tabela~\ref{tab:linear-1000-100000} resume os tempos médios (em segundos) para três tamanhos representativos de entrada (\(N = 1000, 10000, 100000\)).

\begin{table}[H]
\centering
\caption{Tempos médios (s) dos algoritmos lineares para \(N \in \{1000, 10000, 100000\}\).}
\label{tab:linear-1000-100000}
\begin{tabular}{lccc}
\toprule
\textbf{Algoritmo} & \textbf{1000} & \textbf{10000} & \textbf{100000} \\
\midrule
Counting Sort       & 3{,}366923 & 3{,}652079 & 4{,}305130 \\
Pigeonhole Sort     & 9{,}793329 & 10{,}560131 & 11{,}149377 \\
Bucket Sort Uniform & 0{,}000381 & 0{,}004276 & 0{,}097624 \\
Bucket Sort Integer & 9{,}681769 & 10{,}654496 & 10{,}354099 \\
LSD Radix Sort      & 0{,}003049 & 0{,}069737 & 0{,}512917 \\
Spreadsort          & 0{,}000080 & 0{,}003647 & 0{,}021677 \\
Burstsort           & 0{,}000078 & 0{,}004772 & 0{,}021670 \\
Flashsort           & 0{,}000076 & 0{,}002308 & 0{,}021493 \\
Postman Sort        & 0{,}000076 & 0{,}004090 & 0{,}021751 \\
\bottomrule
\end{tabular}
\end{table}

Os resultados evidenciam dois grupos bem distintos:

\begin{itemize}
    \item \textbf{Algoritmos dependentes do intervalo \(M\)} (\textit{Counting Sort}, \textit{Pigeonhole Sort}, \textit{Bucket Sort Integer}): apresentam tempos praticamente constantes em função de \(N\), variando entre aproximadamente \(3{,}3\) s e \(11{,}1\) s, mesmo quando o tamanho da entrada cresce de \(1000\) para \(100000\). Isso ocorre porque o custo dominante é proporcional a \(M\), que é da ordem de \(3\times 10^7\) para todas as instâncias, tornando o termo \(N\) irrelevante na prática. O desvio padrão é relativamente alto (por exemplo, \(\text{DP} \approx 0{,}9\) s para \textit{Counting Sort} em \(N=100000\)), indicando forte influência de fatores externos (alocação de grandes vetores, gerenciamento de memória e cache).
    \item \textbf{Algoritmos lineares independentes de \(M\)} (\textit{Bucket Sort Uniform}, \textit{LSD Radix Sort}, \textit{Spreadsort}, \textit{Burstsort}, \textit{Flashsort}, \textit{Postman Sort}): exibem tempos muito baixos para \(N=1000\) (na ordem de \(10^{-4}\) s para \textit{Spreadsort}, \textit{Burstsort}, \textit{Flashsort} e \textit{Postman Sort}) e crescem de forma aproximadamente linear com \(N\). Por exemplo, \textit{Spreadsort} passa de \(0{,}000080\) s (\(N=1000\)) para \(0{,}021677\) s (\(N=100000\)), um fator de cerca de \(270\), compatível com o aumento de \(N\) em \(100\times\) e com overheads constantes.
\end{itemize}

O comportamento de \textit{Bucket Sort Uniform} é intermediário: o tempo cresce de \(0{,}000381\) s (\(N=1000\)) para \(0{,}097624\) s (\(N=100000\)), o que é significativamente maior que o grupo \textit{Spreadsort}/\textit{Burstsort}/\textit{Flashsort}/\textit{Postman Sort}, sugerindo overhead adicional na distribuição em baldes e na ordenação interna de cada balde.

A Figura~\ref{fig:linear-loglog} apresenta um gráfico em escala logarítmica (em ambos os eixos) para os algoritmos lineares, destacando a diferença entre os algoritmos dominados por \(M\) e os demais.

\begin{figure}[H]
\centering
\begin{tikzpicture}
\begin{axis}[
    width=0.8\textwidth,
    height=0.55\textwidth,
    xmode=log,
    ymode=log,
    log basis x=10,
    log basis y=10,
    legend pos=outer north east,
    xlabel={Tamanho da entrada \(N\)},
    ylabel={Tempo médio (s)},
    % legend style={at={(0.02,0.02)},anchor=south west,font=\scriptsize},
    grid=both,
    xmin=900, xmax=110000,
    ymin=1e-4, ymax=2e1
]
% Dados: N = 1000, 10000, 100000
\addplot+[mark=o] coordinates {
    (1000, 3.366923)
    (10000, 3.652079)
    (100000, 4.305130)
};
\addlegendentry{\footnotesize Counting Sort}

\addplot+[mark=triangle] coordinates {
    (1000, 9.793329)
    (10000, 10.560131)
    (100000, 11.149377)
};
\addlegendentry{\footnotesize Pigeonhole Sort}

\addplot+[mark=square] coordinates {
    (1000, 9.681769)
    (10000, 10.654496)
    (100000, 10.354099)
};
\addlegendentry{\footnotesize Bucket Sort Integer}

\addplot+[mark=*] coordinates {
    (1000, 0.000381)
    (10000, 0.004276)
    (100000, 0.097624)
};
\addlegendentry{\footnotesize Bucket Sort Uniform}

\addplot+[mark=x] coordinates {
    (1000, 0.003049)
    (10000, 0.069737)
    (100000, 0.512917)
};
\addlegendentry{\footnotesize LSD Radix Sort}

\addplot+[mark=+] coordinates {
    (1000, 0.000080)
    (10000, 0.003647)
    (100000, 0.021677)
};
\addlegendentry{\footnotesize Spreadsort}

\addplot+[mark=diamond] coordinates {
    (1000, 0.000078)
    (10000, 0.004772)
    (100000, 0.021670)
};
\addlegendentry{\footnotesize Burstsort}

\addplot+[mark=pentagon] coordinates {
    (1000, 0.000076)
    (10000, 0.002308)
    (100000, 0.021493)
};
\addlegendentry{\footnotesize Flashsort}

\addplot+[mark=star] coordinates {
    (1000, 0.000076)
    (10000, 0.004090)
    (100000, 0.021751)
};
\addlegendentry{\footnotesize Postman Sort}

\end{axis}
\end{tikzpicture}
\caption{Escalonamento dos algoritmos lineares em escala log--log .}
\label{fig:linear-loglog}
\end{figure}

Na Figura~\ref{fig:linear-loglog}, as curvas de \textit{Counting Sort}, \textit{Pigeonhole Sort} e \textit{Bucket Sort Integer} aparecem praticamente horizontais, confirmando que o custo é dominado por \(M\). Em contraste, os algoritmos independentes de \(M\) exibem inclinações compatíveis com crescimento aproximadamente linear em \(N\).

Os desvios padrão dos algoritmos rápidos (\textit{Spreadsort}, \textit{Burstsort}, \textit{Flashsort}, \textit{Postman Sort}) são da ordem de \(10^{-6}\) a \(10^{-4}\) s, indicando alta consistência entre as 10 repetições. Já os algoritmos baseados em grandes estruturas auxiliares (\textit{Counting Sort}, \textit{Pigeonhole Sort}, \textit{Bucket Sort Integer}) apresentam desvios padrão entre aproximadamente \(0{,}5\) s e \(1{,}4\) s, o que sugere forte sensibilidade a variações de alocação de memória, paginação e efeitos de cache.

\subsection{Algoritmos Logarítmicos \(O(N\log N)\)}

Os algoritmos \(O(N\log N)\) avaliados foram: \textit{Quicksort}, \textit{Merge Sort}, \textit{Heapsort}, \textit{Introsort}, \textit{Timsort}, \textit{CubeSort}, \textit{In-Place MergeSort}, \textit{Tournament Sort}, \textit{Tree Sort}, \textit{Block Sort}, \textit{Patience Sorting} e \textit{Smooth Sort}. A Tabela~\ref{tab:nlogn-1000-100000} apresenta os tempos médios para \(N = 1000, 10000, 100000\).

\begin{table}[H]
\centering
\caption{Tempos médios (s) dos algoritmos \(O(N\log N)\) para \(N \in \{1000, 10000, 100000\}\).}
\label{tab:nlogn-1000-100000}
\begin{tabular}{lccc}
\toprule
\textbf{Algoritmo} & \textbf{1000} & \textbf{10000} & \textbf{100000} \\
\midrule
Quicksort          & 0{,}001288 & 0{,}022199 & 0{,}235103 \\
Merge Sort         & 0{,}000991 & 0{,}013221 & 0{,}117483 \\
Heapsort           & 0{,}002970 & 0{,}040548 & 0{,}506438 \\
Introsort          & 0{,}001087 & 0{,}015543 & 0{,}192261 \\
Timsort            & 0{,}000081 & 0{,}001355 & 0{,}017197 \\
CubeSort           & 0{,}000120 & 0{,}001146 & 0{,}017340 \\
In-Place MergeSort & 0{,}014808 & 1{,}374327 & 160{,}801455 \\
Tournament Sort    & 0{,}007784 & 0{,}754737 & 83{,}495839 \\
Tree Sort          & 0{,}001520 & 0{,}019379 & 0{,}321939 \\
Block Sort         & 0{,}004370 & 0{,}109210 & 3{,}757724 \\
Patience Sorting   & 0{,}004246 & 0{,}133945 & 4{,}848978 \\
Smooth Sort        & 0{,}000079 & 0{,}001237 & 0{,}023330 \\
\bottomrule
\end{tabular}
\end{table}

Os dados mostram uma separação clara entre:

\begin{itemize}
    \item \textbf{Implementações eficientes}: \textit{Timsort}, \textit{CubeSort}, \textit{Smooth Sort}, \textit{Merge Sort}, \textit{Quicksort} e \textit{Introsort} apresentam tempos que crescem de forma compatível com \(N\log N\) e mantêm valores absolutos baixos. Por exemplo, \textit{Timsort} cresce de \(0{,}000081\) s (\(N=1000\)) para \(0{,}017197\) s (\(N=100000\)), um fator de cerca de \(212\), enquanto \(N\) cresce \(100\times\) e \(\log_2 N\) cresce de aproximadamente \(10\) para \(17\). \textit{Merge Sort} e \textit{Quicksort} exibem fatores de crescimento semelhantes, com tempos absolutos maiores, mas ainda muito inferiores aos algoritmos quadráticos.
    \item \textbf{Implementações com overhead extremo}: \textit{In-Place MergeSort} e \textit{Tournament Sort} apresentam tempos que crescem muito mais rapidamente. \textit{In-Place MergeSort} passa de \(0{,}014808\) s (\(N=1000\)) para \(160{,}801455\) s (\(N=100000\)), um fator superior a \(10^4\). \textit{Tournament Sort} cresce de \(0{,}007784\) s para \(83{,}495839\) s no mesmo intervalo. Esses valores indicam que, embora a complexidade assintótica seja \(O(N\log N)\), o custo constante e a estrutura de dados utilizada (árvores de torneio, fusão in-place complexa) tornam essas implementações impraticáveis para grandes \(N\).
\end{itemize}

A Figura~\ref{fig:nlogn-loglog} mostra o comportamento em escala log--log para um subconjunto representativo de algoritmos \(O(N\log N)\).

\begin{figure}[H]
\centering
\begin{tikzpicture}
\begin{axis}[
    width=0.85\textwidth,
    height=0.55\textwidth,
    xmode=log,
    ymode=log,
    log basis x=10,
    log basis y=10,
    xlabel={Tamanho da entrada \(N\)},
    ylabel={Tempo médio (s)},
    % legend style={at={(0.02,0.02)},anchor=south west,font=\scriptsize},
    legend pos=outer north east,
    grid=both,
    xmin=900, xmax=110000,
    ymin=5e-5, ymax=5e2
]
% Dados: N = 1000, 10000, 100000
\addplot+[mark=o] coordinates {
    (1000, 0.001288)
    (10000, 0.022199)
    (100000, 0.235103)
};
\addlegendentry{\footnotesize Quicksort}

\addplot+[mark=triangle] coordinates {
    (1000, 0.000991)
    (10000, 0.013221)
    (100000, 0.117483)
};
\addlegendentry{\footnotesize Merge Sort}

\addplot+[mark=square] coordinates {
    (1000, 0.000081)
    (10000, 0.001355)
    (100000, 0.017197)
};
\addlegendentry{\footnotesize Timsort}

\addplot+[mark=diamond] coordinates {
    (1000, 0.000079)
    (10000, 0.001237)
    (100000, 0.023330)
};
\addlegendentry{\footnotesize Smooth Sort}

\addplot+[mark=*] coordinates {
    (1000, 0.014808)
    (10000, 1.374327)
    (100000, 160.801455)
};
\addlegendentry{\footnotesize In-Place MergeSort}

\addplot+[mark=x] coordinates {
    (1000, 0.007784)
    (10000, 0.754737)
    (100000, 83.495839)
};
\addlegendentry{Tournament Sort}

\end{axis}
\end{tikzpicture}
\caption{Escalonamento de algoritmos \(O(N\log N)\) em escala log--log.}
\label{fig:nlogn-loglog}
\end{figure}

Na Figura~\ref{fig:nlogn-loglog}, \textit{Timsort}, \textit{Smooth Sort}, \textit{Merge Sort} e \textit{Quicksort} formam um grupo com inclinações semelhantes e tempos absolutos baixos. \textit{In-Place MergeSort} e \textit{Tournament Sort} aparecem deslocados para cima, com inclinações maiores, refletindo overhead significativo.

Os desvios padrão dos algoritmos eficientes são pequenos em relação ao tempo médio (por exemplo, \(\text{DP} = 0{,}001151\) s para \textit{Timsort} em \(N=100000\), frente a \(0{,}017197\) s de média), indicando boa estabilidade. Em contraste, \textit{In-Place MergeSort} em \(N=100000\) apresenta \(\text{DP} = 32{,}884374\) s, cerca de \(20\%\) do tempo médio, sugerindo forte variabilidade possivelmente associada a padrões de acesso à memória e à complexidade da fusão in-place.

\subsection{Algoritmos Quadráticos \(O(N^2)\)}

Os algoritmos quadráticos avaliados incluem: \textit{Shell Sort}, \textit{Comb Sort}, \textit{Insertion Sort}, \textit{Bubble Sort}, \textit{Gnome Sort}, \textit{Odd-Even Sort}, \textit{Selection Sort}, \textit{Cocktail Shaker Sort}, \textit{Strand Sort}, \textit{Exchange Sort}, \textit{Cycle Sort}, \textit{Recombinant Sort}, \textit{I Can't Believe It Can Sort}, \textit{Spaghetti Sort}, \textit{Sorting Network}, \textit{Bitonic Sorter} e \textit{Pancake Sort}. A Tabela~\ref{tab:quadratic-1000-100000} resume os tempos médios para \(N = 1000, 10000, 100000\).

\begin{table}[H]
\centering
\caption{Tempos médios (s) dos algoritmos quadráticos para \(N \in \{1000, 10000, 100000\}\).}
\label{tab:quadratic-1000-100000}
\begin{tabular}{lccc}
\toprule
\textbf{Algoritmo} & \textbf{1000} & \textbf{10000} & \textbf{100000} \\
\midrule
Shell Sort                 & 0{,}001899 & 0{,}036669 & 0{,}605234 \\
Comb Sort                  & 0{,}003401 & 0{,}047168 & 0{,}671009 \\
Insertion Sort             & 0{,}022584 & 3{,}155436 & 267{,}710087 \\
Bubble Sort                & 0{,}056984 & 7{,}345069 & 631{,}495185 \\
Gnome Sort                 & 0{,}099419 & 10{,}902124 & 985{,}644044 \\
Odd-Even Sort              & 0{,}093009 & 6{,}518017 & 671{,}287938 \\
Selection Sort             & 0{,}031081 & 1{,}772198 & 181{,}699649 \\
Cocktail Shaker Sort       & 0{,}063204 & 5{,}938991 & 558{,}530873 \\
Strand Sort                & 0{,}005845 & 0{,}196898 & 6{,}753527 \\
Exchange Sort              & 0{,}034308 & 3{,}119438 & 326{,}186921 \\
Cycle Sort                 & 0{,}058878 & 5{,}774102 & 700{,}019764 \\
Recombinant Sort           & 0{,}004593 & 0{,}055336 & 0{,}709018 \\
I Can't Believe It Can Sort& 0{,}022668 & 2{,}222203 & 248{,}102844 \\
Spaghetti Sort             & 0{,}000116 & 0{,}001150 & 0{,}017013 \\
Sorting Network            & 0{,}009661 & 0{,}139157 & 1{,}911960 \\
Bitonic Sorter             & 0{,}009298 & 0{,}137937 & 1{,}903500 \\
Pancake Sort               & 0{,}016571 & 1{,}424791 & 251{,}886030 \\
\bottomrule
\end{tabular}
\end{table}

Os resultados mostram que:

\begin{itemize}
    \item \textbf{Algoritmos quadráticos clássicos} (\textit{Bubble Sort}, \textit{Gnome Sort}, \textit{Odd-Even Sort}, \textit{Selection Sort}, \textit{Insertion Sort}, \textit{Exchange Sort}, \textit{Cycle Sort}, \textit{Cocktail Shaker Sort}) tornam-se rapidamente impraticáveis. Por exemplo, \textit{Bubble Sort} cresce de \(0{,}056984\) s (\(N=1000\)) para \(7{,}345069\) s (\(N=10000\)) e atinge \(631{,}495185\) s (\(N=100000\)), um aumento de mais de \(10^4\) vezes, compatível com o fator \(100^2\) esperado para algoritmos \(O(N^2)\).
    \item \textbf{Algoritmos quadráticos otimizados} (\textit{Shell Sort}, \textit{Comb Sort}, \textit{Recombinant Sort}, \textit{Strand Sort}, \textit{Spaghetti Sort}, \textit{Sorting Network}, \textit{Bitonic Sorter}) apresentam tempos muito menores. \textit{Shell Sort} e \textit{Comb Sort} mantêm tempos abaixo de \(1\) s mesmo para \(N=100000\) (\(0{,}605234\) s e \(0{,}671009\) s, respectivamente), o que os coloca próximos, em termos absolutos, de alguns algoritmos \(O(N\log N)\) menos otimizados. \textit{Spaghetti Sort} é um caso extremo: \(0{,}000116\) s (\(N=1000\)), \(0{,}001150\) s (\(N=10000\)) e \(0{,}017013\) s (\(N=100000\)), valores comparáveis aos de \textit{Timsort} e \textit{Smooth Sort}.
\end{itemize}

A Figura~\ref{fig:quadratic-loglog} ilustra o comportamento em escala log--log para alguns algoritmos quadráticos representativos.

\begin{figure}[H]
\centering
\begin{tikzpicture}
\begin{axis}[
    width=0.9\textwidth,
    height=0.55\textwidth,
    xmode=log,
    ymode=log,
    log basis x=10,
    log basis y=10,
    xlabel={Tamanho da entrada \(N\)},
    ylabel={Tempo médio (s)},
    % legend style={at={(0.02,0.02)},anchor=south west,font=\scriptsize},
     legend pos=outer north east,
    grid=both,
    xmin=900, xmax=110000,
    ymin=1e-4, ymax=2e3
]
% Dados: N = 1000, 10000, 100000
\addplot+[mark=o] coordinates {
    (1000, 0.056984)
    (10000, 7.345069)
    (100000, 631.495185)
};
\addlegendentry{\footnotesize Bubble Sort}

\addplot+[mark=triangle] coordinates {
    (1000, 0.099419)
    (10000, 10.902124)
    (100000, 985.644044)
};
\addlegendentry{\footnotesize Gnome Sort}

\addplot+[mark=square] coordinates {
    (1000, 0.031081)
    (10000, 1.772198)
    (100000, 181.699649)
};
\addlegendentry{\footnotesize Selection Sort}

\addplot+[mark=diamond] coordinates {
    (1000, 0.001899)
    (10000, 0.036669)
    (100000, 0.605234)
};
\addlegendentry{\footnotesize Shell Sort}

\addplot+[mark=*] coordinates {
    (1000, 0.003401)
    (10000, 0.047168)
    (100000, 0.671009)
};
\addlegendentry{\footnotesize Comb Sort}

\addplot+[mark=x] coordinates {
    (1000, 0.000116)
    (10000, 0.001150)
    (100000, 0.017013)
};
\addlegendentry{\footnotesize Spaghetti Sort}

\end{axis}
\end{tikzpicture}
\caption{Escalonamento de algoritmos quadráticos em escala log--log.}
\label{fig:quadratic-loglog}
\end{figure}

Na Figura~\ref{fig:quadratic-loglog}, \textit{Bubble Sort}, \textit{Gnome Sort} e \textit{Selection Sort} exibem inclinações compatíveis com \(O(N^2)\), com crescimento muito mais acentuado que os algoritmos \(O(N\log N)\). \textit{Shell Sort}, \textit{Comb Sort} e \textit{Spaghetti Sort} aparecem bem abaixo, com tempos absolutos próximos aos de algoritmos mais eficientes, o que evidencia o impacto de otimizações práticas mesmo em algoritmos com pior complexidade assintótica.

Os desvios padrão dos algoritmos quadráticos clássicos são relativamente pequenos em relação aos tempos médios (por exemplo, \(\text{DP} = 10{,}445157\) s para \textit{Bubble Sort} em \(N=100000\), frente a \(631{,}495185\) s de média), indicando que, embora os tempos sejam altos, o comportamento é consistente entre as repetições. Para algoritmos muito rápidos como \textit{Spaghetti Sort}, os desvios padrão são da ordem de \(10^{-4}\) s, o que reforça a estabilidade.

\subsection{Comparação Entre Classes de Complexidade}

Para comparar diretamente as três classes de algoritmos, a Tabela~\ref{tab:comparacao-classes-100000} apresenta, para \(N=100000\), alguns algoritmos representativos de cada classe.

\begin{table}[H]
\centering
\caption{Comparação de tempos médios (s) para \(N=100000\) entre classes de complexidade.}
\label{tab:comparacao-classes-100000}
\begin{tabular}{llc}
\toprule
\textbf{Classe} & \textbf{Algoritmo} & \textbf{Tempo médio (s)} \\
\midrule
\(O(N)\)        & Spreadsort          & 0{,}021677 \\
\(O(N)\)        & Flashsort           & 0{,}021493 \\
\(O(N)\)        & Bucket Sort Uniform & 0{,}097624 \\
\(O(N)\)        & Counting Sort       & 4{,}305130 \\
\(O(N)\)        & Pigeonhole Sort     & 11{,}149377 \\
\midrule
\(O(N\log N)\)  & Timsort             & 0{,}017197 \\
\(O(N\log N)\)  & Smooth Sort         & 0{,}023330 \\
\(O(N\log N)\)  & Merge Sort          & 0{,}117483 \\
\(O(N\log N)\)  & Quicksort           & 0{,}235103 \\
\(O(N\log N)\)  & In-Place MergeSort  & 160{,}801455 \\
\midrule
\(O(N^2)\)      & Spaghetti Sort      & 0{,}017013 \\
\(O(N^2)\)      & Shell Sort          & 0{,}605234 \\
\(O(N^2)\)      & Comb Sort           & 0{,}671009 \\
\(O(N^2)\)      & Bubble Sort         & 631{,}495185 \\
\(O(N^2)\)      & Gnome Sort          & 985{,}644044 \\
\bottomrule
\end{tabular}
\end{table}

A Tabela~\ref{tab:comparacao-classes-100000} revela vários aspectos importantes:

\begin{itemize}
    \item \textbf{Sobreposição entre classes}: \textit{Spaghetti Sort} (\(O(N^2)\)) apresenta tempo médio de \(0{,}017013\) s, praticamente idêntico a \textit{Timsort} (\(0{,}017197\) s) e inferior a \textit{Spreadsort} e \textit{Flashsort}. Isso mostra que, para \(N=100000\) e para esta implementação específica, a constante oculta e a estrutura do algoritmo podem compensar a pior complexidade assintótica.
    \item \textbf{Algoritmos lineares dominados por \(M\)}: \textit{Counting Sort} e \textit{Pigeonhole Sort} são mais lentos que \textit{Merge Sort} e \textit{Quicksort}, apesar de sua complexidade teórica \(O(N+M)\). O intervalo de valores (\(M \approx 3\times 10^7\)) torna o termo \(M\) dominante, resultando em tempos de vários segundos.
    \item \textbf{Implementações \(O(N\log N)\) impraticáveis}: \textit{In-Place MergeSort} é mais lento que \textit{Bubble Sort} e \textit{Gnome Sort} para \(N=100000\) (160{,}801455 s contra 631{,}495185 s e 985{,}644044 s, respectivamente), mas ainda está na mesma ordem de grandeza de algoritmos quadráticos clássicos, o que o torna pouco competitivo na prática.
\end{itemize}

A Figura~\ref{fig:comparacao-geral-loglog} apresenta um gráfico geral em escala log--log comparando algoritmos representativos das três classes.

\begin{figure}[H]
\centering
\begin{tikzpicture}
\begin{axis}[
    width=0.74\textwidth,
    height=0.55\textwidth,
    xmode=log,
    ymode=log,
    log basis x=10,
    log basis y=10,
    xlabel={Tamanho da entrada \(N\)},
    ylabel={Tempo médio (s)},
    % legend style={at={(0.02,0.02)},anchor=south west,font=\scriptsize},
     legend pos=outer north east,
    grid=both,
    xmin=900, xmax=110000,
    ymin=5e-5, ymax=2e3
]
% Dados: N = 1000, 10000, 100000

% O(N): Spreadsort
\addplot+[mark=o] coordinates {
    (1000, 0.000080)
    (10000, 0.003647)
    (100000, 0.021677)
};
\addlegendentry{\footnotesize Spreadsort (O(N))}

% O(N): Counting Sort
\addplot+[mark=triangle] coordinates {
    (1000, 3.366923)
    (10000, 3.652079)
    (100000, 4.305130)
};
\addlegendentry{\footnotesize Counting Sort (O(N+M))}

% O(N log N): Timsort
\addplot+[mark=square] coordinates {
    (1000, 0.000081)
    (10000, 0.001355)
    (100000, 0.017197)
};
\addlegendentry{\footnotesize Timsort (O(N\textbackslash log N))}

% O(N log N): In-Place MergeSort
\addplot+[mark=diamond] coordinates {
    (1000, 0.014808)
    (10000, 1.374327)
    (100000, 160.801455)
};
\addlegendentry{\footnotesize In-Place MergeSort (O(N\textbackslash log N))}

% O(N^2): Spaghetti Sort
\addplot+[mark=*] coordinates {
    (1000, 0.000116)
    (10000, 0.001150)
    (100000, 0.017013)
};
\addlegendentry{\footnotesize Spaghetti Sort (O(N\textsuperscript{2}))}

% O(N^2): Bubble Sort
\addplot+[mark=x] coordinates {
    (1000, 0.056984)
    (10000, 7.345069)
    (100000, 631.495185)
};
\addlegendentry{\footnotesize Bubble Sort (O(N\textsuperscript{2}))}

\end{axis}
\end{tikzpicture}
\caption{Comparação geral entre algoritmos representativos das três classes em escala log--log.}
\label{fig:comparacao-geral-loglog}
\end{figure}

Na Figura~\ref{fig:comparacao-geral-loglog} observam-se as seguintes tendências:

\begin{itemize}
    \item \textbf{Hierarquia esperada em termos de inclinação}: as curvas de \textit{Spreadsort} e \textit{Counting Sort} (quando se ignora o efeito de \(M\)) são menos inclinadas que as de \textit{Timsort} e \textit{In-Place MergeSort}, que por sua vez são menos inclinadas que as de \textit{Bubble Sort}. Isso é compatível com as ordens \(O(N)\), \(O(N\log N)\) e \(O(N^2)\).
    \item \textbf{Impacto das constantes}: \textit{Spaghetti Sort} (quadrático) acompanha de perto \textit{Timsort} (logarítmico) em toda a faixa de \(N\), e ambos são mais rápidos que \textit{Counting Sort} para \(N \geq 1000\). Isso evidencia que, para os tamanhos de entrada considerados e para este conjunto de dados, as constantes e o modelo de implementação podem ser mais determinantes que a classe assintótica isoladamente.
    \item \textbf{Gargalos de memória e estrutura de dados}: \textit{Counting Sort} e \textit{In-Place MergeSort} ilustram gargalos distintos. O primeiro é penalizado pelo tamanho do intervalo \(M\), que exige estruturas auxiliares enormes, enquanto o segundo sofre com a complexidade da fusão in-place, resultando em tempos comparáveis ou piores que algoritmos quadráticos clássicos.
\end{itemize}

\subsection{Consistência, Desvios Padrão e Outliers}

A análise dos desvios padrão permite avaliar a estabilidade dos algoritmos:

\begin{itemize}
    \item \textbf{Algoritmos muito rápidos} (\textit{Spreadsort}, \textit{Burstsort}, \textit{Flashsort}, \textit{Postman Sort}, \textit{Timsort}, \textit{Smooth Sort}, \textit{Spaghetti Sort}) apresentam desvios padrão muito pequenos em relação ao tempo médio (tipicamente abaixo de \(10\%\) e frequentemente abaixo de \(5\%\)), o que indica comportamento altamente determinístico e pouco sensível a variações do ambiente.
    \item \textbf{Algoritmos com grandes estruturas auxiliares} (\textit{Counting Sort}, \textit{Pigeonhole Sort}, \textit{Bucket Sort Integer}, \textit{In-Place MergeSort}, \textit{Tournament Sort}, \textit{Bubble Sort}, \textit{Gnome Sort}) exibem desvios padrão mais elevados, especialmente para grandes \(N\). Exemplos extremos incluem \textit{In-Place MergeSort} em \(N=100000\) (\(\text{DP} = 32{,}884374\) s) e \textit{Tournament Sort} em \(N=50000\) (\(\text{DP} = 1{,}143221\) s), sugerindo que o custo é fortemente afetado por detalhes de alocação, cache e possíveis interferências do sistema operacional.
\end{itemize}

Não foram observados valores isolados que destoassem completamente das tendências gerais; os tempos médios crescem de forma monotônica com \(N\) para todos os algoritmos, e os desvios padrão, embora altos em alguns casos, permanecem proporcionais aos tempos médios, o que indica ausência de outliers extremos nas 10 repetições.

\subsection{Síntese das Tendências Observadas}

Com base exclusivamente nos dados experimentais:

\begin{itemize}
    \item Algoritmos lineares independentes de \(M\) (\textit{Spreadsort}, \textit{Burstsort}, \textit{Flashsort}, \textit{Postman Sort}) e algoritmos \(O(N\log N)\) bem otimizados (\textit{Timsort}, \textit{Smooth Sort}, \textit{Merge Sort}, \textit{Quicksort}) apresentam os melhores tempos absolutos para todas as instâncias testadas.
    \item Algoritmos lineares dependentes de \(M\) (\textit{Counting Sort}, \textit{Pigeonhole Sort}, \textit{Bucket Sort Integer}) tornam-se impraticáveis neste conjunto de dados devido ao intervalo de valores (\(M \approx 3\times 10^7\)), superando em tempo até mesmo algoritmos \(O(N\log N)\) e alguns \(O(N^2)\) otimizados.
    \item Entre os algoritmos quadráticos, \textit{Shell Sort}, \textit{Comb Sort}, \textit{Recombinant Sort}, \textit{Strand Sort}, \textit{Spaghetti Sort}, \textit{Sorting Network} e \textit{Bitonic Sorter} destacam-se por tempos significativamente menores que os quadráticos clássicos, chegando a competir com algoritmos \(O(N\log N)\) para \(N=100000\).
    \item A hierarquia assintótica \(O(N) \subseteq O(N\log N) \subseteq O(N^2)\) é confirmada em termos de inclinação das curvas em escala log--log, mas há sobreposição substancial em termos de tempos absolutos, especialmente entre algoritmos quadráticos otimizados e algoritmos \(O(N\log N)\) com alto overhead, bem como entre algoritmos lineares dominados por \(M\) e algoritmos \(O(N\log N)\) eficientes.
\end{itemize}

\subsection{Limitações e justificativas}

Nesta seção são detalhados os itens que sofreram ajustes em relação à proposta inicial, bem como as justificativas para tais modificações. Durante o desenvolvimento do projeto, algumas alterações de escopo tornaram-se necessárias, especialmente no item referente à experimentação computacional.

Inicialmente, pretendia-se investigar o desempenho dos algoritmos utilizando instâncias de maior porte disponibilizadas pelo banco de vetores adotado. Entretanto, uma rodada experimental preliminar evidenciou que, mesmo para algoritmos com tempo de execução linear, o custo computacional já era elevado. Dessa forma, estender os testes para algoritmos de ordem logarítmica ou quadrática mostrou-se inviável dentro do prazo disponível para elaboração do estudo. Por esse motivo, conforme descrito no capítulo de Experimentos Computacionais, optou-se pela seleção de subconjuntos de instâncias com tamanhos específicos, suficientes para ilustrar o comportamento relativo entre os algoritmos, mas compatíveis com o tempo de execução acessível.

Outro ajuste necessário foi a decisão de realizar a comparação empírica apenas com as implementações em linguagem Python. Embora o projeto conte com implementações em outras linguagens, a execução de testes completos para todas elas demandaria recursos computacionais e temporais adicionais. Considerando que o objetivo central da análise é comparar o comportamento assintótico dos algoritmos, e que esse comportamento tende a se manter independentemente da linguagem utilizada, optou-se por concentrar a experimentação em uma única linguagem, decisão considerada suficiente e adequada ao propósito do estudo.

Por fim, apesar de terem sido implementados os algoritmos classificados como “miscelâneos”, não foi possível incluí-los na análise computacional. Alguns desses algoritmos apresentam tempo de pior caso não limitado superiormente — como o \textit{BogoSort} — tornando sua execução inviável no período disponível. Assim, este tópico e os outros citados acima permanecem como oportunidade para trabalhos futuros.

\section{Conclusões}
A análise experimental apresentada ao longo deste trabalho permitiu compreender de maneira aprofundada o comportamento prático dos diferentes algoritmos de ordenação quando submetidos a um conjunto padronizado de instâncias. Embora a teoria da complexidade ofereça uma previsão geral sobre o desempenho esperado, os experimentos revelam nuances importantes que somente a avaliação empírica é capaz de demonstrar.

Inicialmente, observou-se que os algoritmos lineares que não dependem do intervalo dos valores — como FlashSort, SpreadSort e Burstsort — apresentam desempenho excepcionalmente eficiente, mantendo tempos de execução muito baixos mesmo para entradas de cem mil elementos. Esse comportamento confirma não apenas a linearidade teórica, mas também evidencia a eficácia prática dessas implementações. Por outro lado, algoritmos lineares cuja complexidade envolve o parâmetro adicional \(M\), como Counting Sort e Pigeonhole Sort, mostraram-se impraticáveis diante de um intervalo extremamente amplo de valores. Este fenômeno ressalta a importância de analisar não apenas a ordem assintótica, mas também as condições reais de uso, especialmente quando o desempenho depende de características específicas dos dados.

Nos algoritmos classificados como \(O(N \log N)\), verificou-se que a diferença entre implementações pode ser significativa. Timsort, por exemplo, destacou-se pela estabilidade e excelente desempenho, muito em função de sua estratégia híbrida e da capacidade de explorar trechos já ordenados. Em contraste, algoritmos como In-place MergeSort e Tournament Sort apresentaram overhead elevado, evidenciando que uma boa complexidade teórica não garante necessariamente bom desempenho prático quando os custos internos da implementação são altos. Assim, fica claro que, dentro da mesma classe assintótica, a escolha do algoritmo mais adequado depende diretamente do contexto e das características específicas da aplicação.

Os algoritmos quadráticos, como previsto, demonstraram limitações severas conforme o tamanho das entradas aumentava. Bubble Sort e Gnome Sort tornaram-se inviáveis para instâncias maiores, com tempos de execução crescendo de forma abrupta. No entanto, algoritmos como Shell Sort e Comb Sort apresentaram desempenho consideravelmente melhor, mostrando que estratégias de melhoria podem reduzir drasticamente o impacto prático da complexidade quadrática pura. Mesmo assim, sua utilização em cenários de grande escala deve ser evitada quando alternativas mais eficientes estão disponíveis.

De modo geral, os resultados experimentais confirmam a hierarquia esperada entre as classes de complexidade, mas também demonstram que a prática envolve fatores adicionais, como otimizações internas, overhead computacional, consumo de memória e organização dos dados. Essa constatação reforça a importância de complementar a análise teórica com experimentos empíricos, garantindo que o algoritmo selecionado seja não apenas assintoticamente eficiente, mas também adequado às condições reais de uso.

Por fim, este estudo abre caminho para investigações futuras. Avaliações específicas do impacto de \(M\) nos algoritmos lineares baseados em contagem, análises detalhadas do consumo de memória ou ainda estudos comparativos em linguagens compiladas podem oferecer insights adicionais sobre o comportamento desses algoritmos em diferentes cenários. Ademais, é possível explorar as limitações ciadas no capítulo anterior, para complementação dos resultados já obtidos. A combinação entre análise teórica e experimental mostra-se, portanto, fundamental para a compreensão completa do desempenho de algoritmos de ordenação.
\chapter{Conclusões}
A análise experimental apresentada ao longo deste trabalho permitiu compreender de maneira aprofundada o comportamento prático dos diferentes algoritmos de ordenação quando submetidos a um conjunto padronizado de instâncias. Embora a teoria da complexidade ofereça uma previsão geral sobre o desempenho esperado, os experimentos revelam nuances importantes que somente a avaliação empírica é capaz de demonstrar.

Inicialmente, observou-se que os algoritmos lineares que não dependem do intervalo dos valores — como FlashSort, SpreadSort e Burstsort — apresentam desempenho excepcionalmente eficiente, mantendo tempos de execução muito baixos mesmo para entradas de cem mil elementos. Esse comportamento confirma não apenas a linearidade teórica, mas também evidencia a eficácia prática dessas implementações. Por outro lado, algoritmos lineares cuja complexidade envolve o parâmetro adicional \(M\), como Counting Sort e Pigeonhole Sort, mostraram-se impraticáveis diante de um intervalo extremamente amplo de valores. Este fenômeno ressalta a importância de analisar não apenas a ordem assintótica, mas também as condições reais de uso, especialmente quando o desempenho depende de características específicas dos dados.

Nos algoritmos classificados como \(O(N \log N)\), verificou-se que a diferença entre implementações pode ser significativa. Timsort, por exemplo, destacou-se pela estabilidade e excelente desempenho, muito em função de sua estratégia híbrida e da capacidade de explorar trechos já ordenados. Em contraste, algoritmos como In-place MergeSort e Tournament Sort apresentaram overhead elevado, evidenciando que uma boa complexidade teórica não garante necessariamente bom desempenho prático quando os custos internos da implementação são altos. Assim, fica claro que, dentro da mesma classe assintótica, a escolha do algoritmo mais adequado depende diretamente do contexto e das características específicas da aplicação.

Os algoritmos quadráticos, como previsto, demonstraram limitações severas conforme o tamanho das entradas aumentava. Bubble Sort e Gnome Sort tornaram-se inviáveis para instâncias maiores, com tempos de execução crescendo de forma abrupta. No entanto, algoritmos como Shell Sort e Comb Sort apresentaram desempenho consideravelmente melhor, mostrando que estratégias de melhoria podem reduzir drasticamente o impacto prático da complexidade quadrática pura. Mesmo assim, sua utilização em cenários de grande escala deve ser evitada quando alternativas mais eficientes estão disponíveis.

De modo geral, os resultados experimentais confirmam a hierarquia esperada entre as classes de complexidade, mas também demonstram que a prática envolve fatores adicionais, como otimizações internas, overhead computacional, consumo de memória e organização dos dados. Essa constatação reforça a importância de complementar a análise teórica com experimentos empíricos, garantindo que o algoritmo selecionado seja não apenas assintoticamente eficiente, mas também adequado às condições reais de uso.

Por fim, este estudo abre caminho para investigações futuras. Avaliações específicas do impacto de \(M\) nos algoritmos lineares baseados em contagem, análises detalhadas do consumo de memória ou ainda estudos comparativos em linguagens compiladas podem oferecer insights adicionais sobre o comportamento desses algoritmos em diferentes cenários. Ademais, é possível explorar as limitações ciadas no capítulo anterior, para complementação dos resultados já obtidos. A combinação entre análise teórica e experimental mostra-se, portanto, fundamental para a compreensão completa do desempenho de algoritmos de ordenação.

\nocite{*}
\bibliographystyle{alpha}
\bibliography{biblio}

\end{document}