\documentclass[12pt]{book}
\usepackage{amsmath}
\usepackage{amsfonts}
\usepackage{amssymb}
\usepackage{graphicx}
\usepackage{epsfig}
\usepackage[boxruled,vlined,boxed,portuguese,algochapter]{algorithm2e}

\usepackage[utf8]{inputenc}
\usepackage[brazil]{babel}
\usepackage{hyperref}
\usepackage{tcolorbox}
\usepackage{epigraph} 
\usepackage{quotchap}
\usepackage{mathtools}
\usepackage{multirow}

\topskip 0pt
\headsep 30pt
\headheight 1pt
\oddsidemargin 0pt
\evensidemargin 0pt
\textwidth 6.5in
\topmargin 0pt
\textheight 9.0in

\usepackage{xcolor}
% Definindo novas cores
\definecolor{verde}{rgb}{0.25,0.5,0.35}
\definecolor{jpurple}{rgb}{0.5,0,0.35}
% Configurando layout para mostrar codigos em Python
\usepackage{listings}
\lstset{
  language=C,
  basicstyle=\ttfamily\small, 
  keywordstyle=\color{blue}\bfseries, %jpurple
  stringstyle=\color{red},
  commentstyle=\color{verde},
  morecomment=[s][\color{blue}]{/**}{*/},
  extendedchars=true, 
  showspaces=false, 
  showstringspaces=false, 
  numbers=left,
  numberstyle=\tiny,
  breaklines=true, 
  backgroundcolor=\color{cyan!10}, 
  breakautoindent=true, 
  captionpos=b,
  xleftmargin=0pt,
  tabsize=4
}

\usepackage{listings}
\lstset{
  language=Python,
  basicstyle=\ttfamily\small, 
  keywordstyle=\color{blue}\bfseries, %jpurple
  stringstyle=\color{red},
  commentstyle=\color{verde},
  morecomment=[s][\color{blue}]{/**}{*/},
  extendedchars=true, 
  showspaces=false, 
  showstringspaces=false, 
  numbers=left,
  numberstyle=\tiny,
  breaklines=true, 
  backgroundcolor=\color{cyan!10}, 
  breakautoindent=true, 
  captionpos=b,
  xleftmargin=0pt,
  tabsize=4
}

\newtheorem{thm}{Teorema}
\newtheorem{defn}{Definição}[chapter]
\newtheorem{exerc}{Exercício}[chapter]
\newtheorem{exmp}{Exemplo}[chapter]
\newtheorem{prob}{Problema.}[chapter]
\newtheorem{property}{Propriedade.}[chapter]
\newcommand{\R}{\mathbb{R}}

\title{\LARGE \bf \textcolor{blue}{Título provisório do trabalho:} Algoritmos de ordenação, suas análises de complexidade de tempo, espaço, implementações e experimentos computações}
\author{Alunas: Aline Milene Martins dos Santos\\
                Camila ...\\
                Fernanda Sayuri \\\\
Supervisor: Cláudio Nogueira de Meneses}
% {\footnotesize claudio.meneses@ufabc.edu.br}


\begin{document}
\maketitle
\listofalgorithms
\tableofcontents

\chapter{Introducao}

VEJA \href{https://en.wikipedia.org/wiki/Sorting_algorithm}{\textcolor{blue}{Sort algorithms}}

Nesta monografia apresentamos ?? algoritmos de ordenação conhecidos, suas implementações em três linguagens de programação (C, C++ e Python), realizamos experimentos computacionais dessas implementações sobre conjuntos de instâncias do problema de ordenação. Além disso, discutimos os experimentos etc..    

Estes algoritmos podem ser divididos em três tipos: (a) para realizar a ordenação são usadas operações de comparação, (b) não são usadas operações de comparação para realizar a ordenação e (c) algoritmos híbridos. Alguns exemplos são:
\begin{itemize}
    \item Baseados em comparações:  selection sort, bubble sort, insertion sort, merge sort, quick sort, heap sort, cycle sort, 3-way merge sort;

    \item Não baseados em comparações: counting sort, radix sort, bucket sort, tim sort, comb sort, pigeonhole sort;

    \item Híbridos: intro sort, tim sort.
\end{itemize}

\section{Características}

\begin{itemize}
    \item In-place: Um algoritmo de ordenação é "in-place" quando não necessita de espaço de memória adicional para armazenar o array de entrada, ou seja, utiliza o mesmo espaço de memória do array original para realizar a ordenação.
    \item Estável: Um algoritmo é "estável" quando mantém a ordem relativa dos elementos que possuem valores iguais. Por exemplo, se dois elementos A e B são iguais, e A aparece antes de B no array original, um algoritmo estável garantirá que A continue aparecendo antes de B após a ordenação.
    \item Adaptativo: Um algoritmo de ordenação é "adaptativo" quando leva em consideração a ordenação inicial do array de entrada. O tempo de execução desse tipo de algoritmo pode melhorar se o array já estiver parcialmente ordenado.
    \item Recursivo vs. Não Recursivo:
        \begin{itemize}
            \item Recursivo: São algoritmos que se chamam para resolver subproblemas, como o Quicksort e o Mergesort.
            \item Não Recursivo: São algoritmos que utilizam laços (loops) em vez de recursão para processar os dados, como o Selection Sort e o Insertion Sort.
        \end{itemize}
    \item Serial vs. Paralelo:
        \begin{itemize}
            \item Serial: São os algoritmos tradicionais que executam uma sequência de operações, uma após a outra.
            \item Paralelo: Projetados para rodar em arquiteturas de múltiplos processadores.
        \end{itemize}
    \item Online vs. Offline:
        \begin{itemize}
            \item Online: Um algoritmo online pode processar a entrada à medida que ela chega, sem precisar de todos os dados de uma vez.
            \item Offline: Um algoritmo offline requer que toda a entrada esteja disponível desde o início para poder funcionar.
        \end{itemize}
\end{itemize}

\section{Notações e Conceitos básicos}
\textcolor{blue}{
Denotamos por
$\mathbb{R}$ o conjunto dos números reais, 
$\mathbb{R}_{\geq 0}$ o conjunto dos números reais não negativos e
$\mathbb{R}_{>0}$ o conjunto dos números reais positivos.
}
\begin{defn} [Notação \textit{``Big O''}]
Dadas as funções $f:\mathbb{R}_{\geq 0}\rightarrow \mathbb{R}_{\geq 0}$ e $g:\mathbb{R}_{\geq 0}\rightarrow \mathbb{R}_{\geq 0}$, dizemos que $f(n)$ é $O(g(n))$ se existem constantes $c\in\mathbb{R}_{>0}$, $n_0\in \mathbb{R}_{\geq 0}$ tal que $f(n) \leq c\cdot g(n)$  para todo $n \geq n_0$. 
\end{defn}
\vspace{0.2cm}

Uma maneira equivalente de definir a notação \textit{``Big O''} é a seguinte:\vspace{0.2cm}

\begin{defn}
$O(g(n)) = \{ f(n):$ existem constantes $c\in\mathbb{R}_{>0}$, $n_0\in \mathbb{R}_{\geq 0}$ tal que $0 \leq f(n) \leq c\cdot g(n)$ para todo $n \geq n_0.\}$
\end{defn}
\vspace{0.3cm}

A notação ``Big O'' representa um \textcolor{blue}{limite superior no tempo de execução} de um algoritmo.  Assim, ela fornece a \textcolor{blue}{complexidade de tempo de pior caso} do algoritmo.

\begin{exmp}%\footnotesize
Seja $f(n):\mathbb{R}_{\geq 0}\rightarrow \mathbb{R}_{\geq 0}$ definida por $f(n) = 3 n^2+ 4 n^{3/2}$. Prove que $f(n)\in O(n^2)$. 

\noindent{\textbf{Prova:}} \\
Para provar que $f(n)\in O(n^2)$, precisamos mostrar que existem constantes $c\in \mathbb{R}_{>0},n_0 \in \mathbb{R}_{\geq 0}$ tal que $f(n) \leq c\cdot n^2$ para $n\geq n_0$.\vspace{0.3cm} 

Como $n^{3/2} \leq n^2$ para $n\geq 0$, então 
$$f(n)=3 n^2+ 4 n^{3/2} \leq 3 n^2 + 4 n^2 = 7\cdot n^2 \quad \forall\; n\geq 0.$$  
Fazendo $c=7$ e $n_0=0$ obtemos $f(n)\leq c\cdot n^2$ para $n\geq n_0$.  Portanto, $f(n)\in O(n^2)$.   
$\hfill\Box$
% usar uns exemplos mostrados no artigo \href{https://www.cs.toronto.edu/~vassos/teaching/c73/handouts/brief-complexity.pdf}{dd}
\end{exmp}

\begin{exmp}
Seja $f(n):\mathbb{R}_{\geq 0}\rightarrow \mathbb{R}_{\geq 0}$ definida por $f(n)=(n-5)^2$. Prove que $f(n)\in O(n^2)$. 
\vspace{0.2cm}

\noindent{\textbf{Prova:}} \\

Para provar que $f(n)\in O(n^2)$, precisamos mostrar que existem constantes $c\in \mathbb{R}_{>0},n_0 \in \mathbb{R}_{\geq 0}$ tal que $f(n) \leq c\cdot n^2$ para $n\geq n_0$.\vspace{0.3cm}

Como $(n-5)^2=n^2 -10n +25$ e $-10n + 25 \leq n^2$ para $n\geq 3$, então
$$
f(n)=(n-5)^2=n^2 -10n +25 \leq 2\cdot n^2 \quad \forall\; n\geq 3.
$$ 

Fazendo $c=2,n_0 =3$ obtemos $f(n)\leq c\cdot n^2$ 
 para $n\geq n_0$.  Portanto, $f(n)\in O(n^2)$.  
$\hfill\Box$
\end{exmp}

\begin{property}
Seja $f(n):\mathbb{R}_{\geq 0}\rightarrow \mathbb{R}_{\geq 0}$ uma função. Se $f$ puder ser escrita como uma soma finita de outras funções, então a que cresce mais rápido determina a ordem de $f(n)$. 
\end{property}\vspace{0.2cm}

\begin{exmp}
Seja $f(n):\mathbb{R}_{\geq 0}\rightarrow \mathbb{R}_{\geq 0}$ definida por $f(n)=9\log n + 5(\log n)^3 +3n^2 + 2n^3$. Então $f(n)\in O(n^3)$ quando $n\rightarrow \infty$.    
\vspace{0.2cm}

\noindent{\textbf{Prova:}} \\

Para provar que $f(n)\in O(n^{3})$, precisamos mostrar que existem constantes $c\in\mathbb{R}_{>0}$, $n_{0}\in\mathbb{R}_{\ge0}$ tal que $f(n)\le c\cdot n^{3}$ para $n\ge n_{0}$.

Para valores de $n$ suficientemente grandes, podemos observar a relação de crescimento dos termos:
\begin{itemize}
    \item $9\log n \le 9n^{3}$
    \item $5(\log n)^{3} \le 5n^{3}$
    \item $3n^{2} \le 3n^{3}$
    \item $2n^{3} \le 2n^{3}$
\end{itemize}

Somando as desigualdades para um $n$ suficientemente grande (por exemplo, $n \ge 1$), obtemos:
$$
f(n) = 9\log n+5(\log n)^{3}+3n^{2}+2n^{3} \le 9n^{3} + 5n^{3} + 3n^{3} + 2n^{3} = 19n^{3}
$$

Fazendo $c=19$ e $n_{0}=1$, obtemos $f(n)\le c\cdot n^{3}$ para $n\ge n_{0}$. Portanto, $f(n)\in O(n^{3})$.
\end{exmp}

\begin{exerc}
Seja $f(n):\mathbb{R}_{\geq 0}\rightarrow \mathbb{R}_{\geq 0}$ definida por $f(n)=9\log n + 5(\log n)^4 +3n^2 + 2n^3$. É verdade que $f(n)\in O(n^3)$ quando $n\rightarrow \infty$.    
\end{exerc}

\begin{defn} [Notação \textit{``little o''}]
Dadas as funções $f:\mathbb{R}_{\geq 0}\rightarrow \mathbb{R}_{\geq 0}$ e $g:\mathbb{R}_{\geq 0}\rightarrow \mathbb{R}_{\geq 0}$, dizemos que $f(n)$ é $o(g(n))$ se existem constantes $c\in\mathbb{R}_{>0}$, $n_0\in \mathbb{R}_{\geq 0}$ tal que $f(n) < c\cdot g(n)$ para todo $n \geq n_0$. 
\end{defn}
\vspace{0.2cm}
% \begin{exmp}
% Se $f(n) = n^2$, então $f(n) = O(n^2)$.
% usar uns exemplos mostrados no artigo \href{https://www.cs.toronto.edu/~vassos/teaching/c73/handouts/brief-complexity.pdf}{dd}
% \end{exmp}

Uma maneira equivalente de definir a notação \textit{``little o''} é a seguinte:
\vspace{0.3cm}

\begin{defn}
$o(g(n)) = \{ f(n):$ existem constantes $c\in\mathbb{R}_{>0}$, $n_0\in \mathbb{R}_{\geq 0}$ tal que $0 \leq f(n) < c\cdot g(n)$ para todo $n \geq n_0.\}$
\end{defn}\vspace{0.3cm}

A notação \textit{``little o''}  representa um \textcolor{blue}{limite estritamente superior no tempo de execução} de um algoritmo. 

\begin{defn} [Notação \textit{``Big omega''}]
Dadas as funções $f:\mathbb{R}_{\geq 0}\rightarrow \mathbb{R}_{\geq 0}$ e $g:\mathbb{R}_{\geq 0}\rightarrow \mathbb{R}_{\geq 0}$, dizemos que $f(n) \in \Omega(g(n))$ se existem constantes $c\in\mathbb{R}_{>0}$, $n_0\in \mathbb{R}_{\geq 0}$ tal que $0\leq c\cdot g(n) \leq f(n)$  para todo $n \geq n_0$. 
\end{defn}   \vspace{0.2cm}

Uma maneira equivalente de definir a notação \textit{``Big omega''}  é a seguinte:
\vspace{0.3cm}

\begin{defn}
$\Omega(g(n)) = \{ f(n):$ existem constantes $c\in\mathbb{R}_{>0}$, $n_0\in \mathbb{R}_{\geq 0}$ tal que $0 \leq c\cdot g(n) \leq f(n)$ para todo $n \geq n_0.\}$
\end{defn}\vspace{0.3cm}

A notação \textit{``Big omega''} representa um \textcolor{blue}{limite inferior no tempo de execução} de um algoritmo.  Assim, ela fornece a \textcolor{blue}{complexidade de tempo de melhor caso} do algoritmo.

\begin{defn} [Notação \textit{``little omega''}]
Dadas as funções $f:\mathbb{R}_{\geq 0}\rightarrow \mathbb{R}_{\geq 0}$ e $g:\mathbb{R}_{\geq 0}\rightarrow \mathbb{R}_{\geq 0}$, dizemos que $f(n)$ é $\omega(g(n))$ se existem constantes $c\in\mathbb{R}_{>0}$, $n_0\in \mathbb{R}_{\geq 0}$ tal que $0\leq c\cdot g(n) < f(n)$  para todo $n \geq n_0$. 
\end{defn}   \vspace{0.2cm}

Uma maneira equivalente de definir a notação \textit{``little omega''} é a seguinte:
\vspace{0.3cm}    

\begin{defn}
$\omega(g(n)) = \{ f(n):$ existem constantes $c\in\mathbb{R}_{>0}$, $n_0\in \mathbb{R}_{\geq 0}$ tal que $0 \leq c\; g(n) < f(n)$ para todo $n \geq n_0.\}$
\end{defn}\vspace{0.3cm}

A notação \textit{``little omega''}  representa um \textcolor{blue}{limite estritamente inferior no tempo de execução} de um algoritmo. 

\begin{exmp}
Seja $f(n):\mathbb{R}_{\geq 0}\rightarrow \mathbb{R}_{\geq 0}$ definida por $f(n) = n^2$. Então $f(n)=n^2 \in \omega(n)$ porque fazendo $c=1$ e $n_0=2$, temos $n < c\cdot n^2$ para todo $n\geq n_0$.
\end{exmp}

A notação ``Big O'' faz parte de uma família de notações criadas pelos matemáticos alemães Paul Bachmann e Edmund Landau, comumente chamadas de notação Bachmann–Landau ou notação assintótica.  
A letra $O$ representa ``ordem de aproximação''.
\vspace{0.5cm}

Na ciência da computação, a notação ``Big O'' é usada para classificar algoritmos de acordo com a forma como seus requisitos de tempo de execução ou espaço crescem conforme o tamanho da entrada cresce.   

\begin{defn} [Notação \textit{``theta''}]
Dadas as funções $f:\mathbb{R}_{\geq 0}\rightarrow \mathbb{R}_{\geq 0}$ e $g:\mathbb{R}_{\geq 0}\rightarrow \mathbb{R}_{\geq 0}$, dizemos que $f(n)$ é $\Theta(g(n))$ se existem constantes $c_1\in\mathbb{R}_{>0},c_2\in\mathbb{R}_{>0}$, $n_0\in \mathbb{R}_{\geq 0}$ tal que $0\leq c_1\cdot g(n) \leq f(n) \leq c_2\cdot g(n)$  para todo $n \geq n_0$. 
\end{defn}   \vspace{0.2cm}

Uma maneira equivalente de definir a notação \textit{``theta''} é a seguinte:
\vspace{0.3cm}    

\begin{defn}
$\Theta(g(n)) = \{ f(n):$ existem constantes $c_1\in\mathbb{R}_{>0},c_2\in\mathbb{R}_{>0}$, $n_0\in \mathbb{R}_{\geq 0}$ tal que $0 \leq c_1\cdot g(n) \leq f(n) \leq c_2\cdot g(n)$ para todo $n \geq n_0.\}$
\end{defn}\vspace{0.3cm}

A notação \textit{``theta''}  representa um \textcolor{blue}{limite inferior e um superior no tempo de execução} de um algoritmo. 

\begin{table}
    \centering
    \begin{tabular}{l|l}
         Notação     & Significado \\\hline
         $O(1)$      & Constante \\
         $O(\log n))$ & logarítmica\\
         $O(n)$      & linear \\
         $O(n\log n)$ & log linear\\
         $O(n^2)$    & quadrática \\
         $O(n^3)$    & cúbica\\
          $O(nK)$    & pseudo-polinomial\\
         $O(2^n)$    & exponencial \\
         $O(n!)$     & fatorial \\\hline
    \end{tabular}
    \caption{Ordens de complexidades tempo/espaço de algoritmos, onde $n$ e $K$ são parâmetros de entrada de um problema para o qual os algoritmos foram desenvolvidos.}
    \label{tab:my_label}
\end{table}

\noindent{Complexidade de tempo constante, O(1)}

Um algoritmo tem complexidade de tempo constante, representado por $O(1)$ em notação \textit{Big O}, quando o tempo de execução do algoritmo não depende do tamanho da instância do problema.\vspace{0.2cm}

\begin{exmp}
Suponha que $v$ é um vetor com $n$ números reais e que $v$ esteja ordenado. Assuma que $v=[v_1,v_2,\ldots,v_n]$.  
Dado este vetor ordenado, o problema de encontrar o valor mediano dos números neste vetor pode ser resolvido em tempo constante da seguinte maneira.  
Se $n$ é ímpar, então o valor mediano dos elementos em $v$ é $v_{\lfloor n/2 \rfloor +1}$.  
Se $n$ é par, então o valor mediano é $\frac{v_{n/2} + v_{(n+1)/2}}{2}$.        
\end{exmp}


\noindent{Complexidade de tempo linear, O(n)}

Um algoritmo tem complexidade de tempo linear quando o tempo de execução do algoritmo cresce linearmente com tamanho da instância do problema.

\noindent{Complexidade de tempo logarítmica, $O(\log n)$}

Dizer que um algoritmo tem complexidade de tempo $O(\log n)$ significa que o tempo de execução do algoritmo cresce linearmente quando o tamanho da instância do problema cresce exponencialmente. \vspace{0.3cm}

\begin{exmp}
Suponha que a complexidade de tempo de um algoritmo é dada por uma função logarítmica em termos dos dados de entrada do problema.  
Para efeito de explicação, suponha que o(s) logaritmo(s) está(ão) na base 10. 
Se o algoritmo processa 10 elementos em um segundo, então ele processará 100 elementos em 2 segundos e processará 1000 elementos em três segundos.
\end{exmp}

\noindent{Complexidade de tempo $O(\log n)$}

\begin{exerc}
Suponha que temos um vetor, $arr$, com números racionais e que ele esteja ordenado em ordem não crescente. Dado um número racional $y$, como podemos verificar se $y$ está ou não em $arr$ não tendo que examinar cada elemento do vetor. 
\end{exerc}


\noindent{\textbf{Solução:}}
Podemos resolver este problema usando um algoritmo que realiza uma pesquisa binária em $arr$ a procura por $y$.$\hfill\Box$ 

\noindent{Complexidade de tempo $O(\log n)$}
...

\noindent{Complexidade de tempo quadrática, $O(n^2)$}

\begin{exmp}
O algoritmo abaixo tem complexidade de tempo $O(n^2)$, para um dado número inteiro positivo $n$:
\end{exmp}

Observe que um algoritmo tem complexidade de tempo não linear quando o tempo de execução do algoritmo cresce não linearmente com o tamanho da instância do problema.

\noindent{Complexidade de tempo quadrática, $O(n^3)$}

\begin{exmp}
Suponha um conjunto finito não vazio com $n$ elementos. Suponha também que desejamos enumerar todas as triplas, $(a,b,c)$, de elementos do conjunto. Um algoritmo para resolver este problema terá necessariamente complexidade de tempo $O(n^3)$.   
\end{exmp}

\noindent{Complexidade de tempo quadrática, $O(2^n)$}

\begin{exmp}
Suponha um conjunto finito não vazio, $S$, contendo $n$ elementos. Suponha também que desejamos enumerar todos os subconjuntos de $S$. Como há $2^n$ subconjuntos de $S$, então um algoritmo para enumerar esses subconjuntos terá necessariamente complexidade de tempo $O(2^n)$.   
\end{exmp}

\noindent{Complexidade de tempo fatorial, $O(n!)$}

\begin{exmp}
Suponha um conjunto finito não vazio, $S$, contendo $n$ elementos. Suponha também que desejamos enumerar todas as permutações com $n$ elementos do conjunto $S$. Como há $n!$ permutações com $n$ elementos do conjunto $S$, então um algoritmo para enumerar essas permutações terá necessariamente complexidade de tempo $O(n!)$.   
\end{exmp}

\section{Lista dos algoritmos desenvolvidos em cada década}

\begin{itemize}
    \item 1941--1950: Insertion sort, Merge sort
    \item 1951--1960: Bubble sort, Counting sort, Selection sort, Shell sort, Quicksort, Shaker sort
    \item 1961--1970: Heapsort, Radix sort, Odd-Even sort
    \item 1971--1980: Bucket sort, Gnome Sort, Pancake Sort
    \item 1981--1990: Combsort
    \item 1991--2000: Introsort
    \item 2001--2010: Timsort
    \item 2011--2020: IPS\textsuperscript{4}o (In-place Parallel Super Scalar Samplesort)
    \item 2021--2025: Versões modernas de Radix Sort em GPU
\end{itemize}

Ver:
\href{https://craftofcoding.wordpress.com/2021/09/22/the-origins-of-bubble-sort/}{The origins of Bubble sort}

Ver: \href{https://users.cs.duke.edu/~ola/bubble/bubble.html}{Bubble Sort: An Archaeological Algorithmic Analysis}

Neste link, veja a seção sobre Measuring Ease of Coding, especificamente a métrica  Halstead Metrics.

\textcolor{blue}{Sugiro que vocês leiam:}
\begin{itemize}
\item 
\href{https://ieeexplore.ieee.org/document/10444609}{Performance Analysis of Various Sorting Algorithms: Comparison and Optimization}

\item
\href{https://www.devzery.com/post/fastest-sorting-algorithms-performance-guide-2025}{Fastest Sorting Algorithms: Performance Guide \& Comparison 2025}
\end{itemize}


\chapter{Algoritmos com complexidades de tempo 
\texorpdfstring{$O(n)$}{O(n)}}

Nesta seção apresentamos algoritmos de ordenação cuja complexidade de tempo pode ser considerada linear em determinados cenários. Esses métodos exploram propriedades dos dados de entrada, como o tamanho do domínio ou a representação numérica. 

\textcolor{blue}{Sugiro que vocês expressem os cálculos das complexidades considerando as constantes envolvidas nas funções. Posso explicar isto na aula da segunda semana de outubro.}

\section{Counting sort}
\textbf{Descrição:} O Counting Sort não utiliza comparações entre elementos. Em vez disso, ele conta quantas vezes cada valor aparece e, com base nessas contagens, reconstrói o vetor ordenado. É especialmente eficiente quando os valores de entrada pertencem a um intervalo pequeno em relação ao número de elementos.

\begin{exmp}
Considere ordenar o vetor $A = [4, 2, 2, 8, 3, 3, 1]$ usando o \textit{Counting Sort}.  

\begin{enumerate}
    \item \textbf{Contagem da frequência:}  
    Criamos um vetor $count$ com tamanho igual ao maior valor de $A$ (neste caso, 8) e inicializamos com zeros.  
    Em seguida, contamos quantas vezes cada número aparece:  count = [0, 1, 2, 2, 1, 0, 0, 0, 1], onde $count[i]$ indica quantas vezes o número $i$ aparece em $A$.

    \item \textbf{Cálculo das posições acumuladas:}  
    Transformamos o vetor $count$ em um vetor de posições acumuladas, somando os valores anteriores: count = [0, 1, 3, 5, 6, 6, 6, 6, 7]. Agora, $count[i]$ indica a posição final do último elemento $i$ no vetor ordenado.

    \item \textbf{Reconstrução do vetor ordenado:}  
    Percorremos o vetor $A$ da direita para a esquerda e colocamos cada elemento na posição correta no vetor $output$:  output = [1, 2, 2, 3, 3, 4, 8]. Finalmente, copiamos o vetor $output$ de volta para $A$.

\end{enumerate}
\end{exmp}

\begin{center}
\begin{minipage}{.9\linewidth}
\begin{algorithm}[H]
\DontPrintSemicolon
\textbf{countingSort(values: array of int, n: integer, k: integer)}

\For{$i \gets 0$ \KwTo $k$}{
    $count[i] \gets 0$\;
}
\For{$i \gets 0$ \KwTo $n-1$}{
    $count[values[i]] \gets count[values[i]] + 1$\;
}
\For{$i \gets 1$ \KwTo $k$}{
    $count[i] \gets count[i] + count[i-1]$\;
}
\For{$i \gets n-1$ \KwTo $0$}{
    $output[count[values[i]]-1] \gets values[i]$\;
    $count[values[i]] \gets count[values[i]] - 1$\;
}
\For{$i \gets 0$ \KwTo $n-1$}{
    $values[i] \gets output[i]$\;
}
\caption{Counting sort.}
\label{lab:alg-countingSort}
\end{algorithm}
\end{minipage}
\end{center}

\vspace{1em}

\begin{lstlisting}[language=C,caption={Counting sort em C},captionpos=t]
void countingSort(int arr[], int n, int k) {
    int count[k+1];
    int output[n];
    for(int i=0; i<=k; i++) count[i] = 0;
    for(int i=0; i<n; i++) count[arr[i]]++;
    for(int i=1; i<=k; i++) count[i] += count[i-1];
    for(int i=n-1; i>=0; i--) {
        output[count[arr[i]]-1] = arr[i];
        count[arr[i]]--;
    }
    for(int i=0; i<n; i++) arr[i] = output[i];
}
\end{lstlisting}

\begin{lstlisting}[language=C++,caption={Counting sort em C++},captionpos=t]
#include <vector>
using namespace std;

void countingSort(vector<int>& arr, int k) {
    int n = arr.size();
    vector<int> count(k + 1, 0), output(n);

    for (int num : arr)
        count[num]++;

    for (int i = 1; i <= k; i++)
        count[i] += count[i - 1];

    for (int i = n - 1; i >= 0; i--) {
        output[count[arr[i]] - 1] = arr[i];
        count[arr[i]]--;
    }

    arr = output;
}
\end{lstlisting}

\begin{lstlisting}[language=Python,caption={Counting sort em Python},captionpos=t]
def counting_sort(arr, k):
    count = [0] * (k+1)
    output = [0] * len(arr)
    for num in arr:
        count[num] += 1
    for i in range(1, len(count)):
        count[i] += count[i-1]
    for num in reversed(arr):
        output[count[num]-1] = num
        count[num] -= 1
    return output
\end{lstlisting}

\subsection{Análise de complexidade do algoritmo \ref{lab:alg-countingSort}}

Consideramos que o vetor $values$ tem comprimento $n$ e que os valores são inteiros no intervalo $[0, k]$. O algoritmo possui os seguintes \textit{loops}:

\begin{enumerate}
    \item Inicialização do vetor $count$ (tamanho $k+1$):
    \[
        \text{executa } k+1 \text{ atribuições em } O(1) \Rightarrow O(k)
    \]
    
    \item Contagem das ocorrências dos elementos do vetor $values$:
    \[
        n \text{ iterações, cada uma com operação } O(1) \Rightarrow O(n)
    \]
    
    \item Acúmulo das contagens no vetor $count$:
    \[
        k \text{ iterações, cada uma com operação } O(1) \Rightarrow O(k)
    \]
    
    \item Construção do vetor ordenado $output$:
    \[
        n \text{ iterações, cada uma com operação } O(1) \Rightarrow O(n)
    \]
    
    \item Cópia do vetor $output$ de volta para $values$:
    \[
        n \text{ iterações, cada uma com operação } O(1) \Rightarrow O(n)
    \]
\end{enumerate}

Portanto, o número total de operações é proporcional a:
\[
T(n, k) = O(k) + O(n) + O(k) + O(n) + O(n) = O(n + k)
\]

\noindent Em termos assintóticos, podemos escrever:
\[
T(n, k) \leq c_1 \cdot n + c_2 \cdot k
\]
para constantes $c_1, c_2 > 0$. Logo:
\[
T(n, k) \in O(n + k)
\]

\begin{itemize}
    \item \textbf{Melhor caso:} $O(n + k)$, pois todos os \textit{loops} sempre executam a mesma quantidade de iterações, independentemente da ordem dos elementos.
    
    \item \textbf{Caso médio:} $O(n + k)$, pois a complexidade depende apenas do tamanho do vetor e do domínio $k$, não da disposição dos elementos.
    
    \item \textbf{Pior caso:} $O(n + k)$, mesmo raciocínio: a ordem dos elementos não altera o número de operações.
\end{itemize}


\section{Radix sort}
\textbf{Descrição:} O Radix Sort ordena números inteiros processando dígito por dígito, da menor para a maior ordem, utilizando Counting Sort como sub-rotina estável. É eficiente quando o número de dígitos $d$ não é muito grande e o domínio dos dígitos é pequeno.

\begin{exmp}
Considere ordenar o vetor $A = [170, 45, 75, 90, 802, 24, 2, 66]$ usando o \textit{Radix Sort}.  

\begin{enumerate}
    \item \textbf{Ordenação pelo dígito das unidades:}  
    Extraímos o dígito das unidades de cada elemento e aplicamos Counting Sort:  
    dígitos das unidades = [0,5,5,0,2,4,2,6]  
    Após o Counting Sort pelos dígitos das unidades, o vetor fica: $[170, 90, 802, 2, 24, 45, 75, 66]$.

    \item \textbf{Ordenação pelo dígito das dezenas:}  
    Extraímos o dígito das dezenas e aplicamos Counting Sort:  
    dígitos das dezenas = [7,9,0,0,2,4,7,6]  
    Após o Counting Sort pelos dígitos das dezenas, o vetor fica: $[802, 2, 24, 45, 66, 170, 75, 90]$.

    \item \textbf{Ordenação pelo dígito das centenas:}  
    Extraímos o dígito das centenas e aplicamos Counting Sort:  
    dígitos das centenas = [1,0,0,0,8,0,0,0]  
    Após o Counting Sort pelos dígitos das centenas, o vetor final ordenado é: $[2, 24, 45, 66, 75, 90, 170, 802]$.
\end{enumerate}
\end{exmp}


\begin{center}
\begin{minipage}{.9\linewidth}
\begin{algorithm}[H]
\DontPrintSemicolon
\textbf{radixSort(values: array of int, n: integer)}

$m \gets$ maior valor em values\;
$exp \gets 1$\;
\While{$m / exp > 0$}{
    countingSort(values, n, exp)\;
    $exp \gets exp \times 10$\;
}
\caption{Radix sort.}
\label{lab:alg-radixSort}
\end{algorithm}
\end{minipage}
\end{center}

\begin{lstlisting}[language=C,caption={Radix sort em C},captionpos=t]
void countingSortRadix(int arr[], int n, int exp) {
    int output[n], count[10] = {0};
    for(int i=0; i<n; i++)
        count[(arr[i]/exp)%10]++;
    for(int i=1; i<10; i++)
        count[i] += count[i-1];
    for(int i=n-1; i>=0; i--) {
        output[count[(arr[i]/exp)%10]-1] = arr[i];
        count[(arr[i]/exp)%10]--;
    }
    for(int i=0; i<n; i++) arr[i] = output[i];
}

void radixSort(int arr[], int n) {
    int m = arr[0];
    for(int i=1; i<n; i++)
        if(arr[i] > m) m = arr[i];
    int exp = 1;
    while(m/exp > 0) {
        countingSort(arr, n, exp);
        exp *= 10;
    }
}
\end{lstlisting}

\begin{lstlisting}[language=C++,caption={Radix sort em C++},captionpos=t]
#include <vector>
#include <algorithm>
using namespace std;

void countingSortRadix(vector<int>& arr, int exp) {
    int n = arr.size();
    vector<int> output(n);
    int count[10] = {0};

    for (int num : arr)
        count[(num / exp) % 10]++;

    for (int i = 1; i < 10; i++)
        count[i] += count[i - 1];

    for (int i = n - 1; i >= 0; i--) {
        int idx = (arr[i] / exp) % 10;
        output[count[idx] - 1] = arr[i];
        count[idx]--;
    }

    arr = output;
}

void radixSort(vector<int>& arr) {
    int m = *max_element(arr.begin(), arr.end());
    for (int exp = 1; m / exp > 0; exp *= 10)
        countingSortRadix(arr, exp);
}
\end{lstlisting}

\begin{lstlisting}[language=Python,caption={Radix sort em Python},captionpos=t]
def counting_sort_radix(arr, exp):
    n = len(arr)
    output = [0] * n
    count = [0] * 10
    for num in arr:
        index = (num // exp) % 10
        count[index] += 1
    for i in range(1, 10):
        count[i] += count[i-1]
    for num in reversed(arr):
        index = (num // exp) % 10
        output[count[index]-1] = num
        count[index] -= 1
    return output

def radix_sort(arr):
    m = max(arr)
    exp = 1
    while m // exp > 0:
        arr = counting_sort_radix(arr, exp)
        exp *= 10
    return arr
\end{lstlisting}

\subsection{Análise de complexidade do algoritmo \ref{lab:alg-radixSort}}

Seja $n$ o número de elementos e $d$ o número de dígitos no maior valor. Cada passo de Counting Sort tem complexidade $O(n + k)$, onde $k=10$ (base decimal). Assim:

\[
T(n) = d \cdot O(n + k) = O(d \cdot (n + k))
\]

\begin{itemize}
    \item \textbf{Melhor caso:} $O(d \cdot (n+k))$, todos os elementos já ordenados não alteram o número de iterações.
    \item \textbf{Caso médio:} $O(d \cdot (n+k))$, a complexidade depende do número de dígitos $d$.
    \item \textbf{Pior caso:} $O(d \cdot (n+k))$, ordem dos elementos não influencia.
\end{itemize}

Complexidade de espaço: $O(n+k)$ para armazenar o vetor de contagem e o vetor de saída.

\section{Bucket sort}
\textbf{Descrição:} O Bucket Sort distribui os elementos em "baldes" (\textit{buckets}), ordena cada balde individualmente (geralmente com Insertion Sort) e depois concatena os baldes. Funciona melhor quando os elementos estão distribuídos uniformemente.

\begin{exmp}
Considere ordenar o vetor $A = [0.78, 0.17, 0.39, 0.26, 0.72, 0.94, 0.21, 0.12, 0.23, 0.68]$ usando o \textit{Bucket Sort}.  

\begin{enumerate}
    \item \textbf{Distribuição nos baldes:}  
    Criamos $n = 10$ baldes (um para cada faixa de 0.0 a 1.0) e distribuímos cada elemento de $A$ no balde correspondente:  
    \begin{itemize}
        \item Balde 0: [0.12]  
        \item Balde 1: [0.17, 0.21, 0.23, 0.26]  
        \item Balde 2: [0.39]  
        \item Balde 6: [0.68, 0.72, 0.78]  
        \item Balde 9: [0.94]
    \end{itemize}

    \item \textbf{Ordenação de cada balde:}  
    Aplicamos Insertion Sort dentro de cada balde:
    \begin{itemize}
        \item Balde 1: [0.17, 0.21, 0.23, 0.26] (já ordenado)  
        \item Balde 6: [0.68, 0.72, 0.78] (já ordenado)  
        \item Os demais baldes contêm apenas um elemento.
    \end{itemize}

    \item \textbf{Concatenação dos baldes:}  
    Unimos todos os baldes em ordem, obtendo o vetor final ordenado:  
    $[0.12, 0.17, 0.21, 0.23, 0.26, 0.39, 0.68, 0.72, 0.78, 0.94]$.
\end{enumerate}
\end{exmp}

\begin{center}
\begin{minipage}{.9\linewidth}
\begin{algorithm}[H]
\DontPrintSemicolon
\textbf{bucketSort(values: array of float, n: integer)}

\For{$i \gets 0$ \KwTo $n-1$}{
    distribuir values[i] em um balde\;
}
\For{cada balde $b$}{
    ordenar $b$ com insertionSort\;
}
concatenar todos os baldes em ordem\;

\caption{Bucket sort.}
\label{lab:alg-bucketSort}
\end{algorithm}
\end{minipage}
\end{center}

\begin{lstlisting}[language=C,caption={Bucket sort em C},captionpos=t]
void insertionSort(float arr[], int n) {
    for(int i=1;i<n;i++){
        float key=arr[i];
        int j=i-1;
        while(j>=0 && arr[j]>key){
            arr[j+1]=arr[j];
            j--;
        }
        arr[j+1]=key;
    }
}

void bucketSort(float arr[], int n){
    float buckets[n][n]; int count[n]={0};
    for(int i=0;i<n;i++){
        int index = n*arr[i];
        buckets[index][count[index]++] = arr[i];
    }
    for(int i=0;i<n;i++)
        insertionSort(buckets[i], count[i]);
    int idx=0;
    for(int i=0;i<n;i++)
        for(int j=0;j<count[i];j++)
            arr[idx++] = buckets[i][j];
}
\end{lstlisting}

\begin{lstlisting}[language=C++,caption={Bucket sort em C++},captionpos=t]
#include <vector>
#include <algorithm>
using namespace std;

void insertionSort(vector<float>& arr) {
    for (int i = 1; i < arr.size(); i++) {
        float key = arr[i];
        int j = i - 1;
        while (j >= 0 && arr[j] > key) {
            arr[j + 1] = arr[j];
            j--;
        }
        arr[j + 1] = key;
    }
}

void bucketSort(vector<float>& arr) {
    int n = arr.size();
    vector<vector<float>> buckets(n);

    for (float num : arr) {
        int idx = n * num;
        buckets[idx].push_back(num);
    }

    for (auto& b : buckets)
        insertionSort(b);

    arr.clear();
    for (auto& b : buckets)
        arr.insert(arr.end(), b.begin(), b.end());
}
\end{lstlisting}

\begin{lstlisting}[language=Python,caption={Bucket sort em Python},captionpos=t]
def insertion_sort(arr):
    for i in range(1, len(arr)):
        key = arr[i]
        j = i-1
        while j >=0 and arr[j] > key:
            arr[j+1] = arr[j]
            j -= 1
        arr[j+1] = key
    return arr

def bucket_sort(arr):
    n = len(arr)
    buckets = [[] for _ in range(n)]
    for num in arr:
        index = int(n*num)
        buckets[index].append(num)
    for i in range(n):
        buckets[i] = insertion_sort(buckets[i])
    sorted_arr = []
    for bucket in buckets:
        sorted_arr.extend(bucket)
    return sorted_arr
\end{lstlisting}

\subsection{Análise de complexidade do algoritmo \ref{lab:alg-bucketSort}}

Seja $n$ o número de elementos:

\begin{itemize}
    \item Distribuição nos baldes: $O(n)$
    \item Ordenação dos baldes (Insertion Sort): no caso médio $O(n)$, assumindo distribuição uniforme
    \item Concatenação dos baldes: $O(n)$
\end{itemize}

Portanto, a complexidade total no **caso médio** é:

\[
T(n) = O(n) + O(n) + O(n) = O(n)
\]

\begin{itemize}
    \item \textbf{Melhor caso:} $O(n)$, quando os elementos são distribuídos uniformemente.
    \item \textbf{Caso médio:} $O(n)$, para distribuição uniforme.
    \item \textbf{Pior caso:} $O(n^2)$, se todos os elementos caírem no mesmo balde (Insertion Sort domina).
\end{itemize}

Complexidade de espaço: $O(n + k)$, onde $k$ é o número de baldes (geralmente $k=n$).
\vspace{1cm}

\textcolor{blue}{Aline, leia os textos nos \textit{links} abaixo e tente melhor o seu texto neste capítulo}: 
\begin{itemize}
    \item 
\href{https://www.ic.unicamp.br/~ra063658/disciplinas/stco02_2025s1/sort_linear.pdf}{Sorting in Linear Time};

\item \href{https://ocw.mit.edu/courses/6-006-introduction-to-algorithms-fall-2011/bf7d79105762bf79bbc0925438e1468a_MIT6_006F11_lec07.pdf}{Linear-time sorting};

\item  \href{https://iudatastructurescourse.github.io/course-web-page-fall-2024/lectures/sort-linear.html}{Sorting in linear time};

\item \href{http://personal.kent.edu/~amohamm4/daa-f2019/slides/ch4-2%20LinearTime%20Sorting.pdf}{Linear-time sorting};

\item
\href{https://www.dcc.fc.up.pt/~pribeiro/aulas/aed2425/slides/4_sorting.pdf}{Sorting Algorithms}.
\end{itemize}

\chapter{Algoritmos com complexidades de tempo $O(n^2)$}

Neste capítulo apresentamos algoritmos de ordenação cujos números de operações de comparações são da forma $cn^2$, onde $n$ é o número de elementos a serem ordenados no multiconjunto e $c$ é um número real positivo.

\section{Bubble sort}
\textbf{Descrição:} O Bubble Sort compara e troca elementos adjacentes repetidamente até que a lista esteja ordenada.

\begin{exmp}
Considere ordenar $A = [5, 1, 4, 2, 8]$ usando Bubble Sort:

\begin{enumerate}
    \item Primeira passagem: comparações e trocas:
    \[
    [5,1,4,2,8] \rightarrow [1,5,4,2,8] \rightarrow [1,4,5,2,8] \rightarrow [1,4,2,5,8] \rightarrow [1,4,2,5,8]
    \]

    \item Segunda passagem:
    \[
    [1,4,2,5,8] \rightarrow [1,4,2,5,8] \rightarrow [1,2,4,5,8] \rightarrow [1,2,4,5,8]
    \]

    \item Terceira passagem:
    \[
    [1,2,4,5,8] \text{ (não há trocas, fim da ordenação)}
    \]

\end{enumerate}
Vetor final: $[1,2,4,5,8]$.
\end{exmp}

\begin{center}
\begin{minipage}{.9\linewidth}
\begin{algorithm}[H]
\DontPrintSemicolon
\textbf{bubbleSort(values: array of any, n: integer)}

\For{$i\gets 1$ \KwTo $n-1$}{
    \For{$j\gets 1$ \KwTo $n-i-1$}{
       \If{$values[j] > values[j+1]$}{
          $swap(values[j], values[j+1]$)
       }
    }
  }
\caption{Bubble sort.}
\label{lab:alg-bubbleSort}
\end{algorithm}
\end{minipage}
\end{center}

\begin{lstlisting}[language=C,caption={Bubble sort em C},captionpos=t]
void bubbleSort(int values[], int n){
    for(int i = 0; i < n - 1; i++){
        for(int j = 0; j < n - i - 1; j++){
            if (values[j] > values[j + 1]){
                swap(values[j], values[j + 1]);
            }
        }
    }
}
\end{lstlisting}

Para tratar o caso do vetor estar inicialmente ordenado, e assim não precisar ordená-lo, podemos usar a seguinte versão: 
\begin{lstlisting}[language=C,caption={Bubble sort otimizado em C},captionpos=t]
void bubbleSort(int values[], int n){
    bool swapped = true;
    for(int i = 0; i < n - 1; i++){
        swapped = false;
        for(int j = 0; j < n - i - 1; j++){
            if (values[j] > values[j + 1]){
                swap(values[j], values[j + 1]);
                swapped = true;
            }
        }
        if (swapped == false) break;
    }
}
\end{lstlisting}

\begin{lstlisting}[language=C++,caption={Bubble sort em C++},captionpos=t]
#include <vector>
using namespace std;

void bubbleSort(vector<int>& arr) {
    int n = arr.size();
    for (int i = 0; i < n - 1; i++) {
        bool swapped = false;
        for (int j = 0; j < n - i - 1; j++) {
            if (arr[j] > arr[j + 1]) {
                swap(arr[j], arr[j + 1]);
                swapped = true;
            }
        }
        if (!swapped) break;
    }
}
\end{lstlisting}

\begin{lstlisting}[language=Python,caption={Bubble sort otimizado em Python},captionpos=t]
def bubbleSort(values, n):
    for i in range(0,n):
        swapped = False  
        for j in range(0,n-i-1):
            if values[j] > values[j+1]:
                values[j],values[j+1] = values[j+1], values[j]
                swapped = True
        if not swapped: break
\end{lstlisting}

\noindent{Vale a pena executar o bubble sort sobre  diferentes vetores de entrada para diferentes valores de n. Para calcular os números de operaçoes de comparação e de troca, sugiro que você inclua no \textbf{código de teste} as variáveis count\_comp e count\_swap. Com estes valores você poderia desenhar uns gráficos.}

\subsection{Análise de complexidade do algoritmo \ref{lab:alg-bubbleSort}}

Consideramos que o vetor $values$ tem comprimento $n$.
O \textit{loop} externo itera sobre o vetor, e o \textit{loop} interno compara elementos adjacentes e os trocam se necessário. Assumimos que qualquer operação de comparação ou troca é feita em tempo constante (isto é, $O(1)$).

O \textit{loop} interno realiza $n-i-1$ operações de comparação e no máximo $n-i-1$ operações de troca. O \textit{loop} externo é executado $n-1$ vezes. Portanto, o número máximo de operações de comparação e de troca é:
\begin{align*}
T(n) & = (n-1) [(n-1) + (n-1)]\\
     & = (n-1) (2n-2)\\
     & = 2n^2 - 4n + 2   \quad\quad \text{para } n\geq 1 
\end{align*}

Em termos assintóticos, podemos escrever:
$$T(n)=2n^2 - 4n + 2 \leq 2n^2 + n^2 + n^2 \leq 4n^2\quad \forall n\geq 1.$$
Se \textcolor{blue}{$c_1=4$ e $n_0=1$} então $T(n)\in O(n^2)$ . 

A seguir fazemos uma análise de melhor, médio e pior casos. Isto é, ...
\begin{itemize}
    \item Melhor caso: $O(n)$, vetor já ordenado, apenas verificações sem trocas.

     \item Caso médio: $O(n^2)$, swaps e comparações para posições aleatórias.

     \item Pior caso: $O(n^2)$, vetor inversamente ordenado.
\end{itemize}


\section{Insertion sort}
\textbf{Descrição:} O Insertion Sort é um algoritmo de ordenação que constrói a sequência ordenada gradualmente, inserindo cada elemento na posição correta em relação aos anteriores.

\begin{exmp}
Ordenar $A = [5, 2, 4, 6, 1, 3]$:

\begin{enumerate}
    \item Inserir 2: $[2,5,4,6,1,3]$
    \item Inserir 4: $[2,4,5,6,1,3]$
    \item Inserir 6: $[2,4,5,6,1,3]$
    \item Inserir 1: $[1,2,4,5,6,3]$
    \item Inserir 3: $[1,2,3,4,5,6]$
\end{enumerate}

\end{exmp}

\begin{center}
\begin{minipage}{.9\linewidth}
\begin{algorithm}[H]
\DontPrintSemicolon
\textbf{insertionSort(values: array of any, n: integer)}

\If{$n\leq 1$}{return\quad \tcp{array is already sorted}}
\For{$i\gets 1$ \KwTo $n-1$}{
    $x\gets values[i]$\;
    $j\gets i-1$\;
    \While{$j\geq 0$ \textbf{and} $x < values[j]$}{
       $values[j+1] \gets values[j]$\;
       $j\gets j-1$\;
    }
    $values[j+1]\gets x$\;
  }
\caption{Insertion sort.}
\label{lab:alg-insertionSort}
\end{algorithm}
\end{minipage}
\end{center}
    
\begin{lstlisting}[language=C,caption={insertion sort em C},captionpos=t]
void insertionSort(int values[], int n) {
    if (n <= 1){
       return   // array is already sorted
    }  
    for (int i = 1; i < n; i++) {
        int x = values[i];
        int j = i - 1;
        while (j >= 0 &&  x < values[j]) {
            values[j + 1] = values[j];
            j = j - 1;
        }
        values[j + 1] = x;
    }
}    
\end{lstlisting}

\begin{lstlisting}[language=C++,caption={Insertion sort em C++},captionpos=t]
#include <vector>
using namespace std;

void insertionSort(vector<int>& arr) {
    for (int i = 1; i < arr.size(); i++) {
        int key = arr[i];
        int j = i - 1;
        while (j >= 0 && arr[j] > key) {
            arr[j + 1] = arr[j];
            j--;
        }
        arr[j + 1] = key;
    }
}
\end{lstlisting}

\begin{lstlisting}[language=python,caption={insertion sort em Python},captionpos=t]
def insertionSort(values, n):
    if n <= 1: return  # array is already sorted
    for i in range(1, n):  
        x = values[i]  
        j = i-1
        while j >= 0 and x < values[j]:  
            values[j+1] = values[j]  
            j -= 1
        values[j+1] = x  
\end{lstlisting}

\subsection{Análise de complexidade do algoritmo \ref{lab:alg-insertionSort}}
Consideremos um vetor $values$ de tamanho $n$. O Insertion Sort percorre o vetor a partir do segundo elemento (índice 1) e, para cada elemento, compara e desloca os elementos anteriores até encontrar a posição correta. Assumimos que cada operação de comparação ou deslocamento é $O(1)$.

\begin{enumerate}
    \item Para $i = 1$, o elemento $values[1]$ pode exigir até 1 comparação e 1 deslocamento.
    \item Para $i = 2$, o elemento $values[2]$ pode exigir até 2 comparações e 2 deslocamentos.
    \item Para $i = 3$, até 3 comparações e 3 deslocamentos.
    \item \dots
    \item Para $i = n-1$, até $n-1$ comparações e $n-1$ deslocamentos.
\end{enumerate}

Portanto, o número máximo de operações de comparação e deslocamento é:

\[
T(n) = 1 + 2 + 3 + \dots + (n-1) + 1 + 2 + \dots + (n-1) = 2 \sum_{i=1}^{n-1} i = 2 \cdot \frac{(n-1)n}{2} = n(n-1)
\]

\[
T(n) = n^2 - n \quad\quad \text{para } n \ge 1
\]

Em termos assintóticos, podemos escrever:

\[
T(n) \le n^2 \quad \forall n \ge 1 \Rightarrow T(n) \in O(n^2)
\]

\begin{itemize}
    \item \textbf{Melhor caso:} $O(n)$, quando o vetor já está ordenado. Nesse caso, o \textit{while} interno nunca executa, apenas uma comparação por iteração.
    
    \item \textbf{Caso médio:} $O(n^2)$, considerando que, em média, cada elemento precisa ser comparado com metade dos elementos anteriores.
    
    \item \textbf{Pior caso:} $O(n^2)$, quando o vetor está inversamente ordenado. Cada elemento será comparado e deslocado até a primeira posição.
\end{itemize}

\noindent\textbf{Espaço auxiliar:} $O(1)$, já que as operações são realizadas \textit{in-place}, apenas uma variável temporária $x$ é necessária. 

\section{Comb sort}
\textbf{Descrição:} O Comb Sort é uma melhoria do Bubble Sort que tenta eliminar \textit{turtles} (elementos pequenos próximos ao final do vetor) usando um \textit{gap} inicial maior que 1, que vai sendo reduzido até chegar em 1. Quando o gap é 1, o Comb Sort se comporta como o Bubble Sort tradicional.

\begin{exmp}
Ordenar $A = [5, 1, 4, 2, 8]$ usando Comb Sort:

\begin{enumerate}
    \item Gap inicial: $5/1.3 \approx 3$ \\
    Comparações: $(5,2)$ e $(1,8)$. Trocas: $[5,1,2,4,8]$.
    \item Gap reduzido para 2. Comparações e trocas: $[2,1,5,4,8] \rightarrow [2,1,4,5,8]$.
    \item Gap reduzido para 1. Agora é igual ao Bubble Sort: $[1,2,4,5,8]$.
\end{enumerate}

Vetor final: $[1,2,4,5,8]$.
\end{exmp}

\begin{center}
\begin{minipage}{.9\linewidth}
\begin{algorithm}[H]
\DontPrintSemicolon
\textbf{combSort(values: array of any, n: integer)}

$gap \gets n$\;
$shrink \gets 1.3$\;
$sorted \gets false$\;

\While{not $sorted$}{
    $gap \gets \lfloor gap/shrink \rfloor$\;
    \If{$gap \leq 1$}{
        $gap \gets 1$\;
        $sorted \gets true$\;
    }
    \For{$i \gets 0$ \KwTo $n-gap-1$}{
        \If{$values[i] > values[i+gap]$}{
            $swap(values[i], values[i+gap])$\;
            $sorted \gets false$\;
        }
    }
}
\caption{Comb sort.}
\label{lab:alg-combSort}
\end{algorithm}
\end{minipage}
\end{center}

\begin{lstlisting}[language=C,caption={Comb sort em C},captionpos=t]
void combSort(int arr[], int n) {
    int gap = n;
    const float shrink = 1.3;
    int sorted = 0;

    while (!sorted) {
        gap = (int)(gap / shrink);
        if (gap <= 1) {
            gap = 1;
            sorted = 1;
        }
        sorted = 1;
        for (int i = 0; i + gap < n; i++) {
            if (arr[i] > arr[i + gap]) {
                int temp = arr[i];
                arr[i] = arr[i + gap];
                arr[i + gap] = temp;
                sorted = 0;
            }
        }
    }
}
\end{lstlisting}

\begin{lstlisting}[language=C++,caption={Comb sort em C++},captionpos=t]
#include <vector>
#include <cmath>
using namespace std;

void combSort(vector<int>& arr) {
    int n = arr.size();
    int gap = n;
    bool swapped = true;

    while (gap > 1 || swapped) {
        gap = max(1, (int)(gap / 1.3));
        swapped = false;
        for (int i = 0; i + gap < n; i++) {
            if (arr[i] > arr[i + gap]) {
                swap(arr[i], arr[i + gap]);
                swapped = true;
            }
        }
    }
}
\end{lstlisting}

\begin{lstlisting}[language=python,caption={Comb sort em Python},captionpos=t]
def combSort(values):
    n = len(values)
    gap = n
    shrink = 1.3
    sorted = False
    while not sorted:
        gap = int(gap / shrink)
        if gap <= 1:
            gap = 1
            sorted = True
        i = 0
        while i + gap < n:
            if values[i] > values[i + gap]:
                values[i], values[i + gap] = values[i + gap], values[i]
                sorted = False
            i += 1
\end{lstlisting}

\subsection{Análise de complexidade do algoritmo \ref{lab:alg-combSort}}
Considere um vetor $values$ de tamanho $n$. O Comb Sort executa passagens sobre o vetor com gaps decrescentes até 1.

\begin{itemize}
    \item \textbf{Número de comparações:}  
    Aproximadamente $O(n \log n)$ no melhor caso (pois os gaps grandes reduzem rapidamente as inversões) e até $O(n^2)$ no pior caso, pois quando $gap=1$ o algoritmo se comporta como Bubble Sort.
    
    \item \textbf{Melhor caso:} $O(n \log n)$, vetor já ordenado.
    \item \textbf{Caso médio:} Melhor que Bubble Sort, mas ainda quadrático. Aproximadamente $O(n^2 / 2^p)$, dependendo do fator \textit{shrink}.
    \item \textbf{Pior caso:} $O(n^2)$, vetor inversamente ordenado.
    \item \textbf{Espaço auxiliar:} $O(1)$, ordenação in-place.
\end{itemize}

\section{Selection sort}
\textbf{Descrição:} O Selection Sort percorre o vetor repetidamente, encontrando o menor elemento e colocando-o na posição correta. É um algoritmo simples, porém pouco eficiente em grandes conjuntos.

\begin{exmp}
Ordenar $A = [29, 10, 14, 37, 13]$:

\begin{enumerate}
    \item Encontrar o menor elemento (10) e trocar com o primeiro: $[10, 29, 14, 37, 13]$
    \item Encontrar o menor do restante (13) e trocar com o segundo: $[10, 13, 14, 37, 29]$
    \item Encontrar o menor do restante (14), já está na posição correta.
    \item Encontrar o menor do restante (29) e trocar com o quarto: $[10, 13, 14, 29, 37]$
    \item Último elemento já está correto.
\end{enumerate}

Vetor final: $[10, 13, 14, 29, 37]$.
\end{exmp}

\begin{center}
\begin{minipage}{.9\linewidth}
\begin{algorithm}[H]
\DontPrintSemicolon
\textbf{selectionSort(values: array of any, n: integer)}

\For{$i \gets 0$ \KwTo $n-2$}{
    $minIndex \gets i$\;
    \For{$j \gets i+1$ \KwTo $n-1$}{
        \If{$values[j] < values[minIndex]$}{
            $minIndex \gets j$\;
        }
    }
    \If{$minIndex \neq i$}{
        $swap(values[i], values[minIndex])$\;
    }
}
\caption{Selection sort.}
\label{lab:alg-selectionSort}
\end{algorithm}
\end{minipage}
\end{center}

\begin{lstlisting}[language=C,caption={Selection sort em C},captionpos=t]
void selectionSort(int values[], int n){
    for(int i = 0; i < n-1; i++){
        int minIndex = i;
        for(int j = i+1; j < n; j++){
            if(values[j] < values[minIndex]){
                minIndex = j;
            }
        }
        if(minIndex != i){
            int temp = values[i];
            values[i] = values[minIndex];
            values[minIndex] = temp;
        }
    }
}
\end{lstlisting}

\begin{lstlisting}[language=C++,caption={Selection sort em C++},captionpos=t]
#include <vector>
using namespace std;

void selectionSort(vector<int>& arr) {
    int n = arr.size();
    for (int i = 0; i < n - 1; i++) {
        int minIndex = i;
        for (int j = i + 1; j < n; j++) {
            if (arr[j] < arr[minIndex])
                minIndex = j;
        }
        swap(arr[i], arr[minIndex]);
    }
}
\end{lstlisting}

\begin{lstlisting}[language=python,caption={Selection sort em Python},captionpos=t]
def selectionSort(values, n):
    for i in range(n):
        min_index = i
        for j in range(i+1, n):
            if values[j] < values[min_index]:
                min_index = j
        values[i], values[min_index] = values[min_index], values[i]
\end{lstlisting}

\subsection{Análise de complexidade do algoritmo \ref{lab:alg-selectionSort}}
Considere um vetor $values$ de tamanho $n$.

\begin{itemize}
    \item O \textit{loop} externo executa $n-1$ vezes.
    \item O \textit{loop} interno executa $(n-1)+(n-2)+\dots+1 = n(n-1)/2$ comparações.
\end{itemize}

\noindent\textbf{Número de comparações:}  
\[
C(n) = \frac{n(n-1)}{2} \approx \frac{n^2}{2}
\]

\noindent\textbf{Número de trocas:}  
No máximo $n-1$, uma por iteração do loop externo.

\begin{itemize}
    \item \textbf{Melhor caso:} $O(n^2)$ comparações, $O(n)$ trocas.
    \item \textbf{Caso médio:} $O(n^2)$ comparações, $O(n)$ trocas.
    \item \textbf{Pior caso:} $O(n^2)$ comparações, $O(n)$ trocas.
\end{itemize}

\noindent\textbf{Espaço auxiliar:} $O(1)$, pois a ordenação é feita in-place.

\section{Shell sort}
\textbf{Descrição:} O Shell Sort é uma generalização do Insertion Sort que permite comparações entre elementos distantes, utilizando um \textit{gap} que vai sendo reduzido até 1. Com isso, o algoritmo reduz o número de movimentações, tornando-o mais eficiente em relação ao Insertion Sort puro.

\begin{exmp}
Ordenar $A = [23, 12, 1, 8, 34, 54, 2, 3]$:

\begin{enumerate}
    \item Gap inicial $= 4$: comparações entre pares separados por 4 posições.
    Após ajustes: $[23, 12, 1, 3, 34, 54, 2, 8]$.
    \item Gap reduzido para $2$: $[1, 3, 2, 8, 23, 12, 34, 54]$.
    \item Gap reduzido para $1$: funciona como Insertion Sort.
    Resultado final: $[1, 2, 3, 8, 12, 23, 34, 54]$.
\end{enumerate}
\end{exmp}

\begin{center}
\begin{minipage}{.9\linewidth}
\begin{algorithm}[H]
\DontPrintSemicolon
\textbf{shellSort(values: array of int, n: integer)}

\For{$gap \gets n/2$ \KwTo $1$ passo $gap/2$}{
    \For{$i \gets gap$ \KwTo $n-1$}{
        $temp \gets values[i]$\;
        $j \gets i$\;
        \While{$j \geq gap$ \textbf{and} $values[j-gap] > temp$}{
            $values[j] \gets values[j-gap]$\;
            $j \gets j-gap$\;
        }
        $values[j] \gets temp$\;
    }
}
\caption{Shell sort.}
\label{lab:alg-shellSort}
\end{algorithm}
\end{minipage}
\end{center}

\begin{lstlisting}[language=C,caption={Shell sort em C},captionpos=t]
void shellSort(int arr[], int n) {
    for (int gap = n/2; gap > 0; gap /= 2) {
        for (int i = gap; i < n; i++) {
            int temp = arr[i];
            int j = i;
            while (j >= gap && arr[j-gap] > temp) {
                arr[j] = arr[j-gap];
                j -= gap;
            }
            arr[j] = temp;
        }
    }
}
\end{lstlisting}

\begin{lstlisting}[language=C++,caption={Shell sort em C++},captionpos=t]
#include <vector>
using namespace std;

void shellSort(vector<int>& arr) {
    int n = arr.size();
    for (int gap = n / 2; gap > 0; gap /= 2) {
        for (int i = gap; i < n; i++) {
            int temp = arr[i];
            int j = i;
            while (j >= gap && arr[j - gap] > temp) {
                arr[j] = arr[j - gap];
                j -= gap;
            }
            arr[j] = temp;
        }
    }
}
\end{lstlisting}

\begin{lstlisting}[language=python,caption={Shell sort em Python},captionpos=t]
def shellSort(arr):
    n = len(arr)
    gap = n // 2
    while gap > 0:
        for i in range(gap, n):
            temp = arr[i]
            j = i
            while j >= gap and arr[j-gap] > temp:
                arr[j] = arr[j-gap]
                j -= gap
            arr[j] = temp
        gap //= 2
\end{lstlisting}

\subsection{Análise de complexidade do algoritmo \ref{lab:alg-shellSort}}
A análise depende da sequência de gaps escolhida.
\begin{itemize}
    \item \textbf{Melhor caso:} $O(n \log n)$
    \item \textbf{Caso médio:} entre $O(n^{1.25})$ e $O(n^{1.5})$, dependendo da sequência
    \item \textbf{Pior caso:} $O(n^2)$
    \item \textbf{Espaço auxiliar:} $O(1)$
\end{itemize}


\section{Gnome sort}
\textbf{Descrição:} O Gnome Sort funciona de forma semelhante ao Insertion Sort, mas realiza trocas locais repetidamente, como um "gnomo organizando vasos". Avança se os elementos estão em ordem e retrocede trocando quando não estão.

\begin{exmp}
Ordenar $A = [34, 2, 10, -9]$:

\begin{enumerate}
    \item $34 > 2$, troca: $[2, 34, 10, -9]$
    \item $34 > 10$, troca: $[2, 10, 34, -9]$
    \item $34 > -9$, troca: $[2, 10, -9, 34]$
    \item $10 > -9$, troca: $[2, -9, 10, 34]$
    \item $2 > -9$, troca: $[-9, 2, 10, 34]$
\end{enumerate}
Vetor final: $[-9, 2, 10, 34]$.
\end{exmp}

\begin{center}
\begin{minipage}{.9\linewidth}
\begin{algorithm}[H]
\DontPrintSemicolon
\textbf{gnomeSort(values: array of int, n: integer)}

$i \gets 0$\;
\While{$i < n$}{
    \If{$i == 0$ or $values[i] \geq values[i-1]$}{
        $i \gets i+1$\;
    }
    \Else{
        $swap(values[i], values[i-1])$\;
        $i \gets i-1$\;
    }
}
\caption{Gnome sort.}
\label{lab:alg-gnomeSort}
\end{algorithm}
\end{minipage}
\end{center}

\begin{lstlisting}[language=C,caption={Gnome sort em C},captionpos=t]
void gnomeSort(int arr[], int n) {
    int i = 0;
    while (i < n) {
        if (i == 0 || arr[i] >= arr[i-1]) {
            i++;
        } else {
            int temp = arr[i];
            arr[i] = arr[i-1];
            arr[i-1] = temp;
            i--;
        }
    }
}
\end{lstlisting}

\begin{lstlisting}[language=C++,caption={Gnome sort em C++},captionpos=t]
#include <vector>
using namespace std;

void gnomeSort(vector<int>& arr) {
    int n = arr.size();
    int i = 0;
    while (i < n) {
        if (i == 0 || arr[i] >= arr[i - 1])
            i++;
        else {
            swap(arr[i], arr[i - 1]);
            i--;
        }
    }
}
\end{lstlisting}

\begin{lstlisting}[language=python,caption={Gnome sort em Python},captionpos=t]
def gnomeSort(arr):
    i = 0
    n = len(arr)
    while i < n:
        if i == 0 or arr[i] >= arr[i-1]:
            i += 1
        else:
            arr[i], arr[i-1] = arr[i-1], arr[i]
            i -= 1
\end{lstlisting}

\subsection{Análise de complexidade do algoritmo \ref{lab:alg-gnomeSort}}
\begin{itemize}
    \item \textbf{Melhor caso:} $O(n)$ (vetor já ordenado)
    \item \textbf{Caso médio:} $O(n^2)$
    \item \textbf{Pior caso:} $O(n^2)$ (vetor inversamente ordenado)
    \item \textbf{Espaço auxiliar:} $O(1)$
\end{itemize}


\section{Shaker sort}
\textbf{Descrição:} O Shaker Sort (ou Cocktail Sort) é uma variação do Bubble Sort que percorre o vetor em ambas as direções alternadamente. Isso faz com que elementos pequenos "subam" rapidamente para o início, e elementos grandes "desçam" para o final.

\begin{exmp}
Ordenar $A = [5, 1, 4, 2, 8]$:

\begin{enumerate}
    \item Passagem esquerda $\to$ direita: $[1, 4, 2, 5, 8]$
    \item Passagem direita $\to$ esquerda: $[1, 2, 4, 5, 8]$
\end{enumerate}
Vetor final: $[1, 2, 4, 5, 8]$.
\end{exmp}

\begin{center}
\begin{minipage}{.9\linewidth}
\begin{algorithm}[H]
\DontPrintSemicolon
\textbf{shakerSort(values: array of int, n: integer)}

$swapped \gets true$\;
$start \gets 0$\;
$end \gets n-1$\;

\While{$swapped$}{
    $swapped \gets false$\;
    \For{$i \gets start$ \KwTo $end-1$}{
        \If{$values[i] > values[i+1]$}{
            $swap(values[i], values[i+1])$\;
            $swapped \gets true$\;
        }
    }
    $end \gets end-1$\;

    \For{$i \gets end-1$ \KwTo $start$}{
        \If{$values[i] > values[i+1]$}{
            $swap(values[i], values[i+1])$\;
            $swapped \gets true$\;
        }
    }
    $start \gets start+1$\;
}
\caption{Shaker sort.}
\label{lab:alg-shakerSort}
\end{algorithm}
\end{minipage}
\end{center}

\begin{lstlisting}[language=C,caption={Shaker sort em C},captionpos=t]
void shakerSort(int arr[], int n) {
    int start = 0, end = n - 1;
    int swapped = 1;
    while (swapped) {
        swapped = 0;
        for (int i = start; i < end; i++) {
            if (arr[i] > arr[i+1]) {
                int temp = arr[i];
                arr[i] = arr[i+1];
                arr[i+1] = temp;
                swapped = 1;
            }
        }
        end--;
        for (int i = end-1; i >= start; i--) {
            if (arr[i] > arr[i+1]) {
                int temp = arr[i];
                arr[i] = arr[i+1];
                arr[i+1] = temp;
                swapped = 1;
            }
        }
        start++;
    }
}
\end{lstlisting}

\begin{lstlisting}[language=C++,caption={Shaker sort em C++},captionpos=t]
#include <vector>
using namespace std;

void shakerSort(vector<int>& arr) {
    int n = arr.size();
    bool swapped = true;
    int start = 0, end = n - 1;

    while (swapped) {
        swapped = false;
        for (int i = start; i < end; i++) {
            if (arr[i] > arr[i + 1]) {
                swap(arr[i], arr[i + 1]);
                swapped = true;
            }
        }
        if (!swapped) break;
        swapped = false;
        end--;
        for (int i = end - 1; i >= start; i--) {
            if (arr[i] > arr[i + 1]) {
                swap(arr[i], arr[i + 1]);
                swapped = true;
            }
        }
        start++;
    }
}
\end{lstlisting}

\begin{lstlisting}[language=python,caption={Shaker sort em Python},captionpos=t]
def shakerSort(arr):
    start = 0
    end = len(arr) - 1
    swapped = True
    while swapped:
        swapped = False
        for i in range(start, end):
            if arr[i] > arr[i+1]:
                arr[i], arr[i+1] = arr[i+1], arr[i]
                swapped = True
        end -= 1
        for i in range(end-1, start-1, -1):
            if arr[i] > arr[i+1]:
                arr[i], arr[i+1] = arr[i+1], arr[i]
                swapped = True
        start += 1
\end{lstlisting}

\subsection{Análise de complexidade do algoritmo \ref{lab:alg-shakerSort}}
\begin{itemize}
    \item \textbf{Melhor caso:} $O(n)$ (já ordenado)
    \item \textbf{Caso médio:} $O(n^2)$
    \item \textbf{Pior caso:} $O(n^2)$
    \item \textbf{Espaço auxiliar:} $O(1)$
\end{itemize}


\section{Odd-Even sort}
\textbf{Descrição:} O Odd-Even Sort, também chamado de Brick Sort, é uma variação do Bubble Sort que alterna duas fases: uma onde compara pares de índices ímpares e outra de índices pares. Repete até o vetor estar ordenado.

\begin{exmp}
Ordenar $A = [5, 3, 8, 4, 2]$:

\begin{enumerate}
    \item Fase ímpar: compara (5,3), (8,4) → $[3, 5, 4, 8, 2]$
    \item Fase par: compara (5,4), (8,2) → $[3, 4, 5, 2, 8]$
    \item Repetindo → $[2, 3, 4, 5, 8]$
\end{enumerate}
\end{exmp}

\begin{center}
\begin{minipage}{.9\linewidth}
\begin{algorithm}[H]
\DontPrintSemicolon
\textbf{oddEvenSort(values: array of int, n: integer)}

$sorted \gets false$\;
\While{$\neg sorted$}{
    $sorted \gets true$\;

    \For{$i \gets 1$ \KwTo $n-2$ passo $2$}{
        \If{$values[i] > values[i+1]$}{
            $swap(values[i], values[i+1])$\;
            $sorted \gets false$\;
        }
    }

    \For{$i \gets 0$ \KwTo $n-2$ passo $2$}{
        \If{$values[i] > values[i+1]$}{
            $swap(values[i], values[i+1])$\;
            $sorted \gets false$\;
        }
    }
}
\caption{Odd-Even sort.}
\label{lab:alg-oddEvenSort}
\end{algorithm}
\end{minipage}
\end{center}

\begin{lstlisting}[language=C,caption={Odd-Even sort em C},captionpos=t]
void oddEvenSort(int arr[], int n) {
    int sorted = 0;
    while (!sorted) {
        sorted = 1;
        for (int i = 1; i < n-1; i += 2) {
            if (arr[i] > arr[i+1]) {
                int temp = arr[i];
                arr[i] = arr[i+1];
                arr[i+1] = temp;
                sorted = 0;
            }
        }
        for (int i = 0; i < n-1; i += 2) {
            if (arr[i] > arr[i+1]) {
                int temp = arr[i];
                arr[i] = arr[i+1];
                arr[i+1] = temp;
                sorted = 0;
            }
        }
    }
}
\end{lstlisting}

\begin{lstlisting}[language=C++,caption={Odd-even sort em C++},captionpos=t]
#include <vector>
using namespace std;

void oddEvenSort(vector<int>& arr) {
    int n = arr.size();
    bool sorted = false;

    while (!sorted) {
        sorted = true;
        for (int i = 1; i < n - 1; i += 2) {
            if (arr[i] > arr[i + 1]) {
                swap(arr[i], arr[i + 1]);
                sorted = false;
            }
        }
        for (int i = 0; i < n - 1; i += 2) {
            if (arr[i] > arr[i + 1]) {
                swap(arr[i], arr[i + 1]);
                sorted = false;
            }
        }
    }
}
\end{lstlisting}

\begin{lstlisting}[language=python,caption={Odd-Even sort em Python},captionpos=t]
def oddEvenSort(arr):
    n = len(arr)
    sorted = False
    while not sorted:
        sorted = True
        for i in range(1, n-1, 2):
            if arr[i] > arr[i+1]:
                arr[i], arr[i+1] = arr[i+1], arr[i]
                sorted = False
        for i in range(0, n-1, 2):
            if arr[i] > arr[i+1]:
                arr[i], arr[i+1] = arr[i+1], arr[i]
                sorted = False
\end{lstlisting}

\subsection{Análise de complexidade do algoritmo \ref{lab:alg-oddEvenSort}}
\begin{itemize}
    \item \textbf{Melhor caso:} $O(n)$ (já ordenado)
    \item \textbf{Caso médio:} $O(n^2)$
    \item \textbf{Pior caso:} $O(n^2)$
    \item \textbf{Espaço auxiliar:} $O(1)$
\end{itemize}


\section{Pancake sort}
\textbf{Descrição:} O Pancake Sort é inspirado em virar panquecas com uma espátula: a única operação permitida é inverter um prefixo do vetor. O algoritmo encontra o maior elemento e o leva ao topo, depois inverte novamente para levá-lo à posição final correta.

\begin{exmp}
Ordenar $A = [3, 6, 1, 10, 2]$:

\begin{enumerate}
    \item Maior elemento é 10 (índice 3). Inverte prefixo até 3: $[10, 1, 6, 3, 2]$
    \item Inverte prefixo até fim: $[2, 3, 6, 1, 10]$
    \item Maior dos restantes é 6. Inverte prefixo até 2: $[6, 3, 2, 1, 10]$
    \item Inverte prefixo até 3: $[1, 2, 3, 6, 10]$
\end{enumerate}
\end{exmp}

\begin{center}
\begin{minipage}{.9\linewidth}
\begin{algorithm}[H]
\DontPrintSemicolon
\textbf{pancakeSort(values: array of int, n: integer)}

\For{$currSize \gets n$ \KwTo $1$}{
    $maxIndex \gets indexOfMax(values, currSize)$\;
    \If{$maxIndex \neq currSize-1$}{
        flip(values, maxIndex)\;
        flip(values, currSize-1)\;
    }
}
\caption{Pancake sort.}
\label{lab:alg-pancakeSort}
\end{algorithm}
\end{minipage}
\end{center}

\begin{lstlisting}[language=C,caption={Pancake sort em C},captionpos=t]
void flip(int arr[], int i) {
    int start = 0;
    while (start < i) {
        int temp = arr[start];
        arr[start] = arr[i];
        arr[i] = temp;
        start++;
        i--;
    }
}
int findMaxIndex(int arr[], int n) {
    int mi = 0;
    for (int i = 1; i < n; i++)
        if (arr[i] > arr[mi]) mi = i;
    return mi;
}
void pancakeSort(int arr[], int n) {
    for (int currSize = n; currSize > 1; currSize--) {
        int mi = findMaxIndex(arr, currSize);
        if (mi != currSize-1) {
            flip(arr, mi);
            flip(arr, currSize-1);
        }
    }
}
\end{lstlisting}

\begin{lstlisting}[language=C++,caption={Pancake sort em C++},captionpos=t]
#include <vector>
#include <algorithm>
using namespace std;

int findMaxIndex(vector<int>& arr, int n) {
    int mi = 0;
    for (int i = 1; i < n; i++)
        if (arr[i] > arr[mi]) mi = i;
    return mi;
}

void flip(vector<int>& arr, int i) {
    reverse(arr.begin(), arr.begin() + i + 1);
}

void pancakeSort(vector<int>& arr) {
    for (int curr_size = arr.size(); curr_size > 1; curr_size--) {
        int mi = findMaxIndex(arr, curr_size);
        if (mi != curr_size - 1) {
            flip(arr, mi);
            flip(arr, curr_size - 1);
        }
    }
}
\end{lstlisting}

\begin{lstlisting}[language=python,caption={Pancake sort em Python},captionpos=t]
def flip(arr, i):
    arr[:i+1] = arr[:i+1][::-1]

def findMaxIndex(arr, n):
    mi = 0
    for i in range(1, n):
        if arr[i] > arr[mi]:
            mi = i
    return mi

def pancakeSort(arr):
    n = len(arr)
    for currSize in range(n, 1, -1):
        mi = findMaxIndex(arr, currSize)
        if mi != currSize-1:
            flip(arr, mi)
            flip(arr, currSize-1)
\end{lstlisting}

\subsection{Análise de complexidade do algoritmo \ref{lab:alg-pancakeSort}}
\begin{itemize}
    \item \textbf{Melhor caso:} $O(n)$ (quando já está ordenado)
    \item \textbf{Caso médio:} $O(n^2)$
    \item \textbf{Pior caso:} $O(n^2)$ (aproximadamente $2n$ flips por iteração)
    \item \textbf{Espaço auxiliar:} $O(1)$
\end{itemize}


\section{Cocktail Sort}

\section{Resumo}

\begin{center}
\begin{tabular}{||c|c|c|c||}
\hline
\multicolumn{4}{|c|}{Complexidades de tempo em termos de comparações} \\
\hline
Algoritmo & Pior caso & Melhor caso & Caso médio \\
\hline
bubble      & $O(n^2)$       & $O(n)$          & $O(n^2)$ \\
insertion   & $O(n^2)$       & $O(n)$          & $O(n^2)$ \\
combsort    & $O(n^2)$       & $O(n\log n)$    & $\approx O(n \log n)$ \\
selection   & $O(n^2)$       & $O(n^2)$        & $O(n^2)$ \\
shellsort   & $O(n^2)$       & $\Omega(n\log n)$ & $O(n^{3/2})$ \\
gnome       & $O(n^2)$       & $O(n)$          & $O(n^2)$ \\
shaker      & $O(n^2)$       & $O(n)$          & $O(n^2)$ \\
odd-even    & $O(n^2)$       & $O(n)$          & $O(n^2)$ \\
pancake (lançamentos)     & $O(n)$  & $O(n)$  & $O(n)$  \\
\hline
\end{tabular}
\end{center}

\begin{center}
\begin{tabular}{||c|c|c|c||}
\hline
\multicolumn{4}{|c|}{Complexidades de espaço} \\
\hline
Algoritmo & Pior caso & Melhor caso & Caso médio \\
\hline
bubble      & $O(1)$ & $O(1)$ & $O(1)$ \\
insertion   & $O(1)$ & $O(1)$ & $O(1)$ \\
combsort    & $O(1)$ & $O(1)$ & $O(1)$ \\
selection   & $O(1)$ & $O(1)$ & $O(1)$ \\
shellsort   & $O(1)$ & $O(1)$ & $O(1)$ \\
gnome       & $O(1)$ & $O(1)$ & $O(1)$ \\
shaker      & $O(1)$ & $O(1)$ & $O(1)$ \\
odd-even    & $O(1)$ & $O(1)$ & $O(1)$ \\
pancake     & $O(1)$ & $O(1)$ & $O(1)$ \\
\hline
\end{tabular}
\end{center}





\chapter{Algoritmos que usam operações de comparação e têm complexidade de tempo $O(n\log n)$}

Neste capítulo apresentamos algoritmos de ordenação cujos números de operações de comparações são da forma $c \cdot n \log n$, onde $n$ é o número de elementos a serem ordenados no multiconjunto e $c$ é um número real positivo.  

\section{Merge Sort}

\textbf{Descrição:} O Merge Sort é um algoritmo de ordenação baseado na estratégia \textit{dividir para conquistar}. Ele divide recursivamente o vetor em duas metades, ordena cada metade e depois intercala as duas partes em um vetor ordenado. É estável, mas não é in-place, pois exige memória auxiliar proporcional a $n$.

\begin{exmp}
Considere ordenar o vetor $A = [38, 27, 43, 3, 9, 82, 10]$ com o \textit{Merge Sort}.

\begin{enumerate}
    \item O vetor é recursivamente dividido ao meio até que os subvetores tenham tamanho 1:  
    $[38, 27, 43, 3, 9, 82, 10] \to [38, 27, 43]$, $[3, 9, 82, 10]$, e assim por diante.
    
    \item Em seguida, os subvetores são intercalados em ordem crescente:  
    $[27, 38, 43]$ e $[3, 9, 10, 82]$.
    
    \item Finalmente, os resultados são mesclados em $[3, 9, 10, 27, 38, 43, 82]$.
\end{enumerate}
\end{exmp}

\begin{algorithm}[H]
\DontPrintSemicolon
\textbf{mergeSort(A: array, l: int, r: int)}\;
\If{$l < r$}{
    $m \gets (l+r)/2$\;
    mergeSort(A, l, m)\;
    mergeSort(A, m+1, r)\;
    merge(A, l, m, r)\;
}
\caption{Merge Sort}
\label{lab:alg-mergeSort}
\end{algorithm}

\begin{lstlisting}[language=C, caption={Implementação do Merge Sort em C}, label=code:mergeSort]
#include <stdio.h>

void merge(int arr[], int l, int m, int r) {
    int n1 = m - l + 1, n2 = r - m;
    int L[n1], R[n2];
    for (int i=0; i<n1; i++) L[i] = arr[l+i];
    for (int j=0; j<n2; j++) R[j] = arr[m+1+j];
    int i=0, j=0, k=l;
    while (i<n1 && j<n2)
        arr[k++] = (L[i]<=R[j]) ? L[i++] : R[j++];
    while (i<n1) arr[k++] = L[i++];
    while (j<n2) arr[k++] = R[j++];
}

void mergeSort(int arr[], int l, int r) {
    if (l < r) {
        int m = l+(r-l)/2;
        mergeSort(arr, l, m);
        mergeSort(arr, m+1, r);
        merge(arr, l, m, r);
    }
}
\end{lstlisting}

\begin{lstlisting}[language=Python, caption={Merge Sort em Python}, label=code:mergeSortPy]
def merge_sort(arr):
    if len(arr) > 1:
        mid = len(arr)//2
        L, R = arr[:mid], arr[mid:]
        merge_sort(L); merge_sort(R)
        i = j = k = 0
        while i < len(L) and j < len(R):
            if L[i] <= R[j]:
                arr[k] = L[i]; i += 1
            else:
                arr[k] = R[j]; j += 1
            k += 1
        while i < len(L): arr[k] = L[i]; i += 1; k += 1
        while j < len(R): arr[k] = R[j]; j += 1; k += 1
\end{lstlisting}

\subsection{Análise de complexidade do algoritmo \ref{lab:alg-mergeSort}}
A cada nível de recursão, o vetor é dividido em duas partes. A mesclagem (\textit{merge}) de dois subvetores de tamanho $n/2$ custa $O(n)$. O número de níveis da árvore de recursão é $\log n$. Logo:
\[
T(n) = n \log n
\]
\begin{itemize}
    \item Melhor caso: $O(n \log n)$
    \item Caso médio: $O(n \log n)$
    \item Pior caso: $O(n \log n)$
\end{itemize}
Espaço auxiliar: $O(n)$ devido ao vetor temporário.

\textcolor{blue}{Vejam os conteúdos nos links abaixo e entendam a história por trás da criação do algoritmo:}
\begin{itemize}
    \item 
      \href{https://compileralchemy.substack.com/p/merge-sort-and-its-early-history}{Merge Sort And It's Early History}
    \item 
      \href{https://compileralchemy.substack.com/p/merge-sort-and-its-early-history}{Merge sort (von Neumann)}
\end{itemize}


\section{Quicksort}

\textbf{Descrição:} O Quicksort também utiliza a técnica de \textit{dividir para conquistar}. O algoritmo escolhe um pivô, particiona o vetor em dois subvetores — um com elementos menores ou iguais ao pivô e outro com elementos maiores — e então ordena cada subvetor recursivamente. É rápido na prática, mas pode ter pior caso quadrático se os pivôs forem mal escolhidos.

\begin{exmp}
Considere ordenar o vetor $A = [10, 80, 30, 90, 40, 50, 70]$.  
Escolhendo o pivô como o último elemento ($70$), após a partição temos $[10, 30, 40, 50]$, $70$, $[80, 90]$.  
Recursivamente, o vetor será ordenado até se tornar $[10, 30, 40, 50, 70, 80, 90]$.
\end{exmp}

\begin{algorithm}[H]
\DontPrintSemicolon
\textbf{quickSort(A: array, low: int, high: int)}\;
\If{$low < high$}{
    $p \gets partition(A, low, high)$\;
    quickSort(A, low, p-1)\;
    quickSort(A, p+1, high)\;
}
\caption{Quicksort}
\label{lab:alg-quickSort}
\end{algorithm}

\begin{lstlisting}[language=C, caption={Implementação do Quicksort em C}, label=code:quickSort]
#include <stdio.h>

int partition(int arr[], int low, int high) {
    int pivot = arr[high], i = low-1;
    for (int j=low; j<high; j++) {
        if (arr[j] <= pivot) {
            i++;
            int tmp = arr[i]; arr[i] = arr[j]; arr[j] = tmp;
        }
    }
    int tmp = arr[i+1]; arr[i+1] = arr[high]; arr[high] = tmp;
    return i+1;
}

void quickSort(int arr[], int low, int high) {
    if (low < high) {
        int p = partition(arr, low, high);
        quickSort(arr, low, p-1);
        quickSort(arr, p+1, high);
    }
}
\end{lstlisting}

\begin{lstlisting}[language=Python, caption={Quicksort em Python}, label=code:quickSortPy]
def quicksort(arr):
    if len(arr) <= 1:
        return arr
    pivot = arr[-1]
    left  = [x for x in arr[:-1] if x <= pivot]
    right = [x for x in arr[:-1] if x > pivot]
    return quicksort(left) + [pivot] + quicksort(right)
\end{lstlisting}

\subsection{Análise de complexidade do algoritmo \ref{lab:alg-quickSort}}
No caso médio, a cada partição o vetor é dividido em duas partes quase iguais, e o custo total é:
\[
T(n) = n \log n
\]
\begin{itemize}
    \item Melhor caso: $O(n \log n)$
    \item Caso médio: $O(n \log n)$
    \item Pior caso: $O(n^2)$ (quando sempre escolhe o pior pivô, por exemplo vetor já ordenado).
\end{itemize}
Espaço auxiliar: $O(\log n)$ devido à pilha de recursão.

---


\section{Heapsort}

\textbf{Descrição:} O Heapsort é baseado na estrutura de dados heap (mais especificamente a \textit{max-heap}). O algoritmo primeiro constrói a heap a partir do vetor de entrada e, em seguida, extrai repetidamente o maior elemento, reconstruindo a heap a cada extração, até que todos os elementos estejam ordenados. É in-place e possui complexidade $O(n \log n)$ em todos os casos, mas não é estável.


\begin{exmp}
Para o vetor $A = [4, 10, 3, 5, 1]$, a construção da heap resulta em $[10, 5, 3, 4, 1]$.  
Extraindo sucessivamente o maior elemento e ajustando a heap, obtemos a ordenação final $[1, 3, 4, 5, 10]$.
\end{exmp}

\begin{algorithm}[H]
\DontPrintSemicolon
\textbf{heapSort(A: array, n: int)}\;
construirMaxHeap(A)\;
\For{$i \gets n-1$ \KwTo $1$}{
    trocar $A[0]$ com $A[i]$\;
    reduzir tamanho da heap em 1\;
    maxHeapify(A, 0)\;
}
\caption{Heapsort}
\label{lab:alg-heapSort}
\end{algorithm}

\begin{lstlisting}[language=C, caption={Implementação do Heapsort em C}, label=code:heapSort]
#include <stdio.h>

void heapify(int arr[], int n, int i) {
    int largest = i, l = 2*i+1, r = 2*i+2;
    if (l<n && arr[l]>arr[largest]) largest = l;
    if (r<n && arr[r]>arr[largest]) largest = r;
    if (largest != i) {
        int tmp = arr[i]; arr[i] = arr[largest]; arr[largest] = tmp;
        heapify(arr, n, largest);
    }
}

void heapSort(int arr[], int n) {
    for (int i=n/2-1; i>=0; i--) heapify(arr, n, i);
    for (int i=n-1; i>=0; i--) {
        int tmp = arr[0]; arr[0] = arr[i]; arr[i] = tmp;
        heapify(arr, i, 0);
    }
}
\end{lstlisting}

\begin{lstlisting}[language=Python, caption={Heapsort em Python}, label=code:heapSortPy]
def heapify(arr, n, i):
    largest = i; l, r = 2*i+1, 2*i+2
    if l < n and arr[l] > arr[largest]: largest = l
    if r < n and arr[r] > arr[largest]: largest = r
    if largest != i:
        arr[i], arr[largest] = arr[largest], arr[i]
        heapify(arr, n, largest)

def heapsort(arr):
    n = len(arr)
    for i in range(n//2-1, -1, -1): heapify(arr, n, i)
    for i in range(n-1, 0, -1):
        arr[0], arr[i] = arr[i], arr[0]
        heapify(arr, i, 0)
\end{lstlisting}

\subsection{Análise de complexidade do algoritmo \ref{lab:alg-heapSort}}
A construção da heap custa $O(n)$. Cada remoção do máximo exige $O(\log n)$ para reequilibrar a heap. Como são feitas $n$ remoções:
\[
T(n) = O(n \log n)
\]
\begin{itemize}
    \item Melhor caso: $O(n \log n)$
    \item Caso médio: $O(n \log n)$
    \item Pior caso: $O(n \log n)$
\end{itemize}
Espaço auxiliar: $O(1)$ (in-place).

---


\section{Introsort}

\textbf{Descrição:} O Introsort combina Quicksort, Heapsort e Insertion Sort. Ele começa como um Quicksort, mas monitora a profundidade da recursão; se ultrapassar um limite (tipicamente $2 \log n$), muda para Heapsort, garantindo $O(n \log n)$ no pior caso. Para subvetores pequenos, usa Insertion Sort. É utilizado em bibliotecas padrão como C++ STL.

\begin{exmp}
O Introsort começa como um Quicksort. Caso a profundidade de recursão ultrapasse um limite (tipicamente $2\log n$), ele muda para Heapsort. Em subvetores pequenos, pode usar Insertion Sort. Assim, o Introsort combina a velocidade média do Quicksort com a garantia de $O(n \log n)$ do Heapsort.
\end{exmp}

\begin{algorithm}[H]
\DontPrintSemicolon
\textbf{introSort(A: array, n: int)}\;
profundidadeMax $\gets 2 \cdot \lfloor \log n \rfloor$\;
introSortRec(A, 0, n-1, profundidadeMax)\;
\caption{Introsort}
\label{lab:alg-introSort}
\end{algorithm}

\begin{lstlisting}[language=C, caption={Estrutura simplificada do Introsort em C}, label=code:introSort]
#include <stdio.h>
#include <math.h>

void introSort(int arr[], int n) {
    int depthLimit = 2 * log(n);
    introSortRec(arr, 0, n-1, depthLimit);
}

// introSortRec usa quicksort como padrao,
// mas troca para heapsort se ultrapassar depthLimit
// e insertionSort para pequenos subarrays.
\end{lstlisting}

\begin{lstlisting}[language=Python, caption={Estrutura simplificada do Introsort em Python}, label=code:introSortPy]
def introsort(arr):
    depth_limit = 2 * int(math.log2(len(arr)))
    introsort_rec(arr, 0, len(arr)-1, depth_limit)
# introsort_rec usa quicksort, mas muda para heapsort
# se a profundidade exceder depth_limit
# e insertion sort para subarrays pequenos
\end{lstlisting}

\subsection{Análise de complexidade do algoritmo \ref{lab:alg-introSort}}
\begin{itemize}
    \item Melhor caso: $O(n \log n)$ (como Quicksort eficiente).
    \item Caso médio: $O(n \log n)$.
    \item Pior caso: $O(n \log n)$ (garantido pela troca para Heapsort).
\end{itemize}
Espaço auxiliar: $O(\log n)$ devido à recursão.

---


\section{Timsort}

\href{https://www.algowalker.com/tim-sort.html}{Veja Tim sort}

\textbf{Descrição:} O Timsort é um algoritmo híbrido que combina Insertion Sort e Merge Sort. Ele foi projetado para lidar bem com dados parcialmente ordenados. O vetor é dividido em \textit{runs} (sequências já ordenadas), que são refinadas por Insertion Sort (se pequenas) e depois mescladas por Merge Sort. É o algoritmo padrão em linguagens como Python e Java.

\begin{exmp}
O Timsort divide o vetor em \textit{runs} (subvetores já ordenados). Cada run é ordenada por Insertion Sort (se pequena) e então as runs são mescladas por Merge Sort.  
Por exemplo, o vetor $[5, 21, 7, 23, 19]$ gera runs $[5, 21]$, $[7, 23]$, $[19]$, que são ordenadas e mescladas até formar $[5, 7, 19, 21, 23]$.
\end{exmp}

\begin{algorithm}[H]
\DontPrintSemicolon
\textbf{timSort(A: array, n: int)}\;
dividir A em runs de tamanho fixo\;
ordenar cada run com insertionSort\;
mesclar runs sucessivamente com merge\;
\caption{Timsort}
\label{lab:alg-timSort}
\end{algorithm}

\begin{lstlisting}[language=Python, caption={Estrutura simplificada do Timsort em Python}, label=code:timSort]
def timsort(arr):
    runs = split_into_runs(arr)
    for run in runs:
        insertion_sort(run)
    while len(runs) > 1:
        merged = merge(runs.pop(), runs.pop())
        runs.append(merged)
    return runs[0]
\end{lstlisting}

\begin{lstlisting}[language=Python, caption={Estrutura simplificada do Timsort em Python}, label=code:timSortPy]
def timsort(arr):
    runs = split_into_runs(arr)
    for run in runs:
        insertion_sort(run)
    while len(runs) > 1:
        runs.append(merge(runs.pop(), runs.pop()))
    return runs[0]
\end{lstlisting}

\subsection{Análise de complexidade do algoritmo \ref{lab:alg-timSort}}
\begin{itemize}
    \item Melhor caso: $O(n)$, quando o vetor já está quase ordenado (runs grandes).
    \item Caso médio: $O(n \log n)$.
    \item Pior caso: $O(n \log n)$.
\end{itemize}
Espaço auxiliar: $O(n)$, pois depende de vetores temporários na mesclagem.

\section{Tournament sort}

\section{Tree sort}

\section{Library sort}

\section{Shellsort}

\section{Cube Sort}

\section{Tree Sort}




\section{Resumo}

\begin{center}
\begin{tabular}{||c|c|c|c||}
\hline
\multicolumn{4}{|c|}{Complexidades de tempo em termos de comparações} \\\hline
Algoritmo & Pior caso & Melhor caso & Caso médio\\
\hline
Merge Sort & $O(n\log n)$ & $O(n\log n)$ & $O(n\log n)$ \\
Quick Sort & $O(n^2)$    & $O(n\log n)$ & $O(n\log n)$ \\
Heap Sort  & $O(n\log n)$ & $O(n\log n)$ & $O(n\log n)$ \\
Intro Sort & $O(n\log n)$ & $O(n\log n)$ & $O(n\log n)$ \\
TimSort    & $O(n\log n)$ & $O(n)$       & $O(n\log n)$ \\\hline
\end{tabular}
\end{center}

\begin{center}
\begin{tabular}{||c|c|c|c||}
\hline
\multicolumn{4}{|c|}{Complexidades de espaço} \\\hline
Algoritmo & Pior caso & Melhor caso & Caso médio\\
\hline
Merge Sort & $O(n)$      & $O(n)$      & $O(n)$ \\
Quick Sort & $O(\log n)$ & $O(\log n)$ & $O(\log n)$ \\
Heap Sort  & $O(1)$      & $O(1)$      & $O(1)$ \\
Intro Sort & $O(\log n)$ & $O(\log n)$ & $O(\log n)$ \\
TimSort    & $O(n)$      & $O(n)$      & $O(n)$ \\\hline
\end{tabular}
\end{center}

\chapter{Experimentos computacionais}

Todos os algoritmos descritos nos CAPITULOS anteriores foram implementados nas linguagens de programação C e Python. Utilizamos ... para 
Neste capítulo, analisamos experimentos 
\section{Ambiente Computacional}
\label{sec:Env}
Os experimentos foram executados em um ambiente computacional com a seguinte configuração: USAR O SEGUINTE MODELO 
\begin{itemize}
    \item[-] Computador: Processador Intel Xeon(R) CPU E5-1620 v2, 4 Núcleos de 3,70GHz e 8 GB de Memória RAM;
    \item[-] Sistema Operacional: Linux Ubuntu 16.04 LTS, 64 bits;
    \item[-] Linguagens de programação: \texttt{C} e \texttt{Python};
    \item[-] Compilador da linguagem C: gcc 5.4.0;
    \item[-] Interpretador da linguagem Python: ...;
    \item[-] Função para coleta do tempo de processamento na CPU na linguagem C: \texttt{gettimeofday()};
    \item[-] Função para a coleta do tempo de processamento na CPU na linguagem Python: \texttt{...}.
\end{itemize}

%Os Algoritmos foram implementados na linguagem de programação C e os tempos de processamento foram coletados usando a função \textit{gettimeofday()}.

Na próxima seção descrevemos as instâncias utilizadas nesta pesquisa.

\section{Instâncias}
\label{sec:Inst}
Para a realização dos experimentos, utilizamos os conjuntos de dados:
\begin{itemize}
\item 
\href{https://www.kaggle.com/datasets/bekiremirhanakay/benchmark-dataset-for-sorting-algorithms}{Benchmark Dataset for Sorting Algorithms}

\item 
\href{https://ieeexplore.ieee.org/document/7280062}{Analysis and Testing of Sorting Algorithms on a Standard Dataset}

\item 
\href{https://link.springer.com/chapter/10.1007/978-981-97-5703-9_8}{A Comparative Analysis of Sorting Algorithms for Large-Scale Data: Performance Metrics and Language Efficiency}

\item 
\href{https://www.kaggle.com/datasets/wazahathussain/data-set-for-sorting-algorithm}{Data set for sorting algorithm}

\end{itemize}

\section{Análises}
Criar tabelas, gráficos de barra, gráficos de pizza etc.. 

\chapter{Comentários finais}

\appendix
\chapter{Apêndice A}

\nocite{manber1989introduction,bubble_sort,Harder2020,Astrachan,Halstead1977,Purnomo_Putra_2023}
\bibliographystyle{alpha}
\bibliography{biblio}

\end{document}